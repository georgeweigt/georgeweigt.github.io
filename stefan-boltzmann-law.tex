\documentclass[12pt]{article}
\usepackage[margin=2cm]{geometry}
\usepackage{amsmath}

\begin{document}

\noindent
Josef Stefan determined from experimental data that the total power
emitted by a radiant object is proportional
to the fourth power of its absolute temperature $T$.
Five years later Ludwig Boltzmann showed how to derive the same relation from principles of thermodynamics.
The modern form of the Stefan-Boltzmann law is
$$
P=A\varepsilon\sigma T^4
$$
where $P$ is total power, $A$ is surface area, $\varepsilon$ is an emissivity constant,
and $\sigma$ is the Stefan--Boltzmann constant
$$
\sigma=5.67\times10^{-8}\,\text{watt}\,\text{meter}^{-2}\,\text{kelvin}^{-4}
$$
For example, consider a one cubic centimeter block of wrought iron at 1000 kelvin.
The emissivity constant for wrought iron is $\varepsilon=0.94$
hence the total radiant power is
\begin{equation*}
P=
\underset{\text{surface area 1 cm cube}}
{(6\times10^{-4}\,\text{meter}^2)}
\times
\underset{\varepsilon}
{0.94}
\times
\underset{\sigma}
{(5.67\times10^{-8}\,\text{watt}\,\text{meter}^{-2}\,\text{kelvin}^{-4})}
\times
\underset{T^4}
{1000^4\,\text{kelvin}^4}
=32\,\text{watt}
\end{equation*}

\end{document}
