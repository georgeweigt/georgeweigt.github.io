\documentclass[12pt]{article}
\usepackage{fancyvrb,amsmath,amsfonts,amssymb,graphicx,listings}
\usepackage[usenames,dvipsnames,svgnames,table]{xcolor}
%\usepackage{parskip}
\usepackage{slashed}

% change margins
\addtolength{\oddsidemargin}{-.875in}
\addtolength{\evensidemargin}{-.875in}
\addtolength{\textwidth}{1.75in}
\addtolength{\topmargin}{-.875in}
\addtolength{\textheight}{1.75in}

\hyphenpenalty=10000

\begin{document}

\begin{center}
{\sc stefan-boltzmann law}
\end{center}

\noindent
Josef Stefan determined from experimental data that the total power
emitted by a radiant object is proportional
to the fourth power of its absolute temperature $T$.
Five years later Ludwig Boltzmann showed how to derive the same relation from principles of thermodynamics.
The modern form of the Stefan-Boltzmann law is
$$
P=A\varepsilon\sigma T^4
$$
where $P$ is total power, $A$ is surface area, $\varepsilon$ is an emissivity constant,
and $\sigma$ is the Stefan--Boltzmann constant
$$
\sigma=5.67\times10^{-8}\,{\rm W}\,{\rm m}^{-2}\,{\rm K}^{-4}
$$
For example, consider a one cubic centimeter block of wrought iron at 1000 K.
The emissivity constant of wrought iron is $\varepsilon=0.94$
hence the total radiant power is
$$
P=(6\times10^{-4}\,{\rm m}^2)\times0.94\times
(5.67\times10^{-8}\,{\rm W}\,{\rm m}^{-2}\,{\rm K}^{-4})
\times1000^4\,{\rm K}^4
=32\,{\rm W}
$$
\begin{Verbatim}[formatcom=\color{blue}]
-- www.eigenmath.org/stefan-boltzmann-law.txt
A = 6 10^(-4) meter^2
epsilon = 0.94
sigma = 5.67 10^(-8) watt meter^(-2) kelvin^(-4)
T = 1000 kelvin
A epsilon sigma T^4
\end{Verbatim}

\end{document}
