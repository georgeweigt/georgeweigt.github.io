\documentclass[12pt]{article}
\usepackage{fancyvrb,amsmath,amsfonts,amssymb,graphicx,listings}
\usepackage[usenames,dvipsnames,svgnames,table]{xcolor}
%\usepackage{parskip}
\usepackage{slashed}
\usepackage{tikz}
\usepackage{mathrsfs}

% change margins
\addtolength{\oddsidemargin}{-.875in}
\addtolength{\evensidemargin}{-.875in}
\addtolength{\textwidth}{1.75in}
\addtolength{\topmargin}{-.875in}
\addtolength{\textheight}{1.75in}

\hyphenpenalty=10000

\begin{document}

\begin{center}
{\sc gordon decomposition}
\end{center}

\noindent
Prove the following Gordon decomposition by direct calculation.
Momentum vectors $p_1$ and $p_2$ have the same rest mass $m$.
Each of the spins $s_1$ and $s_2$ can be either up or down.
$$
\bar{u}(p_2,s_2)\gamma^\mu u(p_1,s_1)=
\bar{u}(p_2,s_2)
\left[
\frac{(p_2+p_1)^\mu}{2m}+i\sigma^{\mu\nu}\frac{(p_2-p_1)_\nu}{2m}
\right]
u(p_1,s_1)
$$
The following vectors and spinors are used.
Spinors $u_{11}$ and $u_{21}$ are spin up, $u_{12}$ and $u_{22}$ are spin down.
\begin{gather*}
p_1=\begin{pmatrix}E_1\\p_{1x}\\p_{1y}\\p_{1z}\end{pmatrix}\quad
u_{11}=\begin{pmatrix}E_1+m\\0\\p_{1z}\\p_{1x}+ip_{1y}\end{pmatrix}\quad
u_{12}=\begin{pmatrix}0\\E_1+m\\p_{1x}-ip_{1y}\\-p_{1z}\end{pmatrix}\quad
E_1=\sqrt{p_{1x}^2+p_{1y}^2+p_{1z}^2+m^2}\\
p_2=\begin{pmatrix}E_2\\p_{2x}\\p_{2y}\\p_{2z}\end{pmatrix}\quad
u_{21}=\begin{pmatrix}E_2+m\\0\\p_{2z}\\p_{2x}+ip_{2y}\end{pmatrix}\quad
u_{22}=\begin{pmatrix}0\\E_2+m\\p_{2x}-ip_{2y}\\-p_{2z}\end{pmatrix}\quad
E_2=\sqrt{p_{2x}^2+p_{2y}^2+p_{2z}^2+m^2}
\end{gather*}
Tensor $\sigma^{\mu\nu}$ is defined as
$$
\sigma^{\mu\nu}=\frac{i}{2}\left[\gamma^\mu,\gamma^\nu\right]
=\frac{i}{2}\left(\gamma^\mu\gamma^\nu-\gamma^\nu\gamma^\mu\right)
$$
In component notation we have
$$
\sigma^{\mu\nu\alpha}{}_\beta
=\frac{i}{2}\left(\gamma^{\mu\alpha}{}_\rho\gamma^{\nu\rho}{}_\beta
-\gamma^{\nu\alpha}{}_\rho\gamma^{\mu\rho}{}_\beta\right)
$$
Let $T^{\mu\nu}=\gamma^\mu\gamma^\nu$.
Transpose the first two indices of $\gamma^{\nu\rho}{}_\beta$ to form a dot product.
$$
T^{\mu\nu\alpha}{}_\beta
=\gamma^{\mu\alpha}{}_\rho\gamma^{\rho\nu}{}_\beta
$$
Convert to code.
The transpose on the second and third indices interchanges $\alpha$ and $\nu$.
\begin{equation*}
T^{\mu\nu\alpha}{}_\beta
=\text{\tt transpose(dot(gamma,transpose(gamma)),2,3)}
\end{equation*}
Hence
\begin{equation*}
\sigma^{\mu\nu}=\text{\tt i/2 (T - transpose(T))}
\end{equation*}
where $\text{\tt T}=T^{\mu\nu\alpha}{}_\beta$.
%
Now convert $\sigma^{\mu\nu}(p_2-p_1)_\nu$ to code.
$$
\sigma^{\mu\nu}(p_2-p_1)_\nu
=\sigma^{\mu\alpha}{}_\beta{}^{\nu}g_{\nu\rho}(p_2-p_1)^\rho
=\text{\tt dot(S,gmunu,p2 - p1)}
$$
where $\text{\tt S}=\sigma^{\mu\alpha}{}_\beta{}^\nu
=\text{\tt transpose(transpose(sigmamunu,2,3),3,4)}$.

\newpage
\VerbatimInput[formatcom=\color{blue},samepage=true]{gordon-decomposition.txt}

\end{document}
