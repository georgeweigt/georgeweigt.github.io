\input{preamble}

\section*{Symbols}

Symbols are defined with an equals sign.

{\color{blue}
\begin{verbatim}
N = 212^17
\end{verbatim}
}

No result is printed when a symbol is defined.
To see the value of a symbol, just evaluate it.

{\color{blue}
\begin{verbatim}
N
\end{verbatim}
}

$\displaystyle N=3529471145760275132301897342055866171392$

\bigskip

Symbols can have more that one letter.
Everything after the first letter is displayed as a subscript.

{\color{blue}
\begin{verbatim}
NA = 6.02214 10^23
NA
\end{verbatim}
}

$\displaystyle N_A=6.02214\times10^{23}$

\bigskip

A symbol can be the name of a Greek letter.

{\color{blue}
\begin{verbatim}
xi = 1/2
xi
\end{verbatim}
}

$\displaystyle \xi=\tfrac{1}{2}$

\bigskip

Greek letters can appear in subscripts.

{\color{blue}
\begin{verbatim}
Amu = 2.0
Amu
\end{verbatim}
}

$\displaystyle A_\mu=2.0$

\bigskip

The following example shows how a symbol is scanned to find Greek letters.

{\color{blue}
\begin{verbatim}
alphamunu = 1
alphamunu
\end{verbatim}
}

$\displaystyle \alpha_{\mu\nu}=1$

\bigskip

Symbol definitions are evaluated serially until a terminal symbol is reached.
The following example sets $A=B$ followed by $B=C$.
Then when $A$ is evaluated, the result is $C$.

{\color{blue}
\begin{verbatim}
A = B
B = C
A
\end{verbatim}
}

$\displaystyle A=C$

\bigskip

Although $A=C$ is printed,
inside the program the binding of $A$ is still $B$, as can be seen with
the \verb$binding$ function.

{\color{blue}
\begin{verbatim}
binding(A)
\end{verbatim}
}

$\displaystyle B$

\bigskip

The \verb$quote$ function returns its argument unevaluated
and can be used to clear a symbol.
The following example clears $A$ so that its evaluation goes back to
being $A$ instead of $C$.

{\color{blue}
\begin{verbatim}
A = quote(A)
A
\end{verbatim}
}

$\displaystyle A$

\end{document}
