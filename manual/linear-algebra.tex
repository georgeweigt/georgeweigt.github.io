\documentclass[12pt]{article}
\usepackage[margin=1in]{geometry}
\usepackage{amsmath}
\usepackage{amssymb} % mathbb
\usepackage{graphicx}
\usepackage{xcolor}
\parindent=0pt
\begin{document}

\section*{Linear algebra}

The \verb$dot$ function is used to multiply vectors, matrices, and tensors.
For example, let
\begin{equation*}
A=\begin{pmatrix}1&2\\3&4\end{pmatrix},
\quad
x=\begin{pmatrix}x_1\\x_2\end{pmatrix}
\end{equation*}

The product $Ax$ is computed as follows.

{\color{blue}
\begin{verbatim}
A = ((1,2),(3,4))
x = (x1,x2)
dot(A,x)
\end{verbatim}
}

$\displaystyle
\begin{bmatrix}
x_1+2x_2
\\[1ex]
3x_1+4x_2
\end{bmatrix}
$

\bigskip

The following example shows how to use \verb$dot$ and \verb$inv$ to solve for
vector $X$ in $AX=B$.

{\color{blue}
\begin{verbatim}
A = ((3,7),(1,-9))
B = (16,-22)
X = dot(inv(A),B)
X
\end{verbatim}
}

$\displaystyle
X=
\begin{bmatrix}
-\tfrac{5}{17}
\\
\\
\tfrac{41}{17}
\end{bmatrix}
$

\bigskip

The \verb$dot$ function can have more than two arguments.
For example, \verb$dot(A,B,C)$ can be used for the dot product of three tensors.

\bigskip

Square brackets are used for component access.
Index numbering starts with 1.

{\color{blue}
\begin{verbatim}
A = ((a,b),(c,d))
A[1,2] = -A[1,1]
A
\end{verbatim}
}

$\displaystyle
\begin{bmatrix}
a & -a
\\[1ex]
c & d
\end{bmatrix}
$

\bigskip

The following example demonstrates the relation
$A^{-1}=(\operatorname{det}A)^{-1}\operatorname{adj}A$.

{\color{blue}
\begin{verbatim}
A = ((a,b),(c,d))
inv(A) == adj(A) / det(A)
\end{verbatim}
}

$\displaystyle 1$

\bigskip

Sometimes a calculation will be simpler if it can be reorganized to use
\verb$adj$ instead of \verb$inv$.
The main idea is to try to prevent the determinant from appearing as a
divisor.
For example, suppose for matrices $A$ and $B$ you want to show that
\begin{equation*}
{A}-{B}^{-1}=0
\end{equation*}

Depending on the complexity of $\mathop{\rm det}B$, the software
may not be able to find a simplification that yields zero.
Should that occur, the following alternative formulation can be tried.
\begin{equation*}
A\operatorname{det}B-\operatorname{adj}B=0
\end{equation*}

\end{document}
