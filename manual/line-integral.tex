\documentclass[12pt]{article}
\usepackage[margin=1in]{geometry}
\usepackage{amsmath}
\usepackage{xcolor}
\parindent=0pt
\pagestyle{empty}
\begin{document}

\section*{Line integral}

There are two kinds of line integrals,
one for scalar fields and one for vector fields.
The following table shows how both are based on the calculation of
arc length.

\begin{center}
\begin{tabular}{|l|l|l|}
\hline
& Abstract form
& Computable form
\\
\hline
 & &\\
Arc length
& $\displaystyle{\int_C ds}$
& $\displaystyle{\int_a^b |g'(t)|\,dt}$\\
 & &\\
\hline
 & & \\
Line integral, scalar field
& $\displaystyle{\int_C f\,ds}$
& $\displaystyle{\int_a^b f(g(t))\,|g'(t)|\,dt}$\\
& &\\
\hline
 & & \\
Line integral, vector field
& $\displaystyle{\int_C(F\cdot u)\,ds}$
& $\displaystyle{\int_a^b F(g(t))\cdot g'(t)\,dt}$\\
 & & \\
\hline
\end{tabular}
\end{center}

Note that for the measure $ds$ we have
\begin{equation*}
ds=|g'(t)|\,dt
\end{equation*}

For vector fields, symbol $u$ is the unit tangent vector
\begin{equation*}
u=\frac{g'(t)}{|g'(t)|}
\end{equation*}

Note that $u$ cancels with $ds$ as follows.
\begin{equation*}
\int_C(F\cdot u)\,ds
=\int_a^b
\left(F(g(t))\cdot\frac{g'(t)}{|g'(t)|}\right)
|g'(t)|\,dt
=\int_a^b F(g(t))\cdot g'(t)\,dt
\end{equation*}

Example 1. Evaluate
\begin{equation*}
\int_Cx\,ds\quad\hbox{and}\quad\int_Cx\,dx
\end{equation*}

where $C$ is a straight line from $(0,0)$ to $(1,1)$.

\bigskip
Although the integrals appear similar,
the first is over a scalar field and the second is over a vector field.

\bigskip
For $\int_Cx\,ds$ we have

{\color{blue}
\begin{verbatim}
x = t
y = t
g = (x,y)
defint(x abs(d(g,t)), t, 0, 1)
\end{verbatim}}

$\displaystyle \frac{1}{2^{1/2}}$

\bigskip
For $\int_Cx\,dx$ we have

{\color{blue}
\begin{verbatim}
x = t
y = t
g = (x,y)
F = (x,0)
defint(dot(F,d(g,t)), t, 0, 1)
\end{verbatim}}

$\displaystyle \tfrac{1}{2}$

\bigskip
The following line integral problems are from
{\it Advanced Calculus, Fifth Edition} by Wilfred Kaplan.

\bigskip
Example 2. Evaluate $\int y^2\,dx$ along the straight
line from $(0,0)$ to $(2,2)$.

\bigskip
The following solution parametrizes $x$ and $y$ so that
the endpoint $(2,2)$ corresponds to $t=1$.

{\color{blue}
\begin{verbatim}
x = 2 t
y = 2 t
g = (x,y)
F = (y^2,0)
defint(dot(F,d(g,t)), t, 0, 1)
\end{verbatim}}

$\displaystyle \tfrac{8}{3}$

\bigskip
Example 3. Evaluate $\int z\,dx+x\,dy+y\,dz$
along the path
$x=2t+1$, $y=t^2$, $z=1+t^3$, $0\le t\le 1$.

{\color{blue}
\begin{verbatim}
x = 2 t + 1
y = t^2
z = 1 + t^3
g = (x,y,z)
F = (z,x,y)
defint(dot(F,d(g,t)), t, 0, 1)
\end{verbatim}}

$\displaystyle \tfrac{163}{30}$

\end{document}
