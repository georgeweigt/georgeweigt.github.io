\documentclass[12pt]{article}
\usepackage[margin=1in]{geometry}
\usepackage{amsmath}
\usepackage{xcolor}
\parindent=0pt
\begin{document}

\section*{Line integral}

There are two different kinds of line integrals,
one for scalar fields and one
for vector fields.
The following table shows how both are based on the calculation of
arc length.

\begin{center}
\begin{tabular}{|l|l|l|}
\hline
& Abstract form
& Computable form
\\
\hline
 & &\\
Arc length
& $\displaystyle{\int_C ds}$
& $\displaystyle{\int_a^b |g'(t)|\,dt}$\\
 & &\\
\hline
 & & \\
Line integral, scalar field
& $\displaystyle{\int_C f\,ds}$
& $\displaystyle{\int_a^b f(g(t))\,|g'(t)|\,dt}$\\
& &\\
\hline
 & & \\
Line integral, vector field
& $\displaystyle{\int_C(F\cdot u)\,ds}$
& $\displaystyle{\int_a^b F(g(t))\cdot g'(t)\,dt}$\\
 & & \\
\hline
\end{tabular}
\end{center}

For the vector field form, the symbol $u$ is the unit tangent vector
$$u=\frac{g'(t)}{|g'(t)|}$$
The length of the tangent vector cancels with $ds$
as follows.
$$\int_C(F\cdot u)\,ds
=\int_a^b\bigg(F(g(t))\cdot\frac{g'(t)}{|g'(t)|}\bigg)\,\bigg(|g'(t)|\,dt\bigg)
=\int_a^b F(g(t))\cdot g'(t)\,dt
$$

Evaluate
$$\int_Cx\,ds\quad\hbox{and}\quad\int_Cx\,dx$$
where $C$ is a straight line from $(0,0)$ to $(1,1)$.

\bigskip
What a difference the measure makes.
The first integral is over a scalar field and the second is over a vector field.
This can be understood when we recall that
$$ds=|g'(t)|\,dt
$$
Hence for $\int_Cx\,ds$ we have

{\color{blue}
\begin{verbatim}
x = t
y = t
g = (x,y)
defint(x abs(d(g,t)),t,0,1)
\end{verbatim}
}

$\displaystyle \frac{1}{2^{1/2}}$

\bigskip
For $\int_Cx\,dx$ we have

{\color{blue}
\begin{verbatim}
x = t
y = t
g = (x,y)
F = (x,0)
defint(dot(F,d(g,t)),t,0,1)
\end{verbatim}
}

$\displaystyle \tfrac{1}{2}$

\bigskip
The following line integral problems are from
{\it Advanced Calculus, Fifth Edition} by Wilfred Kaplan.

\bigskip
Evaluate $\int y^2\,dx$ along the straight
line from $(0,0)$ to $(2,2)$.

{\color{blue}
\begin{verbatim}
x = 2t
y = 2t
g = (x,y)
F = (y^2,0)
defint(dot(F,d(g,t)),t,0,1)
\end{verbatim}
}

$\displaystyle \tfrac{8}{3}$

\bigskip
Evaluate $\int z\,dx+x\,dy+y\,dz$
along the path
$x=2t+1$, $y=t^2$, $z=1+t^3$, $0\le t\le 1$.

{\color{blue}
\begin{verbatim}
x = 2t+1
y = t^2
z = 1+t^3
g = (x,y,z)
F = (z,x,y)
defint(dot(F,d(g,t)),t,0,1)
\end{verbatim}
}

$\displaystyle \tfrac{163}{30}$

\end{document}
