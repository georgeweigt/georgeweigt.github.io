\input{preamble}

\section*{Green's theorem}

This is Green's theorem.
\begin{equation*}
\oint\left(P\,dx+Q\,dy\right)
=\int\!\!\!\int
\left(\frac{\partial Q}{\partial x}-\frac{\partial P}{\partial y}\right)
\,dx\,dy
\end{equation*}

In words, a line integral and a surface integral can yield
the same result.

\bigskip
Example 1.
The following exercise is from {\it Advanced Calculus}
by Wilfred Kaplan, p.~287.
Evaluate $\oint (2x^3-y^3)\,dx+(x^3+y^3)\,dy$ around the circle
$x^2+y^2=1$ using Green's theorem.

\bigskip
Use polar coordinates to solve the surface integral.

{\color{blue}
\begin{verbatim}
P = 2x^3 - y^3
Q = x^3 + y^3
f = d(Q,x) - d(P,y)
x = r cos(theta)
y = r sin(theta)
defint(f r, r, 0, 1, theta, 0, 2 pi)
\end{verbatim}
}

$\displaystyle \tfrac{3}{2}\pi$

\bigskip
The \verb$defint$ integrand is $f\,r$ because $r\,dr\,d\theta=dx\,dy$.

\bigskip
Now let us try computing the line integral side of Green's theorem
and see if we get the same result.
We need to use the trick of converting sine and cosine to exponentials
so that Eigenmath can find the solution.

{\color{blue}
\begin{verbatim}
x = cos(t)
y = sin(t)
P = 2x^3 - y^3
Q = x^3 + y^3
f = P d(x,t) + Q d(y,t)
f = circexp(f)
defint(f, t, 0, 2 pi)
\end{verbatim}
}

$\displaystyle \tfrac{3}{2}\pi$

\bigskip
Example 2.
Compute both sides of Green's theorem for
$F=(1-y,x)$ over the disk $x^2+y^2\le4$.

\bigskip
First compute the line integral along the boundary of the disk.
Note that the radius of the disk is 2.

{\color{blue}
\begin{verbatim}
-- Line integral
P = 1 - y
Q = x
x = 2 cos(t)
y = 2 sin(t)
defint(P d(x,t) + Q d(y,t),t,0,2pi)
\end{verbatim}
}

$\displaystyle 8\pi$

{\color{blue}
\begin{verbatim}
-- Surface integral
x = quote(x) -- clear x
y = quote(y) -- clear y
h = sqrt(4 - x^2)
defint(d(Q,x) - d(P,y), y, -h, h, x, -2, 2)
\end{verbatim}
}

$\displaystyle 8\pi$

{\color{blue}
\begin{verbatim}
-- use polar coordinates
P = 1 - y
Q = x
f = d(Q,x) - d(P,y) -- do before change of coordinates
x = r cos(theta)
y = r sin(theta)
defint(f r, r, 0, 2, theta, 0, 2 pi)
\end{verbatim}
}

$\displaystyle 8\pi$

{\color{blue}
\begin{verbatim}
defint(f r, theta, 0, 2 pi, r, 0, 2) -- integrate over theta first
\end{verbatim}
}

$\displaystyle 8\pi$

\bigskip
In this case, Eigenmath solved both forms of the polar integral.
However, in cases where Eigenmath fails to solve a double integral, try
changing the order of integration.

\end{document}
