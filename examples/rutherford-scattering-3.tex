\input{preamble}

\section*{Rutherford scattering 3}

Rutherford scattering is the interaction
$e^-+\text{atom}\rightarrow e^-+\text{atom}$.

\begin{center}
\begin{tikzpicture}
\draw[dashed] (0,0) circle (0.5cm);
\draw (0,0) node {atom};
\draw[thick,->] (-2,0) node[anchor=east] {$e^-$} -- (-0.6,0);
\draw[thick,->] (0.40,0.40) -- (1.3,1.3) node[anchor=south west] {$e^-$};
\draw (1,0.5) node {$\theta$};
\end{tikzpicture}
\end{center}

Momentum vectors for Rutherford scattering where $E=\sqrt{p^2+m^2}$.
\begin{equation*}
p_1=\underset{\substack{\text{inbound}\\ \text{electron}}}
{\begin{pmatrix}E\\0\\0\\p\end{pmatrix}},
\quad
p_2=\underset{\substack{\text{outbound}\\ \text{electron}}}
{\begin{pmatrix}
E\\
p\sin\theta\cos\phi\\
p\sin\theta\sin\phi\\
p\cos\theta
\end{pmatrix}}
\end{equation*}

Spinors for the inbound electron.
\begin{equation*}
u_{11}=\frac{1}{\sqrt{E+m}}
\underset{\substack{\text{inbound electron}\\ \text{spin up}}}
{\begin{pmatrix}E+m\\0\\p\\0\end{pmatrix}},
\quad
u_{12}=\frac{1}{\sqrt{E+m}}
\underset{\substack{\text{inbound electron}\\ \text{spin down}}}
{\begin{pmatrix}0\\E+m\\0\\-p\end{pmatrix}}
\end{equation*}

Spinors for the outbound electron.
\begin{equation*}
u_{21}=\frac{1}{\sqrt{E+m}}
\underset{\substack{\text{outbound electron}\\ \text{spin up}}}
{\begin{pmatrix}E+m\\0\\p_{2z}\\p_{2x}+ip_{2y}\end{pmatrix}},
\quad
u_{22}=\frac{1}{\sqrt{E+m}}
\underset{\substack{\text{outbound electron}\\ \text{spin down}}}
{\begin{pmatrix}0\\E+m\\p_{2x}-ip_{2y}\\-p_{2z}\end{pmatrix}}
\end{equation*}

The probability density $|\mathcal M_{ab}|^2$ for spin state $ab$.
\begin{equation*}
|\mathcal{M}_{ab}|^2=\frac{Z^2e^4}{q^4}\left|\bar{u}_{2b}\gamma^0 u_{1a}\right|^2
\end{equation*}

$q$ is momentum transfer such that
\begin{equation*}
q^4=(p_1-p_2)^4=\left[(p_1-p_2)^\mu g_{\mu\nu}(p_1-p_2)^\nu\right]^2=4p^4(\cos\theta-1)^2
\end{equation*}

The expected probability density
$\langle\vert\mathcal{M}\vert^2\rangle$
is the average of spin states.
\begin{equation*}
\langle\vert\mathcal{M}\vert^2\rangle
=\frac{1}{2}\sum_{a=1}^2\sum_{b=1}^2\left|\mathcal{M}_{ab}\right|^2
\end{equation*}

The Casimir trick uses matrix arithmetic to sum over spin states.
\begin{equation*}
\langle\vert\mathcal{M}\vert^2\rangle
=\frac{Z^2e^4}{2q^4}\mathop{\rm Tr}\left[(\slashed{p}_1+m)\gamma^0(\slashed{p}_2+m)\gamma^0\right]
\end{equation*}

The result is
\begin{equation*}
\langle\vert\mathcal{M}\vert^2\rangle
=\frac{2Z^2e^4}{q^4}\left(E^2+m^2+p^2\cos\theta\right)
\end{equation*}

For low energy experiments such that $p\ll m$ we can use the approximation
\begin{equation*}
E^2+m^2+p^2\cos\theta\approx2m^2
\end{equation*}

Hence
\begin{equation*}
\langle|\mathcal{M}|^2\rangle=\frac{4Z^2e^4m^2}{q^4}
\end{equation*}

Substituting $e^4=16\pi^2\alpha^2$ and $q^4=4p^4(\cos\theta-1)^2$ we have
\begin{equation*}
\langle|\mathcal{M}|^2\rangle=\frac{16\pi^2Z^2\alpha^2m^2}{p^4(\cos\theta-1)^2}
\end{equation*}

\subsubsection*{Cross section}
The differential cross section for Rutherford scattering is
\begin{equation*}
\frac{d\sigma}{d\Omega}=\frac{\langle|\mathcal{M}|^2\rangle}{16\pi^2}
\end{equation*}

For low energy experiments we have
\begin{equation*}
\langle|\mathcal{M}|^2\rangle=\frac{16\pi^2Z^2\alpha^2m^2}{p^4(\cos\theta-1)^2}
\end{equation*}

Hence for low energy experiments
\begin{equation*}
\frac{d\sigma}{d\Omega}=\frac{Z^2\alpha^2m^2}{p^4(\cos\theta-1)^2}
\end{equation*}

Noting that
\begin{equation*}
d\Omega=\sin\theta\,d\theta\,d\phi
\end{equation*}

we also have
\begin{equation*}
d\sigma=\frac{Z^2\alpha^2m^2}{p^4(\cos\theta-1)^2}
\sin\theta\,d\theta\,d\phi
\end{equation*}

Let $S(\theta_1,\theta_2)$ be the following surface integral of $d\sigma$.
\begin{equation*}
S(\theta_1,\theta_2)=\int_0^{2\pi}\int_{\theta_1}^{\theta_2}d\sigma
\end{equation*}

The solution is
\begin{equation*}
S(\theta_1,\theta_2)=\frac{2\pi Z^2\alpha^2m^2}{p^4}\bigl(I(\theta_2)-I(\theta_1)\bigr)
\end{equation*}

where
\begin{equation*}
I(\theta)=\frac{1}{\cos\theta-1}
\end{equation*}

The cumulative distribution function is
\begin{equation*}
F(\theta)
=\frac{S(a,\theta)}{S(a,\pi)}
=\frac{I(\theta)-I(a)}{I(\pi)-I(a)}
=\frac{2(\cos a-\cos\theta)}{(1+\cos a)(1-\cos\theta)},
\quad
a\le\theta\le\pi
\end{equation*}

Angular support is reduced by an arbitrary angle $a>0$ because $I(0)$ is undefined.

\bigskip
The probability of observing scattering events
in the interval $\theta_1$ to $\theta_2$ is
\begin{equation*}
P(\theta_1\le\theta\le\theta_2)=F(\theta_2)-F(\theta_1)
\end{equation*}

The probability density function is
\begin{equation*}
f(\theta)=\frac{dF(\theta)}{d\theta}=\frac{1}{I(\pi)-I(a)}\frac{1}{(\cos\theta-1)^2}\sin\theta
\end{equation*}

\subsubsection*{Notes}
1. The original Rutherford scattering experiment in 1911 used alpha particles, not electrons.
However, scattering of any charged particle by Coulomb interaction
is now known as Rutherford scattering.
The first Rutherford scattering experiment using electrons appears to have
been done by F.~L.~Arnot, then a student of Rutherford, in 1929.

\bigskip
2. Lancaster and Blundell page 356 has
\begin{equation*}
\frac{d\sigma}{d\Omega}
=\frac{Z^2\alpha^2}{4m^2\mathbf v^4\sin^4(\theta/2)}
\end{equation*}

Noting that
\begin{equation*}
\frac{1}{m^2\mathbf v^4}=\frac{m^2}{m^4\mathbf v^4}=\frac{m^2}{p^4}
\end{equation*}

and
\begin{equation*}
4\sin^4(\theta/2)=(\cos\theta-1)^2
\end{equation*}

we have
\begin{equation*}
\frac{Z^2\alpha^2}{4m^2\mathbf v^4\sin^4(\theta/2)}
=\frac{Z^2\alpha^2m^2}{p^4(\cos\theta-1)^2}
\end{equation*}

\end{document}
