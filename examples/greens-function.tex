\input{preamble}

\section*{Green's function}

In this section we will find the Green's function $G(\mathbf x)$ such that
\begin{equation*}
(\nabla^2+k^2)G(\mathbf x)=\delta^3(\mathbf x)
\tag{1}
\end{equation*}

Let $g(\mathbf y)$ be the Fourier transform of $G(\mathbf x)$ such that
\begin{equation*}
G(\mathbf x)=\frac{1}{(2\pi)^\frac{3}{2}}\int\exp(i\mathbf x\cdot\mathbf y)g(\mathbf y)\,d\mathbf y
\tag{2}
\end{equation*}

Substitute (2) into (1) to obtain
\begin{equation*}
(\nabla^2+k^2)
\left[\frac{1}{(2\pi)^\frac{3}{2}}\int\exp(i\mathbf x\cdot\mathbf y)g(\mathbf y)\,d\mathbf y\right]
=\delta(\mathbf x)
\end{equation*}

By linearity of differentiation the $(\nabla^2+k^2)$ can be moved inside the integral.
\begin{equation*}
\frac{1}{(2\pi)^\frac{3}{2}}
\int(\nabla^2+k^2)\exp(i\mathbf x\cdot\mathbf y)g(\mathbf y)\,d\mathbf y
=\delta(\mathbf x)
\end{equation*}

Noting that
\begin{equation*}
\nabla^2\exp(i\mathbf x\cdot\mathbf y)=-y^2\exp(i\mathbf x\cdot\mathbf y)
\end{equation*}

and
\begin{equation*}
\delta(\mathbf x)=\frac{1}{(2\pi)^3}\int\exp(i\mathbf x\cdot\mathbf y)\,d\mathbf y
\end{equation*}

we have
\begin{equation*}
\frac{1}{(2\pi)^\frac{3}{2}}
\int(-y^2+k^2)\exp(i\mathbf x\cdot\mathbf y)g(\mathbf y)\,d\mathbf y
=\frac{1}{(2\pi)^3}\int\exp(i\mathbf x\cdot\mathbf y)\,d\mathbf y
\end{equation*}

Hence
\begin{equation*}
g(\mathbf y)=\frac{1}{(2\pi)^\frac{3}{2}(k^2-y^2)}
\end{equation*}

Substitute for $g(\mathbf y)$ in (2) to obtain
\begin{equation*}
G(\mathbf x)=\frac{1}{(2\pi)^3}\int\frac{\exp(i\mathbf x\cdot\mathbf y)}{k^2-y^2}\,d\mathbf y
\end{equation*}

Change to polar coordinates where $x=|\mathbf x|$, $y=|\mathbf y|$,
and $\theta$ and $\phi$ are the angular distance from $\mathbf x$ to $\mathbf y$.
\begin{equation*}
G(\mathbf x)=\frac{1}{(2\pi)^3}
\int_0^\infty\int_0^\pi\int_0^{2\pi}
\frac{\exp(ixy\cos\theta)}{k^2-y^2}\,y^2\sin\theta\,dy\,d\theta\,d\phi
\end{equation*}

For the integrals over $\theta$ and $\phi$ we have
\begin{equation*}
\int_0^\pi\int_0^{2\pi}\exp(ixy\cos\theta)\sin\theta\,d\theta\,d\phi=\frac{4\pi\sin(xy)}{xy}
\end{equation*}

Hence
\begin{equation*}
G(\mathbf x)=\frac{1}{2\pi^2x}
\int_0^\infty\frac{y\sin(xy)}{k^2-y^2}\,dy
\end{equation*}

Noting that $y\sin(xy)$ is an even function of $y$ we can change the integral limits as follows.
\begin{equation*}
G(\mathbf x)=\frac{1}{4\pi^2x}
\int_{-\infty}^\infty\frac{y\sin(xy)}{k^2-y^2}\,dy
\end{equation*}

Negate the denominator.
\begin{equation*}
G(\mathbf x)=\frac{1}{4\pi^2x}
\int_{-\infty}^\infty-\frac{y\sin(xy)}{y^2-k^2}\,dy
\end{equation*}

Change the sine function to exponential form and factor the denominator.
\begin{equation*}
G(\mathbf x)
=\frac{i}{8\pi^2x}
\left[
\int_{-\infty}^\infty\frac{y\exp(ixy)}{(y-k)(y+k)}\,dy
-\int_{-\infty}^\infty\frac{y\exp(-ixy)}{(y-k)(y+k)}\,dy
\right]
\end{equation*}

By Cauchy's integral formula we have
\begin{equation*}
\int_{-\infty}^\infty\frac{y\exp(ixy)}{y+k}\frac{1}{y-k}\,dy
=i\pi\exp(ikx)
\end{equation*}

and
\begin{equation*}
\int_{-\infty}^\infty\frac{y\exp(-ixy)}{y-k}\frac{1}{y+k}\,dy
=-i\pi\exp(ikx)
\end{equation*}

Hence
\begin{equation*}
\boxed{
G(\mathbf x)
=-\frac{\exp(ikx)}{4\pi x}
}
\tag{3}
\end{equation*}

where
\begin{equation*}
x=|\mathbf x|
\end{equation*}

Verify that (3) satisfies (1).

\bigskip
We will need the following formula from Griffiths and Schroeter problem 10.8.
\begin{equation*}
\nabla^2(1/r)=-4\pi\delta^3(\mathbf r)
\tag{4}
\end{equation*}

Recall that $\nabla^2=\nabla\cdot\nabla$ and
\begin{equation*}
\nabla\cdot(f\mathbf A)=\nabla f\cdot\mathbf A+f\nabla\mathbf\cdot\mathbf A
\end{equation*}

Substituting $\mathbf r$ for $\mathbf x$ in (3) we have for the Laplacian of $G(\mathbf r)$
\begin{align*}
\nabla^2G(\mathbf r)&=-\frac{1}{4\pi}\nabla\cdot\nabla\left(\frac{e^{ikr}}{r}\right)
\\
&=-\frac{1}{4\pi}\nabla\cdot\left(\frac{1}{r}\nabla e^{ikr}+e^{ikr}\nabla\frac{1}{r}\right)
\\
&=-\frac{1}{4\pi}\biggl(
\underset{\text{subst.~(6)}}{\nabla\frac{1}{r}\cdot\nabla e^{ikr}}
+\underset{\text{subst.~(7)}}{\frac{1}{r}\nabla^2e^{ikr}}
+\underset{\text{subst.~(6)}}{\nabla e^{ikr}\cdot\nabla\frac{1}{r}}
+\underset{\text{subst.~(4)}}{e^{ikr}\nabla^2\frac{1}{r}}
\biggr)\tag{5}
\end{align*}

In spherical coordinates
\begin{equation*}
\nabla\frac{1}{r}\cdot\nabla e^{ikr}
=\left(-\frac{1}{r^2}\mathbf e_r+0\mathbf e_\theta+0\mathbf e_\phi\right)
\cdot
\left(ike^{ikr}\mathbf e_r+0\mathbf e_\theta+0\mathbf e_\phi\right)
=-\frac{ike^{ikr}}{r^2}
\tag{6}
\end{equation*}

and
\begin{align*}
\frac{1}{r}\nabla^2e^{ikr}&=\frac{1}{r^2}\frac{\partial^2}{\partial r^2}(re^{ikr})
\\
&=\frac{1}{r^2}\frac{\partial}{\partial r}\left(e^{ikr}+ikre^{ikr}\right)
\\
&=\frac{1}{r^2}\left(2ike^{ikr}-k^2re^{ikr}\right)
\\
&=\frac{2ike^{ikr}}{r^2}-\frac{k^2e^{ikr}}{r}
\tag{7}
\end{align*}

Substitute (4), (6), and (7) into (5) to obtain
\begin{equation*}
\nabla^2G(\mathbf r)=\frac{k^2e^{ikr}}{4\pi r}+\delta^3(\mathbf r)e^{ikr}
\end{equation*}

Then by equation (3)
\begin{equation*}
\nabla^2G(\mathbf r)=-k^2G(\mathbf r)+\delta^3(\mathbf r)e^{ikr}
\end{equation*}

Noting that $e^{ikr}=1$ at $r=0$, the $e^{ikr}$ term can be discarded leaving
\begin{equation*}
\nabla^2G(\mathbf r)=-k^2G(\mathbf r)+\delta^3(\mathbf r)
\tag{8}
\end{equation*}

Hence
\begin{equation*}
(\nabla^2+k^2)G(\mathbf r)=\delta^3(\mathbf r)
\end{equation*}

\href{https://georgeweigt.github.io/examples/greens-function-demo.html}{Eigenmath code}

\end{document}
