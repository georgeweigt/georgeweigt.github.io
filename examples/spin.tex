\input{preamble}

% see Fitzpatrick problem 10-2

\section*{Spin}

Spin state $|s\rangle$ is a normalized vector in $\mathbb C^2$.
\begin{equation*}
|s\rangle=\begin{pmatrix}c_+\\c_-\end{pmatrix},\quad|c_+|^2+|c_-|^2=1
\end{equation*}

Spin measurement probabilities are the transition probabilities from $|s\rangle$ to an eigenstate.

\bigskip
For spin measurements in the $z$ direction we have
\begin{align*}
\Pr\left(S_z=+\tfrac{\hbar}{2}\right)&=|\langle z_+|s\rangle|^2
\\
\Pr\left(S_z=-\tfrac{\hbar}{2}\right)&=|\langle z_-|s\rangle|^2
\end{align*}

Define the $z$ eigenstates as
\begin{equation*}
|z_+\rangle=\begin{pmatrix}1\\0\end{pmatrix},\quad
|z_-\rangle=\begin{pmatrix}0\\1\end{pmatrix}
\end{equation*}

By definition of expectation value we have
\begin{equation*}
\langle S_z\rangle
=\tfrac{\hbar}{2}\Pr\left(S_z=+\tfrac{\hbar}{2}\right)
-\tfrac{\hbar}{2}\Pr\left(S_z=-\tfrac{\hbar}{2}\right)
\end{equation*}

Rewrite as
\begin{equation*}
\langle S_z\rangle=\tfrac{\hbar}{2}|\langle z_+|s\rangle|^2-\tfrac{\hbar}{2}|\langle z_-|s\rangle|^2
\end{equation*}

Rewrite again as
\begin{equation*}
\langle S_z\rangle
=\tfrac{\hbar}{2}\langle s|z_+\rangle\langle z_+|s\rangle
-\tfrac{\hbar}{2}\langle s|z_-\rangle\langle z_-|s\rangle
\end{equation*}

Then by
\begin{equation*}
\langle S_z\rangle=\langle s|S_z|s\rangle
\end{equation*}

we have
\begin{equation*}
S_z=\tfrac{\hbar}{2}|z_+\rangle\langle z_+|-\tfrac{\hbar}{2}|z_-\rangle\langle z_-|
=\frac{\hbar}{2}\begin{pmatrix}1&0\\0&-1\end{pmatrix}
\end{equation*}

From the commutator
\begin{equation*}
S_+S_--S_-S_+=2\hbar S_z
\end{equation*}

we have
\begin{equation*}
S_+S_--S_-S_+
=\hbar^2|z_+\rangle\langle z_+|-\hbar^2|z_-\rangle\langle z_-|
\end{equation*}

Rewrite as
\begin{equation*}
S_+S_--S_-S_+
=\hbar^2|z_+\rangle\langle z_-|z_-\rangle\langle z_+|
-\hbar^2|z_-\rangle\langle z_+|z_+\rangle\langle z_-|
\end{equation*}

Hence
\begin{align*}
S_+&=\hbar|z_+\rangle\langle z_-|=\hbar\begin{pmatrix}0&1\\0&0\end{pmatrix}
\\
S_-&=\hbar|z_-\rangle\langle z_+|=\hbar\begin{pmatrix}0&0\\1&0\end{pmatrix}
\end{align*}

Then by
\begin{align*}
S_+&=S_x+iS_y
\\
S_-&=S_x-iS_y
\end{align*}

we obtain
\begin{align*}
S_x&=\frac{S_++S_-}{2}=\frac{\hbar}{2}\begin{pmatrix}0&1\\1&0\end{pmatrix}
\\
S_y&=\frac{S_+-S_-}{2i}=\frac{\hbar}{2}\begin{pmatrix}0&-i\\i&0\end{pmatrix}
\end{align*}

By solving for the eigenstates in
\begin{align*}
S_x|x_\pm\rangle&=\pm\tfrac{\hbar}{2}|x_\pm\rangle
\\
S_y|y_\pm\rangle&=\pm\tfrac{\hbar}{2}|y_\pm\rangle
\end{align*}

we obtain
\begin{align*}
|x_+\rangle&=\frac{|z_+\rangle+|z_-\rangle}{\sqrt2}=\frac{1}{\sqrt2}\begin{pmatrix}1\\1\end{pmatrix}
\\
|x_-\rangle&=\frac{|z_+\rangle-|z_-\rangle}{\sqrt2}=\frac{1}{\sqrt2}\begin{pmatrix}1\\-1\end{pmatrix}
\end{align*}

and
\begin{align*}
|y_+\rangle&=\frac{|z_+\rangle+i|z_-\rangle}{\sqrt2}=\frac{1}{\sqrt2}\begin{pmatrix}1\\i\end{pmatrix}
\\
|y_-\rangle&=\frac{|z_+\rangle-i|z_-\rangle}{\sqrt2}=\frac{1}{\sqrt2}\begin{pmatrix}1\\-i\end{pmatrix}
\end{align*}

The expected spin direction vector is
\begin{equation*}
\langle\mathbf S\rangle=\begin{pmatrix}
\langle s|S_x|s\rangle\\
\langle s|S_y|s\rangle\\
\langle s|S_z|s\rangle
\end{pmatrix},\quad
|\langle\mathbf S\rangle|=\frac{\hbar}{2}
\end{equation*}

To convert a direction vector to a spin state, let
\begin{equation*}
\langle\mathbf S\rangle=\frac{\hbar}{2}\begin{pmatrix}
\sin\theta\cos\phi\\
\sin\theta\sin\phi\\
\cos\theta
\end{pmatrix}
\end{equation*}

Then
\begin{equation*}
|s\rangle=\begin{pmatrix}\cos(\theta/2)\\\sin(\theta/2)\exp(i\phi)\end{pmatrix}
\end{equation*}

By the identities
\begin{equation*}
\cos^2(\theta/2)=\frac{\cos\theta+1}{2},\quad
\sin^2(\theta/2)=\frac{1-\cos\theta}{2}
\end{equation*}

and noting that $0\le\theta\le\pi$ we have
\begin{equation*}
\cos(\theta/2)=\sqrt{\frac{\langle z\rangle+1}{2}},\quad
\sin(\theta/2)\exp(i\phi)=\sqrt{\frac{1-\langle z\rangle}{2}}
\frac{\langle x\rangle+i\langle y\rangle}{\sqrt{\langle x\rangle^2+\langle y\rangle^2}}
\end{equation*}

where
\begin{align*}
\langle x\rangle&=\tfrac{2}{\hbar}\langle S_x\rangle
\\
\langle y\rangle&=\tfrac{2}{\hbar}\langle S_y\rangle
\\
\langle z\rangle&=\tfrac{2}{\hbar}\langle S_z\rangle
\end{align*}

\newpage
\fbox{\parbox{\dimexpr\linewidth-2\fboxsep-2\fboxrule}{
1. Verify that
\begin{align*}
S_x&=\tfrac{\hbar}{2}(|x_+\rangle\langle x_+|-|x_-\rangle\langle x_-|)
\\
S_y&=\tfrac{\hbar}{2}(|y_+\rangle\langle y_+|-|y_-\rangle\langle y_-|)
\\
S_z&=\tfrac{\hbar}{2}(|z_+\rangle\langle z_+|-|z_-\rangle\langle z_-|)
\end{align*}
}}

{\footnotesize\begin{verbatim}
zp = (1,0)
zm = (0,1)

xp = (zp + zm) / sqrt(2)
xm = (zp - zm) / sqrt(2)

yp = (zp + i zm) / sqrt(2)
ym = (zp - i zm) / sqrt(2)

Sx = hbar / 2 ((0,1),(1,0))
Sy = hbar / 2 ((0,-i),(i,0))
Sz = hbar / 2 ((1,0),(0,-1))

check(Sx == hbar / 2 (outer(xp,conj(xp)) - outer(xm,conj(xm))))
check(Sy == hbar / 2 (outer(yp,conj(yp)) - outer(ym,conj(ym))))
check(Sz == hbar / 2 (outer(zp,conj(zp)) - outer(zm,conj(zm))))
\end{verbatim}}

\newpage
\fbox{\parbox{\dimexpr\linewidth-2\fboxsep-2\fboxrule}{
2. Let $|s\rangle$ be the following spin state.
\begin{equation*}
|s\rangle=\begin{pmatrix}\frac{1}{3}-\frac{2}{3}i\\[1ex]\frac{2}{3}\end{pmatrix}
\end{equation*}

Verify that $|s\rangle$ is normalized and that
\begin{equation*}
\langle\mathbf S\rangle
=\langle s|\mathbf S|s\rangle
=\frac{\hbar}{2}\begin{pmatrix}\frac{4}{9}\\[1ex]\frac{8}{9}\\[1ex]\frac{1}{9}\end{pmatrix}
\end{equation*}

where
\begin{equation*}
\mathbf S=\begin{pmatrix}S_x\\S_y\\S_z\end{pmatrix}
\end{equation*}

Note: In component form we have
\begin{equation*}
\langle s|\mathbf S|s\rangle
=s_\beta^*{S^{\alpha\beta}}_\gamma s^\gamma
\end{equation*}

Eigenmath requires a transpose so that the $\beta$ indices are adjacent.
\begin{equation*}
\langle s|\mathbf S|s\rangle
=s_\beta^*{S^{\beta\alpha}}_\gamma s^\gamma
\end{equation*}
}}

{\footnotesize\begin{verbatim}
s = (1/3 - 2/3 i, 2/3)

check(dot(conj(s),s) == 1)

Sx = hbar / 2 ((0,1),(1,0))
Sy = hbar / 2 ((0,-i),(i,0))
Sz = hbar / 2 ((1,0),(0,-1))

S = (Sx,Sy,Sz)

check(dot(conj(s),transpose(S),s) == hbar / 2 (4/9, 8/9, 1/9))
\end{verbatim}}

\fbox{\parbox{\dimexpr\linewidth-2\fboxsep-2\fboxrule}{
3. Let $|s\rangle$ be the following spin state.
\begin{equation*}
|s\rangle=\begin{pmatrix}\frac{1}{3}-\frac{2}{3}i\\[1ex]\frac{2}{3}\end{pmatrix}
\end{equation*}

Verify the following measurement probabilities for $|s\rangle$.
\begin{align*}
\Pr\left(S_x=+\tfrac{\hbar}{2}\right)&=|\langle x_+|s\rangle|^2=\tfrac{13}{18}
\\
\Pr\left(S_x=-\tfrac{\hbar}{2}\right)&=|\langle x_-|s\rangle|^2=\tfrac{5}{18}
\\
\\
\Pr\left(S_y=+\tfrac{\hbar}{2}\right)&=|\langle y_+|s\rangle|^2=\tfrac{17}{18}
\\
\Pr\left(S_y=-\tfrac{\hbar}{2}\right)&=|\langle y_-|s\rangle|^2=\tfrac{1}{18}
\\
\\
\Pr\left(S_z=+\tfrac{\hbar}{2}\right)&=|\langle z_+|s\rangle|^2=\tfrac{5}{9}
\\
\Pr\left(S_z=-\tfrac{\hbar}{2}\right)&=|\langle z_-|s\rangle|^2=\tfrac{4}{9}
\end{align*}
}}

{\footnotesize\begin{verbatim}
s = (1/3 - 2/3 i, 2/3)

zp = (1,0)
zm = (0,1)

xp = (zp + zm) / sqrt(2)
xm = (zp - zm) / sqrt(2)

yp = (zp + i zm) / sqrt(2)
ym = (zp - i zm) / sqrt(2)

Pr(a,b) = dot(conj(a),b) dot(conj(b),a)

check(Pr(xp,s) == 13/18)
check(Pr(xm,s) == 5/18)

check(Pr(yp,s) == 17/18)
check(Pr(ym,s) == 1/18)

check(Pr(zp,s) == 5/9)
check(Pr(zm,s) == 4/9)
\end{verbatim}}

\fbox{\parbox{\dimexpr\linewidth-2\fboxsep-2\fboxrule}{
4. Let $|s\rangle$ be the following spin state.
\begin{equation*}
|s\rangle=\begin{pmatrix}\frac{1}{3}-\frac{2}{3}i\\[1ex]\frac{2}{3}\end{pmatrix}
\end{equation*}

Verify that the following spin state $|\chi\rangle$ is indistinguishable from $|s\rangle$.
\begin{equation*}
|\chi\rangle=\begin{pmatrix}\cos(\theta/2)\\\sin(\theta/2)\exp(i\phi)\end{pmatrix}
\end{equation*}

where
\begin{equation*}
\cos(\theta/2)=\sqrt{\frac{\langle z\rangle+1}{2}}=\frac{\sqrt5}{3}
\end{equation*}

and
\begin{equation*}
\sin(\theta/2)\exp(i\phi)
=\sqrt{\frac{1-\langle z\rangle}{2}}
\frac{\langle x\rangle+i\langle y\rangle}{\sqrt{\langle x\rangle^2+\langle y\rangle^2}}
=\frac{2+4i}{3\sqrt5}
\end{equation*}

with
\begin{align*}
\langle x\rangle&=\tfrac{2}{\hbar}\langle S_x\rangle
\\
\langle y\rangle&=\tfrac{2}{\hbar}\langle S_y\rangle
\\
\langle z\rangle&=\tfrac{2}{\hbar}\langle S_z\rangle
\end{align*}
}}

{\footnotesize\begin{verbatim}
s = (1/3 - 2/3 i, 2/3)

Sx = hbar / 2 ((0,1),(1,0))
Sy = hbar / 2 ((0,-i),(i,0))
Sz = hbar / 2 ((1,0),(0,-1))

S = (Sx,Sy,Sz)

x = 2 / hbar dot(conj(s),Sx,s)
y = 2 / hbar dot(conj(s),Sy,s)
z = 2 / hbar dot(conj(s),Sz,s)

cp = sqrt((z + 1) / 2)
cm = sqrt((1 - z) / 2) (x + i y) / sqrt(x^2 + y^2)

check(cp == sqrt(5) / 3)
check(cm == (2 + 4 i) / (3 sqrt(5)))

chi = (cp,cm)

check(dot(conj(s),transpose(S),s) == dot(conj(chi),transpose(S),chi))
\end{verbatim}}

\newpage
\fbox{\parbox{\dimexpr\linewidth-2\fboxsep-2\fboxrule}{
5. Verify the following spin commutation relations using
$\mathbf S\psi=(\mathbf r\times\mathbf p)\psi$.
\begin{align*}
[S_x,S_y]&=i\hbar S_z
\\
[S_y,S_z]&=i\hbar S_x
\\
[S_z,S_x]&=i\hbar S_y
\\[1ex]
[S^2,S_x]&=0
\\
[S^2,S_y]&=0
\\
[S^2,S_z]&=0
\\[1ex]
[S_+,S_-]&=2\hbar S_z
\end{align*}

where
\begin{equation*}
S^2=S_x^2+S_y^2+S_z^2
\end{equation*}

and
\begin{align*}
S_+&=S_x+iS_y
\\
S_-&=S_x-iS_y
\end{align*}
}}

{\footnotesize\begin{verbatim}
Sx(psi) = -i hbar (y d(psi,z) - z d(psi,y))
Sy(psi) = -i hbar (z d(psi,x) - x d(psi,z))
Sz(psi) = -i hbar (x d(psi,y) - y d(psi,x))

psi = Psi()

check(Sx(Sy(psi)) - Sy(Sx(psi)) == i hbar Sz(psi))
check(Sy(Sz(psi)) - Sz(Sy(psi)) == i hbar Sx(psi))
check(Sz(Sx(psi)) - Sx(Sz(psi)) == i hbar Sy(psi))

S2(psi) = Sx(Sx(psi)) + Sy(Sy(psi)) + Sz(Sz(psi))

check(S2(Sx(psi)) - Sx(S2(psi)) == 0)
check(S2(Sy(psi)) - Sy(S2(psi)) == 0)
check(S2(Sz(psi)) - Sz(S2(psi)) == 0)

Sp(psi) = Sx(psi) + i Sy(psi)
Sm(psi) = Sx(psi) - i Sy(psi)

check(Sp(Sm(psi)) - Sm(Sp(psi)) == 2 hbar Sz(psi))
\end{verbatim}}

\end{document}
