\input{preamble}

\section*{Eigenvalues of angular momentum}

We will derive eigenvalues of $L^2$ and $L_z$ from the following commutation relations.
\begin{align*}
[L_x,L_y]&=i\hbar L_z
\\
[L_y,L_z]&=i\hbar L_x
\\
[L_z,L_x]&=i\hbar L_y
\\[1ex]
[L^2,L_z]&=0
\end{align*}

Start by defining ladder operators $L_+$ and $L_-$.
\begin{align*}
L_+&=L_x+iL_y
\\
L_-&=L_x-iL_y
\end{align*}

We have the following commutation relations for ladder operators.
\begin{align*}
[L_z,L_+]&=[L_z,L_x]+i[L_z,L_y]
\\
&=i\hbar L_y+i(-i\hbar L_x)
\\
&=\hbar L_+
\\[1ex]
[L_z,L_-]&=[L_z,L_x]-i[L_z,L_y]
\\
&=i\hbar L_y-i(-i\hbar L_x)
\\
&=-\hbar L_-
\end{align*}

We also have
\begin{align*}
L_-L_+&=(L_x-iL_y)(L_x+iL_y)
\\
&=L_x^2+L_y^2+i[L_x,L_y]
\\
&=L^2-L_z^2-\hbar L_z
\\[1ex]
L_+L_-&=(L_x+iL_y)(L_x-iL_y)
\\
&=L_x^2+L_y^2-i[L_x,L_y]
\\
&=L^2-L_z^2+\hbar L_z
\end{align*}

Operators $L^2$ and $L_z$ commute hence they share eigenfunctions $\psi$.

\bigskip
Let $\lambda$ be an eigenvalue of $L^2$ and let $\mu$ be an eigenvalue of $L_z$ such that
\begin{equation*}
L^2\psi=\lambda\psi
\end{equation*}

and
\begin{equation*}
L_z\psi=\mu\psi
\end{equation*}

We will now show that
\begin{equation*}
\lambda\ge\mu^2
\end{equation*}

By definition of $L^2$ we have
\begin{equation*}
L^2\psi=\left(L_x^2+L_y^2+L_z^2\right)\psi
\end{equation*}

Substitute $\lambda$ for $L^2$ and $\mu$ for $L_z$ to obtain
\begin{equation*}
\lambda\psi=\left(L_x^2+L_y^2+\mu^2\right)\psi
\end{equation*}

Rewrite as
\begin{equation*}
\left(L_x^2+L_y^2\right)\psi=(\lambda-\mu^2)\psi
\end{equation*}

The eigenvalues of squared Hermitian operators are nonnegative hence $\lambda-\mu^2\ge 0$.
Hence
\begin{equation*}
\lambda\ge\mu^2
\end{equation*}

The property $\lambda\ge\mu^2$ means that $\mu$ has an upper limit.

\bigskip
Let $\mu_m$ be the maximum $\mu$.
Then for eigenfunction $\psi_m$ we have
\begin{equation*}
L_z\psi_m=\mu_m\psi_m
\end{equation*}

Apply $L_+$ to both sides.
\begin{equation*}
L_+L_z\psi_m=\mu_mL_+\psi_m
\end{equation*}

Expand the left hand side.
\begin{equation*}
(L_zL_+-L_zL_++L_+L_z)\psi_m=\mu_mL_+\psi_m
\end{equation*}

Substitute $\hbar L_+$ for $[L_z,L_+]$.
\begin{equation*}
L_zL_+\psi_m-\hbar L_+\psi_m=\mu_mL_+\psi_m
\end{equation*}

Hence
\begin{equation*}
L_zL_+\psi_m=(\mu_m+\hbar)L_+\psi_m
\end{equation*}

Because $\mu_m$ is the maximum eigenvalue and $\mu_m+\hbar>\mu_m$ we must have
\begin{equation*}
L_+\psi_m=0
\end{equation*}

Consequently
\begin{equation*}
L_-L_+\psi_m=0
\end{equation*}

Recalling that
\begin{equation*}
L_-L_+=L^2-L_z^2-\hbar L_z
\end{equation*}

we have
\begin{equation*}
(L^2-L_z^2-\hbar L_z)\psi_m
=(\lambda-\mu_m^2-\hbar\mu_m)\psi_m=0
\end{equation*}

Hence
\begin{equation*}
\lambda=\mu_m^2+\hbar\mu_m
\tag{1}
\end{equation*}

Let $\mu_k$ be the minimum $\mu$.
Then for eigenfunction $\psi_k$ we have
\begin{equation*}
L_z\psi_k=\mu_k\psi_k
\end{equation*}

Apply $L_-$ to both sides.
\begin{equation*}
L_-L_z\psi_k=\mu_kL_-\psi_k
\end{equation*}

Expand the left hand side.
\begin{equation*}
(L_zL_--L_zL_-+L_-L_z)\psi_k=\mu_mL_-\psi_k
\end{equation*}

Substitute $-\hbar L_-$ for $[L_z,L_-]$.
\begin{equation*}
L_zL_-\psi_k+\hbar L_-\psi_k=\mu_kL_-\psi_k
\end{equation*}

Hence
\begin{equation*}
L_zL_-\psi_k=(\mu_k-\hbar)L_-\psi_k
\end{equation*}

Because $\mu_k$ is the minimum eigenvalue and $\mu_k-\hbar<\mu_k$ we must have
\begin{equation*}
L_-\psi_k=0
\end{equation*}

Consequently
\begin{equation*}
L_+L_-\psi_k=0
\end{equation*}

Recalling that
\begin{equation*}
L_+L_-=L^2-L_z^2+\hbar L_z
\end{equation*}

we have
\begin{equation*}
(L^2-L_z^2+\hbar L_z)\psi_k=(\lambda-\mu_k^2+\hbar\mu_k)\psi_k=0
\end{equation*}

Hence
\begin{equation*}
\lambda=\mu_k^2-\hbar\mu_k
\tag{2}
\end{equation*}

By equivalence of (1) and (2) we have
\begin{equation*}
\mu_m^2+\hbar\mu_m-\mu_k^2+\hbar\mu_k=0
\tag{3}
\end{equation*}

By ladder operators there is an integer $n$ such that
\begin{equation*}
\mu_m=\mu_k+n\hbar
\end{equation*}

Substitute $\mu_k+n\hbar$ for $\mu_m$ in (3) to obtain
\begin{equation*}
\mu_k^2+2\mu_kn\hbar+n^2\hbar^2+\hbar\mu_k+n\hbar^2-\mu_k^2+\hbar\mu_k=0
\end{equation*}

Cancel $\mu_k^2$ and rewrite the remaining terms as
\begin{equation*}
2\mu_k(n+1)\hbar+n(n+1)\hbar^2=0
\end{equation*}

Divide through by $(n+1)\hbar$ to obtain
\begin{equation*}
2\mu_k+n\hbar=0
\end{equation*}

Hence
\begin{equation*}
\mu_k=-\frac{n\hbar}{2}
\end{equation*}

and
\begin{equation*}
\mu_m=\mu_k+n\hbar=\frac{n\hbar}{2}
\end{equation*}

Define quantum number $l$ as
\begin{equation*}
l=\frac{n}{2}=0,\tfrac{1}{2},1,\tfrac{3}{2},2,\ldots
\end{equation*}

Then
\begin{equation*}
\mu_m=l\hbar
\end{equation*}

By equation (1) we have
\begin{equation*}
\lambda=\mu_m^2+\mu_m\hbar=(l\hbar)^2+l\hbar^2=l(l+1)\hbar^2
\end{equation*}

Hence $l(l+1)\hbar^2$ are eigenvalues of $L^2$.
\begin{equation*}
L^2\psi=\lambda\psi=l(l+1)\hbar^2\psi
\end{equation*}

For a given $l$, operator $L_z$ has eigenvalues
\begin{equation*}
\mu=\mu_k,\ldots,\mu_m=-l\hbar,(-l+1)\hbar,\ldots,(l-1)\hbar,l\hbar
\end{equation*}

Define quantum number $m$ as
\begin{equation*}
m=-l,-l+1,\dots,l-1,l
\end{equation*}

Then
\begin{equation*}
\mu=m\hbar
\end{equation*}

Hence $m\hbar$ are eigenvalues of $L_z$.
\begin{equation*}
L_z\psi=\mu\psi=m\hbar\psi
\end{equation*}

\href{https://georgeweigt.github.io/examples/eigenvalues-of-angular-momentum-demo.html}{Eigenmath code}

\end{document}
