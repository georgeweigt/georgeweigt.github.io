\input{preamble}

\section*{Atomic transitions 6}

From the previous section the transition rate is
\begin{equation*}
R_{a\rightarrow b}
=\frac{\pi e^2}{3\varepsilon_0\hbar^2}
\bigl|\langle\psi_b|\mathbf r|\psi_a\rangle\bigr|^2\rho(\omega_0)
\end{equation*}

Interchange $\psi_a$ and $\psi_b$ by the identity
\begin{equation*}
\bigl|\langle\psi_a|\mathbf r|\psi_b\rangle\bigr|^2
=\bigl|\langle\psi_b|\mathbf r|\psi_a\rangle\bigr|^2
\end{equation*}

to obtain
\begin{equation*}
R_{b\rightarrow a}
=\frac{\pi e^2}{3\varepsilon_0\hbar^2}
\bigl|\langle\psi_a|\mathbf r|\psi_b\rangle\bigr|^2\rho(\omega_0)
\end{equation*}

The stimulated emission coefficient is
\begin{equation*}
B_{b\rightarrow a}=\frac{R_{b\rightarrow a}}{\rho(\omega_0)}
=\frac{\pi e^2}{3\varepsilon_0\hbar^2}
\bigl|\langle\psi_a|\mathbf r|\psi_b\rangle\bigr|^2
\end{equation*}

The spontaneous emission rate is
\begin{equation*}
A_{b\rightarrow a}=\frac{\hbar\omega_0^3}{\pi^2c^3}B_{b\rightarrow a}
=\frac{e^2\omega_0^3}{3\pi\varepsilon_0\hbar c^3}
\bigl|\langle\psi_a|\mathbf r|\psi_b\rangle\bigr|^2
\tag{1}
\end{equation*}

Verify dimensions.
\begin{equation*}
A_{b\rightarrow a}\propto
\frac{
\begin{matrix}
e^2 & \omega_0^3
\\
\text{C}^2 & \text{s}^{-3}
\end{matrix}
}
{
\begin{matrix}
\epsilon_0 & \hbar & c^3
\\
\text{C}^2\,\text{J}^{-1}\,\text{m}^{-1}
& \text{J}\,\text{s}
& \text{m}^3\,\text{s}^{-3}
\end{matrix}
}
\times
\begin{matrix}
\\
\bigl|\langle\psi_a|\mathbf r|\psi_b\rangle\bigr|^2
\\
\text{m}^2
\end{matrix}
=\text{s}^{-1}
\end{equation*}

\end{document}
