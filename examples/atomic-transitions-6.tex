\input{preamble}

\section*{Atomic transitions 6}

From the previous section the transition rate is
\begin{equation*}
R_{a\rightarrow b}
=\frac{\pi e^2}{3\varepsilon_0\hbar^2}
\bigl|\langle\psi_b|\mathbf r|\psi_a\rangle\bigr|^2\rho(\omega_0)
\end{equation*}

where $E_b>E_a$ and
\begin{equation*}
\omega_0=\frac{E_b-E_a}{\hbar}
\end{equation*}

Interchange $\psi_a$ and $\psi_b$ by the identity
\begin{equation*}
\bigl|\langle\psi_a|\mathbf r|\psi_b\rangle\bigr|^2
=\bigl|\langle\psi_b|\mathbf r|\psi_a\rangle\bigr|^2
\end{equation*}

to obtain
\begin{equation*}
R_{b\rightarrow a}
=\frac{\pi e^2}{3\varepsilon_0\hbar^2}
\bigl|\langle\psi_a|\mathbf r|\psi_b\rangle\bigr|^2\rho(\omega_0)
\end{equation*}

By Planck's law
\begin{equation*}
\rho(\omega_0)=\frac{\hbar\omega_0^3}{\pi^2c^3}
\frac{1}{\exp\left(\frac{\hbar\omega_0}{kT}\right)-1}
\end{equation*}

Hence the absorption rate is
\begin{equation*}
R_{a\rightarrow b}
=\frac{e^2\omega_0^3}{3\pi\varepsilon_0\hbar c^3}
\bigl|\langle\psi_b|\mathbf r|\psi_a\rangle\bigr|^2
\frac{1}{\exp\left(\frac{\hbar\omega_0}{kT}\right)-1}
\tag{1}
\end{equation*}

and the stimulated emission rate is
\begin{equation*}
R_{b\rightarrow a}
=\frac{e^2\omega_0^3}{3\pi\varepsilon_0\hbar c^3}
\bigl|\langle\psi_a|\mathbf r|\psi_b\rangle\bigr|^2
\frac{1}{\exp\left(\frac{\hbar\omega_0}{kT}\right)-1}
\end{equation*}

The spontaneous emission rate is
\begin{equation*}
A_{b\rightarrow a}=R_{b\rightarrow a}
\left[\exp\left(\frac{\hbar\omega_0}{kT}\right)-1\right]
=\frac{e^2\omega_0^3}{3\pi\varepsilon_0\hbar c^3}
\bigl|\langle\psi_a|\mathbf r|\psi_b\rangle\bigr|^2
\tag{2}
\end{equation*}

Verify dimensions.
\begin{equation*}
A_{b\rightarrow a}\propto
\frac{
\begin{matrix}
e^2 & \omega_0^3
\\
\text{C}^2 & \text{s}^{-3}
\end{matrix}
}
{
\begin{matrix}
\epsilon_0 & \hbar & c^3
\\
\text{C}^2\,\text{J}^{-1}\,\text{m}^{-1}
& \text{J}\,\text{s}
& \text{m}^3\,\text{s}^{-3}
\end{matrix}
}
\times
\begin{matrix}
\\
\bigl|\langle\psi_a|\mathbf r|\psi_b\rangle\bigr|^2
\\
\text{m}^2
\end{matrix}
=\text{s}^{-1}
\end{equation*}

We will now show why
\begin{equation*}
A_{b\rightarrow a}=R_{b\rightarrow a}
\left[\exp\left(\frac{\hbar\omega_0}{kT}\right)-1\right]
\end{equation*}

Let $N_a$ be the number of atoms in the lower state and let $N_b$ be
the number of atoms in the upper state.
From thermodynamics
\begin{equation*}
\frac{N_a}{N_b}=\exp\left(\frac{\hbar\omega_0}{kT}\right)
\end{equation*}

At thermal equilibrium
\begin{equation*}
N_aR_{a\rightarrow b}=N_b(A_{b\rightarrow a}+R_{b\rightarrow a})
\end{equation*}

Hence
\begin{equation*}
\frac{N_a}{N_b}=\frac{A_{b\rightarrow a}+R_{b\rightarrow a}}{R_{a\rightarrow b}}
=\exp\left(\frac{\hbar\omega_0}{kT}\right)
\end{equation*}

Solve for $A_{b\rightarrow a}$.
\begin{equation*}
A_{b\rightarrow a}=R_{a\rightarrow b}\exp\left(\frac{\hbar\omega_0}{kT}\right)-R_{b\rightarrow a}
\end{equation*}

Noting that $R_{a\rightarrow b}=R_{b\rightarrow a}$ we have
\begin{equation*}
A_{b\rightarrow a}
=R_{b\rightarrow a}\left[\exp\left(\frac{\hbar\omega_0}{kT}\right)-1\right]
\end{equation*}

\end{document}
