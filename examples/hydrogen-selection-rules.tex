\input{preamble}

\section*{Hydrogen selection rules}

By computing transition elements, verify the following selection rules.
\begin{equation*}
|\Delta l|=1,\quad|\Delta m|\le1
\end{equation*}

The transition element $|R_{i\rightarrow f}|^2$ for spontaneous emission is
\begin{equation*}
|R_{i\rightarrow f}|^2
=|\langle\psi_f|\hat x|\psi_i\rangle|^2
+|\langle\psi_f|\hat y|\psi_i\rangle|^2
+|\langle\psi_f|\hat z|\psi_i\rangle|^2
\end{equation*}

where
\begin{gather*}
\langle\psi_f|\hat x|\psi_i\rangle
=\int_V\psi_f^*\hat x\psi_i\,dV,
\quad
\langle\psi_f|\hat y|\psi_i\rangle
=\int_V\psi_f^*\hat y\psi_i\,dV,
\quad
\langle\psi_f|\hat z|\psi_i\rangle
=\int_V\psi_f^*\hat z\psi_i\,dV
\end{gather*}

For the dipole approximation the operators are
\begin{equation*}
\hat x=r\sin\theta\cos\phi,
\quad
\hat y=r\sin\theta\sin\phi,
\quad
\hat z=r\cos\theta
\end{equation*}

In spherical coordinates the measure is
\begin{equation*}
dV=r^2\sin\theta\,dr\,d\theta\,d\phi
\end{equation*}

The wave function for hydrogen is
\begin{equation*}
\psi_{nlm}(r,\theta,\phi)=R_{nl}(r)Y_{lm}(\theta,\phi)
\end{equation*}

where
\begin{gather*}
R_{nl}(r)=
\frac{2}{n^2}
\sqrt{\frac{(n-l-1)!}{(n+l)!}}
\left(\frac{2r}{na_0}\right)^l
L_{n-l-1}^{2l+1}\left(\frac{2r}{na_0}\right)
\exp\left(-\frac{r}{na_0}\right)
a_0^{-3/2}
\\[1ex]
L_n^m(x)=(n+m)!\sum_{k=0}^n\frac{(-x)^k}{(n-k)!(m+k)!k!}
\\[1ex]
Y_{lm}(\theta,\phi)=(-1)^m
\sqrt{\frac{(2l+1)}{4\pi}
\frac{(l-m)!}{(l+m)!}}
P_l^m(\cos\theta)\exp(im\phi)
\\[1ex]
P_l^m(\cos\theta)=\begin{cases}
\displaystyle
\left(\frac{\sin\theta}{2}\right)^m\,\sum_{k=0}^{l-m}
(-1)^k\frac{(l+m+k)!}{(l-m-k)!(m+k)!k!}
\left(\frac{1-\cos\theta}{2}\right)^k, & m\ge0
\\[3ex]
\displaystyle
(-1)^m\frac{(l+m)!}{(l-m)!}P_l^{|m|}(\cos\theta), & m<0
\end{cases}
\end{gather*}

\end{document}
