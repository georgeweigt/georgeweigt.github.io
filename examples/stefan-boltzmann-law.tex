\input{preamble}

\section*{Stefan-Boltzmann law}

Josef Stefan in 1879 determined from experimental data that the total power
emitted by a radiant object is proportional
to the fourth power of its absolute temperature $T$.
Five years later Ludwig Boltzmann showed how to derive the same relation from principles of thermodynamics.
The modern form of the Stefan-Boltzmann law is
\begin{equation*}
P=A\varepsilon\sigma T^4
\end{equation*}

where $P$ is total power, $A$ is surface area, $\varepsilon$ is an emissivity constant,
and $\sigma$ is the Stefan--Boltzmann constant
\begin{equation*}
\sigma=5.67\times10^{-8}\,\text{watt}\,\text{meter}^{-2}\,\text{kelvin}^{-4}
\end{equation*}

For example, consider a one cubic centimeter block of wrought iron at 1000 kelvin.
The emissivity constant for wrought iron is $\varepsilon=0.94$
hence the total radiant power is
\begin{equation*}
P=\underset{\text{surface area 1 cm cube}}
{6\times10^{-4}\,\text{meter}^2}
\times
0.94
\times
5.67\times10^{-8}\,\text{watt}\,\text{meter}^{-2}\,\text{kelvin}^{-4}
\\
{}\times
1000^4\,\text{kelvin}^4
=32\,\text{watt}
\end{equation*}

\end{document}
