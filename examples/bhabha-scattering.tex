\input{preamble}

\section*{Bhabha scattering}

Bhabha scattering is the interaction $e^-+e^+\rightarrow e^-+e^+$.
%
\begin{center}
\begin{tikzpicture}
\draw[dashed] (0,0) circle (0.5cm);
\draw[thick,->] (2,0) node[anchor=west] {$e^-$} -- (0.6,0);
\draw[thick,->] (-2,0) node[anchor=east] {$e^+$} -- (-0.6,0);
\draw[thick,->] (0.40,0.40) -- (1.3,1.3) node[anchor=south west] {$e^+$};
\draw[thick,->] (-0.4,-0.4) -- (-1.3,-1.3) node[anchor=north east] {$e^-$};
\draw (1,0.5) node {$\theta$};
\end{tikzpicture}
\end{center}
%
In the center-of-mass frame we have the following momentum vectors where $E=\sqrt{p^2+m^2}$.
\begin{equation*}
p_1=\underset{\text{inbound $e^+$}}
{\begin{pmatrix}E\\0\\0\\p\end{pmatrix}}
\qquad
p_2=\underset{\text{inbound $e^-$}}
{\begin{pmatrix}E\\0\\0\\-p\end{pmatrix}}
\qquad
p_3=\underset{\text{outbound $e^+$}}
{\begin{pmatrix}E\\p\sin\theta\cos\phi\\p\sin\theta\sin\phi\\p\cos\theta\end{pmatrix}}
\qquad
p_4=\underset{\text{outbound $e^-$}}
{\begin{pmatrix}E\\-p\sin\theta\cos\phi\\-p\sin\theta\sin\phi\\-p\cos\theta\end{pmatrix}}
\end{equation*}

Spinors for the inbound positron.
\begin{equation*}
v_{11}=\frac{1}{\sqrt{E+m}}
\underset{\substack{\text{inbound $e^+$}\\ \text{spin up}}}
{\begin{pmatrix}p\\0\\E+m\\0\end{pmatrix}}
\qquad
v_{12}=\frac{1}{\sqrt{E+m}}
\underset{\substack{\text{inbound $e^+$}\\ \text{spin down}}}
{\begin{pmatrix}0\\-p\\0\\E+m\end{pmatrix}}
\end{equation*}

Spinors for the inbound electron.
\begin{equation*}
u_{21}=\frac{1}{\sqrt{E+m}}
\underset{\substack{\text{inbound $e^-$}\\ \text{spin up}}}
{\begin{pmatrix}E+m\\0\\-p\\0\end{pmatrix}}
\qquad
u_{22}=\frac{1}{\sqrt{E+m}}
\underset{\substack{\text{inbound $e^-$}\\ \text{spin down}}}
{\begin{pmatrix}0\\E+m\\0\\p\end{pmatrix}}
\end{equation*}

Spinors for the outbound positron.
\begin{equation*}
v_{31}=\frac{1}{\sqrt{E+m}}
\underset{\substack{\text{outbound $e^+$}\\ \text{spin up}}}
{\begin{pmatrix}p_{3z}\\p_{3x}+ip_{3y}\\E+m\\0\end{pmatrix}}
\qquad
v_{32}=\frac{1}{\sqrt{E+m}}
\underset{\substack{\text{outbound $e^+$}\\ \text{spin down}}}
{\begin{pmatrix}p_{3x}-ip_{3y}\\-p_{3z}\\0\\E+m\end{pmatrix}}
\end{equation*}

Spinors for the outbound electron.
\begin{equation*}
u_{41}=\frac{1}{\sqrt{E+m}}
\underset{\substack{\text{outbound $e^-$}\\ \text{spin up}}}
{\begin{pmatrix}E+m\\0\\p_{4z}\\p_{4x}+ip_{4y}\end{pmatrix}}
\qquad
u_{42}=\frac{1}{\sqrt{E+m}}
\underset{\substack{\text{outbound $e^-$}\\ \text{spin down}}}
{\begin{pmatrix}0\\E+m\\p_{4x}-ip_{4y}\\-p_{4z}\end{pmatrix}}
\end{equation*}

The probability amplitude $\mathcal M_{abcd}$ for spin state $abcd$ is
\begin{equation*}
\mathcal M_{abcd}=\mathcal M_{1abcd}+\mathcal M_{2abcd}
\end{equation*}

where
\begin{equation*}
\mathcal M_{1abcd}=-\frac{e^2}{t}
\underset{\text{scattering}}
{(\bar v_{1a}\gamma^\nu v_{3c})(\bar u_{4d}\gamma_\nu u_{2b})},
\quad
\mathcal M_{2abcd}=\frac{e^2}{s}
\underset{\text{annihilation}}
{(\bar v_{1a}\gamma^\mu u_{2b})(\bar u_{4d}\gamma_\mu v_{3c})}
\end{equation*}

Symbol $e$ is elementary charge and
\begin{align*}
s&=(p_1+p_2)^2
\\
t&=(p_1-p_3)^2
\end{align*}

The expected probability density $\langle|\mathcal M|^2\rangle$
is the average of spin states.
\begin{equation*}
\langle|\mathcal M|^2\rangle=\frac{1}{4}
\sum_{a=1}^2\sum_{b=1}^2\sum_{c=1}^2\sum_{d=1}^2
|\mathcal M_{abcd}|^2
\end{equation*}

Hence
\begin{equation*}
\langle|\mathcal{M}|^2\rangle=\frac{1}{4}
\sum_{abcd}
\left(
\mathcal M_{1abcd}\mathcal M_{1abcd}^*
+\mathcal M_{1abcd}\mathcal M_{2abcd}^*
+\mathcal M_{2abcd}\mathcal M_{1abcd}^*
+\mathcal M_{2abcd}\mathcal M_{2abcd}^*
\right)
\end{equation*}

The Casimir trick uses matrix arithmetic to sum over spin states.
\begin{equation*}
\langle|\mathcal{M}|^2\rangle
=\frac{e^4}{4}
\left(
\frac{f_{11}}{s^2}-\frac{2f_{12}}{st}+\frac{f_{22}}{t^2}
\right)
\end{equation*}

where
\begin{align*}
f_{11}&=\operatorname{Tr}\left(
(\slashed p_1-m)\gamma^\mu(\slashed p_3-m)\gamma^\nu
\right)
\operatorname{Tr}\left(
(\slashed p_4+m)\gamma_\mu(\slashed p_2+m)\gamma_\nu
\right)
\\
f_{12}&=\operatorname{Tr}\left(
(\slashed p_1-m)\gamma^\mu(\slashed p_2+m)\gamma^\nu
(\slashed p_4+m)\gamma_\mu(\slashed p_3-m)\gamma_\nu
\right)
\\
f_{22}&=\operatorname{Tr}\left(
(\slashed p_1-m)\gamma^\mu(\slashed p_2+m)\gamma^\nu
\right)
\operatorname{Tr}\left(
(\slashed p_4+m)\gamma_\mu(\slashed p_3-m)\gamma_\nu
\right)
\end{align*}

The following formulas are equivalent to the Casimir trick.
(Recall that $a\cdot b=a^\mu g_{\mu\nu}b^\nu$)
\begin{align*}
f_{11}&=
32(p_1\cdot p_2)^2
+32(p_1\cdot p_4)^2
-64 m^2(p_1\cdot p_3)
+64 m^4
\\
f_{12}&=
-32 (p_1\cdot p_4)^2
-32 m^2 (p_1\cdot p_2)
+32 m^2 (p_1\cdot p_3)
-32 m^2 (p_1\cdot p_4)
-32 m^4
\\
f_{22}&=
32(p_1\cdot p_3)^2
+32(p_1\cdot p_4)^2
+64 m^2(p_1\cdot p_2)
+64 m^4
\end{align*}

In Mandelstam variables
\begin{align*}
f_{11} &= 8 u^2 + 8 s^2 - 64 u m^2 - 64 s m^2 + 192 m^4
\\
f_{12} &= -8 u^2 + 64 u m^2 - 96 m^4
\\
f_{22} &= 8 u^2 + 8 t^2 - 64 u m^2 - 64 t m^2 + 192 m^4
\end{align*}

For $E\gg m$ a useful approximation is to set $m=0$ and obtain
\begin{align*}
f_{11}&=8u^2+8s^2
\\
f_{12}&=-8u^2
\\
f_{22}&=8u^2+8t^2
\end{align*}

For $m=0$ the Mandelstam variables are
\begin{align*}
s&=4E^2
\\
t&=-2E^2(1-\cos\theta)
\\
u&=-2E^2(1+\cos\theta)
\end{align*}

Hence
\begin{align*}
\langle|\mathcal{M}|^2\rangle
&=\frac{e^4}{4}
\left(\frac{f_{11}}{s^2}-\frac{2f_{12}}{st}+\frac{f_{22}}{t^2}\right)
\\
&=2e^4\left(\frac{u^2+s^2}{t^2}+\frac{2u^2}{st}+\frac{u^2+t^2}{s^2}\right)
\\
&=e^4\biggl(
\underset{\text{scattering}}
{\frac{2(1+\cos\theta)^2+8}{(1-\cos\theta)^2}}
-\underset{\text{interaction}}
{\frac{2(1+\cos\theta)^2}{1-\cos\theta}}
+\underset{\text{annihilation}}
{1+\cos^2\theta}
\biggr)
\end{align*}

The expected probability density can be written more compactly as
\begin{equation*}
\langle|\mathcal{M}|^2\rangle=e^4\left(\frac{\cos^2\theta+3}{\cos\theta-1}\right)^2
\end{equation*}

\subsubsection*{Cross section}

The differential cross section is
\begin{equation*}
\frac{d\sigma}{d\Omega}=\frac{\langle|\mathcal{M}|^2\rangle}{4(4\pi\varepsilon_0)^2s}
\end{equation*}

where
\begin{equation*}
s=(p_1+p_2)^2=4E^2
\end{equation*}

For high energy experiments we have
\begin{equation*}
\langle|\mathcal{M}|^2\rangle=e^4\left(\frac{\cos^2\theta+3}{\cos\theta-1}\right)^2
\end{equation*}

Hence for high energy experiments
\begin{equation*}
\frac{d\sigma}{d\Omega}=\frac{e^4}{4(4\pi\varepsilon_0)^2s}
\left(\frac{\cos^2\theta+3}{\cos\theta-1}\right)^2
\end{equation*}

Noting that
\begin{equation*}
e^2=4\pi\varepsilon_0\alpha\hbar c
\end{equation*}

we have
\begin{equation*}
\frac{d\sigma}{d\Omega}
=\frac{\alpha^2(\hbar c)^2}{4s}
\left(\frac{\cos^2\theta+3}{\cos\theta-1}\right)^2
\end{equation*}

Noting that
\begin{equation*}
d\Omega=\sin\theta\,d\theta\,d\phi
\end{equation*}

we also have
\begin{equation*}
d\sigma=\frac{\alpha^2(\hbar c)^2}{4s}
\left(\frac{\cos^2\theta+3}{\cos\theta-1}\right)^2
\sin\theta\,d\theta\,d\phi
\end{equation*}

Let $S(\theta_1,\theta_2)$ be the following integral of $d\sigma$.
\begin{equation*}
S(\theta_1,\theta_2)=\int_0^{2\pi}\int_{\theta_1}^{\theta_2}d\sigma
\end{equation*}

The solution is
\begin{equation*}
S(\theta_1,\theta_2)=\frac{\pi\alpha^2(\hbar c)^2}{2s}[I(\theta_2)-I(\theta_1)]
\end{equation*}

where
\begin{equation*}
I(\theta)=\frac{16}{\cos\theta-1}-\frac{\cos^3\theta}{3}-\cos^2\theta-9\cos\theta-16\log(1-\cos\theta)
\end{equation*}

The cumulative distribution function is
\begin{equation*}
F(\theta)
=\frac{S(a,\theta)}{S(a,\pi)}
=\frac{I(\theta)-I(a)}{I(\pi)-I(a)},
\quad
a\le\theta\le\pi
\end{equation*}

Angular support is reduced by an arbitrary angle $a>0$ because $I(0)$ is undefined.

\bigskip
The probability of observing scattering events in the interval $\theta_1$ to $\theta_2$ is
\begin{equation*}
P(\theta_1<\theta\le\theta_2)=F(\theta_2)-F(\theta_1)
\end{equation*}

The probability density function is
\begin{equation*}
f(\theta)=\frac{dF(\theta)}{d\theta}
=\frac{1}{I(\pi)-I(a)}
\left(\frac{\cos^2\theta+3}{\cos\theta-1}\right)^2
\sin\theta
\end{equation*}

\subsubsection*{Data from SLAC SPEAR experiment}

The following Bhabha scattering data is from SLAC-PUB-1501.
\begin{equation*}
\begin{matrix}
k & x_k & x_{k+1} & y\\
\phantom01 & \phantom+0.6 & \phantom+0.5 & 4432\\
\phantom02 & \phantom+0.5 & \phantom+0.4 & 2841\\
\phantom03 & \phantom+0.4 & \phantom+0.3 & 2045\\
\phantom04 & \phantom+0.3 & \phantom+0.2 & 1420\\
\phantom05 & \phantom+0.2 & \phantom+0.1 & 1136\\
\phantom06 & \phantom+0.1 & \phantom+0.0 & \phantom{0}852\\
\phantom07 & \phantom+0.0 & -0.1 & \phantom{0}656\\
\phantom08 & -0.1 & -0.2 & \phantom{0}625\\
\phantom09 & -0.2 & -0.3 & \phantom{0}511\\
10 & -0.3 & -0.4 & \phantom{0}455\\
11 & -0.4 & -0.5 & \phantom{0}402\\
12 & -0.5 & -0.6 & \phantom{0}398\\
\end{matrix}
\end{equation*}

Column $k$ is the bin number, column $y$ is the number of scattering events, and
\begin{equation*}
x_k=\cos\theta_k
\end{equation*}

The cumulative distribution function for this experiment is
\begin{equation*}
F(\theta)=\frac{I(\theta)-I(\theta_1)}
{I(\theta_{13})-I(\theta_1)}
\end{equation*}
where
\begin{equation*}
\theta_{13}=\arccos(-0.6),
\quad
\theta_1=\arccos(0.6)
\end{equation*}

The scattering probability $P_k$ is
\begin{equation*}
P_k=F\left(\arccos(x_{k+1})\right)-F\left(\arccos(x_k)\right)
\end{equation*}

Multiply $P_k$ by total scattering events to obtain predicted number of events $\hat y_k$.
\begin{equation*}
\sum y_k=15773,\quad \hat y_k=15773\,P_k
\end{equation*}

The following table shows the predicted scattering events $\hat y$.
\begin{equation*}
\begin{matrix}
k & x_k & x_{k+1} & y & \hat y\\
\phantom01 & \phantom+0.6 & \phantom+0.5 & 4432 & 4598\\
\phantom02 & \phantom+0.5 & \phantom+0.4 & 2841 & 2880\\
\phantom03 & \phantom+0.4 & \phantom+0.3 & 2045 & 1955\\
\phantom04 & \phantom+0.3 & \phantom+0.2 & 1420 & 1410\\
\phantom05 & \phantom+0.2 & \phantom+0.1 & 1136 & 1068\\
\phantom06 & \phantom+0.1 & \phantom+0.0 & \phantom0852 & \phantom0843\\
\phantom07 & \phantom+0.0 & -0.1 & \phantom0656 & \phantom0689\\
\phantom08 & -0.1 & -0.2 & \phantom0625 & \phantom0582\\
\phantom09 & -0.2 & -0.3 & \phantom0511 & \phantom0505\\
10 & -0.3 & -0.4 & \phantom0455 & \phantom0450\\
11 & -0.4 & -0.5 & \phantom0402 & \phantom0411\\
12 & -0.5 & -0.6 & \phantom0398 & \phantom0382
\end{matrix}
\end{equation*}

The coefficient of determination $R^2$ measures how well predicted values fit the data.
\begin{equation*}
R^2=1-\frac{\sum(y-\hat y)^2}{\sum(y-\bar y)^2}=0.997
\end{equation*}

The result indicates that $F(\theta)$ explains
99.7\% of the variance in the data.

\subsubsection*{Data from DESY PETRA experiment}

See www.hepdata.net/record/ins191231, Table 3, 14.0 GeV.
\begin{equation*}
\begin{matrix}
x & y\\
-0.7300 & 0.10115\\
-0.6495 & 0.12235\\
-0.5495 & 0.11258\\
-0.4494 & 0.09968\\
-0.3493 & 0.14749\\
-0.2491 & 0.14017\\
-0.1490 & 0.18190\\
-0.0488 & 0.22964\\
\phantom+0.0514 & 0.25312\\
\phantom+0.1516 & 0.30998\\
\phantom+0.2520 & 0.40898\\
\phantom+0.3524 & 0.62695\\
\phantom+0.4529 & 0.91803\\
\phantom+0.5537 & 1.51743\\
\phantom+0.6548 & 2.56714\\
\phantom+0.7323 & 4.30279
\end{matrix}
\end{equation*}

Data $x$ and $y$ have the following relationship with the cross section formula.
\begin{equation*}
x=\cos\theta,
\quad
y=\frac{d\sigma}{d\Omega}\text{ in units of nanobarns}
\end{equation*}

The cross section formula is
\begin{equation*}
\frac{d\sigma}{d\Omega}
=\frac{\alpha^2}{4s}
\left(\frac{\cos^2\theta+3}{\cos\theta-1}\right)^2\times(\hbar c)^2
\end{equation*}

To compute predicted values $\hat y$, multiply by $10^{37}$ to convert square meters to nanobarns.
\begin{equation*}
\hat y
=\frac{\alpha^2}{4s}
\left(\frac{x^2+3}{x-1}\right)^2
\times(\hbar c)^2
\times10^{37}
\end{equation*}

The following table shows predicted values $\hat y$ for $s=(14.0\,\text{GeV})^2$.
\begin{equation*}
\begin{matrix}
x & y & \hat y\\
-0.7300 & 0.10115 & 0.110296\\
-0.6495 & 0.12235 & 0.113816\\
-0.5495 & 0.11258 & 0.120101\\
-0.4494 & 0.09968 & 0.129075\\
-0.3493 & 0.14749 & 0.141592\\
-0.2491 & 0.14017 & 0.158934\\
-0.1490 & 0.18190 & 0.182976\\
-0.0488 & 0.22964 & 0.216737\\
\phantom{+}0.0514 & 0.25312 & 0.264989\\
\phantom{+}0.1516 & 0.30998 & 0.335782\\
\phantom{+}0.2520 & 0.40898 & 0.443630\\
\phantom{+}0.3524 & 0.62695 & 0.615528\\
\phantom{+}0.4529 & 0.91803 & 0.907700\\
\phantom{+}0.5537 & 1.51743 & 1.451750\\
\phantom{+}0.6548 & 2.56714 & 2.609280\\
\phantom{+}0.7323 & 4.30279 & 4.615090
\end{matrix}
\end{equation*}

The coefficient of determination $R^2$ measures how well predicted values fit the data.
\begin{equation*}
R^2=1-\frac{\sum(y-\hat y)^2}{\sum(y-\bar y)^2}=0.995
\end{equation*}

The result indicates that the model $d\sigma$ explains 99.5\% of the variance in the data.

\end{document}
