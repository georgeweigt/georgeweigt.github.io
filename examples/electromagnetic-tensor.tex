\documentclass[12pt]{article}
\usepackage[margin=1in]{geometry}
\usepackage{amsmath}
\parindent=0pt
\begin{document}

\section*{Electromagnetic tensor}

This is the standard model for an EM field.
\begin{equation*}
\mathbf A=\begin{pmatrix}A_x\\A_y\\A_z\end{pmatrix},\quad
A^\mu=\begin{pmatrix}\phi\\A_x\\A_y\\A_z\end{pmatrix},\quad
A_\mu=g_{\mu\nu}A^\nu=\begin{pmatrix}\phi\\-A_x\\-A_y\\-A_z\end{pmatrix}
\end{equation*}
{\footnotesize
\begin{verbatim}
gmunu = ((1,0,0,0),(0,-1,0,0),(0,0,-1,0),(0,0,0,-1))

A = (Ax(),Ay(),Az())
Au = (phi(),Ax(),Ay(),Az())
Ad = dot(gmunu,Au)
\end{verbatim}}

\begin{equation*}
\mathbf B=\nabla\times\mathbf A,\quad
\mathbf E=-\nabla\phi-\frac{\partial\mathbf A}{\partial t}
\end{equation*}
{\footnotesize
\begin{verbatim}
B = curl(A)
E = -d(phi(),(x,y,z)) - d(A,t)

Bx = B[1]
By = B[2]
Bz = B[3]

Ex = E[1]
Ey = E[2]
Ez = E[3]
\end{verbatim}}

This is the electromagnetic tensor.
\begin{equation*}
F_{\mu\nu}
=\partial_\mu A_\nu-\partial_\nu A_\mu
=A_{\nu,\mu}-A_{\mu,\nu}
=\begin{pmatrix}
0 & E_x & E_y & E_z
\\
-E_x & 0 & -B_z & B_y
\\
-E_y & B_z & 0 & -B_x
\\
-E_z & -B_y & B_x & 0
\end{pmatrix}
\end{equation*}
{\footnotesize
\begin{verbatim}
X = (t,x,y,z)
Fdd = d(Ad,X)
Fdd = transpose(Fdd) - Fdd

T = ((0, Ex, Ey, Ez),
     (-Ex, 0, -Bz, By),
     (-Ey, Bz, 0, -Bx),
     (-Ez, -By, Bx, 0))

check(Fdd == T)
\end{verbatim}}

Check the following relations.
\begin{equation*}
F_{\mu\nu}F^{\mu\nu}=2\bigl(\mathbf B^2-\mathbf E^2\bigl),\quad
\det(F_{\mu\nu})=\det(F^{\mu\nu})=(\mathbf B\cdot\mathbf E)^2
\end{equation*}
{\footnotesize
\begin{verbatim}
Fuu = dot(gmunu,Fdd,gmunu)
T = contract(dot(transpose(Fdd),Fuu))
check(T == 2 dot(B,B) - 2 dot(E,E))

check(det(Fdd) == dot(B,E)^2)
check(det(Fuu) == dot(B,E)^2)
\end{verbatim}}

This is the vector current.
\begin{equation*}
J^\nu
=\partial_\mu F^{\mu\nu}
=F^{\mu\nu}{}_{,\mu}
\end{equation*}

Gradient increases rank by one.
The new index is the rightmost index,
hence the contraction is over the first and third indices.
{\footnotesize
\begin{verbatim}
Ju = contract(d(Fuu,X),1,3)
\end{verbatim}}

Check the following relations.
\begin{equation*}
\partial_\mu J^\mu=J^\mu{}_{,\mu}=0,
\quad\mathbf J=\nabla\times\mathbf B-\frac{\partial\mathbf E}{\partial t}
\end{equation*}
{\footnotesize
\begin{verbatim}
check(contract(d(Ju,X)) == 0)

Jx = Ju[2]
Jy = Ju[3]
Jz = Ju[4]
J = (Jx,Jy,Jz)
check(J == curl(B) - d(E,t))
\end{verbatim}}

\end{document}
