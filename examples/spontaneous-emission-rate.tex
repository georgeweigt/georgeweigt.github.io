\input{preamble}

\section*{Spontaneous emission rate}

Find the spontaneous emission rate for hydrogen state $2p\rightarrow1s$.

\bigskip
The wave function for hydrogen is
\begin{equation*}
\psi_{nlm}(r,\theta,\phi)=R_{nl}(r)Y_{lm}(\theta,\phi)
\end{equation*}

where
\begin{gather*}
R_{nl}(r)=
\frac{2}{n^2}
\sqrt{\frac{(n-l-1)!}{(n+l)!}}
\left(\frac{2r}{na_0}\right)^l
L_{n-l-1}^{2l+1}\left(\frac{2r}{na_0}\right)
\exp\left(-\frac{r}{na_0}\right)
a_0^{-3/2}
\\[1ex]
L_n^m(x)=(n+m)!\sum_{k=0}^n\frac{(-x)^k}{(n-k)!(m+k)!k!}
\\[1ex]
Y_{lm}(\theta,\phi)=(-1)^m
\sqrt{\frac{(2l+1)}{4\pi}
\frac{(l-m)!}{(l+m)!}}
P_l^m(\cos\theta)\exp(im\phi)
\\[1ex]
P_l^m(\cos\theta)=\begin{cases}
\displaystyle
\left(\frac{\sin\theta}{2}\right)^m\,\sum_{k=0}^{l-m}
(-1)^k\frac{(l+m+k)!}{(l-m-k)!(m+k)!k!}
\left(\frac{1-\cos\theta}{2}\right)^k, & m\ge0
\\[3ex]
\displaystyle
(-1)^m\frac{(l+m)!}{(l-m)!}P_l^{|m|}(\cos\theta), & m<0
\end{cases}
\end{gather*}

State $2p$ is shorthand for $n=2$ and $l=1$.
For $l=1$ there are three ways to choose $m$ hence all of the following processes correspond to the transition
$2p\rightarrow1s$.
It turns out that all three processes have the same transition rate.
\begin{equation*}
\left.\begin{aligned}
&\psi_{2,1,1}
\\
&\psi_{2,1,0}
\\
&\psi_{2,1,-1}
\end{aligned}\right\}\rightarrow\psi_{100}+\text{photon}
\end{equation*}

The spontaneous emission rate is
\begin{equation*}
A_{21}=\frac{e^2}{3\pi\varepsilon_0\hbar c^3}\omega_{21}^3|r_{21}|^2
\end{equation*}

Noting that
\begin{equation*}
e^2=4\pi\varepsilon_0\hbar c\alpha
\end{equation*}

we can also write
\begin{equation*}
A_{21}=\frac{4\alpha}{3c^2}\omega_{21}^3|r_{21}|^2
\tag{1}
\end{equation*}

Verify dimensions:
\begin{equation*}
A_{21}\propto(\text{m/s})^{-2}\times\text{s}^{-3}\times\text{m}^2=\text{s}^{-1}\,\text{(or hertz)}
\end{equation*}

For angular frequency $\omega_{21}$ we have
\begin{equation*}
\omega_{21}=\frac{E_2-E_1}{\hbar},\quad E_n=-\frac{\alpha\hbar c}{2n^2a_0}
\end{equation*}

For displacement $r_{21}$ we have
\begin{equation*}
|r_{21}|^2=|x_{21}|^2+|y_{21}|^2+|z_{21}|^2
\end{equation*}

where
\begin{equation*}
x_{21}=\int\limits_{0}^{\infty}\int\limits_{0}^{\pi}\int\limits_{0}^{2\pi}xf_{21}\,dV,
\quad
y_{21}=\int\limits_{0}^{\infty}\int\limits_{0}^{\pi}\int\limits_{0}^{2\pi}yf_{21}\,dV,
\quad
z_{21}=\int\limits_{0}^{\infty}\int\limits_{0}^{\pi}\int\limits_{0}^{2\pi}zf_{21}\,dV
\end{equation*}

and
\begin{gather*}
f_{21}=\psi_{100}^*\psi_{210}%=\frac{r\cos\theta}{4\sqrt2\pi a_0^4}\exp\left(-\frac{3r}{2a_0}\right)
\\[1ex]
x=r\sin\theta\cos\phi,
\quad
y=r\sin\theta\sin\phi,
\quad
z=r\cos\theta
\\[1ex]
dV=r^2\sin\theta\,dr\,d\theta\,d\phi
\end{gather*}

The integrals work out to be
\begin{equation*}
x_{21}=0,
\quad
y_{21}=0,
\quad
z_{21}=\frac{2^7}{3^5}\sqrt2a_0
\end{equation*}
hence
\begin{equation*}
|r_{21}|^2=|z_{21}|^2=\frac{2^{15}}{3^{10}}a_0^2
\end{equation*}

We also obtain
\begin{equation*}
\omega_{21}=\frac{3\alpha c}{8a_0}
\end{equation*}

By equation (1) the spontaneous emission rate is
\begin{equation*}
A_{21}=\frac{2^8}{3^8}\frac{\alpha^4c}{a_0}=6.26\times10^8\,\text{s}^{-1}
\end{equation*}

Noting that
\begin{equation*}
a_0=\frac{\hbar}{\alpha\mu c}
\end{equation*}

we can also write
\begin{equation*}
A_{21}=\frac{2^8}{3^8}\frac{\alpha^5\mu c^2}{\hbar}=6.26\times10^8\,\text{s}^{-1}
\end{equation*}

where $\mu$ is reduced electron mass
\begin{equation*}
\mu=\frac{m_em_p}{m_e+m_p}
\end{equation*}

\end{document}
