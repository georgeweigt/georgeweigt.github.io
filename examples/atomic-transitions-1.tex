\input{preamble}

\section*{Atomic transitions 1}

Let $\Psi(\mathbf r,t)$ be the following linear combination of two atomic states.
\begin{equation*}
\Psi(\mathbf r,t)=\psi_a(\mathbf r)c_a(t)\exp\left(-\tfrac{i}{\hbar}E_at\right)+
\psi_b(\mathbf r)c_b(t)\exp\left(-\tfrac{i}{\hbar}E_bt\right)
\end{equation*}

Let the Hamiltonian be
\begin{equation*}
H(\mathbf r,t)=H_0(\mathbf r)+H_1(\mathbf r,t)
\end{equation*}

where
\begin{equation*}
H_0\psi_a=E_a\psi_a,\quad H_0\psi_b=E_b\psi_b
\end{equation*}

We want to find solutions for $c_a(t)$ and $c_b(t)$.
Start with the Schr\"odinger equation.
\begin{equation*}
i\hbar\frac{\partial}{\partial t}\Psi=H\Psi
\end{equation*}

Evaluate the left side of the Schr\"odinger equation.
\begin{multline*}
i\hbar\frac{\partial}{\partial t}\Psi
=\overbrace{E_a\psi_a(\mathbf r)c_a(t)\exp\left(-\tfrac{i}{\hbar}E_at\right)
+E_b\psi_b(\mathbf r)c_b(t)\exp\left(-\tfrac{i}{\hbar}E_bt\right)}^\text{cancels with right side}
\\
+i\hbar\psi_a(\mathbf r)\dot c_a(t)\exp\left(-\tfrac{i}{\hbar}E_at\right)
+i\hbar\psi_b(\mathbf r)\dot c_b(t)\exp\left(-\tfrac{i}{\hbar}E_bt\right)
\end{multline*}

Evaluate the right side of the Schr\"odinger equation.
\begin{equation*}
H\Psi
=\overbrace{E_a\psi_a(\mathbf r)c_a(t)\exp\left(-\tfrac{i}{\hbar}E_at\right)
+E_b\psi_b(\mathbf r)c_b(t)\exp\left(-\tfrac{i}{\hbar}E_bt\right)}^\text{cancels with left side}
+H_1\Psi
\end{equation*}

After cancellations
\begin{equation*}
i\hbar\psi_a(\mathbf r)\dot c_a(t)\exp\left(-\tfrac{i}{\hbar}E_at\right)
+i\hbar\psi_b(\mathbf r)\dot c_b(t)\exp\left(-\tfrac{i}{\hbar}E_bt\right)
=H_1\Psi
\tag{1}
\end{equation*}

Take the inner product $\psi_a$ and equation (1) to obtain
\begin{multline*}
i\hbar\dot c_a(t)\exp\left(-\tfrac{i}{\hbar}E_at\right)
\\
=\langle\psi_a|H_1|\Psi\rangle
=\langle\psi_a|H_1|\psi_a\rangle c_a(t)\exp\left(-\tfrac{i}{\hbar}E_at\right)
+\langle\psi_a|H_1|\psi_b\rangle c_b(t)\exp\left(-\tfrac{i}{\hbar}E_bt\right)
\tag{2}
\end{multline*}

Take the inner product of $\psi_b$ and equation (1) to obtain
\begin{multline*}
i\hbar\dot c_b(t)\exp\left(-\tfrac{i}{\hbar}E_bt\right)
\\
=\langle\psi_b|H_1|\Psi\rangle
=\langle\psi_b|H_1|\psi_a\rangle c_a(t)\exp\left(-\tfrac{i}{\hbar}E_at\right)
+\langle\psi_b|H_1|\psi_b\rangle c_b(t)\exp\left(-\tfrac{i}{\hbar}E_bt\right)
\tag{3}
\end{multline*}

Let it be the case that diagonal elements vanish, that is,
\begin{equation*}
\langle\psi_a|H_1|\psi_a\rangle=\langle\psi_b|H_1|\psi_b\rangle=0
\end{equation*}

Then (2) and (3) simplify as
\begin{equation*}
\begin{aligned}
i\hbar\dot c_a(t)\exp\left(-\tfrac{i}{\hbar}E_at\right)
&=\langle\psi_a|H_1|\psi_b\rangle c_b(t)\exp\left(-\tfrac{i}{\hbar}E_bt\right)
\\
i\hbar\dot c_b(t)\exp\left(-\tfrac{i}{\hbar}E_bt\right)
&=\langle\psi_b|H_1|\psi_a\rangle c_a(t)\exp\left(-\tfrac{i}{\hbar}E_at\right)
\end{aligned}
\tag{4}
\end{equation*}

Let $E_b>E_a$ and let
\begin{equation*}
\omega_0=\frac{E_b-E_a}{\hbar}
\end{equation*}

Rewrite (4) as
\begin{align*}
\dot c_a(t)&=-\frac{i}{\hbar}\langle\psi_a|H_1|\psi_b\rangle c_b(t)\exp(-i\omega_0t)
\\
\dot c_b(t)&=-\frac{i}{\hbar}\langle\psi_b|H_1|\psi_a\rangle c_a(t)\exp(i\omega_0t)
\end{align*}

Let the initial conditions be $c_a(0)=1$ and $c_b(0)=0$.
It was shown in ``Perturbation example'' that the first-order perturbation solutions are
\begin{align*}
c_a(t)&=1
\\
c_b(t)&=-\frac{i}{\hbar}\int_0^t
\langle\psi_b|H_1(\mathbf r,t')|\psi_a\rangle\exp(i\omega_0t')\,dt'
\end{align*}

\end{document}
