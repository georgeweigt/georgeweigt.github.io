\input{preamble}

% see Griffiths and Schroeter problem 5.35

\section*{White dwarf}

The radius of a white dwarf can be estimated using the electron gas model of a solid.

\bigskip
The total electron energy $E$ of a spherical electron gas is
\begin{equation*}
E=\left(\frac{3\pi^2}{2}\right)^\frac{1}{3}
\frac{9\hbar^2N^\frac{5}{3}}{20m_eR^2}
\end{equation*}

where $R$ is the radius, $N$ is the number of free electrons, and $m_e$ is electron mass.

\bigskip
The gravitational energy $U$ of a sphere with mass $M$ and uniform density is
\begin{equation*}
U=-\frac{3GM^2}{5R}
\end{equation*}

Minimize the total energy by finding $R$ such that
\begin{equation*}
\frac{d}{dR}(E+U)=0
\end{equation*}

Hence
\begin{equation*}
-\left(\frac{3\pi^2}{2}\right)^\frac{1}{3}
\frac{9\hbar^2N^\frac{5}{3}}{10m_eR^3}+\frac{3GM^2}{5R^2}=0
\end{equation*}

Multiply both sides by $R^3$.
\begin{equation*}
-\left(\frac{3\pi^2}{2}\right)^\frac{1}{3}
\frac{9\hbar^2N^\frac{5}{3}}{10m_e}+\frac{3GM^2}{5}R=0
\end{equation*}

Hence
\begin{equation*}
R=\left(\frac{3\pi^2}{2}\right)^\frac{1}{3}
\frac{9\hbar^2N^\frac{5}{3}}{10m_e}
\frac{5}{3GM^2}
=\left(\frac{3\pi^2}{2}\right)^\frac{1}{3}
\frac{3\hbar^2N^\frac{5}{3}}{2m_eGM^2}
\tag{1}
\end{equation*}

The number of free electrons is estimated to be one-half the number of nucleons.
\begin{equation*}
N=\frac{M}{2m_p}\tag{2}
\end{equation*}

Substitute (2) into (1) to obtain
\begin{equation*}
R=\frac{3\hbar^2}{8Gm_e}\left(\frac{3\pi^2}{Mm_p^5}\right)^\frac{1}{3}\tag{3}
\end{equation*}

For one solar mass $M=M_\odot=2\times10^{30}\,\text{kg}$ we have
\begin{equation*}
R=7146\,\text{km}
\end{equation*}

\end{document}
