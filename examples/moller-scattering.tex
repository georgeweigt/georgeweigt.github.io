\input{preamble}

\section*{Moller scattering}

Moller scattering is the process $e^-+e^-\rightarrow e^-+e^-$.

\begin{center}
\begin{tikzpicture}
\draw[dashed] (0,0) circle (0.5cm);
\draw[thick,->] (2,0) node[anchor=west] {$e^-$} -- (0.6,0);
\draw[thick,->] (-2,0) node[anchor=east] {$e^-$} -- (-0.6,0);
\draw[thick,->] (0.40,0.40) -- (1.3,1.3) node[anchor=south west] {$e^-$};
\draw[thick,->] (-0.4,-0.4) -- (-1.3,-1.3) node[anchor=north east] {$e^-$};
\draw (1,0.5) node {$\theta$};
\end{tikzpicture}
\end{center}

The following momentum vectors are for the center-of-mass frame with $E=\sqrt{p^2+m^2}$.
\begin{equation*}
p_1=\underset{e^- \, \longrightarrow}
{\begin{pmatrix}E\\0\\0\\p\end{pmatrix}}
\qquad
p_2=\underset{\longleftarrow \, e^-}
{\begin{pmatrix}E\\0\\0\\-p\end{pmatrix}}
\qquad
p_3=\underset{\substack{\phantom{e^-} \, \nearrow\\e^- \, \phantom{\nearrow}}}
{\begin{pmatrix}
E\\
p\sin\theta\cos\phi\\
p\sin\theta\sin\phi\\
p\cos\theta
\end{pmatrix}}
\qquad
p_4=\underset{\substack{\phantom{\swarrow} \, e^-\\\swarrow \, \phantom{e^-}}}
{\begin{pmatrix}
E\\
-p\sin\theta\cos\phi\\
-p\sin\theta\sin\phi\\
-p\cos\theta
\end{pmatrix}}
\end{equation*}

Spinors for $p_1$.
\begin{equation*}
u_{11}=\frac{1}{\sqrt{E+m}}
\underset{\text{spin up }}
{\begin{pmatrix}E+m\\0\\p\\0\end{pmatrix}}
\qquad
u_{12}=\frac{1}{\sqrt{E+m}}
\underset{\text{spin down }}
{\begin{pmatrix}0\\E+m\\0\\-p\end{pmatrix}}
\end{equation*}

Spinors for $p_2$.
\begin{equation*}
u_{21}=\frac{1}{\sqrt{E+m}}
\underset{\text{spin up}}
{\begin{pmatrix}E+m\\0\\-p\\0\end{pmatrix}}
\qquad
u_{22}=\frac{1}{\sqrt{E+m}}
\underset{\text{spin down}}
{\begin{pmatrix}0\\E+m\\0\\p\end{pmatrix}}
\end{equation*}

Spinors for $p_3$.
\begin{equation*}
u_{31}=\frac{1}{\sqrt{E+m}}
\underset{\text{spin up}}
{\begin{pmatrix}E+m\\0\\p_{3z}\\p_{3x}+ip_{3y}\end{pmatrix}}
\qquad
u_{32}=\frac{1}{\sqrt{E+m}}
\underset{\text{spin down}}
{\begin{pmatrix}0\\E+m\\p_{3x}-ip_{3y}\\-p_{3z}\end{pmatrix}}
\end{equation*}

Spinors for $p_4$.
\begin{equation*}
u_{41}=\frac{1}{\sqrt{E+m}}
\underset{\text{spin up}}
{\begin{pmatrix}E+m\\0\\p_{4z}\\p_{4x}+ip_{4y}\end{pmatrix}}
\qquad
u_{42}=\frac{1}{\sqrt{E+m}}
\underset{\text{spin down}}
{\begin{pmatrix}0\\E+m\\p_{4x}-ip_{4y}\\-p_{4z}\end{pmatrix}}
\end{equation*}

The scattering amplitude $\mathcal M_{abcd}$ for spin state $abcd$ is
\begin{equation*}
\mathcal M_{abcd}=\mathcal M_{1abcd}+\mathcal M_{2abcd}
\end{equation*}

where
\begin{equation*}
\mathcal M_{1abcd}=\frac{e^2}{t}
\underset{\text{no electron interchange}}
{(\bar{u}_{3c}\gamma^\mu u_{1a})(\bar{u}_{4d}\gamma_\mu u_{2b})},
\quad
\mathcal M_{2abcd}=-\frac{e^2}{u}
\underset{\text{electron interchange}}
{(\bar{u}_{4d}\gamma^\nu u_{1a})(\bar{u}_{3c}\gamma_\nu u_{2b})}
\end{equation*}

Symbols $t$ and $u$ are Mandelstam variables.
\begin{align*}
t&=(p_1-p_3)^2
\\
u&=(p_1-p_4)^2
\end{align*}

The expected probability density $\langle|\mathcal M|^2\rangle$
is the sum of squared amplitudes divided by the number of inbound states.
\begin{equation*}
\langle|\mathcal M|^2\rangle=\frac{1}{4}\sum_{abcd}|\mathcal M_{abcd}|^2
\end{equation*}

Expand the summand and label the terms.
By hermiticity $\boxed{\scriptstyle2}=\boxed{\scriptstyle3}$.
\begin{equation*}
\langle|\mathcal{M}|^2\rangle=\frac{1}{4}
\sum_{abcd}
\bigl(
\underset{\boxed{\scriptstyle1}}
{\mathcal M_{1abcd}\mathcal M_{1abcd}^*}+
\underset{\boxed{\scriptstyle2}}
{\mathcal M_{1abcd}\mathcal M_{2abcd}^*}+
\underset{\boxed{\scriptstyle3}}
{\mathcal M_{2abcd}\mathcal M_{1abcd}^*}+
\underset{\boxed{\scriptstyle4}}
{\mathcal M_{2abcd}\mathcal M_{2abcd}^*}
\bigr)
\end{equation*}

The following Casimir trick uses matrix arithmetic to sum over spin states.
\begin{align*}
\sum_{abcd}\boxed{\scriptstyle1}&=\frac{e^4}{t^2}
\operatorname{Tr}\left[
(\slashed p_3+m)\gamma^\mu(\slashed p_1+m)\gamma^\nu
\right]
\operatorname{Tr}\left[
(\slashed p_4+m)\gamma_\mu(\slashed p_2+m)\gamma_\nu
\right]
\\
\sum_{abcd}\boxed{\scriptstyle2}&=-\frac{e^4}{tu}
\operatorname{Tr}\left[
(\slashed p_3+m)\gamma^\mu(\slashed p_1+m)\gamma^\nu
(\slashed p_4+m)\gamma_\mu(\slashed p_2+m)\gamma_\nu
\right]
\\
\sum_{abcd}\boxed{\scriptstyle4}&=\frac{e^4}{u^2}
\operatorname{Tr}\left[
(\slashed p_4+m)\gamma^\mu(\slashed p_1+m)\gamma^\nu
\right]
\operatorname{Tr}\left[
(\slashed p_3+m)\gamma_\mu(\slashed p_2+m)\gamma_\nu
\right]
\end{align*}

Let
\begin{align*}
f_{11}&=
\operatorname{Tr}\left[
(\slashed p_3+m)\gamma^\mu(\slashed p_1+m)\gamma^\nu
\right]
\operatorname{Tr}\left[
(\slashed p_4+m)\gamma_\mu(\slashed p_2+m)\gamma_\nu
\right]
\\
f_{12}&=
-\operatorname{Tr}\left[
(\slashed p_3+m)\gamma^\mu(\slashed p_1+m)\gamma^\nu
(\slashed p_4+m)\gamma_\mu(\slashed p_2+m)\gamma_\nu
\right]
\\
f_{22}&=
\operatorname{Tr}\left[
(\slashed p_4+m)\gamma^\mu(\slashed p_1+m)\gamma^\nu
\right]
\operatorname{Tr}\left[
(\slashed p_3+m)\gamma_\mu(\slashed p_2+m)\gamma_\nu
\right]
\end{align*}

so that
\begin{equation*}
\langle|\mathcal{M}|^2\rangle
=\frac{e^4}{4}
\left(
\frac{f_{11}}{t^2}+\frac{2f_{12}}{tu}+\frac{f_{22}}{u^2}
\right)
\end{equation*}

The following formulas are equivalent to the Casimir trick.
(Recall that $a\cdot b=a^\mu g_{\mu\nu}b^\nu$)
\begin{align*}
f_{11}&=32 (p_1\cdot p_2)^2 + 32 (p_1\cdot p_4)^2 - 64 (p_1\cdot p_2) m^2 + 64 (p_1\cdot p_4) m^2
\\
f_{12}&=32 (p_1\cdot p_2)^2 - 64 (p_1\cdot p_2) m^2
\\
f_{22}&=32 (p_1\cdot p_2)^2 + 32 (p_1\cdot p_3)^2 - 64 (p_1\cdot p_2) m^2 + 64 (p_1\cdot p_3) m^2
\end{align*}

In Mandelstam variables
\begin{align*}
f_{11} &= 8 s^2 + 8 u^2 - 64 s m^2 - 64 u m^2 + 192 m^4
\\
f_{12} &= 8 s^2 - 64 s m^2 + 96 m^4
\\
f_{22} &= 8 s^2 + 8 t^2 - 64 s m^2 - 64 t m^2 + 192 m^4
\end{align*}

For $E\gg m$ a useful approximation is to set $m=0$ and obtain
\begin{align*}
f_{11}&=8s^2+8u^2\\
f_{12}&=8s^2\\
f_{22}&=8s^2+8t^2
\end{align*}

For $m=0$ the Mandelstam variables are
\begin{align*}
s&=4E^2
\\
t&=-2E^2(1-\cos\theta)
\\
u&=-2E^2(1+\cos\theta)
\end{align*}

Hence
\begin{align*}
\langle|\mathcal{M}|^2\rangle
&=\frac{e^4}{4}
\left(
\frac{f_{11}}{t^2}+\frac{2f_{12}}{tu}+\frac{f_{22}}{u^2}
\right)
\\
&=2e^4
\left(
\frac{s^2+u^2}{t^2}+\frac{2s^2}{tu}+\frac{s^2+t^2}{u^2}
\right)
\\
&=2e^4\biggl(
\underset{\text{no electron interchange}}
{\frac{1+\cos^4(\theta/2)}{\sin^4(\theta/2)}}
+
\underset{\text{interaction term}}
{\frac{2}{\sin^2(\theta/2)\cos^2(\theta/2)}}
+
\underset{\text{electron interchange}}
{\frac{1+\sin^4(\theta/2)}{\cos^4(\theta/2)}}
\biggr)
\end{align*}

The expected probability density can be written more compactly as
\begin{equation*}
\langle|\mathcal{M}|^2\rangle=4e^4\frac{(\cos^2\theta+3)^2}{\sin^4\theta}
\end{equation*}

\iffalse
Hence
\begin{align*}
\langle|\mathcal{M}|^2\rangle
&=\frac{e^4}{4}
\left(
\frac{f_{11}}{t^2}-\frac{f_{12}}{tu}-\frac{f_{12}^*}{tu}+\frac{f_{22}}{u^2}
\right)
\\
&=\frac{e^4}{4}
\left(
\frac{8s^2+8u^2}{t^2}-\frac{-8s^2}{tu}-\frac{-8s^2}{tu}+\frac{8s^2+8t^2}{u^2}
\right)
\\
&=2e^4
\left(
\frac{s^2+u^2}{t^2}+\frac{2s^2}{tu}+\frac{s^2+t^2}{u^2}
\right)
\end{align*}

Combine terms so $\langle|\mathcal{M}|^2\rangle$ has a common denominator.
\begin{equation*}
\langle|\mathcal{M}|^2\rangle
=2e^4
\left(
\frac{u^2(s^2+u^2)+2s^2tu+t^2(s^2+t^2)}{t^2u^2}
\right)
\end{equation*}

For $m=0$ the Mandelstam variables are
\begin{align*}
s&=4E^2
\\
t&=2E^2(\cos\theta-1)
\\
u&=-2E^2(\cos\theta+1)
\end{align*}

Hence
\begin{align*}
\langle|\mathcal{M}|^2\rangle
&=2e^4
\left(
\frac{32E^8\cos^4\theta+192E^8\cos^2\theta+288E^8}{16E^8(\cos\theta-1)^2(\cos\theta+1)^2}
\right)
\\
&=4e^4\frac{\left(\cos^2\theta+3\right)^2}{(\cos\theta-1)^2(\cos\theta+1)^2}
\\
&=4e^4
\frac{(\cos^2\theta+3)^2}{\sin^4\theta}
\end{align*}

The following equivalent formula can also be used.
\begin{align*}
\langle|\mathcal{M}|^2\rangle
&=2e^4
\left(
\frac{s^2+u^2}{t^2}+\frac{2s^2}{tu}+\frac{s^2+t^2}{u^2}
\right)
\\
&=2e^4\biggl(
\underset{\substack{\\[1ex]\text{no electron interchange}}}
{\frac{1+\cos^4(\theta/2)}{\sin^4(\theta/2)}}
+
\underset{\substack{\\[1ex]\text{interaction term}}}
{\frac{2}{\sin^2(\theta/2)\cos^2(\theta/2)}}
+
\underset{\substack{\\[1ex]\text{electron interchange}}}
{\frac{1+\sin^4(\theta/2)}{\cos^4(\theta/2)}}
\biggr)
\end{align*}
\fi

\subsubsection*{Cross section}

The differential cross section is
\begin{equation*}
\frac{d\sigma}{d\Omega}=\frac{\langle|\mathcal{M}|^2\rangle}{4(4\pi\varepsilon_0)^2s}
\end{equation*}

where
\begin{equation*}
s=(p_1+p_2)^2=4E^2
\end{equation*}

For high energy experiments we have
\begin{equation*}
\langle|\mathcal{M}|^2\rangle=4e^4\frac{(\cos^2\theta+3)^2}{\sin^4\theta}
\end{equation*}

Hence
\begin{equation*}
\frac{d\sigma}{d\Omega}=\frac{e^4}{(4\pi\varepsilon_0)^2s}\frac{(\cos^2\theta+3)^2}{\sin^4\theta}
\end{equation*}

Noting that
\begin{equation*}
e^2=4\pi\varepsilon_0\alpha\hbar c
\end{equation*}

we have
\begin{equation*}
\frac{d\sigma}{d\Omega}=\frac{\alpha^2(\hbar c)^2}{s}
\frac{(\cos^2\theta+3)^2}{\sin^4\theta}
\end{equation*}

Noting that
\begin{equation*}
d\Omega=\sin\theta\,d\theta\,d\phi
\end{equation*}

we also have
\begin{equation*}
d\sigma=\frac{\alpha^2(\hbar c)^2}{s}
\frac{(\cos^2\theta+3)^2}{\sin^4\theta}
\sin\theta\,d\theta\,d\phi
\end{equation*}

Let $S(\theta_1,\theta_2)$ be the following integral of $d\sigma$.
\begin{equation*}
S(\theta_1,\theta_2)=\int_0^{2\pi}\int_{\theta_1}^{\theta_2}d\sigma
\end{equation*}

The solution is
\begin{equation*}
S(\theta_1,\theta_2)=\frac{2\pi\alpha^2(\hbar c)^2}{s}[I(\theta_2)-I(\theta_1)]
\end{equation*}

where
\begin{equation*}
I(\theta)=-\frac{8\cos\theta}{\sin^2\theta}-\cos\theta
\end{equation*}

The cumulative distribution function is
\begin{equation*}
F(\theta)
=\frac{S(a,\theta)}{S(a,\pi-a)}
=\frac{I(\theta)-I(a)}{I(\pi-a)-I(a)},
\quad
a\le\theta\le\pi-a
\end{equation*}

Angular support is reduced by an arbitrary angle $a>0$ because $I(0)$ and $I(\pi)$ are undefined.

\bigskip
The probability of observing scattering events in the interval $\theta_1$ to $\theta_2$ is
\begin{equation*}
P(\theta_1<\theta\le\theta_2)=F(\theta_2)-F(\theta_1)
\end{equation*}

The probability density function is
\begin{equation*}
f(\theta)=\frac{dF(\theta)}{d\theta}
=\frac{1}{I(\pi-a)-I(a)}
\frac{\left(\cos^2\theta+3\right)^2}{\sin^4\theta}\sin\theta
\end{equation*}

\subsubsection*{Note}

A.~Zee page 134 has the cross section
\begin{equation*}
\frac{d\sigma}{d\Omega}=\left(\frac{e^2}{4\pi}\right)^2\frac{1}{8E^2}f(\theta)
\end{equation*}
where $f(\theta)$ is the probability density function
\begin{equation*}
f(\theta)=
\frac{1+\cos^4(\theta/2)}{\sin^4(\theta/2)}
+\frac{2}{\sin^2(\theta/2)\cos^2(\theta/2)}
+\frac{1+\sin^4(\theta/2)}{\cos^4(\theta/2)}
\end{equation*}

The probability density function is equivalent to
\begin{equation*}
f(\theta)=\frac{2(\cos^2\theta+3)^2}{\sin^4\theta}
\end{equation*}

Hence for natural units $\varepsilon_0=\hbar=c=1$ and $e^2=4\pi\alpha$ the above cross section is equivalent to
\begin{equation*}
\frac{d\sigma}{d\Omega}=\frac{\alpha^2(\hbar c)^2}{4E^2}
\frac{(\cos^2\theta+3)^2}{\sin^4\theta}
\end{equation*}

\end{document}
