\input{preamble}

\section*{Rotating wave approximation}

Let $\Psi(\mathbf r,t)$ be the following wave function for a two state system.
\begin{equation*}
\Psi(\mathbf r,t)
=c_a(t)\psi_a(\mathbf r)\exp(-\tfrac{i}{\hbar}E_at)
+c_b(t)\psi_b(\mathbf r)\exp(-\tfrac{i}{\hbar}E_bt)
\end{equation*}

Let $\hat H(\mathbf r,t)$ be the Hamiltonian
\begin{equation*}
\hat H(\mathbf r,t)=\hat H_0(\mathbf r)+\hat H_1(\mathbf r,t)
\end{equation*}

where
\begin{equation*}
\hat H_0\psi_a=E_a\psi_a,\quad\hat H_0\psi_b=E_b\psi_b
\end{equation*}

From the Schr\"odinger equation
\begin{equation*}
i\hbar\frac{\partial}{\partial t}\Psi=\hat H\Psi
\end{equation*}

we obtain the differential equations
\begin{equation*}
\begin{aligned}
\frac{d}{dt}c_a(t)&=-\frac{i}{\hbar}c_a(t)\langle\psi_a|\hat H_1|\psi_a\rangle
-\frac{i}{\hbar}c_b(t)\langle\psi_a|\hat H_1|\psi_b\rangle\exp(-i\omega_0t)
\\[1ex]
\frac{d}{dt}c_b(t)&=-\frac{i}{\hbar}c_b(t)\langle\psi_b|\hat H_1|\psi_b\rangle
-\frac{i}{\hbar}c_a(t)\langle\psi_b|\hat H_1|\psi_a\rangle\exp(i\omega_0t)
\end{aligned}
\end{equation*}

where
\begin{equation*}
\omega_0=\frac{E_b-E_a}{\hbar}
\end{equation*}

Typically the diagonal elements vanish as
\begin{equation*}
\langle\psi_a|\hat H_1|\psi_a\rangle=\langle\psi_b|\hat H_1|\psi_b\rangle=0
\end{equation*}

and the differential equations become
\begin{align*}
\frac{d}{dt}c_a(t)&=-\frac{i}{\hbar}c_b(t)\langle\psi_a|\hat H_1|\psi_b\rangle\exp(-i\omega_0t)
\tag{1}
\\[1ex]
\frac{d}{dt}c_b(t)&=-\frac{i}{\hbar}c_a(t)\langle\psi_b|\hat H_1|\psi_a\rangle\exp(i\omega_0t)
\tag{2}
\end{align*}

Let $\hat H_1(\mathbf r,t)$ be the perturbation
\begin{equation*}
\hat H_1(\mathbf r,t)=\hat V(\mathbf r)\cos(\omega t)
\end{equation*}

Then
\begin{equation*}
\langle\psi_a|\hat H_1|\psi_b\rangle
=\langle\psi_a|\hat V|\psi_b\rangle
\left(\tfrac{1}{2}\exp(i\omega t)+\tfrac{1}{2}\exp(-i\omega t)\right)
\end{equation*}

The rotating wave approximation discards the second term and asserts
\begin{equation*}
\langle\psi_a|\hat H_1|\psi_b\rangle=\tfrac{1}{2}\langle\psi_a|\hat V|\psi_b\rangle\exp(i\omega t)
\tag{3}
\end{equation*}

Substitute equation (3) into (1) and (2) to obtain
\begin{align*}
\frac{d}{dt}c_a(t)&=-\frac{i}{2\hbar}c_b(t)\langle\psi_a|\hat V|\psi_b\rangle
\exp\bigl(i(\omega-\omega_0)t\bigr)
\tag{4}
\\[1ex]
\frac{d}{dt}c_b(t)&=-\frac{i}{2\hbar}c_a(t)\langle\psi_b|\hat V|\psi_a\rangle
\exp\bigl(i(\omega_0-\omega)t\bigr)
\tag{5}
\end{align*}

Use Laplace transforms to solve for $c_b(t)$ with initial conditions $c_a(0)=1$ and $c_b(0)=0$.
\begin{equation*}
c_b(t)=-\frac{i}{\hbar}\langle\psi_b|\hat V|\psi_a\rangle
\frac{\sin(\omega_rt)}{2\omega_r}
\exp\left(\tfrac{i}{2}(\omega_0-\omega)t\right)
\tag{6}
\end{equation*}

Symbol $\omega_r$ is the Rabi flopping frequency
\begin{equation*}
\omega_r=\frac{1}{2}\sqrt{(\omega_0-\omega)^2
+\bigl|\langle\psi_a|\hat V|\psi_b\rangle\bigr|^2/\hbar^2}
\end{equation*}

Use equation (2) and the solution for $c_b(t)$ to solve for $c_a(t)$.
\begin{equation*}
c_a(t)=\left[\cos(\omega_rt)+i\left(\frac{\omega_0-\omega}{2\omega_r}\right)\sin(\omega_rt)\right]
\exp\left(-\tfrac{i}{2}(\omega_0-\omega)t\right)
\end{equation*}

Rewrite $\omega_r$ as
\begin{equation*}
\omega_r=\frac{1}{2\hbar}\sqrt{\hbar^2(\omega_0-\omega)^2
+\bigl|\langle\psi_a|\hat V|\psi_b\rangle\bigr|^2}
\end{equation*}

and note that for
\begin{equation*}
\hbar^2(\omega_0-\omega)^2\gg\bigl|\langle\psi_a|\hat V|\psi_b\rangle\bigr|^2
\end{equation*}

we have
\begin{equation*}
\omega_r\approx\tfrac{1}{2}|\omega_0-\omega|
\tag{7}
\end{equation*}

Substitute (7) into (6) to obtain
\begin{equation*}
c_b(t)=-\frac{i}{\hbar}\langle\psi_b|\hat V|\psi_a\rangle
\frac{\sin\left(\tfrac{1}{2}|\omega_0-\omega|t\right)}{|\omega_0-\omega|}
\exp\left(\tfrac{i}{2}(\omega_0-\omega)t\right)
\end{equation*}

This is equivalent to $c_b(t)$ obtained from first order perturbation expansion.

\end{document}
