\input{preamble}

\section*{Atomic transitions 3}

From the previous section
\begin{equation*}
c_b(t)=\frac{ieE_0}{m\hbar\omega}
\langle\psi_b|\exp(i\mathbf k\cdot\mathbf r)\boldsymbol{\epsilon}\cdot\mathbf p|\psi_a\rangle
\frac{\sin\bigl(\tfrac{1}{2}(\omega_0-\omega)t\bigr)}{\omega_0-\omega}
\exp\bigl(\tfrac{i}{2}(\omega_0-\omega)t\bigr)
\end{equation*}

The transition probability is
\begin{equation*}
\Pr_{a\rightarrow b}(t)=|c_b(t)|^2
=\frac{e^2E_0^2}{m^2\hbar^2\omega^2}
\bigl|\langle\psi_b|\exp(i\mathbf k\cdot\mathbf r)\boldsymbol{\epsilon}\cdot\mathbf p|\psi_a\rangle\bigr|^2
\frac{\sin^2\bigl(\tfrac{1}{2}(\omega_0-\omega)t\bigr)}{(\omega_0-\omega)^2}
\end{equation*}

We are now going to pivot from $E_0$ to a full radiation field.
Let $u$ be energy density such that
\begin{equation*}
E_0^2=\frac{2u}{\varepsilon_0}
\end{equation*}

By substitution
\begin{equation*}
\Pr_{a\rightarrow b}(t)
=\frac{2u}{\varepsilon_0}
\frac{e^2}{m^2\hbar^2\omega^2}
\bigl|\langle\psi_b|\exp(i\mathbf k\cdot\mathbf r)\boldsymbol{\epsilon}\cdot\mathbf p|\psi_a\rangle\bigr|^2
\frac{\sin^2\bigl(\tfrac{1}{2}(\omega_0-\omega)t\bigr)}{(\omega_0-\omega)^2}
\end{equation*}

For a full radiation field
\begin{equation*}
u=\int_{-\infty}^\infty\rho(\omega)\,d\omega
\end{equation*}

where $\rho(\omega)$ is energy per volume per hertz.
By substitution
\begin{equation*}
\Pr_{a\rightarrow b}(t)
=\frac{2}{\varepsilon_0}
\frac{e^2}{m^2\hbar^2}
\bigl|\langle\psi_b|\exp(i\mathbf k\cdot\mathbf r)\boldsymbol{\epsilon}\cdot\mathbf p|\psi_a\rangle\bigr|^2
\int_{-\infty}^\infty
\frac{\rho(\omega)}{\omega^2}
\frac{\sin^2\bigl(\tfrac{1}{2}(\omega_0-\omega)t\bigr)}{(\omega_0-\omega)^2}\,d\omega
\end{equation*}

Because the integrand is sharply peaked at $\omega=\omega_0$,
we are going to make the following move.
Substitute $\omega_0$ for $\omega$ in $\rho(\omega)/\omega^2$.
That makes the term a constant so it can be moved outside the integral.
We now have
\begin{equation*}
\Pr_{a\rightarrow b}(t)
=\frac{2}{\varepsilon_0}
\frac{e^2}{m^2\hbar^2}
\bigl|\langle\psi_b|\exp(i\mathbf k\cdot\mathbf r)\boldsymbol{\epsilon}\cdot\mathbf p|\psi_a\rangle\bigr|^2
\frac{\rho(\omega_0)}{\omega_0^2}
\int_{-\infty}^\infty
\frac{\sin^2\bigl(\tfrac{1}{2}(\omega_0-\omega)t\bigr)}{(\omega_0-\omega)^2}\,d\omega
\end{equation*}

Apply change of variable
\begin{equation*}
y=\frac{1}{2}(\omega-\omega_0)t,\quad
dy=\frac{t}{2}\,d\omega,\quad
\omega-\omega_0=\frac{2y}{t}
\end{equation*}

to obtain
\begin{equation*}
\int_{-\infty}^\infty
\frac{\sin^2\bigl(\tfrac{1}{2}(\omega_0-\omega)t\bigr)}{(\omega_0-\omega)^2}\,d\omega
=\frac{t}{2}\int_{-\infty}^\infty\frac{\sin^2(-y)}{y^2}\,dy
=\frac{\pi}{2}t
\end{equation*}

Hence
\begin{equation*}
\Pr_{a\rightarrow b}(t)
=\frac{\pi e^2}{\varepsilon_0m^2\hbar^2}
\bigl|\langle\psi_b|\exp(i\mathbf k\cdot\mathbf r)\boldsymbol{\epsilon}\cdot\mathbf p|\psi_a\rangle\bigr|^2
\frac{\rho(\omega_0)}{\omega_0^2}
t
\end{equation*}

The transition rate is
\begin{equation*}
R_{a\rightarrow b}=\frac{d}{dt}\Pr_{a\rightarrow b}(t)
=\frac{\pi e^2}{\varepsilon_0m^2\hbar^2}
\bigl|\langle\psi_b|\exp(i\mathbf k\cdot\mathbf r)\boldsymbol{\epsilon}\cdot\mathbf p|\psi_a\rangle\bigr|^2
\frac{\rho(\omega_0)}{\omega_0^2}
\end{equation*}

Verify dimensions.
\begin{equation*}
R_{a\rightarrow b}\propto
\frac{\begin{matrix}
e^2
\\
\text{C}^2
\end{matrix}
}{\begin{matrix}
\varepsilon_0 & m^2 & \hbar^2
\\
\text{C}^2\,\text{J}^{-1}\,\text{m}^{-1}
& \text{kg}^2 & \text{J}^2\,\text{s}^2
\end{matrix}}
\times
\begin{matrix}
\\
\bigl|\langle\psi_b|\exp(i\mathbf k\cdot\mathbf r)\boldsymbol{\epsilon}\cdot\mathbf p|\psi_a\rangle\bigr|^2
\\
\text{kg}^2\,\text{m}^2\,\text{s}^{-2}
\end{matrix}
\times
\frac{
\begin{matrix}
\rho(\omega_0)
\\
\text{J}\,\text{m}^{-3}\,\text{s}
\end{matrix}
}{
\begin{matrix}
\omega_0^2
\\
\text{s}^{-2}
\end{matrix}
}
=\text{s}^{-1}
\end{equation*}

\end{document}
