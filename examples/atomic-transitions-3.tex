\input{preamble}

\section*{Atomic transitions 3}

In the previous section we obtained
\begin{equation*}
c_b(t)=\frac{ieE_0}{m\hbar\omega}
\langle\psi_b|\boldsymbol{\epsilon}\cdot\mathbf p\exp(i\mathbf k\cdot\mathbf r)|\psi_a\rangle
\frac{\sin\bigl(\tfrac{1}{2}(\omega_0-\omega)t\bigr)}{\omega_0-\omega}
\exp\bigl(\tfrac{i}{2}(\omega_0-\omega)t\bigr)
\tag{1}
\end{equation*}

Use the dipole approximation
\begin{equation*}
\exp(i\mathbf k\cdot\mathbf r)=1+i\mathbf k\cdot\mathbf r+\cdots\approx1
\end{equation*}

to obtain
\begin{equation*}
\langle\psi_b|\boldsymbol{\epsilon}\cdot\mathbf p\exp(i\mathbf k\cdot\mathbf r)|\psi_a\rangle
\approx
\langle\psi_b|\boldsymbol{\epsilon}\cdot\mathbf p|\psi_a\rangle
\end{equation*}

By the identity
\begin{equation*}
\mathbf p=\frac{im}{\hbar}[H_0,\mathbf r]
\tag{2}
\end{equation*}

we have
\begin{align*}
\langle\psi_b|\boldsymbol{\epsilon}\cdot\mathbf p|\psi_a\rangle
&=\frac{im}{\hbar}\langle\psi_b|\boldsymbol{\epsilon}\cdot[H_0,\mathbf r]|\psi_a\rangle
\\
&=\frac{im}{\hbar}\langle\psi_b|\boldsymbol{\epsilon}\cdot H_0\mathbf r|\psi_a\rangle
-\frac{im}{\hbar}\langle\psi_b|\boldsymbol{\epsilon}\cdot\mathbf rH_0|\psi_a\rangle
\\
&=\frac{im}{\hbar}E_b\langle\psi_b|\boldsymbol{\epsilon}\cdot\mathbf r|\psi_a\rangle
-\frac{im}{\hbar}E_a\langle\psi_b|\boldsymbol{\epsilon}\cdot\mathbf r|\psi_a\rangle
\\
&=\frac{im}{\hbar}(E_b-E_a)\langle\psi_b|\boldsymbol{\epsilon}\cdot\mathbf r|\psi_a\rangle
\\
&=im\omega_0\langle\psi_b|\boldsymbol{\epsilon}\cdot\mathbf r|\psi_a\rangle
\end{align*}

Hence
\begin{equation*}
\langle\psi_b|\boldsymbol{\epsilon}\cdot\mathbf p\exp(i\mathbf k\cdot\mathbf r)|\psi_a\rangle
\approx
im\omega_0\langle\psi_b|\boldsymbol{\epsilon}\cdot\mathbf r|\psi_a\rangle
\end{equation*}

Substitute into (1) to obtain
\begin{equation*}
c_b(t)=-\frac{eE_0}{\hbar}\frac{\omega_0}{\omega}
\langle\psi_b|\boldsymbol{\epsilon}\cdot\mathbf r|\psi_a\rangle
\frac{\sin\bigl(\tfrac{1}{2}(\omega_0-\omega)t\bigr)}{\omega_0-\omega}
\exp\bigl(\tfrac{i}{2}(\omega_0-\omega)t\bigr)
\end{equation*}

Verify dimensions.
\begin{equation*}
c_b(t)\propto
\frac{\begin{matrix}
e & E_0
\\
\text{coulomb} & \text{newton}\,\text{coulomb}^{-1}
\end{matrix}}
{\begin{matrix}
\hbar
\\
\text{joule}\,\text{second}
\end{matrix}}
\times
\frac{\begin{matrix}
\omega_0
\\
\text{second}^{-1}
\end{matrix}}
{\begin{matrix}
\omega
\\
\text{second}^{-1}
\end{matrix}}
\times
\frac{\begin{matrix}
\langle\psi_b|\boldsymbol{\epsilon}\cdot\mathbf r|\psi_a\rangle
\\
\text{meter}
\end{matrix}}
{\begin{matrix}
\omega_0-\omega
\\
\text{second}^{-1}
\end{matrix}}
=1
\end{equation*}

Note that for an experiment with $\boldsymbol{\epsilon}\cdot\mathbf r=z$ we have
\begin{equation*}
c_b(t)=-\frac{eE_0}{\hbar}\frac{\omega_0}{\omega}
\langle\psi_b|z|\psi_a\rangle
\frac{\sin\bigl(\tfrac{1}{2}(\omega_0-\omega)t\bigr)}{\omega_0-\omega}
\exp\bigl(\tfrac{i}{2}(\omega_0-\omega)t\bigr)
\end{equation*}

\end{document}
