\input{preamble}

\section*{Harmonic oscillator coherent state}

A coherent state minimizes uncertainty.
The ground state is a coherent state.
To make the ground state more interesting, parameters $r$ and $\theta$ are added
to shift $\langle x\rangle$ and $\langle p\rangle$ from zero.
Time dependence is also added.
\begin{multline*}
\psi_{n,r,\theta}(x,t)
=\frac{1}{\sqrt{2^nn!}}
\left(\frac{m\omega}{\pi\hbar}\right)^\frac{1}{4}
H_n\left(\sqrt{\frac{m\omega}{\hbar}}\left(x-\langle x\rangle\right)\right)
\\{}\times
\exp\left[
-\frac{m\omega}{2\hbar}\left(x-\langle x\rangle\right)^2
+\frac{i}{\hbar}\langle p\rangle\left(x-\frac{\langle x\rangle}{2}\right)
-i\left(n+\frac{1}{2}\right)\omega t
\right]
\end{multline*}

Parameters $r$ and $\theta$ are polar coordinates in phase space such that
\begin{equation*}
\langle x\rangle=\sqrt{\frac{2\hbar}{m\omega}}\,r\cos(\omega t+\theta),\quad
\langle p\rangle=-\sqrt{2m\hbar\omega}\,r\sin(\omega t+\theta)
\end{equation*}

Note that $\psi_{0,0,\theta}(x,0)$ is equivalent to the ordinary ground state.
\begin{equation*}
\psi_{0,0,\theta}(x,0)=\left(\frac{m\omega}{\pi\hbar}\right)^\frac{1}{4}
\exp\left(-\frac{m\omega x^2}{2\hbar}\right)
\end{equation*}

\subsubsection*{Exercises}

1. Verify for $n=1$
\begin{equation*}
i\hbar\frac{d}{dt}\psi(x,t)=\hat H\psi(x,t)
\end{equation*}

2. Verify for $n=1$
\begin{equation*}
\int_{-\infty}^\infty\psi^*(x,t)\psi(x,t)\,dx=1
\end{equation*}

3. Verify for the ground state $\psi_0(x,t)$
\begin{equation*}
\Delta x=\sqrt{\langle x^2\rangle-\langle x\rangle^2}
=\sqrt{\frac{\hbar}{2m\omega}}
\end{equation*}

and
\begin{equation*}
\Delta p=\sqrt{\langle p^2\rangle-\langle p\rangle^2}
=\sqrt{\frac{m\hbar\omega}{2}}
\end{equation*}

Hence $\Delta x\Delta p$ is the minimum allowed by the uncertainty principle.
\begin{equation*}
\Delta x\Delta p=\frac{\hbar}{2}
\end{equation*}

\end{document}
