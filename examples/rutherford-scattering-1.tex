\input{preamble}

% Cf. Feynman and Hibbs problem 6-6

\section*{Rutherford scattering 1}

Find the scattering cross section for the Coulomb potential
\begin{equation*}
V(r)=-\frac{Ze^2}{4\pi\varepsilon_0r}
\end{equation*}

Start with the Born approximation for scattering amplitude
$f(\mathbf p)$ where $\mathbf p$ is momentum transfer.
\begin{equation*}
f(\mathbf p)
=\frac{m}{2\pi\hbar^2}
\int\exp\left(\frac{i\mathbf p\cdot\mathbf r}{\hbar}\right)V(\mathbf r)\,d\mathbf r
\end{equation*}

Convert to polar coordinates where $p=|\mathbf p|$.
\begin{equation*}
f(\mathbf p)
=\frac{m}{2\pi\hbar^2}
\int_0^\infty
\int_0^\pi
\int_0^{2\pi}
\exp\left(\frac{ipr\cos\theta}{\hbar}\right)V(r,\theta,\phi)
\,r^2\sin\theta\,dr\,d\theta\,d\phi
\end{equation*}

Substitute the Coulomb potential
\begin{equation*}
V(r,\theta,\phi)=-\frac{Ze^2}{4\pi\epsilon_0r}
\end{equation*}

to obtain
\begin{equation*}
f(\mathbf p)
=-\frac{mZe^2}{8\pi^2\varepsilon_0\hbar^2}
\int_0^\infty
\int_0^\pi
\int_0^{2\pi}
\exp\left(\frac{ipr\cos\theta}{\hbar}\right)
r\sin\theta\,dr\,d\theta\,d\phi
\end{equation*}

Integrate over $\phi$ (multiply by $2\pi$).
\begin{equation*}
f(\mathbf p)
=-\frac{mZe^2}{4\pi\varepsilon_0\hbar^2}
\int_0^\infty
\int_0^\pi
\exp\left(\frac{ipr\cos\theta}{\hbar}\right)
r\sin\theta\,dr\,d\theta
\end{equation*}

Transform the integral over $\theta$ into an integral over $y$
where $y=\cos\theta$ and $dy=-\sin\theta\,d\theta$.
The minus sign in $dy$ is canceled by interchanging integration limits
$\cos0=1$ and $\cos\pi=-1$.
\begin{equation*}
f(\mathbf p)
=-\frac{mZe^2}{4\pi\varepsilon_0\hbar^2}
\int_0^\infty
\int_{-1}^1
\exp\left(\frac{ipry}{\hbar}\right)
r\,dr\,dy
\end{equation*}

Solve the integral over $y$ and note that $r$ in the integrand cancels.
\begin{equation*}
f(\mathbf p)=-\frac{mZe^2}{4\pi\varepsilon_0\hbar^2}
\int_0^\infty
\frac{\hbar}{ip}
\left[\exp\left(\frac{ipr}{\hbar}\right)-\exp\left(-\frac{ipr}{\hbar}\right)\right]
dr
\end{equation*}

Solve the integral over $r$.
\begin{equation*}
f(\mathbf p)
=-\frac{mZe^2}{4\pi\varepsilon_0\hbar^2}\frac{\hbar}{ip}
\left[
\frac{\hbar}{ip}
\exp\left(\frac{ipr}{\hbar}\right)
+\frac{\hbar}{ip}
\exp\left(-\frac{ipr}{\hbar}\right)
\right]_{r=0}^{r=\infty}
\end{equation*}

The exponentials fail to converge at the upper limit.
The workaround is to go back and multiply the integrand by $\exp(-\epsilon r)\approx1$.
\begin{equation*}
f(\mathbf p)=-\frac{mZe^2}{4\pi\varepsilon_0\hbar^2}
\int_0^\infty
\frac{\hbar}{ip}
\left[\exp\left(\frac{ipr}{\hbar}-\epsilon r\right)
-\exp\left(-\frac{ipr}{\hbar}-\epsilon r\right)\right]
dr
\end{equation*}

Solve the modified integral.
\begin{equation*}
f(\mathbf p)=-\frac{mZe^2}{4\pi\varepsilon_0\hbar^2}
\frac{\hbar}{ip}
\left[
\frac{1}{ip/\hbar-\epsilon}\exp\left(\frac{ipr}{\hbar}-\epsilon r\right)
+\frac{1}{ip/\hbar+\epsilon}\exp\left(-\frac{ipr}{\hbar}-\epsilon r\right)
\right]_{r=0}^{r=\infty}
\end{equation*}

Evaluate the limits. Now the exponentials vanish at the upper limit.
\begin{equation*}
f(\mathbf p)=-\frac{mZe^2}{4\pi\varepsilon_0\hbar^2}
\frac{\hbar}{ip}
\left(-\frac{1}{ip/\hbar-\epsilon}-\frac{1}{ip/\hbar+\epsilon}\right)
%=-\frac{mZe^2}{4\pi\varepsilon_0\hbar^2}\frac{2}{(p/\hbar)^2+\epsilon^2}
\tag{1}
\end{equation*}

Set $\epsilon=0$ to obtain
\begin{equation*}
f(\mathbf p)=-\frac{mZe^2}{2\pi\varepsilon_0p^2}
\end{equation*}

Substitute $e^2=4\pi\varepsilon_0\alpha\hbar c$.
\begin{equation*}
f(\mathbf p)=-\frac{2mZ\alpha\hbar c}{p^2}
\end{equation*}

Note that $\mathbf p$ is momentum transfer such that
\begin{equation*}
p^2=|\mathbf p|^2=4mE(1-\cos\theta)
\end{equation*}

Hence
\begin{equation*}
f(\theta)=-\frac{Z\alpha\hbar c}{2E(1-\cos\theta)}
\end{equation*}

Calculate the cross section.
\begin{equation*}
\frac{d\sigma}{d\Omega}=|f(\theta)|^2
=\frac{Z^2\alpha^2(\hbar c)^2}{4E^2(1-\cos\theta)^2}
\tag{2}
\end{equation*}

Note that
\begin{equation*}
(1-\cos\theta)^2=4\sin^4(\theta/2)
\end{equation*}

Hence equation (2) is equivalent to
\begin{equation*}
\frac{d\sigma}{d\Omega}=\frac{Z^2\alpha^2(\hbar c)^2}{16E^2\sin^4(\theta/2)}
\tag{3}
\end{equation*}

\end{document}
