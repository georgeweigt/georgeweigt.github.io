\input{preamble}

% Feynman and Hibbs problem 6-6

\section*{Rutherford scattering 1}

Use the following formula to compute the cross section for Rutherford scattering.
\begin{equation*}
\frac{d\sigma}{d\Omega}=\frac{1}{64\pi^2\varepsilon_0^2}\left(\frac{mQ}{2\pi\hbar^2}\right)^2,\quad
Q=\int\exp\left(\frac{i\mathbf p\cdot\mathbf r}{\hbar}\right)V(\mathbf r)\,d\mathbf r^3
\end{equation*}

Convert $Q$ to polar coordinates.
\begin{equation*}
Q=\int_0^{2\pi}
\int_0^\pi
\int_0^\infty
\exp\left(\frac{ipr\cos\theta}{\hbar}\right)V(r,\theta,\phi)
\,r^2\sin\theta\,dr\,d\theta\,d\phi
\end{equation*}

For Rutherford scattering $V(\mathbf r)$ is the Coulomb potential.
\begin{equation*}
V(\mathbf r)=V(r)=-\frac{Ze^2}{r}
\end{equation*}

Substitute the Coulomb potential for $V(r,\theta,\phi)$ and note $r^2$ becomes $r$.
\begin{equation*}
Q=-Ze^2
\int_0^{2\pi}
\int_0^\pi
\int_0^\infty
\exp\left(\frac{ipr\cos\theta}{\hbar}\right)
\,r\sin\theta\,dr\,d\theta\,d\phi
\end{equation*}

Integrate over $\phi$.
\begin{equation*}
Q=-2\pi Ze^2
\int_0^\pi
\int_0^\infty
\exp\left(\frac{ipr\cos\theta}{\hbar}\right)
\,r\sin\theta\,dr\,d\theta
\end{equation*}

Change the complex exponential to rectangular form.
\begin{equation*}
Q=-2\pi Ze^2
\int_0^\pi
\int_0^\infty
\left[
\cos\left(\frac{pr\cos\theta}{\hbar}\right)
+i\sin\left(\frac{pr\cos\theta}{\hbar}\right)
\right]
V(r)\,r\sin\theta\,dr\,d\theta
\end{equation*}

By the integrals
\begin{equation*}
\int_0^\pi\cos\bigl(a\cos(\theta)\bigr)\sin\theta\,d\theta=\frac{2\sin a}{a},\quad
\int_0^\pi\sin\bigl(a\cos(\theta)\bigr)\sin\theta\,d\theta=0
\end{equation*}

we obtain (note $r$ in the integrand is canceled)
\begin{equation*}
Q=-\frac{4\pi\hbar Ze^2}{p}
\int_0^\infty\sin\left(\frac{pr}{\hbar}\right)\,dr
\end{equation*}

To solve the integral, multiply the integrand by $\exp(-\epsilon r)$.
\begin{equation*}
Q=-\frac{4\pi\hbar Ze^2}{p}
\int_0^\infty\sin\left(\frac{pr}{\hbar}\right)\exp(-\epsilon r)\,dr
\end{equation*}

Convert the integrand to exponential form.
\begin{equation*}
Q=-\frac{4\pi\hbar Ze^2}{p}
\int_0^\infty\frac{i}{2}\left[
\exp\left(-\frac{ipr}{\hbar}-\epsilon r\right)
-\exp\left(\frac{ipr}{\hbar}-\epsilon r\right)
\right]\,dr
\end{equation*}

Solve the integral.
\begin{equation*}
Q=-\frac{4\pi\hbar Ze^2}{p}\frac{i}{2}
\left(\frac{1}{ip/\hbar+\epsilon}+\frac{1}{ip/\hbar-\epsilon}\right)
\tag{1}
\end{equation*}

Set $\epsilon=0$.
\begin{equation*}
Q=-\frac{4\pi\hbar^2Ze^2}{p^2}
\end{equation*}

Compute the differential cross section.
\begin{equation*}
\frac{d\sigma}{d\Omega}=\frac{1}{64\pi^2\varepsilon_0^2}\left(\frac{mQ}{2\pi\hbar^2}\right)^2
=\frac{1}{16\pi^2\varepsilon_0^2}\frac{m^2Z^2e^4}{p^4}
\tag{2}
\end{equation*}

Substitute $(4\pi\varepsilon_0\alpha\hbar c)^2$ for $e^4$.
\begin{equation*}
\frac{d\sigma}{d\Omega}=\frac{m^2Z^2\alpha^2(\hbar c)^2}{p^4}
\end{equation*}

Symbol $p$ is momentum transfer $|\mathbf p_i|-|\mathbf p_f|$ such that
\begin{equation*}
p^2=2mE(1-\cos\theta)
\end{equation*}

Hence
\begin{equation*}
\frac{d\sigma}{d\Omega}=\frac{Z^2\alpha^2(\hbar c)^2}{4E^2(1-\cos\theta)^2}
\tag{3}
\end{equation*}

Noting that
\begin{equation*}
4\sin^4\frac{\theta}{2}=(1-\cos\theta)^2
\end{equation*}

we have the alternative form of (3)
\begin{equation*}
\frac{d\sigma}{d\Omega}=\frac{Z^2\alpha^2(\hbar c)^2}{16E^2\sin^4(\theta/2)}
\end{equation*}

\subsubsection*{Experimental data}
The following data is from Geiger and Marsden's 1913 paper where
$y$ is the number of scattering events.
\begin{equation*}
\begin{matrix}
\theta & y\\
150 & 22.2\\
135 & 27.4\\
120 & 33.0\\
105 & 47.3\\
75 & 136\\
60 & 320\\
45 & 989\\
37.5 & 1760\\
30 & 5260\\
22.5 & 20300\\
15 & 105400
\end{matrix}
\end{equation*}

Let
\begin{equation*}
x_i=\frac{1}{(1-\cos\theta_i)^2}
\end{equation*}

The scattering probability for angle $\theta_i$ is $x_i$ normalized by $\sum x=4529$.
\begin{equation*}
\Pr(\theta_i)=\frac{x_i}{4529}
\end{equation*}

Predicted values $\hat y_i$ are $\Pr(\theta_i)$
times total scattering events $\sum y=134295$.
\begin{equation*}
\hat y_i=\Pr(\theta_i)\times134295
\end{equation*}

The following table shows the predicted values $\hat y$.
\begin{equation*}
\begin{matrix}
\theta & y & \hat y\\
150 & 22.2 & 34.1\\
135 & 27.4 & 40.7\\
120 & 33.0 & 52.7\\
105 & 47.3 & 74.9\\
75 & 136 & 216\\
60 & 320 & 474\\
45 & 989 & 1383\\
37.5 & 1760 & 2778\\
30 & 5260 & 6608\\
22.5 & 20300 & 20471\\
15 & 105400 & 102162
\end{matrix}
\end{equation*}

The coefficient of determination $R^2$ measures how well predicted values fit the data.
\begin{equation*}
R^2=1-\frac{\sum_i(y_i-\hat y_i)^2}{\sum_i(y_i-\bar y)^2}=0.999
\end{equation*}

The result indicates that $x$ explains 99.9\%
of the variance in the data.

\end{document}
