\input{preamble}

\section*{Atomic transitions 7}

From the previous section the spontaneous emission rate is
\begin{equation*}
A_{b\rightarrow a}=\frac{e^2}{3\pi\varepsilon_0\hbar c^3}\omega_{ba}^3
\bigl|\langle\psi_a|\mathbf r|\psi_b\rangle\bigr|^2
\end{equation*}

For hydrogen $2p\rightarrow1s$ we have
\begin{equation*}
A_{21}=\frac{e^2}{3\pi\varepsilon_0\hbar c^3}\omega_{21}^3
\bigl|\langle\psi_{100}|\mathbf r|\psi_{210}\rangle\bigr|^2
\end{equation*}

Noting that
\begin{equation*}
e^2=4\pi\varepsilon_0\hbar c\alpha
\end{equation*}

we can also write
\begin{equation*}
A_{21}=\frac{4\alpha}{3c^2}\omega_{21}^3
\bigl|\langle\psi_{100}|\mathbf r|\psi_{210}\rangle\bigr|^2
\tag{1}
\end{equation*}

Verify dimensions:
\begin{equation*}
A_{21}\propto(\text{m/s})^{-2}\times\text{s}^{-3}\times\text{m}^2=\text{s}^{-1}\,\text{(or hertz)}
\end{equation*}

For angular frequency $\omega_{21}$ we have
\begin{equation*}
\omega_{21}=\frac{E_2-E_1}{\hbar}
\end{equation*}

and for hydrogen
\begin{equation*}
E_n=-\frac{\alpha\hbar c}{2n^2a_0}
\end{equation*}

Hence
\begin{equation*}
\omega_{21}=\frac{3\alpha c}{8a_0}
\end{equation*}

For the transition ``probability'' we have
\begin{equation*}
\bigl|\langle\psi_{100}|\mathbf r|\psi_{210}\rangle\bigr|^2
=|x_{21}|^2+|y_{21}|^2+|z_{21}|^2
\end{equation*}

where
\begin{align*}
x_{21}&=\int\limits_{0}^{\infty}\int\limits_{0}^{\pi}\int\limits_{0}^{2\pi}
\psi_{100}^*\,x\,\psi_{210}\,dV
\\
y_{21}&=\int\limits_{0}^{\infty}\int\limits_{0}^{\pi}\int\limits_{0}^{2\pi}
\psi_{100}^*\,y\,\psi_{210}\,dV
\\
z_{21}&=\int\limits_{0}^{\infty}\int\limits_{0}^{\pi}\int\limits_{0}^{2\pi}
\psi_{100}^*\,z\,\psi_{210}\,dV
\end{align*}

and
\begin{equation*}
x=r\sin\theta\cos\phi,
\quad
y=r\sin\theta\sin\phi,
\quad
z=r\cos\theta,
\quad
dV=r^2\sin\theta\,dr\,d\theta\,d\phi
\end{equation*}

The integrals work out to be
\begin{equation*}
x_{21}=0,
\quad
y_{21}=0,
\quad
z_{21}=\frac{2^7}{3^5}\sqrt2a_0
\end{equation*}

hence
\begin{equation*}
\bigl|\langle\psi_{100}|\mathbf r|\psi_{210}\rangle\bigr|^2=|z_{21}|^2=\frac{2^{15}}{3^{10}}a_0^2
\end{equation*}

By equation (1) the spontaneous emission rate is
\begin{equation*}
A_{21}=\frac{2^8}{3^8}\frac{\alpha^4c}{a_0}=6.26\times10^8\,\text{s}^{-1}
\end{equation*}

Noting that
\begin{equation*}
a_0=\frac{\hbar}{\alpha\mu c}
\end{equation*}

we can also write
\begin{equation*}
A_{21}=\frac{2^8}{3^8}\frac{\alpha^5\mu c^2}{\hbar}=6.26\times10^8\,\text{s}^{-1}
\end{equation*}

where $\mu$ is reduced electron mass
\begin{equation*}
\mu=\frac{m_em_p}{m_e+m_p}
\end{equation*}

\href{https://georgeweigt.github.io/examples/atomic-transitions-7-demo.html}{Eigenmath code}

\end{document}
