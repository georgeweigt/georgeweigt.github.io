\input{preamble}

% Feynman and Hibbs problem 9-5

\section*{Field momentum}

For a free field show that
\begin{equation*}
\frac{1}{4\pi c}\int\mathbf E\times\mathbf B\,d^3\mathbf r
=i\int\mathbf k\left(\mathbf a_{\mathbf k}^*\cdot\dot{\mathbf a}_{\mathbf k}\right)\,
\frac{d^3\mathbf k}{(2\pi)^3}
\end{equation*}

This is the electric field vector.
\begin{equation*}
\mathbf E(\mathbf r,t)
=\int\left(-i\mathbf k\phi_{\mathbf k}-\sqrt{4\pi}\,\dot{\mathbf a}_{\mathbf k}\right)
\exp(i\mathbf k\cdot\mathbf r)\,
\frac{d^3\mathbf k}{(2\pi)^3}
\end{equation*}

Set $\phi_{\mathbf k}=0$ for no charges.
\begin{equation*}
\mathbf E(\mathbf r,t)
=-\sqrt{4\pi}\int\dot{\mathbf a}_{\mathbf k}
\exp(i\mathbf k\cdot\mathbf r)\,
\frac{d^3\mathbf k}{(2\pi)^3}
\end{equation*}

This is the magnetic field vector.
\begin{equation*}
\mathbf B(\mathbf r,t)
=\sqrt{4\pi}ic\int(\mathbf k\times\mathbf a_{\mathbf k})
\exp(i\mathbf k\cdot\mathbf r)\,
\frac{d^3\mathbf k}{(2\pi)^3}
\end{equation*}

Take the complex conjugate of $\mathbf B$.
\begin{equation*}
\mathbf B(\mathbf r,t)
=-\sqrt{4\pi}ic\int(\mathbf k\times\mathbf a_{\mathbf k}^*)
\exp(-i\mathbf k\cdot\mathbf r)\,
\frac{d^3\mathbf k}{(2\pi)^3}
\end{equation*}

Hence
\begin{equation*}
\mathbf E\times\mathbf B
=4\pi ic\iint\dot{\mathbf a}_{\mathbf k}\times(\mathbf k'\times\mathbf a_{\mathbf k'}^*)
\exp\bigl(i(\mathbf k-\mathbf k')\cdot\mathbf r\bigr)
\,\frac{d^3\mathbf k}{(2\pi)^3}\,\frac{d^3\mathbf k'}{(2\pi)^3}
\end{equation*}

Integrate over $\mathbf r$ to change the exponential to a delta function.
\begin{equation*}
\int\mathbf E\times\mathbf B\,d^3\mathbf r
=4\pi ic\iint\dot{\mathbf a}_{\mathbf k}\times(\mathbf k'\times\mathbf a_{\mathbf k'}^*)
(2\pi)^3\delta(\mathbf k-\mathbf k')
\,\frac{d^3\mathbf k}{(2\pi)^3}\,\frac{d^3\mathbf k'}{(2\pi)^3}
\end{equation*}

The delta function vanishes except for $\mathbf k=\mathbf k'$.
\begin{equation*}
\int\mathbf E\times\mathbf B\,d^3\mathbf r
=4\pi ic\int\dot{\mathbf a}_{\mathbf k}\times(\mathbf k\times\mathbf a_{\mathbf k})
\,\frac{d^3\mathbf k}{(2\pi)^3}
\end{equation*}

By the triple cross product formula
\begin{equation*}
\int\mathbf E\times\mathbf B\,d\mathbf r
=4\pi ic\int
\bigl((\dot{\mathbf a}_{\mathbf k}\cdot\mathbf a_{\mathbf k}^*)\mathbf k
-(\dot{\mathbf a}_{\mathbf k}\cdot\mathbf k)\mathbf a_{\mathbf k}^*\bigr)
\,\frac{d^3\mathbf k}{(2\pi)^3}
\end{equation*}

By orthogonality of $\mathbf E$ and $\mathbf k$ the $\dot{\mathbf a}_{\mathbf k}\cdot\mathbf k$
term vanishes hence
\begin{equation*}
\int\mathbf E\times\mathbf B\,d\mathbf r
=4\pi ic\int
(\dot{\mathbf a}_{\mathbf k}\cdot\mathbf a_{\mathbf k}^*)\mathbf k
\,\frac{d^3\mathbf k}{(2\pi)^3}
\end{equation*}

\end{document}
