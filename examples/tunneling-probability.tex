\input{preamble}

\section*{Tunneling probability}

Consider the following potential energy function.
\begin{equation*}
V(x)=\begin{cases}
0, & x<0
\\
V_0, & 0\le x\le L
\\
0, & x>L
\end{cases}
\end{equation*}

Let a particle with mass $m$ and energy $E<V_0$ travel from left to right
resulting in the following three Schrodinger equations.
\begin{align*}
\frac{\hbar^2}{2m}\frac{d^2}{dx}&\psi_1=E\psi_1, & &x<0
\\[1ex]
\frac{\hbar^2}{2m}\frac{d^2}{dx}&\psi_2+V_0\psi_2=E\psi_2, & &0\le x\le L
\\[1ex]
\frac{\hbar^2}{2m}\frac{d^2}{dx}&\psi_3=E\psi_3, & &x>L
\end{align*}

Let $\psi_1$ and $\psi_3$ have the most general
free-particle solutions.
\begin{align*}
\psi_1(x)=A\exp\left(i\sqrt{\frac{2mE}{\hbar^2}}x\right)
+B\exp\left(-i\sqrt{\frac{2mE}{\hbar^2}}x\right)
\\
\psi_3(x)=F\exp\left(i\sqrt{\frac{2mE}{\hbar^2}}x\right)
+G\exp\left(-i\sqrt{\frac{2mE}{\hbar^2}}x\right)
\end{align*}

Use the WKB approximation to solve for $\psi_2$.
\begin{equation*}
\psi_2(x)\approx
C\exp\left(i\int\sqrt{\frac{2m(E-V_0)}{\hbar^2}}\,dx\right)
+D\exp\left(-i\int\sqrt{\frac{2m(E-V_0)}{\hbar^2}}\,dx\right)
\end{equation*}

Cancel $i$ by swapping $E$ and $V_0$.
\begin{equation*}
\psi_2(x)\approx
C\exp\left(\int\sqrt{\frac{2m(V_0-E)}{\hbar^2}}\,dx\right)
+D\exp\left(-\int\sqrt{\frac{2m(V_0-E)}{\hbar^2}}\,dx\right)
\end{equation*}

Substitute $x$ for $\int dx$.
\begin{equation*}
\psi_2(x)\approx
C\exp\left(\sqrt{\frac{2m(V_0-E)}{\hbar^2}}x\right)
+D\exp\left(-\sqrt{\frac{2m(V_0-E)}{\hbar^2}}x\right)
\end{equation*}

To simplify the formulas let
\begin{equation*}
k=\sqrt{\frac{2mE}{\hbar^2}},\quad\beta=\sqrt{\frac{2m(V_0-E)}{\hbar^2}}
\end{equation*}

and write
\begin{align*}
\psi_1(x)&=A\exp(ikx)+B\exp(-ikx)
\\
\psi_2(x)&=C\exp(\beta x)+D\exp(-\beta x)
\\
\psi_3(x)&=F\exp(ikx)+G\exp(-ikx)
\end{align*}

Exponentials of $-i$ represent particles moving from right to left.
The $B$ exponential represents a particle reflected from the
boundary at $x=0$.
There is no particle moving right to left at $x>L$ hence $G=0$.

\bigskip
Let us now solve for the coefficients using boundary conditions.
Four boundary conditions are needed to ensure continuity
at $x=0$ and $x=L$.
\begin{align*}
\psi_1(0)&=\psi_2(0)
\\
\psi_1'(0)&=\psi_2'(0)
\\
\psi_2(L)&=\psi_3(L)
\\
\psi_2'(L)&=\psi_3'(L)
\end{align*}

From the boundary condition $\psi_2(L)=\psi_3(L)$ we have
\begin{equation*}
C\exp(\beta L)+D\exp(-\beta L)=F\exp(ikL)
\tag{1}
\end{equation*}

From the boundary condition $\psi_2'(L)=\psi_3'(L)$ we have
\begin{equation*}
\beta C\exp(\beta L)-\beta D\exp(-\beta L)
=ikF\exp(ikL)
\tag{2}
\end{equation*}

Add $\beta$ times (1) to (2) to obtain
\begin{equation*}
2\beta C\exp(\beta L)=(\beta+ik)F\exp(ikL)
\end{equation*}

Hence
\begin{equation*}
C=\frac{(\beta+ik)F\exp(ikL-\beta L)}{2\beta}
\tag{3}
\end{equation*}

Add minus $\beta$ times (1) to (2) to obtain
\begin{equation*}
-2\beta D\exp(-\beta L)=(-\beta+ik)F\exp(ikL)
\end{equation*}

Hence
\begin{equation*}
D=\frac{(\beta-ik)F\exp(ikL+\beta L)}{2\beta}
\tag{4}
\end{equation*}

From the boundary condition $\psi_1(0)=\psi_2(0)$ we have
\begin{equation*}
A+B=C+D
\tag{5}
\end{equation*}

From the boundary condition $\psi_1'(0)=\psi_2'(0)$ we have
\begin{equation*}
ik(A-B)=\beta(C-D)
\tag{6}
\end{equation*}

Add $ik$ times (5) to (6) to obtain
\begin{equation*}
2ikA=\beta(C-D)+ik(C-D)
\end{equation*}

Hence
\begin{equation*}
A=\frac{\beta(C-D)}{2ik}+\frac{C+D}{2}
\end{equation*}

Substitute (3) and (4) for $C$ and $D$ to obtain
the simplified form
\begin{equation*}
A = F\exp(ikL)\bigl(\cosh(\beta L)+i(\gamma/2)\sinh(\beta L)\bigr)
\tag{7}
\end{equation*}

where
\begin{equation*}
\gamma=\frac{\beta}{k}-\frac{k}{\beta}
\end{equation*}

The tunneling probability $T$ is the magnitude of the transmitted
wave divided by the magnitude of the inbound wave.
\begin{equation*}
T=\frac{|F|^2}{|A|^2}
=\left|\frac{1}{\exp(ikL)\bigl(\cosh(\beta L)+i(\gamma/2)\sinh(\beta L)\bigr)}\right|^2
\end{equation*}

Hence
\begin{equation*}
T=\frac{1}{\cosh^2(\beta L)+(\gamma/2)^2\sinh^2(\beta L)}
\tag{8}
\end{equation*}

(Ref. ``Quantum Tunneling of Particles through Potential Barriers''
at phys.libretexts.org)

\end{document}
