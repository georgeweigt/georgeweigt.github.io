\input{preamble}

\section*{Statistics for hydrogen atom}

Start with the ground state wave function.
\begin{equation*}
\psi_{100}(r,\theta,\phi)=\frac{1}{\sqrt{\pi a_0^3}}\exp\left(-\frac{r}{a_0}\right)
\end{equation*}

The cumulative distribution function $\Pr(r<a)$ is obtained by integrating
$|\psi_{100}|^2$ over the volume element $r^2\sin\theta\,dr\,d\theta\,d\phi$.
\begin{equation*}
\Pr(r<a)=\frac{1}{\pi a_0^3}
\int_0^a\int_0^\pi\int_0^{2\pi}\exp\left(-\frac{2r}{a_0}\right)
r^2\sin\theta\,dr\,d\theta\,d\phi
\end{equation*}

Integrate over $\phi$ (multiply by $2\pi$).
\begin{equation*}
\Pr(r<a)=\frac{2}{a_0^3}
\int_0^a\int_0^\pi\exp\left(-\frac{2r}{a_0}\right)r^2\sin\theta\,dr\,d\theta
\end{equation*}

Transform the integral over $\theta$ to an integral over $y$ where
$y=\cos\theta$ and $dy=-\sin\theta\,d\theta$.
The minus sign in $dy$ is canceled by interchanging integration limits
$\cos0=1$ and $\cos\pi=-1$.
\begin{equation*}
\Pr(r<a)=\frac{2}{a_0^3}
\int_0^a\int_{-1}^1\exp\left(-\frac{2r}{a_0}\right)r^2\,dr\,dy
\end{equation*}

Integrate over $y$ (multiply by 2).
\begin{equation*}
\Pr(r<a)=\frac{4}{a_0^3}
\int_0^a\exp\left(-\frac{2r}{a_0}\right)r^2\,dr
\end{equation*}

Solve the integral over $r$.
\begin{equation*}
\Pr(r<a)=1-\left(\frac{2a^2}{a_0^2}+\frac{2a}{a_0}+1\right)
\exp\left(-\frac{2a}{a_0}\right)
\tag{1}
\end{equation*}

For $a=a_0$ we have
\begin{equation*}
\Pr(r<a_0)=0.32
\end{equation*}

Hence the probability that the electron is inside the Bohr radius is 32\%.

\bigskip
Another form of the cumulative distribution function is $F(r)$ where
\begin{equation*}
F(r)=\frac{4}{a_0^3}
\int\exp\left(-\frac{2r}{a_0}\right)r^2\,dr
=-\left(\frac{2r^2}{a_0^2}+\frac{2r}{a_0}+1\right)
\exp\left(-\frac{2r}{a_0}\right)
\tag{2}
\end{equation*}

Then
\begin{equation*}
\Pr(a<r<b)=F(b)-F(a)
\end{equation*}

and
\begin{equation*}
\Pr(r<a)=F(a)-F(0)
\end{equation*}

The probability density function $f(r)$ is the derivative of $F(r)$.
\begin{equation*}
f(r)=\frac{dF(r)}{dr}=\frac{4r^2}{a_0^3}\exp\left(-\frac{2r}{a_0}\right)
\tag{3}
\end{equation*}

Use $f(r)$ to compute expectation values.
\begin{align*}
\langle r\rangle&=\int_0^\infty rf(r)\,dr=\tfrac{3}{2}a_0
\\
\langle r^2\rangle&=\int_0^\infty r^2f(r)\,dr=3a_0^2
\end{align*}

Hence
\begin{equation*}
\operatorname{Var}(r)=\langle r^2\rangle-\langle r\rangle^2=\tfrac{3}{4}a_0^2
\end{equation*}

\end{document}
