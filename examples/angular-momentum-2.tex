\input{preamble}

\section*{Angular momentum 2}

The cross product in $\mathbf L=\mathbf r\times\mathbf p$
demands rectangular coordinates.
Hence for a wavefunction $\psi$ in spherical coordinates,
vectors $\mathbf r$ and $\mathbf p$
must be transformed in $\mathbf L\psi$.

\medskip
Vector $\mathbf r$ transforms as
\begin{equation*}
\mathbf r=\begin{pmatrix}x\\y\\z\end{pmatrix}
=\begin{pmatrix}
r\sin\theta\cos\phi
\\
r\sin\theta\sin\phi
\\
r\cos\theta
\end{pmatrix}
\end{equation*}

To transform $\mathbf p$ we have by the chain rule
\begin{align*}
%
%Dx(f) = sin(theta) cos(phi) d(f,r) +
%        cos(theta) cos(phi) / r d(f,theta) -
%        sin(phi) / (r sin(theta)) d(f,phi)
%
\frac{\partial}{\partial x}
&=\sin\theta\cos\phi\frac{\partial}{\partial r}
+\frac{\cos\theta\cos\phi}{r}\frac{\partial}{\partial\theta}
-\frac{\sin\phi}{r\sin\theta}\frac{\partial}{\partial\phi}
\\[1ex]
%
%Dy(f) = sin(theta) sin(phi) d(f,r) +
%        cos(theta) sin(phi) / r d(f,theta) +
%        cos(phi) / (r sin(theta)) d(f,phi)
%
\frac{\partial}{\partial y}
&=\sin\theta\sin\phi\frac{\partial}{\partial r}
+\frac{\cos\theta\sin\phi}{r}\frac{\partial}{\partial\theta}
+\frac{\cos\phi}{r\sin\theta}\frac{\partial}{\partial\phi}
\\[1ex]
%
%Dz(f) = cos(theta) d(f,r) - sin(theta) / r d(f,theta)
%
\frac{\partial}{\partial z}
&=\cos\theta\frac{\partial}{\partial r}
-\frac{\sin\theta}{r}\frac{\partial}{\partial\theta}
\end{align*}

Hence
\begin{align*}
p_x&=-i\hbar\frac{\partial}{\partial x}=-i\hbar\left(
\sin\theta\cos\phi\frac{\partial}{\partial r}
+\frac{\cos\theta\cos\phi}{r}\frac{\partial}{\partial\theta}
-\frac{\sin\phi}{r\sin\theta}\frac{\partial}{\partial\phi}
\right)
\\[1ex]
p_y&=-i\hbar\frac{\partial}{\partial y}=-i\hbar\left(
\sin\theta\sin\phi\frac{\partial}{\partial r}
+\frac{\cos\theta\sin\phi}{r}\frac{\partial}{\partial\theta}
+\frac{\cos\phi}{r\sin\theta}\frac{\partial}{\partial\phi}
\right)
\\[1ex]
p_z&=-i\hbar\frac{\partial}{\partial z}=-i\hbar\left(
\cos\theta\frac{\partial}{\partial r}
-\frac{\sin\theta}{r}\frac{\partial}{\partial\theta}
\right)
\end{align*}

Using the transformed coordinates
\begin{align*}
%
%x = r sin(theta) cos(phi)
%y = r sin(theta) sin(phi)
%z = r cos(theta)
%
x&=r\sin\theta\cos\phi
\\
y&=r\sin\theta\sin\phi
\\
z&=r\cos\theta
\end{align*}

we have in spherical coordinates
\begin{align*}
%
%Lx(f) = i hbar (sin(phi) d(f,theta) +
%        cos(theta) cos(phi) / sin(theta) d(f,phi))
%
L_x&=yp_z-zp_y=i\hbar\left(\sin\phi\frac{\partial}{\partial\theta}
+\frac{\cos\theta\cos\phi}{\sin\theta}\frac{\partial}{\partial\phi}\right)
\\[1ex]
%
%Ly(f) = i hbar (-cos(phi) d(f,theta) +
%        cos(theta) sin(phi) / sin(theta) d(f,phi))
%
L_y&=zp_x-xp_z=i\hbar\left(-\cos\phi\frac{\partial}{\partial\theta}
+\frac{\cos\theta\sin\phi}{\sin\theta}\frac{\partial}{\partial\phi}\right)
\\[1ex]
%
%Lz(f) = -i hbar d(f,phi)
%
L_z&=xp_y-yp_x=-i\hbar\frac{\partial}{\partial\phi}
\end{align*}

and
\begin{equation*}
%
%LL(f) = -hbar^2 (d(f,theta,theta) +
%        cos(theta) / sin(theta) d(f,theta) +
%        1 / sin(theta)^2 d(f,phi,phi))
%
L^2={L_x}^2+{L_y}^2+{L_z}^2=-\hbar^2\left(
\frac{\partial^2}{\partial\theta^2}
+\frac{\cos\theta}{\sin\theta}\frac{\partial}{\partial\theta}
+\frac{1}{\sin^2\theta}\frac{\partial^2}{\partial\phi^2}
\right)
\end{equation*}

\href{https://georgeweigt.github.io/examples/angular-momentum-3-demo.html}{Eigenmath script}

\subsubsection*{Note}

The following construction is from section 4.3.2 of
``Introduction to Quantum Mechanics''
by Griffiths and Schroeter.

\medskip
Noting that in spherical coordinates
\begin{equation*}
\mathbf r=r\mathbf e_r
\end{equation*}

and
\begin{equation*}
\nabla=\mathbf e_r\frac{\partial}{\partial r}
+\mathbf e_\theta\frac{1}{r}\frac{\partial}{\partial\theta}
+\mathbf e_\phi\frac{1}{r\sin\theta}\frac{\partial}{\partial\phi}
\end{equation*}

we have for angular momentum $\mathbf L$
\begin{equation*}
\mathbf L=-i\hbar(\mathbf r\times\nabla)
=-i\hbar\left[
r(\mathbf e_r\times\mathbf e_r)\frac{\partial}{\partial r}
+(\mathbf e_r\times\mathbf e_\theta)\frac{\partial}{\partial\theta}
+(\mathbf e_r\times\mathbf e_\phi)\frac{1}{\sin\theta}\frac{\partial}{\partial\phi}
\right]
\end{equation*}

In rectangular coordinates
\begin{equation*}
%er = (sin(theta) cos(phi), sin(theta) sin(phi), cos(theta))
\mathbf e_r=\begin{pmatrix}
\sin\theta\cos\phi
\\
\sin\theta\sin\phi
\\
\cos\theta
\end{pmatrix},\quad
%etheta = (cos(theta) cos(phi), cos(theta) sin(phi), -sin(theta))
\mathbf e_\theta=\begin{pmatrix}
\cos\theta\cos\phi
\\
\cos\theta\sin\phi
\\
-\sin\theta
\end{pmatrix},\quad
%ephi = (-sin(phi), cos(phi), 0)
\mathbf e_\phi=\begin{pmatrix}
-\sin\phi
\\
\cos\phi
\\
0
\end{pmatrix}
\end{equation*}

Hence for the cross products
\begin{align*}
\mathbf e_r\times\mathbf e_r&=0
\\
\mathbf e_r\times\mathbf e_\theta&=\mathbf e_\phi
\\
\mathbf e_r\times\mathbf e_\phi&=-\mathbf e_\theta
\end{align*}

Angular momentum $\mathbf L$ now reduces to
\begin{equation*}
\mathbf L=-i\hbar\left(
\mathbf e_\phi\frac{\partial}{\partial\theta}
-\mathbf e_\theta\frac{1}{\sin\theta}\frac{\partial}{\partial\phi}
\right)
\end{equation*}

Substitute for the basis vectors.
\begin{equation*}
\mathbf L=-i\hbar
\begin{pmatrix}
-\sin\phi
\\
\cos\phi
\\
0
\end{pmatrix}
\frac{\partial}{\partial\theta}
+i\hbar
\begin{pmatrix}
\cos\theta\cos\phi
\\
\cos\theta\sin\phi
\\
-\sin\theta
\end{pmatrix}
\frac{1}{\sin\theta}\frac{\partial}{\partial\phi}
\end{equation*}

In component form
\begin{align*}
L_x&=i\hbar\left(\sin\phi\frac{\partial}{\partial\theta}
+\frac{\cos\theta\cos\phi}{\sin\theta}\frac{\partial}{\partial\phi}\right)
\\[1ex]
L_y&=i\hbar\left(-\cos\phi\frac{\partial}{\partial\theta}
+\frac{\cos\theta\sin\phi}{\sin\theta}\frac{\partial}{\partial\phi}\right)
\\[1ex]
L_z&=-i\hbar\frac{\partial}{\partial\phi}
\end{align*}

\end{document}
