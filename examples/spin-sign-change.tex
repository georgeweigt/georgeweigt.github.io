\input{preamble}

\section*{Spin sign change}

An electron is at rest in the following magnetic field.
\begin{equation*}
\mathbf B=B_0\cos(\omega t)\begin{pmatrix}0\\0\\1\end{pmatrix}
\end{equation*}

What is the minimum $B_0$ that changes the sign of spin in the $x$ direction?

\bigskip
These are the spin operators.
\begin{equation*}
S_x=\frac{\hbar}{2}\begin{pmatrix}0&1\\1&0\end{pmatrix},\quad
S_y=\frac{\hbar}{2}\begin{pmatrix}0&-i\\i&0\end{pmatrix},\quad
S_z=\frac{\hbar}{2}\begin{pmatrix}1&0\\0&-1\end{pmatrix}
\end{equation*}

This is the spin angular momentum operator.
\begin{equation*}
\mathbf S=\begin{pmatrix}S_x\\S_y\\S_z\end{pmatrix}
\end{equation*}

This is the Hamiltonian.
\begin{equation*}
H=\frac{ge}{2m}\mathbf S\cdot\mathbf B
=\frac{ge}{2m}S_zB_0\cos(\omega t)
\end{equation*}

Let $|s\rangle$ be the spin state
\begin{equation*}
|s\rangle=\begin{pmatrix}c_1(t)\\c_2(t)\end{pmatrix}
\end{equation*}

By the Schrodinger equation
\begin{equation*}
i\hbar\frac{\partial}{\partial t}|s\rangle=H|s\rangle
\end{equation*}

we have
\begin{align*}
i\hbar\frac{\partial}{\partial t}c_1(t)&=\frac{ge\hbar}{4m}B_0\cos(\omega t)c_1(t)
\\
i\hbar\frac{\partial}{\partial t}c_2(t)&=-\frac{ge\hbar}{4m}B_0\cos(\omega t)c_2(t)
\end{align*}

Hence
\begin{equation*}
\begin{aligned}
c_1(t)&=C_1\exp\left(-\frac{ige}{4m\omega}B_0\sin(\omega t)\right)
\\
c_2(t)&=C_2\exp\left(\frac{ige}{4m\omega}B_0\sin(\omega t)\right)
\end{aligned}
\tag{1}
\end{equation*}

where complex coefficients $C_1$ and $C_2$ have the general polar forms
\begin{equation*}
C_1=a_1\exp(i\theta_1),\quad C_2=a_2\exp(i\theta_2),\quad|C_1|^2+|C_2|^2=a_1^2+a_2^2=1
\end{equation*}

For the expectation value in the $x$ direction we have
\begin{equation*}
\langle S_x\rangle=\langle s|S_x|s\rangle
=a_1a_2\hbar\cos\left(\frac{ge}{2m\omega}B_0\sin(\omega t)-\theta_1+\theta_2\right)
\tag{2}
\end{equation*}

Note that if the sign of $\langle S_x\rangle$ is constant in time then we have for all $t$
\begin{equation*}
-\frac{\pi}{2}\le\frac{ge}{2m\omega}B_0\sin(\omega t)-\theta_1+\theta_2\le\frac{\pi}{2}
\end{equation*}

For $t$ such that $\sin(\omega t)=1$ we have
\begin{equation*}
\frac{ge}{2m\omega}B_0-\theta_1+\theta_2\le\frac{\pi}{2}
\end{equation*}

Hence $\langle S_x\rangle$ changes sign for some interval of time when
\begin{equation*}
B_0>\frac{2m\omega}{ge}\left(\frac{\pi}{2}+\theta_1-\theta_2\right)
\tag{3}
\end{equation*}

For $t$ such that $\sin(\omega t)=-1$ we have
\begin{equation*}
-\frac{\pi}{2}\le-\frac{ge}{2m\omega}B_0-\theta_1+\theta_2
\end{equation*}

Hence $\langle S_x\rangle$ changes sign for some interval of time when
\begin{equation*}
B_0>\frac{2m\omega}{ge}\left(\frac{\pi}{2}-\theta_1+\theta_2\right)
\tag{4}
\end{equation*}

The minimum $B_0$ is the minimum of (3) and (4).

\bigskip
See exercise 10.6 of {\it Quantum Mechanics}
(Lulu edition) by Richard Fitzpatrick.

\end{document}
