\input{preamble}

% Richard Fitzpatrick exercise 10-6

\section*{Spin sign change}

An electron is at rest in the following magnetic field.
\begin{equation*}
\mathbf B=B_0\cos(\omega t)\begin{pmatrix}0\\0\\1\end{pmatrix}
\end{equation*}

What is the minimum $B_0$ that changes the sign of spin in the $x$ direction?

\bigskip
These are the spin operators.
\begin{equation*}
S_x=\frac{\hbar}{2}\begin{pmatrix}0&1\\1&0\end{pmatrix},\quad
S_y=\frac{\hbar}{2}\begin{pmatrix}0&-i\\i&0\end{pmatrix},\quad
S_z=\frac{\hbar}{2}\begin{pmatrix}1&0\\0&-1\end{pmatrix}
\end{equation*}

This is the spin angular momentum operator.
\begin{equation*}
\mathbf S=\begin{pmatrix}S_x\\S_y\\S_z\end{pmatrix}
\end{equation*}

This is the Hamiltonian.
\begin{equation*}
H=\frac{ge}{2m}\mathbf S\cdot\mathbf B
=\frac{ge}{2m}S_zB_0\cos(\omega t)
\end{equation*}

Let $s(t)$ be the spin state
\begin{equation*}
s(t)=\begin{pmatrix}c_1(t)\\c_2(t)\end{pmatrix}
\end{equation*}

By the Schrodinger equation
\begin{equation*}
i\hbar\frac{\partial}{\partial t}s(t)=Hs(t)
\end{equation*}

we have
\begin{align*}
i\hbar\frac{\partial}{\partial t}c_1(t)&=\frac{ge\hbar}{4m}B_0\cos(\omega t)c_1(t)
\\
i\hbar\frac{\partial}{\partial t}c_2(t)&=-\frac{ge\hbar}{4m}B_0\cos(\omega t)c_2(t)
\end{align*}

Hence
\begin{equation*}
\begin{aligned}
c_1(t)&=C\exp\left(-\frac{ige}{4m\omega}B_0\sin(\omega t)\right)
\\
c_2(t)&=C\exp\left(\frac{ige}{4m\omega}B_0\sin(\omega t)\right)
\end{aligned}
\tag{1}
\end{equation*}

By the normalization requirement $|s(t)|=1$ we have
\begin{equation*}
C=\frac{1}{\sqrt2}
\end{equation*}

This is the expectation value for $S_x$.
\begin{equation*}
\langle S_x\rangle=\langle s|S_x|s\rangle
=\frac{\hbar}{2}\cos\left(\frac{ge}{2m\omega}B_0\sin(\omega t)\right)
\tag{2}
\end{equation*}

As a result of $-1\le\sin(\omega t)\le1$, the cosine is always positive for
\begin{equation*}
\frac{ge}{2m\omega}B_0\le\frac{\pi}{2}
\end{equation*}

Hence the minimum $B_0$ for changing the sign of $\langle S_x\rangle$ is
\begin{equation*}
B_0=\frac{\pi m\omega}{ge}
\end{equation*}


\end{document}
