\input{preamble}

% see Thomas Jordan problems 19-1 and 22-1

\section*{Bohr model}

By an argument that is no longer relevant the Bohr model
has for hydrogen energy levels
\begin{equation*}
E_n=-\frac{\alpha^2mc^2}{2n^2}
\end{equation*}

By the kinetic energy relation
\begin{equation*}
v^2=-\frac{2E_n}{m}
\end{equation*}

velocity $v$ reduces to
\begin{equation*}
v=\frac{\alpha c}{n}
\end{equation*}

The Bohr model quantizes orbital angular momentum as
\begin{equation*}
mvr_n=n\hbar
\end{equation*}

Hence the radius is
\begin{equation*}
r_n=\frac{n\hbar}{mv}=\frac{n^2\hbar}{\alpha mc}
\end{equation*}

For $n=1$ and $m=m_e$ we have
\begin{equation*}
E_1=-13.6057\,\text{eV},\quad
r_1=5.29177\times10^{-11}\,\text{meter}
\end{equation*}

For reduced electron mass
\begin{equation*}
m=\frac{m_em_p}{m_e+m_p}
\end{equation*}

the result is
\begin{equation*}
E_1=-13.5983\,\text{eV},\quad
r_1=5.29465\times10^{-11}\,\text{meter}
\end{equation*}

The model can be made more convoluted by the substitution
\begin{equation*}
\alpha=\frac{e^2}{4\pi\varepsilon_0\hbar c}
\end{equation*}

leading to
\begin{equation*}
E_n=-\frac{me^4}{2(4\pi\varepsilon_0\hbar)^2n^2}
\end{equation*}

and
\begin{equation*}
r_n=\frac{4\pi\varepsilon_0\hbar^2n^2}{me^2}
\end{equation*}

\end{document}
