\input{preamble}

\section*{Hydrogen transition 1}

Start with the perturbing Hamiltonian ($E_0$ is peak electric field in newtons per coulomb)
\begin{equation*}
H_1(\mathbf r,t)=-\frac{eE_0\boldsymbol{\epsilon}\cdot\mathbf p}{m\omega}
\cos(\mathbf k\cdot\mathbf r-\omega t)
\end{equation*}

In exponential form
\begin{equation*}
H_1(\mathbf r,t)=-\frac{eE_0\boldsymbol{\epsilon}\cdot\mathbf p}{m\omega}
\left(\tfrac{1}{2}\exp(i\mathbf k\cdot\mathbf r-i\omega t)
+\tfrac{1}{2}\exp(-i\mathbf k\cdot\mathbf r+i\omega t)\right)
\end{equation*}

Given the initial condition $c_b(0)=0$ the first-order approximation for $c_b(t)$ is
\begin{equation*}
c_b(t)=-\frac{i}{\hbar}\int_0^t
\langle\psi_b|H_1(\mathbf r,t')|\psi_a\rangle\exp(i\omega_0t')\,dt',\quad
\omega_0=\frac{E_b-E_a}{\hbar}
\end{equation*}

Factor the integrand.
\begin{multline*}
c_b(t)
=\frac{ieE_0}{2\hbar m\omega}
\langle\psi_b|\boldsymbol{\epsilon}\cdot\mathbf p\exp(i\mathbf k\cdot\mathbf r)|\psi_a\rangle
\int_0^t\exp(-i\omega t')\exp(i\omega_0t')\,dt'
\\
+\frac{ieE_0}{2\hbar m\omega}
\langle\psi_b|\boldsymbol{\epsilon}\cdot\mathbf p\exp(-i\mathbf k\cdot\mathbf r)|\psi_a\rangle
\int_0^t\exp(i\omega t')\exp(i\omega_0t')\,dt'
\end{multline*}

Solve the integrals to obtain
\begin{multline*}
c_b(t)=\frac{eE_0}{2\hbar m\omega}
\langle\psi_b|\boldsymbol{\epsilon}\cdot\mathbf p\exp(i\mathbf k\cdot\mathbf r)|\psi_a\rangle
\frac{\exp\bigl(i(\omega_0-\omega)t\bigr)-1}{\omega_0-\omega}
\\
+\frac{eE_0}{2\hbar m\omega}
\langle\psi_b|\boldsymbol{\epsilon}\cdot\mathbf p\exp(-i\mathbf k\cdot\mathbf r)|\psi_a\rangle
\frac{\exp\bigl(i(\omega_0+\omega)t\bigr)-1}{\omega_0+\omega}
\tag{1}
\end{multline*}

As an approximation, discard the second term since the first term
dominates for $\omega\approx\omega_0$.
\begin{equation*}
c_b(t)=\frac{eE_0}{2\hbar m\omega}
\langle\psi_b|\boldsymbol{\epsilon}\cdot\mathbf p\exp(i\mathbf k\cdot\mathbf r)|\psi_a\rangle
\frac{\exp\bigl(i(\omega_0-\omega)t\bigr)-1}{\omega_0-\omega}
\end{equation*}

Rewrite $c_b(t)$ in the form of a sine function.
\begin{equation*}
c_b(t)=\frac{ieE_0}{\hbar m\omega}
\langle\psi_b|\boldsymbol{\epsilon}\cdot\mathbf p\exp(i\mathbf k\cdot\mathbf r)|\psi_a\rangle
\frac{\sin\bigl(\tfrac{1}{2}(\omega_0-\omega)t\bigr)}{\omega_0-\omega}
\exp\bigl(\tfrac{i}{2}(\omega_0-\omega)t\bigr)
\tag{2}
\end{equation*}

Verify dimensions.
\begin{equation*}
\frac{eE_0\boldsymbol{\epsilon}\cdot\mathbf p}{m\omega}
=\frac{
\begin{matrix}
e & E_0 & \boldsymbol{\epsilon}\cdot\mathbf p
\\
\text{coulomb}
& \text{newton}\,\text{coulomb}^{-1}
& \text{momentum}
\end{matrix}
}{
\begin{matrix}
m & \omega
\\
\text{kilogram} & \text{second}^{-1}
\end{matrix}
}=\text{joule}
\end{equation*}
%
\begin{equation*}
c_b(t)=\frac{
\begin{matrix}
e & E_0
\\
\text{coulomb} & \text{newton}\,\text{coulomb}^{-1}
\end{matrix}
}{
\begin{matrix}
\hbar & m & \omega
\\
\text{joule}\,\text{second} & \text{kilogram} &\text{second}^{-1}
\end{matrix}
}
\,
\frac{
\begin{matrix}
\\
\langle\psi_b|\boldsymbol{\epsilon}\cdot\mathbf p\exp(i\mathbf k\cdot\mathbf r)|\psi_a\rangle
\\
\text{momentum}
\end{matrix}
}{
\begin{matrix}
\omega_0-\omega
\\
\text{second}^{-1}
\end{matrix}
}
=1
\end{equation*}

\end{document}
