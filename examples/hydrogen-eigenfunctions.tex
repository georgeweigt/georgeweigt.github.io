\input{preamble}

\section*{Hydrogen eigenfunctions}

The hydrogen atom eigenfunction $\psi_{nlm}(r,\theta,\phi)$
is the product of a radial eigenfunction $R_{nl}(r)$ and a spherical
harmonic eigenfunction $Y_{lm}(\theta,\phi)$.
\begin{equation*}
\psi_{nlm}(r,\theta,\phi)=R_{nl}(r)Y_{lm}(\theta,\phi)
\end{equation*}

Quantum number $n$ is the principal quantum number.
\begin{equation*}
n=1,2,3,\ldots
\end{equation*}

Quantum number $l$ is the angular momentum quantum number.
\begin{equation*}
l=0,1,\ldots,n-1
\end{equation*}

Quantum number $m$ is the magnetic quantum number.
\begin{equation*}
m=-l,\ldots,0,\ldots,l
\end{equation*}

The normalized radial eigenfunction $R_{nl}(r)$ is computed from the following formula.
\begin{equation*}
R_{nl}(r)=
\frac{2}{n^2}
\left(\frac{(n-l-1)!}{(n+l)!}\right)^{1/2}
\left(\frac{2r}{na_0}\right)^l
L_{n-l-1}^{2l+1}\left(\frac{2r}{na_0}\right)
\exp\left(-\frac{r}{na_0}\right)
a_0^{-3/2}
\end{equation*}

Symbol $a_0$ is the Bohr radius.
\begin{equation*}
a_0=\frac{4\pi\varepsilon_0\hbar^2}{e^2\mu}
\approx0.529\times10^{-10}\,\text{meter}
\end{equation*}
Symbol $\mu$ is the reduced mass of the electron.
\begin{equation*}
\mu=\frac{m_e m_p}{m_e+m_p}
\end{equation*}

Symbol $L$ is a Laguerre polynomial computed from the following formula.
\begin{equation*}
L_n^m(x)=(n+m)!\sum_{k=0}^n
\frac{(-x)^k}{(n-k)!(m+k)!k!}
\end{equation*}

The normalized spherical harmonic eigenfunction $Y_{lm}(\theta,\phi)$
is computed from the following formula.
\begin{equation*}
Y_{lm}(\theta,\phi)=(-1)^m
\left(\frac{2l+1}{4\pi}\right)^{1/2}
\left(\frac{(l-m)!}{(l+m)!}\right)^{1/2}
P_l^m(\cos\theta)\exp(im\phi)
\end{equation*}

Symbol $P$ is a Legendre polynomial which can be computed using Rodrigues's formula.
\begin{equation*}
P_n^m(x)=\frac{1}{2^n n!}(1-x^2)^{m/2}
\frac{d^{n+m}}{dx^{n+m}}(x^2-1)^n
\end{equation*}

The eigenfunction $\psi_n\equiv\psi_{nlm}(r,\theta,\phi)$ solves Schrodinger's equation.
\begin{equation*}
\hat{H}\psi_n=E_n\psi_n
\end{equation*}

Symbol $\hat{H}$ is the Hamiltonian operator for the hydrogen atom.
\begin{equation*}
\hat{H}\psi_n=-\frac{\hbar^2}{2\mu}\nabla^2\psi_n-\frac{e^2}{4\pi\varepsilon_0 r}\psi_n
\end{equation*}

Symbol $\nabla^2$ is the Laplacian operator in spherical coordinates.
\begin{equation*}
\nabla^2\psi_n=\frac{1}{r^2}\frac{\partial}{\partial r}
\left(r^2\frac{\partial}{\partial r}\psi_n\right)
+
\frac{1}{r^2\sin\theta}\frac{\partial}{\partial\theta}
\left(\sin\theta\frac{\partial}{\partial\theta}\psi_n\right)
+
\frac{1}{r^2\sin^2\theta}\frac{\partial^2}{\partial\phi^2}\psi_n
\end{equation*}

Symbol $E_n$ is the energy eigenvalue.
\begin{equation*}
E_n=-\frac{\mu}{2n^2}\left(\frac{e^2}{4\pi\varepsilon_0\hbar}\right)^2
\approx-\frac{1}{n^2}\times13.6\,\text{eV}
\end{equation*}

\end{document}
