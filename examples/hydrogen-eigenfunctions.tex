\input{preamble}

\section*{Hydrogen eigenfunctions}

Verify
\begin{equation*}
H\psi_{nlm}(r,\theta,\phi)=E_n\psi_{nlm}(r,\theta,\phi)
\end{equation*}

where $H$ is the Hamiltonian operator
\begin{equation*}
H\psi=-\frac{\hbar^2}{2\mu}\nabla^2\psi-\frac{\hbar^2}{\mu a_0r}\psi
\end{equation*}

and $E_n$ is the energy eigenvalue
\begin{equation*}
E_n=-\frac{\hbar^2}{2\mu a_0^2n^2}
\approx-\frac{1}{n^2}\times13.6\,\text{eV}
\end{equation*}

Symbol $\mu$ is the reduced electron mass
\begin{equation*}
\mu=\frac{m_e m_p}{m_e+m_p}
\end{equation*}

Hydrogen eigenfunctions $\psi_{nlm}$ are formed as
\begin{equation*}
\psi_{nlm}(r,\theta,\phi)=R_{nl}(r)Y_{lm}(\theta,\phi)
\end{equation*}

\iffalse
Quantum number $n$ is the principal quantum number.
\begin{equation*}
n=1,2,3,\ldots
\end{equation*}

Quantum number $l$ is the angular momentum quantum number.
\begin{equation*}
l=0,1,\ldots,n-1
\end{equation*}

Quantum number $m$ is the magnetic quantum number.
\begin{equation*}
m=-l,\ldots,0,\ldots,l
\end{equation*}
\fi

Radial eigenfunction $R_{nl}$ is formed as
\begin{equation*}
R_{nl}(r)=
\frac{2}{n^2}
\sqrt{\frac{(n-l-1)!}{(n+l)!}}
\left(\frac{2r}{na_0}\right)^l
L_{n-l-1}^{2l+1}\left(\frac{2r}{na_0}\right)
\exp\left(-\frac{r}{na_0}\right)
a_0^{-3/2}
\end{equation*}

Symbol $L_n^m$ is the Laguerre polynomial
\begin{equation*}
L_n^m(x)=(n+m)!\sum_{k=0}^n
\frac{(-x)^k}{(n-k)!(m+k)!k!}
\end{equation*}

Symbol $Y_{lm}$ is the spherical harmonic
\begin{equation*}
Y_{lm}(\theta,\phi)=(-1)^m
\sqrt{\frac{(2l+1)}{4\pi}
\frac{(l-m)!}{(l+m)!}}\,
P_l^m(\cos\theta)\exp(im\phi)
\end{equation*}

Legendre polynomial $P_l^m(\cos\theta)$ is formed as (see arxiv.org/abs/1805.12125)
\begin{equation*}
P_l^m(\cos\theta)=\begin{cases}
\displaystyle
\left(\frac{\sin\theta}{2}\right)^m\,\sum_{k=0}^{l-m}
(-1)^k\frac{(l+m+k)!}{(l-m-k)!(m+k)!k!}
\left(\frac{1-\cos\theta}{2}\right)^k, & m\ge0
\\[4ex]
\displaystyle
(-1)^m\frac{(l+m)!}{(l-m)!}P_l^{|m|}(\cos\theta), & m<0
\end{cases}
\end{equation*}

Symbol $\nabla^2$ is the Laplacian operator in spherical coordinates.
\begin{equation*}
\nabla^2\psi=\frac{1}{r^2}\frac{\partial}{\partial r}
\left(r^2\frac{\partial}{\partial r}\psi\right)
+
\frac{1}{r^2\sin\theta}\frac{\partial}{\partial\theta}
\left(\sin\theta\frac{\partial}{\partial\theta}\psi\right)
+
\frac{1}{r^2\sin^2\theta}\frac{\partial^2}{\partial\phi^2}\psi
\end{equation*}

Noting that
\begin{equation*}
a_0=\frac{4\pi\varepsilon_0\hbar^2}{e^2\mu}
\end{equation*}

we have
\begin{align*}
H\psi&=-\frac{\hbar^2}{2\mu}\nabla^2\psi-\frac{e^2}{4\pi\varepsilon_0 r}\psi
\\
E_n&=-\frac{\mu}{2n^2}\left(\frac{e^2}{4\pi\varepsilon_0\hbar}\right)^2
%\approx-\frac{1}{n^2}\times13.6\,\text{eV}
\end{align*}

Noting that
\begin{equation*}
e^2=4\pi\epsilon_0\alpha\hbar c
\end{equation*}

we have
\begin{align*}
a_0&=\frac{\hbar}{\alpha\mu c}
\\
H\psi&=-\frac{\hbar^2}{2\mu}\nabla^2\psi-\frac{\alpha\hbar c}{r}\psi
\\
E_n&=-\frac{\alpha^2\mu c^2}{2n^2}
\end{align*}

\end{document}
