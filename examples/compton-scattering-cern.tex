\input{preamble}

\section*{Compton scattering CERN data}

See ``Compton Scattering of Quasi-Real Virtual Photons at LEP,''
arxiv.org/abs/hep-ex/0504012.
\begin{equation*}
\begin{matrix}
x & y\\
-0.74 & 13380\\
-0.60 & \phantom{0}7720\\
-0.47 & \phantom{0}6360\\
-0.34 & \phantom{0}4600\\
-0.20 & \phantom{0}4310\\
-0.07 & \phantom{0}3700\\
\phantom{+}0.06 & \phantom{0}3640\\
\phantom{+}0.20 & \phantom{0}3340\\
\phantom{+}0.33 & \phantom{0}3500\\
\phantom{+}0.46 & \phantom{0}3010\\
\phantom{+}0.60 & \phantom{0}3310\\
\phantom{+}0.73 & \phantom{0}3330\\
\end{matrix}
\end{equation*}

For columns $x$ and $y$ we have
\begin{equation*}
x=\cos\theta,
\quad
y=\frac{d\sigma}{d\cos\theta}
\end{equation*}

This is the differential cross section in the center of mass frame.
\begin{equation*}
\frac{d\sigma}{d\cos\theta}
=2\pi\frac{d\sigma}{d\Omega}
=\frac{\pi\alpha^2}{s}\left(
\frac{\cos\theta+1}{2}+\frac{2}{\cos\theta+1}
\right)\times(\hbar c)^2
\end{equation*}

Let $\hat y$ be predicted values.
The factor $10^{40}$ converts square meters to picobarns.
\begin{equation*}
\hat y_i
=\left.\frac{d\sigma}{d\cos\theta}\right|_{\cos\theta=x_i}
=
\frac{\pi\alpha^2}{s}
\left(
\frac{x_i+1}{2}+
\frac{2}{x_i+1}
\right)
\times(\hbar c)^2
\times10^{40}
\end{equation*}

The following table shows predicted values for $s=(40\,\text{GeV})^2$.
\begin{equation*}
\begin{matrix}
x & y & \hat{y}\\
-0.74 & 13380 & 12573\\
-0.60 & \phantom{0}7720 & \phantom{0}8358\\
-0.47 & \phantom{0}6360 & \phantom{0}6491\\
-0.34 & \phantom{0}4600 & \phantom{0}5401\\
-0.20 & \phantom{0}4310 & \phantom{0}4661\\
-0.07 & \phantom{0}3700 & \phantom{0}4204\\
\phantom{+}0.06 & \phantom{0}3640 & \phantom{0}3884\\
\phantom{+}0.20 & \phantom{0}3340 & \phantom{0}3643\\
\phantom{+}0.33 & \phantom{0}3500 & \phantom{0}3486\\
\phantom{+}0.46 & \phantom{0}3010 & \phantom{0}3375\\
\phantom{+}0.60 & \phantom{0}3310 & \phantom{0}3295\\
\phantom{+}0.73 & \phantom{0}3330 & \phantom{0}3248\\
\end{matrix}
\end{equation*}

The coefficient of determination $R^2$ measures how well predicted values fit the data.
\begin{equation*}
R^2=1-\frac{\sum(y-\hat{y})^2}{\sum(y-\bar{y})^2}=0.97
\end{equation*}

The result indicates that $d\sigma$ explains 97\% of the variance in the data.

\end{document}
