\input{preamble}

\section*{Compton scattering CERN data}

See ``Compton Scattering of Quasi-Real Virtual Photons at LEP,''
arxiv.org/abs/hep-ex/0504012.

\begin{center}
\begin{tabular}{|c|c|}
\hline
$x$ & $y$\\
\hline
$-0.74$ & $13380$\\
$-0.60$ & $\phantom{0}7720$\\
$-0.47$ & $\phantom{0}6360$\\
$-0.34$ & $\phantom{0}4600$\\
$-0.20$ & $\phantom{0}4310$\\
$-0.07$ & $\phantom{0}3700$\\
$\phantom{+}0.06$ & $\phantom{0}3640$\\
$\phantom{+}0.20$ & $\phantom{0}3340$\\
$\phantom{+}0.33$ & $\phantom{0}3500$\\
$\phantom{+}0.46$ & $\phantom{0}3010$\\
$\phantom{+}0.60$ & $\phantom{0}3310$\\
$\phantom{+}0.73$ & $\phantom{0}3330$\\
\hline
\end{tabular}
\end{center}

The data are for the center of mass frame and have the following relationship with the differential cross section formula.
\begin{equation*}
x=\cos\theta,
\quad
y=\frac{d\sigma}{d\cos\theta}=2\pi\frac{d\sigma}{d\Omega}
\end{equation*}

For the high energy approximation we have
\begin{equation*}
\langle|\mathcal{M}|^2\rangle
=
2e^4\left(
\frac{\cos\theta+1}{2}+\frac{2}{\cos\theta+1}
\right)
\end{equation*}

The corresponding cross section formula is
\begin{equation*}
\frac{d\sigma}{d\Omega}
=\frac{\langle|\mathcal{M}|^2\rangle}{64\pi^2s}
=\frac{e^4}{32\pi^2s}
\left(
\frac{\cos\theta+1}{2}+\frac{2}{\cos\theta+1}
\right),\quad s\gg m
\end{equation*}

Substituting $e^4=16\pi^2\alpha^2$ yields
\begin{equation*}
\frac{d\sigma}{d\Omega}
=\frac{\alpha^2}{2s}
\left(
\frac{\cos\theta+1}{2}+\frac{2}{\cos\theta+1}
\right)
\end{equation*}

Multiply by $2\pi$ to obtain
\begin{equation*}
\frac{d\sigma}{d\cos\theta}
=\frac{\pi\alpha^2}{s}\left(
\frac{\cos\theta+1}{2}+\frac{2}{\cos\theta+1}
\right)
\end{equation*}

To compute predicted values $\hat{y}$ from the above formula,
multiply by $(hc)^2$ to convert to SI
and multiply by $10^{40}$ to convert square meters to picobarns.
\begin{equation*}
\hat{y}
=
\frac{\pi\alpha^2}{s}
\left(
\frac{x+1}{2}+
\frac{2}{x+1}
\right)
\times(hc)^2
\times10^{40}
\end{equation*}

The following table shows $\hat{y}$
for $s=(40\,\text{GeV})^2$.

\begin{center}
\begin{tabular}{|c|c|c|}
\hline
$x$ & $y$ & $\hat{y}$\\
\hline
$-0.74$ & $13380$ & $12573$\\
$-0.60$ & $\phantom{0}7720$ & $\phantom{0}8358$\\
$-0.47$ & $\phantom{0}6360$ & $\phantom{0}6491$\\
$-0.34$ & $\phantom{0}4600$ & $\phantom{0}5401$\\
$-0.20$ & $\phantom{0}4310$ & $\phantom{0}4661$\\
$-0.07$ & $\phantom{0}3700$ & $\phantom{0}4204$\\
$\phantom{+}0.06$ & $\phantom{0}3640$ & $\phantom{0}3884$\\
$\phantom{+}0.20$ & $\phantom{0}3340$ & $\phantom{0}3643$\\
$\phantom{+}0.33$ & $\phantom{0}3500$ & $\phantom{0}3486$\\
$\phantom{+}0.46$ & $\phantom{0}3010$ & $\phantom{0}3375$\\
$\phantom{+}0.60$ & $\phantom{0}3310$ & $\phantom{0}3295$\\
$\phantom{+}0.73$ & $\phantom{0}3330$ & $\phantom{0}3248$\\
\hline
\end{tabular}
\end{center}

The coefficient of determination $R^2$ measures how well predicted values fit the data.
\begin{equation*}
R^2=1-\frac{\sum(y-\hat{y})^2}{\sum(y-\bar{y})^2}=0.97
\end{equation*}

The result indicates that the model $d\sigma$ explains 97\% of the variance in the data.

\end{document}
