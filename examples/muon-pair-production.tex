\input{preamble}

\section*{Muon pair production}

Muon pair production is the interaction $e^-+e^+\rightarrow\mu^-+\mu^+$.
%
\begin{center}
\begin{tikzpicture}
\draw[dashed] (0,0) circle (0.5cm);
\draw[thick,->] (2,0) node[anchor=west] {$e^+$} -- (0.6,0);
\draw[thick,->] (-2,0) node[anchor=east] {$e^-$} -- (-0.6,0);
\draw[thick,->] (0.40,0.40) -- (1.3,1.3) node[anchor=south west] {$\mu^-$};
\draw[thick,->] (-0.4,-0.4) -- (-1.3,-1.3) node[anchor=north east] {$\mu^+$};
\draw (1,0.5) node {$\theta$};
\end{tikzpicture}
\end{center}
%
In the center-of-mass frame we have the following momentum vectors where
$p=\sqrt{E^2-m^2}$ and $\rho=\sqrt{E^2-M^2}$.
\begin{equation*}
p_1=\underset{\text{inbound $e^-$}}
{\begin{pmatrix}E\\0\\0\\p\end{pmatrix}}
\qquad
p_2=\underset{\text{inbound $e^+$}}
{\begin{pmatrix}E\\0\\0\\-p\end{pmatrix}}
\qquad
p_3=\underset{\text{outbound $\mu^-$}}
{\begin{pmatrix}
E\\
\rho\sin\theta\cos\phi\\
\rho\sin\theta\sin\phi\\
\rho\cos\theta
\end{pmatrix}}
\qquad
p_4=
\underset{\text{outbound $\mu^+$}}
{\begin{pmatrix}
E\\
-\rho\sin\theta\cos\phi\\
-\rho\sin\theta\sin\phi\\
-\rho\cos\theta
\end{pmatrix}}
\end{equation*}

Spinors for the inbound electron.
\begin{equation*}
u_{11}=\frac{1}{\sqrt{E+m}}
\underset{\substack{\text{inbound $e^-$}\\ \text{spin up}}}
{\begin{pmatrix}E+m\\0\\p\\0\end{pmatrix}}
\qquad
u_{12}=\frac{1}{\sqrt{E+m}}
\underset{\substack{\text{inbound $e^-$}\\ \text{spin down}}}
{\begin{pmatrix}0\\E+m\\0\\-p\end{pmatrix}}
\end{equation*}

Spinors for the inbound positron.
\begin{equation*}
v_{21}=\frac{1}{\sqrt{E+m}}
\underset{\substack{\text{inbound $e^+$}\\ \text{spin up}}}
{\begin{pmatrix}-p\\0\\E+m\\0\end{pmatrix}}
\qquad
v_{22}=\frac{1}{\sqrt{E+m}}
\underset{\substack{\text{inbound $e^+$}\\ \text{spin down}}}
{\begin{pmatrix}0\\p\\0\\E+m\end{pmatrix}}
\end{equation*}

Spinors for the outbound muon.
\begin{equation*}
u_{31}=\frac{1}{\sqrt{E+M}}
\underset{\substack{\text{outbound $\mu^-$}\\ \text{spin up}}}
{\begin{pmatrix}E+M\\0\\p_{3z}\\p_{3x}+ip_{3y}\end{pmatrix}}
\qquad
u_{32}=\frac{1}{\sqrt{E+M}}
\underset{\substack{\text{outbound $\mu^-$}\\ \text{spin down}}}
{\begin{pmatrix}0\\E+M\\p_{3x}-ip_{3y}\\-p_{3z}\end{pmatrix}}
\end{equation*}

Spinors for the outbound anti-muon.
\begin{equation*}
v_{41}=\frac{1}{\sqrt{E+M}}
\underset{\substack{\text{outbound $\mu^+$}\\ \text{spin up}}}
{\begin{pmatrix}p_4^z\\p_4^x+ip_4^y\\E+M\\0\end{pmatrix}}
\qquad
v_{42}=\frac{1}{\sqrt{E+M}}
\underset{\substack{\text{outbound $\mu^+$}\\ \text{spin down}}}
{\begin{pmatrix}p_4^x-ip_4^y\\-p_4^z\\0\\E+M\end{pmatrix}}
\end{equation*}

The probability amplitude $\mathcal M_{abcd}$ for spin state $abcd$ is
\begin{equation*}
\mathcal{M}_{abcd}=\frac{e^2}{s}(\bar{u}_{3c}\gamma_\mu v_{4d})(\bar{v}_{2b}\gamma^\mu u_{1a})
\end{equation*}

Symbol $e$ is elementary charge and
\begin{equation*}
s=(p_1+p_2)^2=4E^2
\end{equation*}

The expected probability density $\langle|\mathcal{M}|^2\rangle$ is the average of spin states.
\begin{align*}
\langle|\mathcal{M}|^2\rangle
&=\frac{1}{4}\sum_{a=1}^2\sum_{b=1}^2\sum_{c=1}^2\sum_{d=1}^2\big|\mathcal{M}_{abcd}\big|^2
\\
&=\frac{e^4}{64E^4}\sum_{a=1}^2\sum_{b=1}^2\sum_{c=1}^2\sum_{d=1}^2
\left|(\bar{u}_{3c}\gamma_\mu v_{4d})(\bar{v}_{2b}\gamma^\mu u_{1a})\right|^2
\end{align*}

The Casimir trick uses matrix arithmetic to sum over spin states.
\begin{equation*}
\langle|\mathcal{M}|^2\rangle
=\frac{e^4}{64E^4}
\mathop{\rm Tr}\left((\slashed{p}_3+M)\gamma^\mu(\slashed{p}_4-M)\gamma^\nu\right)
\mathop{\rm Tr}\left((\slashed{p}_2-m)\gamma_\mu(\slashed{p}_1+m)\gamma_\nu\right)
\end{equation*}

\iffalse

Another way to compute $\langle|\mathcal{M}|^2\rangle$ is
\begin{equation*}
\langle|\mathcal{M}|^2\rangle
=\frac{e^4}{4s^2}
\left(-8 s^2 + 16 t^2 - 16 s u + (64 s + 32 u) \left(m^2 + M^2\right) - 48 \left(m^2 + M^2\right)^2\right)
\end{equation*}
where
\begin{align*}
s&=(p_1+p_2)^2
\\
t&=(p_1-p_3)^2
\\
u&=(p_1-p_4)^2
\end{align*}

\fi

The result is
\begin{equation*}
\langle|\mathcal{M}|^2\rangle
=e^4\left(1+\cos^2\theta+\frac{m^2+M^2}{E^2}\sin^2\theta+\frac{m^2M^2}{E^4}\cos^2\theta\right)
\end{equation*}

For $E\gg M$ a useful approximation is
\begin{equation*}
\langle|\mathcal{M}|^2\rangle=e^4\left(1+\cos^2\theta\right)
\end{equation*}

\subsubsection*{Cross section}

The differential cross section is
\begin{equation*}
\frac{d\sigma}{d\Omega}
=\frac{\langle|\mathcal{M}|^2\rangle}{4(4\pi\varepsilon_0)^2s}
\end{equation*}

where
\begin{equation*}
s=(p_1+p_2)^2=4E^2
\end{equation*}

For high energy experiments we have
\begin{equation*}
\langle|\mathcal{M}|^2\rangle=e^4\left(1+\cos^2\theta\right)
\end{equation*}

Hence
\begin{equation*}
\frac{d\sigma}{d\Omega}=\frac{e^4}{4(4\pi\varepsilon_0)^2s}\left(1+\cos^2\theta\right)
\end{equation*}

Noting that
\begin{equation*}
e^2=4\pi\varepsilon_0\alpha\hbar c
\end{equation*}

we have
\begin{equation*}
\frac{d\sigma}{d\Omega}=\frac{\alpha^2(\hbar c)^2}{4s}\left(1+\cos^2\theta\right)
\end{equation*}

Noting that
\begin{equation*}
d\Omega=\sin\theta\,d\theta\,d\phi
\end{equation*}

we also have
\begin{equation*}
d\sigma=\frac{\alpha^2(\hbar c)^2}{4s}\left(1+\cos^2\theta\right)
\sin\theta\,d\theta\,d\phi
\end{equation*}

Let $S(\theta_1,\theta_2)$ be the following integral of $d\sigma$.
\begin{equation*}
S(\theta_1,\theta_2)=\int_0^{2\pi}\int_{\theta_1}^{\theta_2}d\sigma
\end{equation*}

The solution is
\begin{equation*}
S(\theta_1,\theta_2)=\frac{2\pi\alpha^2(\hbar c)^2}{4s}[I(\theta_2)-I(\theta_1)]
\end{equation*}

where
\begin{equation*}
I(\theta)=-\frac{\cos^3\theta}{3}-\cos\theta
\end{equation*}

The cumulative distribution function is
\begin{equation*}
F(\theta)
=\frac{S(0,\theta)}{S(0,\pi)}
=\frac{I(\theta)-I(0)}{I(\pi)-I(0)}
=-\frac{\cos^3\theta}{8}-\frac{3\cos\theta}{8}+\frac{1}{2},
\quad
0\le\theta\le\pi
\end{equation*}

The probability of observing scattering events in the interval $\theta_1$ to $\theta_2$ is
\begin{equation*}
P(\theta_1\le\theta\le\theta_2)=F(\theta_2)-F(\theta_1)
\end{equation*}

The probability density function is
\begin{equation*}
f(\theta)=\frac{dF(\theta)}{d\theta}
=\frac{3}{8}
\left(1+\cos^2\theta\right)
\sin\theta
\end{equation*}

\subsubsection*{Data from SLAC PEP experiment}

See www.hepdata.net/record/ins216031, Table 1, $s=(29.0\,\text{GeV})^2$.

\begin{center}
\begin{tabular}{|c|c|}
\hline
$x$ & $y$\\
\hline
$-0.925$ & 67.08\\
$-0.85\phantom{0}$ & 58.67\\
$-0.75\phantom{0}$ & 54.66\\
$-0.65\phantom{0}$ & 51.72\\
$-0.55\phantom{0}$ & 43.70\\
$-0.45\phantom{0}$ & 41.12\\
$-0.35\phantom{0}$ & 39.71\\
$-0.25\phantom{0}$ & 35.34\\
$-0.15\phantom{0}$ & 33.35\\
$-0.05\phantom{0}$ & 34.69\\
$\phantom{+}0.05\phantom{0}$ & 34.05\\
$\phantom{+}0.15\phantom{0}$ & 34.48\\
$\phantom{+}0.25\phantom{0}$ & 34.66\\
$\phantom{+}0.35\phantom{0}$ & 35.23\\
$\phantom{+}0.45\phantom{0}$ & 35.60\\
$\phantom{+}0.55\phantom{0}$ & 40.13\\
$\phantom{+}0.65\phantom{0}$ & 42.56\\
$\phantom{+}0.75\phantom{0}$ & 46.37\\
$\phantom{+}0.85\phantom{0}$ & 49.28\\
$\phantom{+}0.925$ & 55.70\\
\hline
\end{tabular}
\end{center}

Data $x$ and $y$ have the following relationship with the differential cross section formula.
\begin{equation*}
x=\cos\theta,
\quad
y=s\frac{d\sigma}{d\cos\theta}=2\pi s\frac{d\sigma}{d\Omega}
\end{equation*}

The cross section formula is
\begin{equation*}
\frac{d\sigma}{d\Omega}=\frac{\alpha^2}{4s}\left(1+\cos^2\theta\right)\times(\hbar c)^2
\end{equation*}

To compute predicted values $\hat{y}$, multiply by $10^{37}$ to convert square meters to nanobarns.
\begin{equation*}
\hat y=2\pi s\frac{d\sigma}{d\Omega}=\frac{\pi\alpha^2}{2}\left(1+x^2\right)\times(\hbar c)^2\times10^{37}
\end{equation*}

The following table shows predicted values $\hat{y}$.

\begin{center}
\begin{tabular}{|c|c|c|}
\hline
$x$ & $y$ & $\hat{y}$ \\
\hline
$-0.925$ & 67.08 & 60.44\\
$-0.85\phantom{0}$ & 58.67 & 56.10\\
$-0.75\phantom{0}$ & 54.66 & 50.89\\
$-0.65\phantom{0}$ & 51.72 & 46.33\\
$-0.55\phantom{0}$ & 43.70 & 42.42\\
$-0.45\phantom{0}$ & 41.12 & 39.17\\
$-0.35\phantom{0}$ & 39.71 & 36.56\\
$-0.25\phantom{0}$ & 35.34 & 34.61\\
$-0.15\phantom{0}$ & 33.35 & 33.30\\
$-0.05\phantom{0}$ & 34.69 & 32.65\\
$\phantom{+}0.05\phantom{0}$ & 34.05 & 32.65\\
$\phantom{+}0.15\phantom{0}$ & 34.48 & 33.30\\
$\phantom{+}0.25\phantom{0}$ & 34.66 & 34.61\\
$\phantom{+}0.35\phantom{0}$ & 35.23 & 36.56\\
$\phantom{+}0.45\phantom{0}$ & 35.60 & 39.17\\
$\phantom{+}0.55\phantom{0}$ & 40.13 & 42.42\\
$\phantom{+}0.65\phantom{0}$ & 42.56 & 46.33\\
$\phantom{+}0.75\phantom{0}$ & 46.37 & 50.89\\
$\phantom{+}0.85\phantom{0}$ & 49.28 & 56.10\\
$\phantom{+}0.925$ & 55.70 & 60.44\\
\hline
\end{tabular}
\end{center}

The coefficient of determination $R^2$ measures how well predicted values fit the data.
\begin{equation*}
R^2=1-\frac{\sum(y-\hat{y})^2}{\sum(y-\bar{y})^2}=0.87
\end{equation*}

The result indicates that the model $d\sigma$ explains 87\% of the variance in the data.

\subsubsection*{Electroweak model}

The following differential cross section formula from electroweak
theory results in a better fit to the
data.\footnote{F. Mandl and G. Shaw, {\it Quantum Field Theory Revised Edition,} 316.}

\begin{equation*}
\frac{d\sigma}{d\Omega}=F(s)\bigl(1+\cos^2\theta\bigr)+G(s)\cos\theta
\end{equation*}

where
\begin{align*}
F(s)&=\frac{\alpha^2}{4s}
\left(
1+\frac{g_V^2}{\sqrt{2}\pi}\left(\frac{m_Z^2}{s-m_Z^2}\right)\left(\frac{sG}{\alpha}\right)
+\frac{(g_A^2+g_V^2)^2}{8\pi^2}\left(\frac{m_Z^2}{s-m_Z^2}\right)^2\left(\frac{sG}{\alpha}\right)^2
\right)
\\
G(s)&=\frac{\alpha^2}{4s}
\left(
\frac{\sqrt{2}g_A^2}{\pi}\left(\frac{m_Z^2}{s-m_Z^2}\right)\left(\frac{sG}{\alpha}\right)
+\frac{g_A^2g_V^2}{\pi^2}\left(\frac{m_Z^2}{s-m_Z^2}\right)^2\left(\frac{sG}{\alpha}\right)^2
\right)
\end{align*}

and
\begin{align*}
g_A&=-0.5
\\
g_V&=-0.0348
\\
m_Z&=91.17\,\text{GeV}
\\
G&=1.166\times10^{-5}\,\text{GeV}^{-2}
\end{align*}

The corresponding formula for $\hat{y}$ is
\begin{equation*}
\hat{y}=2\pi\left[F(s)(1+x^2)+G(s)x\right]\times(\hbar c)^2\times10^{37}
\end{equation*}

where $\sqrt{s}=29\,\text{GeV}$ is the center of mass collision energy.
Here are the predicted values $\hat{y}$ based on the above formula.

\begin{center}
\begin{tabular}{|c|c|c|}
\hline
$x$ & $y$ & $\hat{y}$ \\
\hline
$-0.925$ & 67.08 & 65.59\\
$-0.85\phantom{0}$ & 58.67 & 60.84\\
$-0.75\phantom{0}$ & 54.66 & 55.07\\
$-0.65\phantom{0}$ & 51.72 & 49.96\\
$-0.55\phantom{0}$ & 43.70 & 45.49\\
$-0.45\phantom{0}$ & 41.12 & 41.69\\
$-0.35\phantom{0}$ & 39.71 & 38.53\\
$-0.25\phantom{0}$ & 35.34 & 36.02\\
$-0.15\phantom{0}$ & 33.35 & 34.17\\
$-0.05\phantom{0}$ & 34.69 & 32.97\\
$\phantom{+}0.05\phantom{0}$ & 34.05 & 32.42\\
$\phantom{+}0.15\phantom{0}$ & 34.48 & 32.53\\
$\phantom{+}0.25\phantom{0}$ & 34.66 & 33.28\\
$\phantom{+}0.35\phantom{0}$ & 35.23 & 34.69\\
$\phantom{+}0.45\phantom{0}$ & 35.60 & 36.75\\
$\phantom{+}0.55\phantom{0}$ & 40.13 & 39.47\\
$\phantom{+}0.65\phantom{0}$ & 42.56 & 42.83\\
$\phantom{+}0.75\phantom{0}$ & 46.37 & 46.85\\
$\phantom{+}0.85\phantom{0}$ & 49.28 & 51.52\\
$\phantom{+}0.925$ & 55.70 & 55.45\\
\hline
\end{tabular}
\end{center}

The coefficient of determination $R^2$ is
\begin{equation*}
R^2=1-\frac{\sum(y-\hat{y})^2}{\sum(y-\bar{y})^2}=0.98
\end{equation*}

The result indicates that electroweak theory explains 98\% of the variance in the data.

\end{document}
