\input{preamble}

% http://kirkmcd.princeton.edu/examples/dirac.pdf

% Lancaster and Blundell p. 333 "In fact, the Dirac equation is correctly interpreted as providing the equation of motion for fermionic quantum fields."

\section*{Dirac from boost}

This is a Dirac spinor that represents an electron at rest with spin up along the $z$ axis.
\begin{equation*}
u_0=\sqrt{2m}\begin{pmatrix}1\\0\\0\\0\end{pmatrix}
\end{equation*}

This matrix boosts a spinor in the $z$ direction where $E^2=p^2+m^2$.
\begin{equation*}
\Lambda=\frac{1}{\sqrt{2m(E+m)}}
\begin{pmatrix}
E+m & 0 & p & 0\\
0 & E+m & 0 & p\\
p & 0 & E+m & 0\\
0 & p & 0 & E+m\\
\end{pmatrix}
\end{equation*}

Hence
\begin{equation*}
u=\Lambda u_0=\frac{1}{\sqrt{E+m}}
\begin{pmatrix}
E+m & 0 & p & 0\\
0 & E+m & 0 & p\\
p & 0 & E+m & 0\\
0 & p & 0 & E+m\\
\end{pmatrix}
\begin{pmatrix}1\\0\\0\\0\end{pmatrix}
=\frac{1}{\sqrt{E+m}}\begin{pmatrix}E+m\\0\\p\\0\end{pmatrix}
\end{equation*}

This is the Dirac equation in spinor form.
\begin{equation*}
\slashed pu=mu
\end{equation*}

Substitute $\Lambda u_0$ for $u$.
\begin{equation*}
\slashed p\Lambda u_0=m\Lambda u_0
\end{equation*}

By the identity $\gamma^0u_0=u_0$ substitute $\gamma^0u_0$ for $u_0$ on the right hand side.
\begin{equation*}
\slashed p\Lambda u_0=m\Lambda\gamma^0u_0
\end{equation*}

Substitute $\Lambda^{-1}u$ for $u_0$.
\begin{equation*}
\slashed p\Lambda\Lambda^{-1}u=m\Lambda\gamma^0\Lambda^{-1}u
\end{equation*}

Cancel $\Lambda\Lambda^{-1}$ and $u$ to obtain
\begin{equation*}
\slashed p=m\Lambda\gamma^0\Lambda^{-1}
\end{equation*}

Multiply both sides by $m^{-1}$ and $\Lambda$.
\begin{equation*}
m^{-1}\slashed p\Lambda=\Lambda\gamma^0
\tag{1}
\end{equation*}

To recover the Dirac equation, start with this identity.
\begin{equation*}
\gamma^0u_0=u_0
\end{equation*}

Boost both sides of the equation.
\begin{equation*}
\Lambda\gamma^0u_0=\Lambda u_0
\end{equation*}

By equation (1) we have
\begin{equation*}
m^{-1}\slashed p\Lambda u_0=\Lambda u_0
\end{equation*}

Hence
\begin{equation*}
\slashed pu=mu
\tag{2}
\end{equation*}

\end{document}
