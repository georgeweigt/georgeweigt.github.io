\documentclass[12pt]{article}
\usepackage[margin=1in]{geometry}
\usepackage{amsmath}
\parindent=0pt
\begin{document}

\section*{What is $d\sigma$?}

$d\sigma$ multiplied by $\sin\theta$ is an
{\it unnormalized} probability density function.
Integrate $d\sigma\sin\theta$ to obtain a
cumulative distribution function $F(\theta)$.
Differentiate $F(\theta)$ to obtain a
{\it normalized} probability density function $f(\theta)$.
\begin{equation*}
f(\theta)=\frac{dF(\theta)}{d\theta}
\propto d\sigma\sin\theta
\end{equation*}

For example, the well-known cross section for Bhabha scattering is
\begin{equation*}
\frac{d\sigma}{d\Omega}
=\frac{\alpha^2(\hbar c)^2}{4s}
\left(\frac{\cos^2\theta+3}{\cos\theta-1}\right)^2
\end{equation*}

Let $I(\theta)$ be the following integral of $d\sigma$.
(The $\sin\theta$ is from $d\Omega=\sin\theta\,d\theta\,d\phi$.)
\begin{equation*}
I(\theta)=\int
\left(
\frac{\cos^2\theta+3}{\cos\theta-1}
\right)^2
\sin\theta\,d\theta
\end{equation*}

The result is
\begin{equation*}
I(\theta)=\frac{16}{\cos\theta-1}-\frac{\cos^3\theta}{3}-\cos^2\theta-9\cos\theta-16\log(1-\cos\theta)
\end{equation*}

The cumulative distribution function is
\begin{equation*}
F(\theta)=\frac{I(\theta)-I(a)}{I(\pi)-I(a)},
\quad
a\le\theta\le\pi
\end{equation*}

Angular support is reduced by an arbitrary angle $a>0$ because $I(0)$ is undefined.

\bigskip

The probability of observing scattering events in the interval $\theta_1$ to $\theta_2$ is
\begin{equation*}
P(\theta_1\le\theta\le\theta_2)=F(\theta_2)-F(\theta_1)
\end{equation*}

Let $N$ be the total number of scattering events from an experiment.
Then the number of scattering events in the interval $\theta_1$
to $\theta_2$ is predicted to be
$$
N\times\bigl(F(\theta_2)-F(\theta_1)\bigr)
$$

The probability density function is
$$
f(\theta)=\frac{dF(\theta)}{d\theta}
=\frac{1}{I(\pi)-I(a)}
\left(\frac{\cos^2\theta+3}{\cos\theta-1}\right)^2
\sin\theta
$$

Note that if we had carried through the $\alpha^2(\hbar c)^2/4s$ in $I(\theta)$,
it would have canceled out in $F(\theta)$.

\bigskip

The raw data from scattering experiments are counts per angular bin.
The raw data are processed to produce numbers that can
be compared directly with $d\sigma$.
For example, here is Bhabha scattering data from DESY.
\begin{equation*}
\begin{matrix}
x & y\\
-0.7300 & 0.10115\\
-0.6495 & 0.12235\\
-0.5495 & 0.11258\\
-0.4494 & 0.09968\\
-0.3493 & 0.14749\\
-0.2491 & 0.14017\\
-0.1490 & 0.18190\\
-0.0488 & 0.22964\\
\phantom{+}0.0514 & 0.25312\\
\phantom{+}0.1516 & 0.30998\\
\phantom{+}0.2520 & 0.40898\\
\phantom{+}0.3524 & 0.62695\\
\phantom{+}0.4529 & 0.91803\\
\phantom{+}0.5537 & 1.51743\\
\phantom{+}0.6548 & 2.56714\\
\phantom{+}0.7323 & 4.30279\\
\end{matrix}
\end{equation*}

Data $x$ and $y$ have the following relationship with the cross section formula.
\begin{equation*}
x=\cos\theta,
\quad
y=\frac{d\sigma}{d\Omega}\text{ in nanobarns}
\end{equation*}

The Bhabha scattering cross section formula is
\begin{equation*}
\frac{d\sigma}{d\Omega}
=\frac{\alpha^2}{4s}
\left(\frac{\cos^2\theta+3}{\cos\theta-1}\right)^2\times(\hbar c)^2
\end{equation*}

To compute predicted values $\hat{y}$, multiply by $10^{37}$ to convert square meters to nanobarns.
\begin{equation*}
\hat{y}
=\frac{\alpha^2}{4s}
\left(\frac{x^2+3}{x-1}\right)^2
\times(\hbar c)^2
\times10^{37}
\end{equation*}

The following table shows predicted values $\hat{y}$ for $s=(14.0\,\text{GeV})^2$.
\begin{equation*}
\begin{matrix}
x & y & \hat{y}\\
-0.7300 & 0.10115 & 0.110296\\
-0.6495 & 0.12235 & 0.113816\\
-0.5495 & 0.11258 & 0.120101\\
-0.4494 & 0.09968 & 0.129075\\
-0.3493 & 0.14749 & 0.141592\\
-0.2491 & 0.14017 & 0.158934\\
-0.1490 & 0.18190 & 0.182976\\
-0.0488 & 0.22964 & 0.216737\\
\phantom{+}0.0514 & 0.25312 & 0.264989\\
\phantom{+}0.1516 & 0.30998 & 0.335782\\
\phantom{+}0.2520 & 0.40898 & 0.443630\\
\phantom{+}0.3524 & 0.62695 & 0.615528\\
\phantom{+}0.4529 & 0.91803 & 0.907700\\
\phantom{+}0.5537 & 1.51743 & 1.451750\\
\phantom{+}0.6548 & 2.56714 & 2.609280\\
\phantom{+}0.7323 & 4.30279 & 4.615090\\
\end{matrix}
\end{equation*}

The coefficient of determination $R^2$ measures how well predicted values fit the data.
\begin{equation*}
R^2=1-\frac{\sum(y-\hat{y})^2}{\sum(y-\bar{y})^2}=0.995
\end{equation*}

The result indicates that 99.5\% of the variance in the data is explained by $d\sigma$.

\end{document}
