\input{preamble}

\section*{Cold emission}

Consider the following potential energy function where $Q$ is a positive constant.
\begin{equation*}
V(x)=
\begin{cases}
0, & x<0
\\
V_0-Qx, & x\ge0
\end{cases}
\end{equation*}

Suppose a particle of mass $m$ and energy $E<V_0$ is traveling from left to right
along the $x$ axis.
The particle is in a potential energy barrier for
\begin{equation*}
E\le V_0-Qx
\end{equation*}

Solving for $x$ we have the particle in the barrier for
\begin{equation*}
x\le\frac{V_0-E}{Q}
\end{equation*}

Hence the particle has a Schrodinger equation for each the following regions.
\begin{align*}
-\frac{\hbar^2}{2m}\frac{d^2}{dx^2}&\psi_1=E\psi_1, & &x<0
\\
-\frac{\hbar^2}{2m}\frac{d^2}{dx^2}&\psi_2+(V_0-Qx)\psi_2=E\psi_2, & &0\le x\le\frac{V_0-E}{Q}
\\
-\frac{\hbar^2}{2m}\frac{d^2}{dx^2}&\psi_3=E\psi_3, & &x>\frac{V_0-E}{Q}
\end{align*}

Let $\psi_1$ and $\psi_3$ have the most general
free-particle solutions.
\begin{align*}
\psi_1(x)=A\exp\left(i\sqrt{\frac{2mE}{\hbar^2}}x\right)
+B\exp\left(-i\sqrt{\frac{2mE}{\hbar^2}}x\right)
\\
\psi_3(x)=F\exp\left(i\sqrt{\frac{2mE}{\hbar^2}}x\right)
+G\exp\left(-i\sqrt{\frac{2mE}{\hbar^2}}x\right)
\end{align*}

Let $W=V_0-E$ and use the WKB approximation to solve for $\psi_2$.
\begin{align*}
\psi_2(x)&=
C\exp\left(-\frac{1}{\hbar}\int\sqrt{2m(W-Qx)}\,dx\right)
+D\exp\left(\frac{1}{\hbar}\int\sqrt{2m(W-Qx)}\,dx\right)
\\
{}&=C\exp\left(\frac{\left(2m(W-Qx)\right)^\frac{3}{2}}{3Qm\hbar}\right)
+D\exp\left(-\frac{\left(2m(W-Qx)\right)^\frac{3}{2}}{3Qm\hbar}\right)
\end{align*}

To simplify the formulas let
\begin{equation*}
k=\frac{\sqrt{2mE}}{\hbar},\quad\beta(x)=\frac{(2m(W-Qx))^\frac{3}{2}}{3Qm\hbar}
\end{equation*}

and write
\begin{align*}
\psi_1(x)&=A\exp(ikx)+B\exp(-ikx)
\\
\psi_2(x)&=C\exp(\beta(x))+D\exp(-\beta(x))
\\
\psi_3(x)&=F\exp(ikx)+G\exp(-ikx)
\end{align*}

Exponentials of $-i$ represent particles moving from right to left.
The $B$ exponential represents a particle reflected from the
boundary at $x=0$.
There is no reflection for $x>(V_0-E)/Q$ hence $G=0$.

\bigskip
Let us now solve for the coefficients using boundary conditions.
Let $L=(V_0-E)/Q$.
Four boundary conditions are needed to ensure continuity
at $x=0$ and $x=L$.
\begin{align*}
\psi_1(0)&=\psi_2(0)
\\
\psi_1'(0)&=\psi_2'(0)
\\
\psi_2(L)&=\psi_3(L)
\\
\psi_2'(L)&=\psi_3'(L)
\end{align*}

From the boundary condition $\psi_2(L)=\psi_3(L)$ we have
\begin{equation*}
C\exp(\beta(L))+D\exp(-\beta(L))=F\exp(ikL)
\tag{1}
\end{equation*}

From the boundary condition $\psi_2'(L)=\psi_3'(L)$ we have
\begin{equation*}
\beta'(L) C\exp(\beta(L))-\beta'(L) D\exp(-\beta(L))
=ikF\exp(ikL)
\tag{2}
\end{equation*}

Add $\beta'(L)$ times (1) to (2) to obtain
\begin{equation*}
2\beta'(L)C\exp(\beta(L))=(\beta'(L)+ik)F\exp(ikL)
\end{equation*}

Hence
\begin{equation*}
C=\frac{(\beta'(L)+ik)F\exp(ikL-\beta(L))}{2\beta'(L)}
\tag{3}
\end{equation*}

\end{document}

Add minus $\beta$ times (1) to (2) to obtain
\begin{equation*}
-2\beta D\exp(-\beta L)=(-\beta+ik)F\exp(ikL)
\end{equation*}

Hence
\begin{equation*}
D=\frac{(\beta-ik)F\exp(ikL+\beta L)}{2\beta}
\tag{4}
\end{equation*}

From the boundary condition $\psi_1(0)=\psi_2(0)$ we have
\begin{equation*}
A+B=C+D
\tag{5}
\end{equation*}

From the boundary condition $\psi_1'(0)=\psi_2'(0)$ we have
\begin{equation*}
ik(A-B)=\beta(C-D)
\tag{6}
\end{equation*}

Add $ik$ times (5) to (6) to obtain
\begin{equation*}
2ikA=\beta(C-D)+ik(C+D)
\end{equation*}

Hence
\begin{equation*}
A=\frac{\beta(C-D)}{2ik}+\frac{C+D}{2}
\end{equation*}

Substitute for $C$ and $D$ to obtain
\begin{equation*}
A=F\exp(ikL)\left(\cosh(\beta L)+\frac{i}{2}\left(\frac{\beta}{k}-\frac{k}{\beta}\right)\sinh(\beta L)\right)
\end{equation*}

It follows that
\begin{equation*}
\frac{|A|^2}{|F|^2}=\cosh^2(\beta L)+\frac{1}{4}\left(\frac{\beta}{k}-\frac{k}{\beta}\right)^2\sinh^2(\beta L)
\end{equation*}

Hence the transmission probability $|T|^2$ is
\begin{equation*}
|T|^2=\frac{|F|^2}{|A|^2}=\frac{1}{\cosh^2(\beta L)+\frac{1}{4}\left(\frac{\beta}{k}
-\frac{k}{\beta}\right)^2\sinh^2(\beta L)}
\end{equation*}

If the transmission probability is small then a useful approximation is
\begin{equation*}
\frac{1}{4}\left(\frac{\beta}{k}-\frac{k}{\beta}\right)^2
\approx-\frac{1}{2}
\end{equation*}

It follows that
\begin{equation*}
|T|^2\approx\frac{1}{\cosh^2(\beta L)-\frac{1}{2}\sinh^2(\beta L)}
=\frac{1}{\frac{1}{8}\exp\left(\frac{4\sqrt{2m}LW^\frac{3}{2}}{3Q\hbar}\right)
+\frac{1}{8}\exp\left(-\frac{4\sqrt{2m}LW^\frac{3}{2}}{3Q\hbar}\right)+\frac{3}{4}}
\end{equation*}

The first exponential term dominates hence
\begin{equation*}
|T|^2\approx 8\exp\left(\frac{4\sqrt{2m}LW^\frac{3}{2}}{3Q\hbar}\right)
\end{equation*}

\end{document}
