\input{preamble}

\section*{Annihilation}
Annihilation is the interaction $e^-+e^+\rightarrow\gamma+\gamma$.

\begin{center}
\begin{tikzpicture}
\draw[dashed] (0,0) circle (0.5cm);
\draw[thick,->] (2,0) node[anchor=west] {$e^+$} -- (0.6,0);
\draw[thick,->] (-2,0) node[anchor=east] {$e^-$} -- (-0.6,0);
\draw[thick,->] (0.40,0.40) -- (1.3,1.3) node[anchor=south west] {$\gamma$};
\draw[thick,->] (-0.4,-0.4) -- (-1.3,-1.3) node[anchor=north east] {$\gamma$};
\draw (1,0.5) node {$\theta$};
\end{tikzpicture}
\end{center}

In the center-of-mass frame we have the following momentum vectors where $E=\sqrt{p^2+m^2}$.
\begin{equation*}
p_1=\underset{\substack{\\ \text{inbound}\\ \text{electron}}}
{\begin{pmatrix}E\\0\\0\\p\end{pmatrix}}
\qquad
p_2=\underset{\substack{\\ \text{inbound}\\ \text{positron}}}
{\begin{pmatrix}E\\0\\0\\-p\end{pmatrix}}
\qquad
p_3=\underset{\substack{\\ \text{outbound}\\ \text{photon}}}
{\begin{pmatrix}E\\ E\sin\theta\cos\phi\\ E\sin\theta\sin\phi\\ E\cos\theta\end{pmatrix}}
\qquad
p_4=\underset{\substack{\\ \text{outbound}\\ \text{photon}}}
{\begin{pmatrix}E\\ -E\sin\theta\cos\phi\\ -E\sin\theta\sin\phi\\ -E\cos\theta\end{pmatrix}}
\end{equation*}

Spinors for the inbound electron.
\begin{equation*}
u_{11}=\frac{1}{\sqrt{E+m}}
\underset{\substack{\\ \text{inbound electron}\\ \text{spin up}}}
{\begin{pmatrix}E+m\\0\\p\\0\end{pmatrix}}
\qquad
u_{12}=\frac{1}{\sqrt{E+m}}
\underset{\substack{\\ \text{inbound electron}\\ \text{spin down}}}
{\begin{pmatrix}0\\E+m\\0\\-p\end{pmatrix}}
\end{equation*}

Spinors for the inbound positron.
\begin{equation*}
v_{21}=\frac{1}{\sqrt{E+m}}
\underset{\substack{\\ \text{inbound positron}\\ \text{spin up}}}
{\begin{pmatrix}-p\\0\\E+m\\0\end{pmatrix}}
\qquad
v_{22}=\frac{1}{\sqrt{E+m}}
\underset{\substack{\\ \text{inbound positron}\\ \text{spin down}}}
{\begin{pmatrix}0\\p\\0\\E+m\end{pmatrix}}
\end{equation*}

Let $a$ be the spin state of the inbound electron and let $b$
be the spin state of the inbound positron
such that subscript $ba\in\{11,12,21,22\}$.
The probability amplitude $\mathcal M_{ba}$ for spin state $ba$ is
\begin{equation*}
\mathcal M_{ba}=\mathcal M_{1ba}+\mathcal M_{2ba}
\end{equation*}

where
\begin{equation*}
\mathcal M_{1ba}=\frac{\bar{v}_{2b}(-ie\gamma^\mu)(\slashed q_1+m)(-ie\gamma^\nu)u_{1a}}{t-m^2},
\quad
\mathcal M_{2ba}=\frac{\bar{v}_{2b}(-ie\gamma^\nu)(\slashed q_2+m)(-ie\gamma^\mu)u_{1a}}{u-m^2}
\end{equation*}

Symbol $e$ is elementary charge and
\begin{gather*}
\begin{aligned}
\slashed q_1&=(p_1-p_3)^\alpha g_{\alpha\beta}\gamma^\beta
\\
\slashed q_2&=(p_1-p_4)^\alpha g_{\alpha\beta}\gamma^\beta
\end{aligned}
\\[1ex]
\begin{aligned}
t&=(p_1-p_3)^2
\\
u&=(p_1-p_4)^2
\end{aligned}
\end{gather*}

The expected probability density $\langle|\mathcal M|^2\rangle$
is the average probability density for all four spin states.
\begin{equation*}
\langle|\mathcal M|^2\rangle=\frac{1}{4}\sum_{a=1}^2\sum_{b=1}^2|\mathcal M_{ba}|^2
\end{equation*}

Hence
\begin{equation*}
\langle|\mathcal{M}|^2\rangle
=\frac{1}{4}\sum_{a=1}^2\sum_{b=1}^2
\left(
\frac{\mathcal M_{1ba}\mathcal M_{1ba}^*}{(t-m^2)^2}
+\frac{\mathcal M_{1ba}\mathcal M_{2ba}^*+\mathcal M_{2ba}\mathcal M_{1ba}^*}{(t-m^2)(u-m^2)}
+\frac{\mathcal M_{2ba}\mathcal M_{2ba}^*}{(u-m^2)^2}
\right)
\end{equation*}

To understand how $\mathcal M_{1ba}\mathcal M_{1ba}^*$ is calculated,
write $\mathcal M_{1ba}$ in component form.
\begin{equation*}
(\mathcal M_{1ba})^{\mu\nu}=\frac{
(\bar{v}_{2b})_\alpha
(-ie\gamma^{\mu\alpha}{}_\beta)
(\slashed q_1+m)^\beta{}_\rho
(-ie\gamma^{\nu\rho}{}_\sigma)
(u_{1a})^\sigma}{t-m^2}
\end{equation*}

Metric tensor $g_{\mu\nu}$ is required to sum over indices $\mu$ and $\nu$.
\begin{equation*}
\mathcal M_{1ba}\mathcal M_{1ba}^*=(\mathcal M_{1ba})^{\mu\nu}(\mathcal M_{1ba}^*)_{\mu\nu}
=(\mathcal M_{1ba})^{\mu\nu}g_{\mu\alpha}(\mathcal M_{1ba}^*)^{\alpha\beta}g_{\beta\nu}
\end{equation*}

Similarly for $\mathcal M_{2ba}\mathcal M_{2ba}^*$.
For $\mathcal M_{2ba}$ the index order is $\nu$ followed by $\mu$ hence
\begin{equation*}
\mathcal M_{1ba}\mathcal M_{2ba}^*=(\mathcal M_{1ba})^{\mu\nu}(\mathcal M_{2ba}^*)_{\nu\mu}
=(\mathcal M_{1ba})^{\mu\nu}g_{\nu\beta}(\mathcal M_{2ba}^*)^{\beta\alpha}g_{\alpha\mu}
\end{equation*}

The Casimir trick uses matrix arithmetic to sum over spin states.
\begin{align*}
f_{11}&=\sum_{a=1}^2\sum_{b=1}^2\mathcal M_{1ba}\mathcal M_{1ba}^*=e^4\operatorname{Tr}
\left(
(\slashed{p}_1+m)\gamma^\mu(\slashed q_1+m)\gamma^\nu(\slashed p_2-m)\gamma_\nu(\slashed q_1+m)\gamma_\mu
\right)
\\
f_{12}&=\sum_{a=1}^2\sum_{b=1}^2\mathcal M_{1ba}\mathcal M_{2ba}^*=e^4\operatorname{Tr}
\left(
(\slashed{p}_1+m)\gamma^\mu(\slashed q_2+m)\gamma^\nu(\slashed p_2-m)\gamma_\mu(\slashed q_1+m)\gamma_\nu
\right)
\\
f_{22}&=\sum_{a=1}^2\sum_{b=1}^2\mathcal M_{2ba}\mathcal M_{2ba}^*=e^4\operatorname{Tr}
\left(
(\slashed{p}_1+m)\gamma^\mu(\slashed q_2+m)\gamma^\nu(\slashed p_2-m)\gamma_\nu(\slashed q_2+m)\gamma_\mu
\right)
\end{align*}

Hence
\begin{equation*}
\langle|\mathcal{M}|^2\rangle
=
\frac{1}{4}
\left(
\frac{f_{11}}{(t-m^2)^2}
+\frac{2f_{12}}{(t-m^2)(u-m^2)}
+\frac{f_{22}}{(u-m^2)^2}
\right)
\end{equation*}

The following formulas are equivalent to the Casimir trick.
(Recall that $a\cdot b=a^\mu g_{\mu\nu}b^\nu$)
\begin{align*}
f_{11}&=e^4\left(
 32 (p_1 \cdot p_3) (p_1 \cdot p_4) -
 32 m^2 (p_1 \cdot p_2) +
 64 m^2 (p_1 \cdot p_3) +
 32 m^2 (p_1 \cdot p_4) - 64 m^4\right)
\\
f_{12}&=e^4\left(
 16 m^2 (p_1 \cdot p_3) +
 16 m^2 (p_1 \cdot p_4) - 32 m^4\right)
\\
f_{22}&=e^4\left(
 32 (p_1 \cdot p_3) (p_1 \cdot p_4) -
 32 m^2 (p_1 \cdot p_2) +
 32 m^2 (p_1 \cdot p_3) +
 64 m^2 (p_1 \cdot p_4) - 64 m^4\right)
\end{align*}

In Mandelstam variables
\begin{align*}
f_{11}&=e^4\left(8 t u - 24 t m^2 - 8 u m^2 - 8 m^4\right)
\\
f_{12}&=e^4\left(8 s m^2 - 32 m^4\right)
\\
f_{22}&=e^4\left(8 t u - 8 t m^2 - 24 u m^2 - 8 m^4\right)
\end{align*}

For high energy experiments such that $E\gg m$ let $m=0$ and obtain
\begin{align*}
f_{11}&=e^4\,8tu
\\
f_{12}&=0
\\
f_{22}&=e^4\,8tu
\end{align*}

Hence
\begin{align*}
\langle|\mathcal{M}|^2\rangle
&=
\frac{e^4}{4}
\left(
\frac{8tu}{t^2}
+\frac{8tu}{u^2}
\right)
\\
&=
2e^4
\left(
\frac{u}{t}
+\frac{t}{u}
\right)
\end{align*}

For $m=0$ the Mandelstam variables are
\begin{align*}
t&=-2E^2(1-\cos\theta)
\\
u&=-2E^2(1+\cos\theta)
\end{align*}

Hence
\begin{equation*}
\langle|\mathcal{M}|^2\rangle
=2e^4\left(
\frac{1+\cos\theta}{1-\cos\theta}+
\frac{1-\cos\theta}{1+\cos\theta}
\right)
\end{equation*}

\subsubsection*{Cross section}
The differential cross section is
\begin{equation*}
\frac{d\sigma}{d\Omega}=\frac{\langle|\mathcal{M}|^2\rangle}{4(4\pi\varepsilon_0)^2s}
\end{equation*}

where
\begin{equation*}
s=(p_1+p_2)^2=4E^2
\end{equation*}

For high energy experiments we have
\begin{equation*}
\langle|\mathcal{M}|^2\rangle=2e^4\left(
\frac{1+\cos\theta}{1-\cos\theta}+
\frac{1-\cos\theta}{1+\cos\theta}
\right)
\end{equation*}

Hence for high energy experiments
\begin{equation*}
\frac{d\sigma}{d\Omega}
=\frac{e^4}{2(4\pi\varepsilon_0)^2s}\left(\frac{1+\cos\theta}{1-\cos\theta}+\frac{1-\cos\theta}{1+\cos\theta}\right)
\end{equation*}

Noting that
\begin{equation*}
e^2=4\pi\varepsilon_0\alpha\hbar c
\end{equation*}

we have
\begin{equation*}
\frac{d\sigma}{d\Omega}
=
\frac{\alpha^2(\hbar c)^2}{2s}
\left(
\frac{1+\cos\theta}{1-\cos\theta}+
\frac{1-\cos\theta}{1+\cos\theta}
\right)
\end{equation*}

Noting that
\begin{equation*}
d\Omega=\sin\theta\,d\theta\,d\phi
\end{equation*}

we also have
\begin{equation*}
d\sigma=
\frac{\alpha^2(\hbar c)^2}{2s}
\left(
\frac{1+\cos\theta}{1-\cos\theta}+
\frac{1-\cos\theta}{1+\cos\theta}
\right)\sin\theta\,d\theta\,d\phi
\end{equation*}

Let $S(\theta_1,\theta_2)$ be the following surface integral of $d\sigma$.
\begin{equation*}
S(\theta_1,\theta_2)=\int_0^{2\pi}\int_{\theta_1}^{\theta_2}d\sigma
\end{equation*}

The solution is
\begin{equation*}
S(\theta_1,\theta_2)=\frac{2\pi\alpha^2(\hbar c)^2}{2s}
\bigl(I(\theta_2)-I(\theta_1)\bigr)
\end{equation*}

where
\begin{equation*}
I(\theta)=2\cos\theta+2\log(1-\cos\theta)-2\log(1+\cos\theta)
\end{equation*}

The cumulative distribution function is
\begin{equation*}
F(\theta)
=\frac{S(a,\theta)}{S(a,\pi-a)}
=\frac{I(\theta)-I(a)}{I(\pi-a)-I(a)},
\quad
a\le\theta\le\pi-a
\end{equation*}

Angular support is reduced by an arbitrary angle $a>0$ because $I(0)$ and $I(\pi)$ are undefined.

\bigskip
The probability of observing scattering events in the interval $\theta_1$ to $\theta_2$ is
\begin{equation*}
P(\theta_1\le\theta\le\theta_2)=F(\theta_2)-F(\theta_1)
\end{equation*}

Let $N$ be the total number of scattering events from an experiment.
Then the number of scattering events in the interval $\theta_1$
to $\theta_2$ is predicted to be
\begin{equation*}
NP(\theta_1\le\theta\le\theta_2)
\end{equation*}

The probability density function is
\begin{equation*}
f(\theta)=\frac{dF(\theta)}{d\theta}
=\frac{1}{I(\pi-a)-I(a)}
\left(\frac{1+\cos\theta}{1-\cos\theta}+\frac{1-\cos\theta}{1+\cos\theta}\right)
\sin\theta
\end{equation*}

\subsubsection*{Data from DESY PETRA experiment}
See www.hepdata.net/record/ins191231, Table 2, 14.0 GeV.
\begin{equation*}
\begin{matrix}
x & y\\
0.0502 & 0.09983\\
0.1505 & 0.10791\\
0.2509 & 0.12026\\
0.3512 & 0.13002\\
0.4516 & 0.17681\\
0.5521 & 0.19570\\
0.6526 & 0.27900\\
0.7312 & 0.33204
\end{matrix}
\end{equation*}

Data $x$ and $y$ have the following relationship
with the differential cross section formula.
\begin{equation*}
x=\cos\theta,
\quad
y=\frac{d\sigma}{d\Omega}
\end{equation*}

The cross section formula is
\begin{equation*}
\frac{d\sigma}{d\Omega}
=
\frac{\alpha^2}{2s}
\left(
\frac{1+\cos\theta}{1-\cos\theta}+
\frac{1-\cos\theta}{1+\cos\theta}
\right)\times(\hbar c)^2
\end{equation*}

To compute predicted values $\hat{y}$,
multiply by $10^{37}$ to convert square meters to nanobarns.
\begin{equation*}
\hat{y}
=
\frac{\alpha^2}{2s}
\left(
\frac{1+x}{1-x}+
\frac{1-x}{1+x}
\right)
\times(\hbar c)^2
\times10^{37}
\end{equation*}

The following table shows predicted values $\hat y$ for $s=(14.0\,\text{GeV})^2$.
\begin{equation*}
\begin{matrix}
x & y & \hat y\\
0.0502 & 0.09983 & 0.106325\\
0.1505 & 0.10791 & 0.110694\\
0.2509 & 0.12026 & 0.120005\\
0.3512 & 0.13002 & 0.135559\\
0.4516 & 0.17681 & 0.159996\\
0.5521 & 0.19570 & 0.198562\\
0.6526 & 0.27900 & 0.262745\\
0.7312 & 0.33204 & 0.348884\\
\end{matrix}
\end{equation*}

The coefficient of determination $R^2$ measures how well predicted values fit the data.
\begin{equation*}
R^2=1-\frac{\sum(y-\hat{y})^2}{\sum(y-\bar{y})^2}=0.98
\end{equation*}

The result indicates that the model $d\sigma$ explains 98\% of the variance in the data.

\end{document}
