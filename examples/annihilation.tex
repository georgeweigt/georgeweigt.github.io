\input{preamble}

\section*{Annihilation}

Annihilation is the process $e^-+e^+\rightarrow\gamma+\gamma$.

\begin{center}
\begin{tikzpicture}
\draw[dashed] (0,0) circle (0.5cm);
\draw[thick,->] (2,0) node[anchor=west] {$e^+$} -- (0.6,0);
\draw[thick,->] (-2,0) node[anchor=east] {$e^-$} -- (-0.6,0);
\draw[thick,->] (0.40,0.40) -- (1.3,1.3) node[anchor=south west] {$\gamma$};
\draw[thick,->] (-0.4,-0.4) -- (-1.3,-1.3) node[anchor=north east] {$\gamma$};
\draw (1,0.5) node {$\theta$};
\end{tikzpicture}
\end{center}

The following center-of-mass momentum vectors have $E=\sqrt{p^2+m^2}$.
\begin{equation*}
p_1=\underset{e^- \, \longrightarrow}
{\begin{pmatrix}E\\0\\0\\p\end{pmatrix}}
\qquad
p_2=\underset{\longleftarrow \, e^+}
{\begin{pmatrix}E\\0\\0\\-p\end{pmatrix}}
\qquad
p_3=\underset{\substack{\phantom{\gamma} \, \nearrow\\\gamma \, \phantom{\nearrow}}}
{\begin{pmatrix}E\\ E\sin\theta\cos\phi\\ E\sin\theta\sin\phi\\ E\cos\theta\end{pmatrix}}
\qquad
p_4=\underset{\substack{\phantom{\swarrow} \, \gamma\\\swarrow \, \phantom{\gamma}}}
{\begin{pmatrix}E\\ -E\sin\theta\cos\phi\\ -E\sin\theta\sin\phi\\ -E\cos\theta\end{pmatrix}}
\end{equation*}

Spinors for $p_1$.
\begin{equation*}
u_{11}=\frac{1}{\sqrt{E+m}}
\underset{\text{spin up}}
{\begin{pmatrix}E+m\\0\\p\\0\end{pmatrix}}
\qquad
u_{12}=\frac{1}{\sqrt{E+m}}
\underset{\text{spin down}}
{\begin{pmatrix}0\\E+m\\0\\-p\end{pmatrix}}
\end{equation*}

Spinors for $p_2$.
\begin{equation*}
v_{21}=\frac{1}{\sqrt{E+m}}
\underset{\text{spin up}}
{\begin{pmatrix}-p\\0\\E+m\\0\end{pmatrix}}
\qquad
v_{22}=\frac{1}{\sqrt{E+m}}
\underset{\text{spin down}}
{\begin{pmatrix}0\\p\\0\\E+m\end{pmatrix}}
\end{equation*}

The scattering amplitude $\mathcal M_{ab}{}^{\mu\nu}$
for spin $ab$ and polarization $\mu\nu$ is
\begin{equation*}
\mathcal M_{ab}{}^{\mu\nu}=\mathcal M_{1ab}{}^{\mu\nu}+\mathcal M_{2ab}{}^{\nu\mu}
\end{equation*}

where
\begin{align*}
\mathcal M_{1ab}{}^{\mu\nu}
&=\frac{\bar v_{2b}(-ie\gamma^\mu)(\slashed q_1+m)(-ie\gamma^\nu)u_{1a}}{t-m^2}
\\
\mathcal M_{2ab}{}^{\nu\mu}
&=\frac{\bar v_{2b}(-ie\gamma^\nu)(\slashed q_2+m)(-ie\gamma^\mu)u_{1a}}{u-m^2}
\end{align*}

Matrices $\slashed q_1$ and $\slashed q_2$ represent momentum transfer.
\begin{align*}
\slashed q_1&=(p_1-p_3)^\alpha g_{\alpha\beta}\gamma^\beta
\\
\slashed q_2&=(p_1-p_4)^\alpha g_{\alpha\beta}\gamma^\beta
\end{align*}

Scalars $t$ and $u$ are Mandelstam variables.
\begin{align*}
t&=(p_1-p_3)^2
\\
u&=(p_1-p_4)^2
\end{align*}

In component form
\begin{align*}
\mathcal M_{1ab}{}^{\mu\nu}&=\frac{
(\bar v_{2b})_\alpha
(-ie\gamma^{\mu\alpha}{}_\beta)
(\slashed q_1+m)^\beta{}_\rho
(-ie\gamma^{\nu\rho}{}_\sigma)
(u_{1a})^\sigma}{t-m^2}
\\
\mathcal M_{2ab}{}^{\nu\mu}&=\frac{
(\bar v_{2b})_\alpha
(-ie\gamma^{\nu\alpha}{}_\beta)
(\slashed q_2+m)^\beta{}_\rho
(-ie\gamma^{\mu\rho}{}_\sigma)
(u_{1a})^\sigma}{u-m^2}
\end{align*}

Expected probability density $\langle|\mathcal M|^2\rangle$
is the sum over squared amplitudes divided by the number of inbound states.
\begin{equation*}
\langle|\mathcal M|^2\rangle
=\frac{1}{4}\sum_{ab}\sum_{\mu\nu}\bigl|\mathcal M_{ab}{}^{\mu\nu}\bigr|^2
\end{equation*}

Summing over $\mu\nu$ requires $g_{\mu\nu}$ to lower indices.
\begin{equation*}
\langle|\mathcal M|^2\rangle
=\frac{1}{4}\sum_{ab}\mathcal M_{ab}{}^{\mu\nu}
\left(g_{\mu\alpha}\mathcal M_{ab}{}^{\alpha\beta}g_{\beta\nu}\right)^*
\end{equation*}

Expand the summand and label the terms.
By positivity $\boxed{\scriptstyle2}=\boxed{\scriptstyle3}$.
\begin{multline*}
\langle|\mathcal{M}|^2\rangle
=\frac{1}{4}\sum_{ab}\Bigl[
\underset{\boxed{\scriptstyle1}}
{\mathcal M_{1ab}{}^{\mu\nu}\left(g_{\mu\alpha}\mathcal M_{1ab}{}^{\alpha\beta}g_{\beta\nu}\right)^*}
+
\underset{\boxed{\scriptstyle2}}
{\mathcal M_{1ab}{}^{\mu\nu}\left(g_{\nu\alpha}\mathcal M_{2ab}{}^{\alpha\beta}g_{\beta\mu}\right)^*}
\\
+
\underset{\boxed{\scriptstyle3}}
{\mathcal M_{2ab}{}^{\nu\mu}\left(g_{\mu\alpha}\mathcal M_{1ab}{}^{\alpha\beta}g_{\beta\nu}\right)^*}
+
\underset{\boxed{\scriptstyle4}}
{\mathcal M_{2ab}{}^{\nu\mu}\left(g_{\nu\alpha}\mathcal M_{2ab}{}^{\alpha\beta}g_{\beta\mu}\right)^*}
\Bigr]
\end{multline*}

The following Casimir trick uses matrix arithmetic to sum over spin and polarization states.
\begin{align*}
\sum_{ab}{\boxed{\scriptstyle1}}&=\frac{e^4}{(t-m^2)^2}\operatorname{Tr}
\left[
(\slashed p_1+m)\gamma^\mu(\slashed q_1+m)\gamma^\nu(\slashed p_2-m)\gamma_\nu(\slashed q_1+m)\gamma_\mu
\right]
\\
\sum_{ab}{\boxed{\scriptstyle2}}&=\frac{e^4}{(t-m^2)(u-m^2)}\operatorname{Tr}
\left[
(\slashed p_1+m)\gamma^\mu(\slashed q_2+m)\gamma^\nu(\slashed p_2-m)\gamma_\mu(\slashed q_1+m)\gamma_\nu
\right]
\\
\sum_{ab}{\boxed{\scriptstyle4}}&=\frac{e^4}{(u-m^2)^2}\operatorname{Tr}
\left[
(\slashed p_1+m)\gamma^\mu(\slashed q_2+m)\gamma^\nu(\slashed p_2-m)\gamma_\nu(\slashed q_2+m)\gamma_\mu
\right]
\end{align*}

Probability density $\langle|\mathcal{M}|^2\rangle$ can be reformulated as
\begin{equation*}
\langle|\mathcal{M}|^2\rangle=\frac{e^4}{4}
\left[
\frac{f_{11}}{(t-m^2)^2}+\frac{2f_{12}}{(t-m^2)(u-m^2)}+\frac{f_{22}}{(u-m^2)^2}
\right]
\end{equation*}

with Casimir trick terms
\begin{align*}
f_{11}&=\operatorname{Tr}
\left[
(\slashed p_1+m)\gamma^\mu(\slashed q_1+m)\gamma^\nu(\slashed p_2-m)\gamma_\nu(\slashed q_1+m)\gamma_\mu
\right]
\\
f_{12}&=\operatorname{Tr}
\left[
(\slashed p_1+m)\gamma^\mu(\slashed q_2+m)\gamma^\nu(\slashed p_2-m)\gamma_\mu(\slashed q_1+m)\gamma_\nu
\right]
\\
f_{22}&=\operatorname{Tr}
\left[
(\slashed p_1+m)\gamma^\mu(\slashed q_2+m)\gamma^\nu(\slashed p_2-m)\gamma_\nu(\slashed q_2+m)\gamma_\mu
\right]
\end{align*}

The following formulas are equivalent to the Casimir trick.
(Recall that $a\cdot b=a^\mu g_{\mu\nu}b^\nu$)
\begin{align*}
f_{11}&=32 (p_1\cdot p_3) (p_1\cdot p_4) + 32 (p_1\cdot p_3) m^2 - 32 m^4
\\
f_{12}&=16 (p_1\cdot p_2) m^2 - 16 m^4
\\
f_{22}&=32 (p_1\cdot p_3) (p_1\cdot p_4) + 32 (p_1\cdot p_4) m^2 - 32 m^4
\end{align*}

In Mandelstam variables
\begin{align*}
f_{11}&=8 t u - 24 t m^2 - 8 u m^2 - 8 m^4
\\
f_{12}&=8 s m^2 - 32 m^4
\\
f_{22}&=8 t u - 8 t m^2 - 24 u m^2 - 8 m^4
\end{align*}

For $E\gg m$ a useful approximation is to set $m=0$ and obtain
\begin{align*}
f_{11}&=8tu
\\
f_{12}&=0
\\
f_{22}&=8tu
\end{align*}

Hence
\begin{align*}
\langle|\mathcal{M}|^2\rangle
&=\frac{e^4}{4}
\left(
\frac{f_{11}}{(t-m^2)^2}
+\frac{2f_{12}}{(t-m^2)(u-m^2)}
+\frac{f_{22}}{(u-m^2)^2}
\right)
\\
&=\frac{e^4}{4}
\left(
\frac{8tu}{t^2}
+\frac{8tu}{u^2}
\right)
\\
&=2e^4
\left(
\frac{u}{t}
+\frac{t}{u}
\right)
\end{align*}

For $m=0$ the Mandelstam variables are
\begin{align*}
t&=-2E^2(1-\cos\theta)
\\
u&=-2E^2(1+\cos\theta)
\end{align*}

Hence
\begin{equation*}
\langle|\mathcal{M}|^2\rangle
=2e^4\left(
\frac{1+\cos\theta}{1-\cos\theta}+
\frac{1-\cos\theta}{1+\cos\theta}
\right)
\end{equation*}

\subsubsection*{Cross section}

The differential cross section is
\begin{equation*}
\frac{d\sigma}{d\Omega}=\frac{\langle|\mathcal{M}|^2\rangle}{4(4\pi\varepsilon_0)^2s}
\end{equation*}

where
\begin{equation*}
s=(p_1+p_2)^2=4E^2
\end{equation*}

For high energy experiments we have
\begin{equation*}
\langle|\mathcal{M}|^2\rangle=2e^4\left(
\frac{1+\cos\theta}{1-\cos\theta}+
\frac{1-\cos\theta}{1+\cos\theta}
\right)
\end{equation*}

Hence
\begin{equation*}
\frac{d\sigma}{d\Omega}
=\frac{e^4}{2(4\pi\varepsilon_0)^2s}\left(\frac{1+\cos\theta}{1-\cos\theta}+\frac{1-\cos\theta}{1+\cos\theta}\right)
\end{equation*}

Noting that
\begin{equation*}
e^2=4\pi\varepsilon_0\alpha\hbar c
\end{equation*}

we have
\begin{equation*}
\frac{d\sigma}{d\Omega}
=
\frac{\alpha^2(\hbar c)^2}{2s}
\left(
\frac{1+\cos\theta}{1-\cos\theta}+
\frac{1-\cos\theta}{1+\cos\theta}
\right)
\end{equation*}

Noting that
\begin{equation*}
d\Omega=\sin\theta\,d\theta\,d\phi
\end{equation*}

we also have
\begin{equation*}
d\sigma=
\frac{\alpha^2(\hbar c)^2}{2s}
\left(
\frac{1+\cos\theta}{1-\cos\theta}+
\frac{1-\cos\theta}{1+\cos\theta}
\right)\sin\theta\,d\theta\,d\phi
\end{equation*}

Let $S(\theta_1,\theta_2)$ be the following integral of $d\sigma$.
\begin{equation*}
S(\theta_1,\theta_2)=\int_0^{2\pi}\int_{\theta_1}^{\theta_2}d\sigma
\end{equation*}

The solution is
\begin{equation*}
S(\theta_1,\theta_2)=\frac{\pi\alpha^2(\hbar c)^2}{s}
[I(\theta_2)-I(\theta_1)]
\end{equation*}

where
\begin{equation*}
I(\theta)=2\cos\theta+2\log(1-\cos\theta)-2\log(1+\cos\theta)
\end{equation*}

The cumulative distribution function is
\begin{equation*}
F(\theta)
=\frac{S(a,\theta)}{S(a,\pi-a)}
=\frac{I(\theta)-I(a)}{I(\pi-a)-I(a)},
\quad
a\le\theta\le\pi-a
\end{equation*}

Angular support is reduced by an arbitrary angle $a>0$ because $I(0)$ and $I(\pi)$ are undefined.

\bigskip
The probability of observing scattering events in the interval $\theta_1$ to $\theta_2$ is
\begin{equation*}
P(\theta_1<\theta\le\theta_2)=F(\theta_2)-F(\theta_1)
\end{equation*}

The probability density function is
\begin{equation*}
f(\theta)=\frac{dF(\theta)}{d\theta}
=\frac{1}{I(\pi-a)-I(a)}
\left(\frac{1+\cos\theta}{1-\cos\theta}+\frac{1-\cos\theta}{1+\cos\theta}\right)
\sin\theta
\end{equation*}

\end{document}
