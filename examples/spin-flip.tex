\input{preamble}

\section*{Spin flip}

From exercise 5.8 of ``Quantum Mechanics'' by Richard Fitzpatrick.

\medskip
Consider an electron at rest in the following magnetic field $\mathbf B$.
\begin{equation*}
\mathbf B=B_0\cos(\omega t)\begin{pmatrix}0\\0\\1\end{pmatrix}
\end{equation*}

Find the minimum $B_0$ such that $\langle S_x\rangle$ ranges from
$-\frac{\hbar}{2}$ to $+\frac{\hbar}{2}$.

\medskip
The Hamiltonian is
\begin{equation*}
H=\frac{e}{m}\mathbf B\cdot\mathbf S
\end{equation*}

Hence by hypothesis
\begin{equation*}
H=\frac{e}{m}B_0\cos(\omega t)S_z
\end{equation*}

Let $|s\rangle$ be the following spin state.
\begin{equation*}
|s\rangle=\begin{pmatrix}c_1(t)\\[1ex]c_2(t)\end{pmatrix}
\end{equation*}

By the Schr\"odinger equation
\begin{equation*}
i\hbar\frac{\partial}{\partial t}|s\rangle=H|s\rangle
\end{equation*}

In component form
\begin{align*}
i\hbar\frac{\partial}{\partial t}c_1(t)&=\frac{e\hbar}{2m}B_0\cos(\omega t)c_1(t)
\\
i\hbar\frac{\partial}{\partial t}c_2(t)&=-\frac{e\hbar}{2m}B_0\cos(\omega t)c_2(t)
\end{align*}

Solve for $c_1(t)$ and $c_2(t)$ and normalize so that $\langle s|s\rangle=|c_1(t)|^2+|c_2(t)|^2=1$.
\begin{equation*}
\begin{aligned}
c_1(t)&=\frac{1}{\sqrt2}\exp\left[-\frac{ie}{2m\omega}B_0\sin(\omega t)\right]
\\
c_2(t)&=\frac{1}{\sqrt2}\exp\left[\frac{ie}{2m\omega}B_0\sin(\omega t)\right]
\end{aligned}
\tag{1}
\end{equation*}

Having obtained $|s\rangle$ we can now solve for $\langle S_x\rangle$.
\begin{equation*}
\langle S_x\rangle=\langle s|S_x|s\rangle
=\frac{\hbar}{2}\cos\left[\frac{e}{m\omega}B_0\sin(\omega t)\right]
\tag{2}
\end{equation*}

At time $t=0$
\begin{equation*}
\langle S_x\rangle=\frac{\hbar}{2}
\end{equation*}

To obtain $\langle S_x\rangle=-\frac{\hbar}{2}$ we must have
\begin{equation*}
\frac{e}{m\omega}B_0\sin(\omega t)=\pi
\end{equation*}

Taking $\sin(\omega t)=1$ we have
\begin{equation*}
B_0=\frac{\pi m\omega}{e}
\end{equation*}

\href{https://georgeweigt.github.io/examples/spin-flip-demo.html}{Eigenmath script}

\end{document}
