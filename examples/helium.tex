\input{preamble}

\section*{Helium}
Consider the simplest Schroedinger equation.
\begin{equation*}
\hat{H}\psi=E\psi
\end{equation*}

For a wavefunction $\psi$ that is an exact solution, energy $E$ is a number.
However, when $\psi$ is an approximate solution, $E$ varies over configuration space.
For example, two electrons have six dimensions in configuration space so we have
\begin{equation*}
\hat{H}\psi=E(r_1,r_2,\theta_1,\theta_2,\phi_1,\phi_2)\,\psi
\end{equation*}

We still want a single number for energy so we use the expected energy $\langle E\rangle$.
The expected energy is the average value of function $E$ over configuration space.
If the wavefunction $\psi$ is a good approximation
then $\langle E\rangle$ should be close to the actual energy determined by experiment.

\bigskip
For a helium atom, find the expected energy $\langle E\rangle$ given the following wavefunction.
\begin{equation*}
\psi=\frac{\alpha^3}{\pi}\exp\bigl(-\alpha (r_1+r_2)\bigr)
\end{equation*}

Recall that $|\psi|^2=\psi^*\psi$ is a probability density function.
Hence the expected energy is
\begin{equation*}
\langle E\rangle=\int E\,|\psi|^2\,dV_1\,dV_2
\end{equation*}

Reorder factors in the integrand.
For the wavefunction given above, $\psi^*=\psi$.
\begin{equation*}
\langle E\rangle=\int\psi E\psi\,dV_1\,dV_2
\end{equation*}

Replace $E\psi$ with $\hat{H}\psi$.
\begin{equation*}
\langle E\rangle=\int\psi\hat{H}\psi\,dV_1\,dV_2
\end{equation*}

This is a simplified Hamiltonian for helium, in atomic units of $\hbar=m_e=e=4\pi\epsilon_0=1$.
\begin{equation*}
\hat{H}=
\underset{\substack{\\[1ex]\text{kinetic}\\\text{energy}\\\text{electron 1}}}{\left(-\frac{1}{2}\nabla^2_1\right)}
+\underset{\substack{\\[1ex]\text{kinetic}\\\text{energy}\\\text{electron 2}}}{\left(-\frac{1}{2}\nabla^2_2\right)}
+\underset{\substack{\\[1ex]\text{potential}\\\text{energy}\\\text{electron 1}}}{\left(-\frac{Z}{r_1}\right)}
+\underset{\substack{\\[1ex]\text{potential}\\\text{energy}\\\text{electron 2}}}{\left(-\frac{Z}{r_2}\right)}
+\underset{\substack{\\[1ex]\text{potential}\\\text{energy}\\\text{inter-electron}}}{\frac{1}{r_{12}}}
\end{equation*}

Hence
\begin{equation*}
\langle E\rangle=
\int\psi\left(
-\frac{1}{2}\nabla^2_1
-\frac{1}{2}\nabla^2_2
-\frac{Z}{r_1}
-\frac{Z}{r_2}
+\frac{1}{r_{12}}
\right)\psi\,dV_1\,dV_2
\end{equation*}

The measure $dV_1\,dV_2$ is the product of two volume elements.
\begin{equation*}
dV_1\,dV_2=r_1^2 r_2^2 \sin\theta_1 \sin\theta_2
\,dr_1\,dr_2\,d\theta_1\,d\theta_2\,d\phi_1\,d\phi_2
\end{equation*}

Since $\psi$ has no angular dependence, we can use the identity
\begin{equation*}
\int_0^{2\pi}\int_0^{2\pi}\int_0^\pi\int_0^\pi\sin\theta_1\sin\theta_2
\,d\theta_1\,d\theta_2\,d\phi_1\,d\phi_2=16\pi^2
\tag{1}
\end{equation*}
to obtain
\begin{equation*}
\langle E\rangle=16\pi^2\int_0^\infty\int_0^\infty
\psi\left(
-\frac{1}{2}\nabla^2_1
-\frac{1}{2}\nabla^2_2
-\frac{Z}{r_1}
-\frac{Z}{r_2}
+\frac{1}{r_{12}}
\right)\psi
\,r_1^2 r_2^2
\,dr_1\,dr_2
\end{equation*}

Solve for each term separately, starting with the kinetic energy of the first electron.
\begin{equation*}
\langle K_1\rangle
=16\pi^2\int_0^\infty\int_0^\infty\psi\left(-\frac{1}{2}\nabla_1^2\right)\psi
\,r_1^2r_2^2
\,dr_1\,dr_2
\end{equation*}

The Laplacian is
\begin{equation*}
\nabla^2=\frac{1}{r^2}\frac{\partial}{\partial r}
\left(r^2\frac{\partial}{\partial r}\right)
+
\frac{1}{r^2\sin\theta}\frac{\partial}{\partial\theta}
\left(\sin\theta\frac{\partial}{\partial\theta}\right)
+
\frac{1}{r^2\sin^2\theta}\frac{\partial^2}{\partial\phi^2}
\end{equation*}

Since $\psi=\alpha^3/\pi\exp(-\alpha(r_1+r_2))$ has no angular dependence we have for the first electron
\begin{equation*}
\nabla^2_1\psi
=\frac{1}{r_1^2}\frac{\partial}{\partial r_1}\left(r_1^2\frac{\partial}{\partial r_1}\right)\psi
\end{equation*}

It follows that
\begin{equation*}
\nabla^2_1\psi=\left(\alpha^2-\frac{2\alpha}{r_1}\right)\psi
\end{equation*}

Hence
\begin{equation*}
\langle K_1\rangle=-8\alpha^6\int_0^\infty\int_0^\infty
\left(\alpha^2-\frac{2\alpha}{r_1}\right)\exp\bigl(-2\alpha(r_1+r_2)\bigr)
\,r_1^2\,r_2^2
\,dr_1\,dr_2
\end{equation*}

The result is
\begin{equation*}
\langle K_1\rangle=\tfrac{1}{2}\alpha^2
\tag{2}
\end{equation*}

By symmetry $\langle K_1\rangle=\langle K_2\rangle$ so we can write
\begin{equation*}
\langle K\rangle=\langle K_1\rangle+\langle K_2\rangle=\alpha^2
\end{equation*}

For the potential energy of the first electron we have
\begin{equation*}
\langle V_1\rangle
=
16\pi^2\int_0^\infty\int_0^\infty\psi\left(-\frac{Z}{r_1}\right)\psi
\,r_1^2r_2^2
\,dr_1\,dr_2
=-Z\alpha
\tag{3}
\end{equation*}

Again by symmetry we can write
\begin{equation*}
\langle V\rangle=\langle V_1\rangle+\langle V_2\rangle=-2Z\alpha
\end{equation*}

Finally, for inter-electron potential energy we have (derivation is below)
\begin{equation*}
\langle U\rangle
=
16\pi^2\int_0^\infty\int_0^\infty\frac{\psi^2}{r_{12}}
\,r_1^2r_2^2
\,dr_1\,dr_2
=\tfrac{5}{8}\alpha
\tag{4}
\end{equation*}

Summing over kinetic and potential energies we have
\begin{equation*}
\langle E\rangle=\langle K\rangle+\langle V\rangle+\langle U\rangle=\alpha^2-2Z\alpha+\tfrac{5}{8}\alpha
\end{equation*}

Next, find $\alpha$ that minimizes $\langle E\rangle$.
\begin{equation*}
\frac{d}{d\alpha}\langle E\rangle=2\alpha-2Z+\tfrac{5}{8}=0
\end{equation*}

Solve for $\alpha$.
\begin{equation*}
\alpha=Z-\tfrac{5}{16}
\end{equation*}

For helium with $Z=2$ we have
\begin{equation*}
\alpha=\tfrac{27}{16}
\end{equation*}

Hence the expected energy is
\begin{equation*}
\langle E\rangle
=\alpha^2-2Z\alpha+\tfrac{5}{8}\alpha
=-\frac{729}{256}\,\text{hartree}
\end{equation*}

The result is in hartrees because we used atomic units.
Convert hartrees to electron volts.
\begin{equation*}
\langle E\rangle=-\frac{729}{256}\,\text{hartree}\times27.2114\,\frac{\text{eV}}{\text{hartree}}=-77.4887\,\text{eV}
\end{equation*}

It turns out that $\langle E\rangle$ differs from the observed value by about 2\%.
\begin{equation*}
\frac{77.4887}{79.0052}=0.98
\end{equation*}

To verify equation (4) we will now show that
\begin{equation*}
\int\frac{\psi_1^2\psi_2^2}{r_{12}}\,dV_1\,dV_2=\tfrac{5}{8}\alpha
\end{equation*}
where
\begin{equation*}
\psi_j=\sqrt{\frac{\alpha^3}{\pi}}\exp\left(-\alpha r_j\right)
\end{equation*}
Symbol $r_{12}$ is the following distance function where $\theta_{12}$ is angular separation.
\begin{equation*}
r_{12}=\sqrt{r_1^2+r_2^2-r_1r_2\cos\theta_{12}}
\end{equation*}

Let $I(r_1)$ be the following integral over $V_2$.
\begin{equation*}
I(r_1)=\int\frac{\psi_2^2}{r_{12}}\,dV_2
\end{equation*}
The measure $dV_2$ is a volume element in spherical coordinates.
\begin{equation*}
dV_2=r_2^2\sin\theta_2\,dr_2\,d\theta_2\,d\phi_2
\end{equation*}

Write out the full integral and make $\theta_2=\theta_{12}$ by independence of the coordinate system.
\begin{equation*}
I(r_1)=\frac{\alpha^3}{\pi}
\int\limits_0^{2\pi}\int\limits_0^\pi\int\limits_0^\infty
\frac{\exp(-2\alpha r_2)}{\sqrt{r_1^2+r_2^2-r_1r_2\cos\theta_2}}
\,r_2^2\sin\theta_2\,dr_2\,d\theta_2\,d\phi_2
\end{equation*}

Integrate over $\phi_2$.
\begin{equation*}
I(r_1)=
2\alpha^3\int\limits_0^\pi\int\limits_0^\infty
\frac{\exp(-2\alpha r_2)}{\sqrt{r_1^2+r_2^2-r_1r_2\cos\theta_2}}
\,r_2^2\sin\theta_2\,dr_2\,d\theta_2
\end{equation*}

Expand $1/r_{12}$ in Legendre polynomials.
The first integral is over $r_2<r_1$ and the second integral is over $r_2>r_1$.
\begin{multline*}
I(r_1)=
2\alpha^3\int\limits_0^\pi\int\limits_0^{r_1}
\exp(-2\alpha r_2)
\left(\sum_{k=0}^\infty\frac{r_2^k}{r_1^{k+1}}P_k(\cos\theta_2)\right)
r_2^2\sin\theta_2\,dr_2\,d\theta_2
\\
+2\alpha^3\int\limits_0^\pi\int\limits_{r_1}^\infty
\exp(-2\alpha r_2)
\left(\sum_{k=0}^\infty\frac{r_1^k}{r_2^{k+1}}P_k(\cos\theta_2)\right)
r_2^2\sin\theta_2\,dr_2\,d\theta_2
\end{multline*}

It turns out that, after integrating over $\theta_2$, all summands vanish except for $k=0$.
\begin{equation*}
\int\limits_0^\pi P_k(\cos\theta_2)\sin\theta_2\,d\theta_2=
\left\{
\begin{aligned}
&2, & k=0
\\
&0, & k>0
\end{aligned}\right.
\tag{5}
\end{equation*}

Hence
\begin{equation*}
I(r_1)=
\frac{4\alpha^3}{r_1}\int\limits_0^{r_1}\exp(-2\alpha r_2)\,r_2^2\,dr_2
+4\alpha^3\int\limits_{r_1}^\infty\exp(-2\alpha r_2)\,r_2\,dr_2
\end{equation*}

Solve the integrals.
\begin{equation*}
I(r_1)=
\frac{4\alpha^3}{r_1}
\left.
\exp(-2\alpha r_2)\left(-\frac{r_2^2}{2\alpha}-\frac{r_2}{2\alpha^2}-\frac{1}{4\alpha^3}
\right)\right|_0^{r_1}
+4\alpha^3\left.\exp(-2\alpha r_2)\left(-\frac{r_2}{2\alpha}-\frac{1}{4\alpha^2}\right)\right|_{r_1}^\infty
\end{equation*}

Evaluate per limits.
\begin{equation*}
I(r_1)=\frac{1}{r_1}-\frac{1}{r_1}\exp(-2\alpha r_1)-\alpha\exp(-2\alpha r_1)
\tag{6}
\end{equation*}

Having obtained $I(r_1)$ we can now evaluate the integral over $V_1$.
\begin{equation*}
I=\frac{\alpha^3}{\pi}\int\limits_0^{2\pi}\int\limits_0^\pi\int\limits_0^\infty
\exp(-2\alpha r_1)I(r_1)\,r_1^2\sin\theta_1\,dr_1\,d\theta_1\,d\phi_1
\end{equation*}

Integrate over $\theta_1$ and $\phi_1$.
\begin{equation*}
I=4\alpha^3\int\limits_0^\infty
\exp(-2\alpha r_1)I(r_1)\,r_1^2\,dr_1
\end{equation*}

Expand the integrand.
\begin{equation*}
I=4\alpha^3\int\limits_0^\infty\exp(-2\alpha r_1)\,r_1\,dr_1
-4\alpha^3\int\limits_0^\infty\exp(-4\alpha r_1)\,r_1\,dr_1
-4\alpha^4\int\limits_0^\infty\exp(-4\alpha r_1)\,r_1^2\,dr_1
\end{equation*}

Solve the integrals.
\begin{multline*}
I=\exp(-2\alpha r_1)\left(-2\alpha^2r_1-\alpha\right)\bigg|_0^\infty
-\exp(-4\alpha r_1)\left(-\alpha^2r_1-\tfrac{1}{4}\alpha\right)\bigg|_0^\infty
\\
{}-\exp(-4\alpha r_1)\left(-\alpha^3r_1^2-\tfrac{1}{2}\alpha^2r_1-\tfrac{1}{8}\alpha\right)\bigg|_0^\infty
\end{multline*}

The result vanishes for $r_1=\infty$ hence
\begin{equation*}
I=0-\left(-\alpha+\tfrac{1}{4}\alpha+\tfrac{1}{8}\alpha\right)=\tfrac{5}{8}\alpha
\tag{7}
\end{equation*}

\end{document}
