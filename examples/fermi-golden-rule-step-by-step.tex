\documentclass[12pt]{article}
\usepackage[margin=1in]{geometry}
\usepackage{amsmath}
\parindent=0pt
\begin{document}

\section*{Fermi golden rule step by step}
Consider the transition rate $\Gamma_{1\rightarrow2}$
where $\psi_1(x)$ is the initial eigenstate
and $\psi_2(x)$ is the final eigenstate.

\bigskip
\textcircled{\scriptsize1}
Let $\Psi(x,t)$ be the following linear combination of eigenstates.
\begin{equation*}
\Psi(x,t)=c_1(t)\psi_1(x)\exp(-i\omega_1t)+c_2(t)\psi_2(x)\exp(-i\omega_2t)
\end{equation*}

We need to solve for $c_2(t)$ to find the transition rate.

\bigskip
\textcircled{\scriptsize2}
Let the perturbing Hamiltonian be
\begin{equation*}
H'(x,t)=2V(x)\cos(\omega t+\phi)
\end{equation*}

\textcircled{\scriptsize3}
From the time-dependent Schrodinger equation we have
\begin{equation*}
i\hbar\frac{\partial}{\partial t}\Psi(x,t)=H_0(x)\Psi(x,t)+H'(x,t)\Psi(x,t)
\end{equation*}

\textcircled{\scriptsize4}
This reduces to
\begin{equation*}
i\hbar\frac{\partial c_1(t)}{\partial t}\psi_1(x)\exp(-i\omega_1t)
+i\hbar\frac{\partial c_2(t)}{\partial t}\psi_2(x)\exp(-i\omega_2t)
=H'(x,t)\Psi(x,t)
\end{equation*}

\textcircled{\scriptsize5}
Take the inner product of $\psi_2^*(x)$ with the above equation to obtain
\begin{equation*}
i\hbar\frac{\partial c_2(t)}{\partial t}\exp(-i\omega_2t)
=2\cos(\omega t+\phi)
\Bigl(
c_1(t)M_{21}\exp(-i\omega_1t)+
c_2(t)M_{22}\exp(-i\omega_2t)
\Bigr)
\end{equation*}

where $M_{21}$ and $M_{22}$ are the matrix elements
\begin{align*}
M_{21}&=\int\psi_2^*(x)V(x)\psi_1(x)\,dx
\\
M_{22}&=\int\psi_2^*(x)V(x)\psi_2(x)\,dx
\end{align*}

\textcircled{\scriptsize6}
Rewrite as
\begin{equation*}
i\hbar\frac{\partial c_2(t)}{\partial t}
=2\cos(\omega t+\phi)
\Bigl(
c_1(t)M_{21}\exp\bigl(i(\omega_2-\omega_1)t\bigr)+2c_2(t)M_{22}
\Bigr)
\end{equation*}

\textcircled{\scriptsize7}
Let the initial state be $\Psi(x,0)=\psi_1(x)$ hence the initial conditions are
\begin{equation*}
c_1(0)=1,\quad c_2(0)=0
\end{equation*}

For small $t$ we use the approximations $c_1(t)=1$ and $c_2(t)=0$ to obtain
\begin{equation*}
i\hbar\frac{\partial c_2(t)}{\partial t}
=2\cos(\omega t+\phi)M_{21}\exp\bigl(i(\omega_2-\omega_1)t\bigr)
\end{equation*}

\textcircled{\scriptsize8}
Solve for $c_2(t)$ by integrating.
\begin{equation*}
c_2(t)=\frac{2M_{21}}{i\hbar}
\int_0^t\cos(\omega t'+\phi)\exp\bigl(i(\omega_2-\omega_1)t'\bigr)\,dt'
\end{equation*}

The solution is
\begin{equation*}
c_2(t)
=-\frac{M_{21}}{\hbar}
\left(
\frac{\exp\bigl(i(\omega_2-\omega_1-\omega) t\bigr)-1}{\omega_2-\omega_1-\omega}
\right)\exp(-i\phi)
-\frac{M_{21}}{\hbar}
\left(
\frac{\exp\bigl(i(\omega_2-\omega_1+\omega) t\bigr)-1}{\omega_2-\omega_1+\omega}
\right)\exp(i\phi)
\end{equation*}

\textcircled{\scriptsize9}
For $\omega$ such that $\omega\approx\omega_2-\omega_1$ the first term
dominates so discard the second term and write
\begin{equation*}
c_2(t)=-\frac{M_{21}}{\hbar}
\left(
\frac{\exp\bigl(i(\omega_2-\omega_1-\omega) t\bigr)-1}{\omega_2-\omega_1-\omega}
\right)\exp(-i\phi)
\end{equation*}

\textcircled{\scriptsize10}
Rewrite using a sinc function.
\begin{equation*}
c_2(t)=-\frac{it}{\hbar}\,M_{21}
\exp\left(i\,\frac{\omega_2-\omega_1-\omega}{2}\,t-i\phi\right)
\operatorname{sinc}\left(\frac{\omega_2-\omega_1-\omega}{2}\,t\right)
\end{equation*}

\textcircled{\scriptsize11}
Hence the transition probability is
\begin{equation*}
P(1\rightarrow2)=|c_2(t)|^2=\frac{t^2}{\hbar^2}\,|M_{21}|^2
\operatorname{sinc}^2\left(\frac{\omega_2-\omega_1-\omega}{2}\,t\right)
\end{equation*}

\textcircled{\scriptsize12}
Integrate $P(1\rightarrow2)$ to obtain the total transition probability
$P_{tot}(1\rightarrow2)$.
\begin{equation*}
P_{tot}(1\rightarrow2)=\frac{t^2}{\hbar^2}|M_{21}|^2
\int_{E-\epsilon}^{E+\epsilon}
\operatorname{sinc}^2\left(\frac{E'/\hbar-\omega}{2}\,t\right)
g(E')\,dE'
\end{equation*}

where
\begin{equation*}
E=\hbar(\omega_2-\omega_1)
\end{equation*}

and $g(E')$ is the density of photon states for energy $E'$.

\bigskip
\textcircled{\scriptsize13}
Use the approximation $g(E')\approx g(\hbar\omega)$ to obtain
\begin{equation*}
P_{tot}(1\rightarrow2)=\frac{t^2}{\hbar^2}|M_{21}|^2g(\hbar\omega)
\int_{E-\epsilon}^{E+\epsilon}
\operatorname{sinc}^2\left(\frac{E'/\hbar-\omega}{2}\,t\right)\,dE'
\end{equation*}

\textcircled{\scriptsize14}
Let
\begin{equation*}
y=\frac{E'/\hbar-\omega}{2}\,t
\end{equation*}

It follows that
\begin{equation*}
E'=\frac{2\hbar y}{t}+\hbar\omega
\end{equation*}

and
\begin{equation*}
dE'=\frac{2\hbar}{t}\,dy
\end{equation*}

The integration limits transform as
\begin{equation*}
E\pm\epsilon
\rightarrow
\frac{(E\pm\epsilon)/\hbar-\omega}{2}\,t
=\frac{Et}{2\hbar}-\frac{\omega t}{2}
\pm\frac{\epsilon t}{2\hbar}
\approx
\pm\frac{\epsilon t}{2\hbar}
\end{equation*}

Hence the integral is transformed as
\begin{equation*}
P_{tot}(1\rightarrow2)=\frac{2t}{\hbar}|M_{21}|^2g(\hbar\omega)
\int_{-\epsilon t/2\hbar}^{\epsilon t/2\hbar}
\operatorname{sinc}^2y\,dy
\end{equation*}

\textcircled{\scriptsize15}
The sinc squared function is very narrow so use the approximation
\begin{equation*}
\int_{-\epsilon t/2\hbar}^{\epsilon t/2\hbar}\operatorname{sinc}^2 y\,dy
\approx
\int_{-\infty}^\infty\operatorname{sinc}^2 y\,dy=\pi
\end{equation*}

to obtain
\begin{equation*}
P_{tot}(1\rightarrow2)=\frac{2\pi t}{\hbar}|M_{21}|^2g(\hbar\omega)
\end{equation*}

\textcircled{\scriptsize16}
The transition rate is the derivative of $P_{tot}(1\rightarrow2)$.
\begin{equation*}
\Gamma_{1\rightarrow2}
=\frac{d}{dt}P_{tot}(1\rightarrow2)=\frac{2\pi}{\hbar}|M_{21}|^2g(\hbar\omega)
\end{equation*}

\end{document}
