\input{preamble}

\section*{Rutherford scattering data}

The following data is from Geiger and Marsden's 1913 paper.\footnote{www.chemteam.info/Chem-History/GeigerMarsden-1913/GeigerMarsden-1913.html}
Column $y$ is number of scattering events for silver foil.
\begin{equation*}
\begin{matrix}
\theta & y\\
150 & 22.2\\
135 & 27.4\\
120 & 33.0\\
105 & 47.3\\
75 & 136\\
60 & 320\\
45 & 989\\
37.5 & 1760\\
30 & 5260\\
22.5 & 20300\\
15 & 105400
\end{matrix}
\end{equation*}

This is the differential cross section for Rutherford scattering.
\begin{equation*}
\frac{d\sigma}{d\Omega}\propto
\frac{1}{(1-\cos\theta)^2}
\end{equation*}

Let $f(k)$ be the probability of scattering into a detector at $\theta_k$.
Then
\begin{equation*}
f(k)=\Pr(\theta=\theta_k)\propto\left.\frac{d\sigma}{d\Omega}\right|_{\theta=\theta_k}
\end{equation*}

Hence
\begin{equation*}
f(k)=\frac{C}{(1-\cos\theta_k)^2}
\end{equation*}

where $C$ is a normalization constant.
To find $C$ let
\begin{equation*}
x_k=\frac{1}{(1-\cos\theta_k)^2}
\end{equation*}

By total probability
\begin{equation*}
\sum_k f(k)=C\sum_k x_k=1
\end{equation*}

It follows that
\begin{equation*}
C=\frac{1}{\sum_k x_k}=\frac{1}{1132}
\end{equation*}

Hence the scattering probability for angle $\theta_k$ is
\begin{equation*}
f(k)=\frac{x_k}{1132}=\frac{1}{1132\,(1-\cos\theta_k)^2}
\end{equation*}

Let $\hat y_k$ be predicted number of scattering events such that
\begin{equation*}
\Pr(y=\hat y_k)=\Pr(\theta=\theta_k)
\end{equation*}

It follows that
\begin{equation*}
\frac{\hat y_k}{\sum y}=f(k)
\end{equation*}

Hence
\begin{equation*}
\hat y_k=f(k)\sum y
\end{equation*}

The following table shows the predicted values.
\begin{equation*}
\begin{matrix}
\theta & y & \hat y\\
150 & 22.2 & 34.1\\
135 & 27.4 & 40.7\\
120 & 33.0 & 52.7\\
105 & 47.3 & 74.9\\
75 & 136 & 216\\
60 & 320 & 474\\
45 & 989 & 1383\\
37.5 & 1760 & 2778\\
30 & 5260 & 6608\\
22.5 & 20300 & 20471\\
15 & 105400 & 102162
\end{matrix}
\end{equation*}

\iffalse

The coefficient of determination $R^2$ measures how well predicted values fit the data.
\begin{equation*}
R^2=1-\frac{\sum (y-\hat y)^2}{\sum (y-\bar y)^2}=0.998
\end{equation*}

The result indicates that $d\sigma$ explains 99.8\%
of the variance in the data.

\fi

\end{document}
