\input{preamble}

\section*{Rutherford scattering data}

The following data is from Geiger and Marsden's 1913 paper
where $y$ is the number of scattering events.
\begin{equation*}
\begin{matrix}
\theta & y\\
150 & 22.2\\
135 & 27.4\\
120 & 33.0\\
105 & 47.3\\
75 & 136\\
60 & 320\\
45 & 989\\
37.5 & 1760\\
30 & 5260\\
22.5 & 20300\\
15 & 105400
\end{matrix}
\end{equation*}

This is the differential cross section for Rutherford scattering.
\begin{equation*}
\frac{d\sigma}{d\Omega}\propto
%\frac{1}{4\sin^4(\theta/2)}
\frac{1}{(1-\cos\theta)^2}
\end{equation*}

Let $f(k)$ be the following probability mass function.
\begin{equation*}
f(k)=\Pr(\theta=\theta_k)\propto\left.\frac{d\sigma}{d\Omega}\right|_{\theta=\theta_k}
\end{equation*}

Hence
\begin{equation*}
f(k)=\frac{C}{(1-\cos\theta_k)^2}
\end{equation*}

where $C$ is a normalization constant.
To find $C$ let
\begin{equation*}
x_i=\frac{1}{(1-\cos\theta_i)^2}
\end{equation*}

By total probability
\begin{equation*}
C\sum_i x_i=1
\end{equation*}

It follows that
\begin{equation*}
C=\frac{1}{\sum_i x_i}
\end{equation*}

Hence the scattering probability for angle $\theta_k$ is
\begin{equation*}
f(k)=\frac{x_k}{\sum_i x_i}
\end{equation*}

Predicted values $\hat y_k$ are computed as
\begin{equation*}
\hat y_k=f(k)\sum_i y_i=\frac{x_k\sum_i y_i}{\sum_i x_i}
\end{equation*}

The following table shows the predicted values $\hat y$.
\begin{equation*}
\begin{matrix}
\theta & y & \hat y\\
150 & 22.2 & 34.1\\
135 & 27.4 & 40.7\\
120 & 33.0 & 52.7\\
105 & 47.3 & 74.9\\
75 & 136 & 216\\
60 & 320 & 474\\
45 & 989 & 1383\\
37.5 & 1760 & 2778\\
30 & 5260 & 6608\\
22.5 & 20300 & 20471\\
15 & 105400 & 102162
\end{matrix}
\end{equation*}

The coefficient of determination $R^2$ measures how well predicted values fit the data.
\begin{equation*}
R^2=1-\frac{\sum_i(y_i-\hat y_i)^2}{\sum_i(y_i-\bar y)^2}=0.999
\end{equation*}

The result indicates that $d\sigma$ explains 99.9\%
of the variance in the data.

\end{document}
