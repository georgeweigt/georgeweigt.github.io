\input{preamble}

\section*{Quantum harmonic oscillator}

{\it Anything quadratic is called harmonic.} ---A. Zee

\bigskip
A harmonic oscillator is anything with potential energy proportional to displacement squared.
\begin{equation*}
V(x)\propto x^2
\end{equation*}

For a quantum harmonic oscillator
\begin{equation*}
V(x)=\frac{m\omega^2 x^2}{2}
\end{equation*}

Hence the hamiltonian
\begin{equation*}
\hat H=\frac{\hat p^2}{2m}+\frac{m\omega^2 x^2}{2}
\end{equation*}

We seek to solve the eigenvalue equation
\begin{equation*}
\hat H\psi_n=E_n\psi_n
\end{equation*}

The solution is
\begin{equation*}
\psi_n(x)=C_n\exp\left(-\frac{m\omega x^2}{2\hbar}\right)
H_n\left(x\sqrt{m\omega/\hbar}\right),\quad n=0,1,2,\ldots
\end{equation*}

$C_n$ is the normalization constant
\begin{equation*}
C_n=\frac{1}{\sqrt{2^nn!}}\biggl(\frac{m\omega}{\pi\hbar}\biggr)^{\frac{1}{4}}
\end{equation*}

$H_n$ is the $n$th hermite polynomial
\begin{equation*}
H_n(y)=(-1)^2\exp\left(y^2\right)\frac{d^n}{dy^n}\exp\left(-y^2\right)
\end{equation*}

The eigenvalues are
\begin{equation*}
E_n=\hbar\omega\left(n+\tfrac{1}{2}\right)
\end{equation*}

The ladder operators are
\begin{align*}
\hat a&=\sqrt{\frac{m\omega}{2\hbar}}\left(x+\frac{i\hat p}{m\omega}\right)
\\[1ex]
\hat a^\dag&=\sqrt{\frac{m\omega}{2\hbar}}\left(x-\frac{i\hat p}{m\omega}\right)
\end{align*}

Operator $\hat a$ lowers $\psi_n$.
\begin{equation*}
\hat a\psi_n=\sqrt n\psi_{n-1}
\end{equation*}

Operator $\hat a^\dag$ raises $\psi_n$.
\begin{equation*}
\hat a^\dag\psi_n=\sqrt{n+1}\psi_{n+1}
\end{equation*}

This is how $\psi_n$ can be obtained from $\psi_0$.
\begin{equation*}
\psi_n=\frac{(\hat a^\dag)^n}{\sqrt{n!}}\psi_0
\end{equation*}

The number operator is the result of lowering then raising.
\begin{equation*}
\hat N=\hat a^\dag\hat a,\quad\hat N\psi_n=n\psi_n
\end{equation*}

\subsubsection*{Exercises}

1. Verify $\psi_n$ and $E_n$.

\bigskip
2. Verify ladder operators.

\bigskip
3. Let
\begin{equation*}
\Psi(x)=\frac{\psi_2(x)+\psi_3(x)}{\sqrt2}
\end{equation*}

Verify that
\begin{align*}
\Pr(x\ge0)=\int_0^\infty \Psi^*\Psi\,dx\approx0.85
\end{align*}

4. Let
\begin{equation*}
m=6.64\times10^{-27}\,\text{kilogram},\quad
V(10^{-6}\,\text{meter})=1\,\text{electronvolt}
\end{equation*}

Verify that
\begin{equation*}
\omega=\sqrt{\frac{2V(x)}{mx^2}}=6.95\times10^9\,\text{second}^{-1}
\end{equation*}

For $\Psi=(\psi_2+\psi_3)/\sqrt2$ verify that
\begin{align*}
\langle x\rangle&=\int_{-\infty}^\infty x\Psi^*\Psi\,dx=1.85\times10^{-9}\,\text{meter}
\\
\langle E\rangle&=\int_{-\infty}^\infty \Psi^*\hat H\Psi\,dx=1.37\times10^{-5}\,\text{electronvolt}
=\frac{E_2+E_3}{2}
\end{align*}

\end{document}
