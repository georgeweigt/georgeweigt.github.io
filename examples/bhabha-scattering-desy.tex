\input{preamble}

\section*{Bhabha scattering DESY data}

See www.hepdata.net/record/ins191231, Table 3, 14.0 GeV.
\begin{equation*}
\begin{matrix}
x & y\\
-0.7300 & 0.10115\\
-0.6495 & 0.12235\\
-0.5495 & 0.11258\\
-0.4494 & 0.09968\\
-0.3493 & 0.14749\\
-0.2491 & 0.14017\\
-0.1490 & 0.18190\\
-0.0488 & 0.22964\\
\phantom+0.0514 & 0.25312\\
\phantom+0.1516 & 0.30998\\
\phantom+0.2520 & 0.40898\\
\phantom+0.3524 & 0.62695\\
\phantom+0.4529 & 0.91803\\
\phantom+0.5537 & 1.51743\\
\phantom+0.6548 & 2.56714\\
\phantom+0.7323 & 4.30279
\end{matrix}
\end{equation*}

Data $x$ and $y$ have the following relationship with the cross section formula.
\begin{equation*}
x=\cos\theta,
\quad
y=\frac{d\sigma}{d\Omega}\text{ in units of nanobarns}
\end{equation*}

The cross section formula is
\begin{equation*}
\frac{d\sigma}{d\Omega}
=\frac{\alpha^2}{4s}
\left(\frac{\cos^2\theta+3}{\cos\theta-1}\right)^2\times(\hbar c)^2
\end{equation*}

To compute predicted values $\hat y$, multiply by $10^{37}$ to convert square meters to nanobarns.
\begin{equation*}
\hat y
=\frac{\alpha^2}{4s}
\left(\frac{x^2+3}{x-1}\right)^2
\times(\hbar c)^2
\times10^{37}
\end{equation*}

The following table shows predicted values $\hat y$ for $s=(14.0\,\text{GeV})^2$.
\begin{equation*}
\begin{matrix}
x & y & \hat y\\
-0.7300 & 0.10115 & 0.110296\\
-0.6495 & 0.12235 & 0.113816\\
-0.5495 & 0.11258 & 0.120101\\
-0.4494 & 0.09968 & 0.129075\\
-0.3493 & 0.14749 & 0.141592\\
-0.2491 & 0.14017 & 0.158934\\
-0.1490 & 0.18190 & 0.182976\\
-0.0488 & 0.22964 & 0.216737\\
\phantom{+}0.0514 & 0.25312 & 0.264989\\
\phantom{+}0.1516 & 0.30998 & 0.335782\\
\phantom{+}0.2520 & 0.40898 & 0.443630\\
\phantom{+}0.3524 & 0.62695 & 0.615528\\
\phantom{+}0.4529 & 0.91803 & 0.907700\\
\phantom{+}0.5537 & 1.51743 & 1.451750\\
\phantom{+}0.6548 & 2.56714 & 2.609280\\
\phantom{+}0.7323 & 4.30279 & 4.615090
\end{matrix}
\end{equation*}

The coefficient of determination $R^2$ measures how well predicted values fit the data.
\begin{equation*}
R^2=1-\frac{\sum(y-\hat y)^2}{\sum(y-\bar y)^2}=0.995
\end{equation*}

The result indicates that the model $d\sigma$ explains 99.5\% of the variance in the data.

\end{document}
