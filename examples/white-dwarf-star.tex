\input{preamble}

% see Griffiths and Schroeter problem 5.35

\section*{White dwarf star}

The size of a white dwarf star can be estimated using the electron gas model of a solid.

\bigskip
The total electron energy of a spherical electron gas with radius $r$ is
\begin{equation*}
E=\left(\frac{3\pi^2}{2}\right)^\frac{1}{3}
\frac{9\hbar^2n^\frac{5}{3}}{20m_er^2}
\end{equation*}

where $n$ is the number of free electrons and $m_e$ is electron mass.

\bigskip
The gravitational energy of a sphere with mass $M$ and uniform density is
\begin{equation*}
U=-\frac{3GM^2}{5r}
\end{equation*}

Minimize the total energy by finding $r$ such that
\begin{equation*}
\frac{d}{dr}(E+U)=0
\end{equation*}

Hence
\begin{equation*}
-\left(\frac{3\pi^2}{2}\right)^\frac{1}{3}
\frac{9\hbar^2n^\frac{5}{3}}{10m_er^3}+\frac{3GM^2}{5r^2}=0
\end{equation*}

Multiply both sides by $r^3$.
\begin{equation*}
-\left(\frac{3\pi^2}{2}\right)^\frac{1}{3}
\frac{9\hbar^2n^\frac{5}{3}}{10m_e}+\frac{3GM^2}{5}r=0
\end{equation*}

Hence
\begin{equation*}
r=\left(\frac{3\pi^2}{2}\right)^\frac{1}{3}
\frac{9\hbar^2n^\frac{5}{3}}{10m_e}
\frac{5}{3GM^2}
=\left(\frac{3\pi^2}{2}\right)^\frac{1}{3}
\frac{3\hbar^2n^\frac{5}{3}}{2m_eGM^2}
\tag{1}
\end{equation*}

The number of free electrons is estimated to be one-half the number of nucleons.
For one solar mass we have
\begin{equation*}
n=\frac{M_{\odot}}{2m_p}=6\times10^{56}
\end{equation*}

For $M=M_{\odot}$ the radius is
\begin{equation*}
r=7146\,\text{km}
\end{equation*}

\end{document}
