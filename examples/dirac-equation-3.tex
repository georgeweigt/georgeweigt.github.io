\input{preamble}

\section*{Dirac equation 3}

From the previous section
\begin{align*}
\psi_1&=\underset{\text{wavefunction for fermion spin up}}
{\frac{e^{-i\xi/\hbar}}{\sqrt{E/c+mc}}\begin{pmatrix}E/c+mc\\0\\p_z\\p_x+ip_y\end{pmatrix}}
&
\psi_2&=\underset{\text{wavefunction for fermion spin down}}
{\frac{e^{-i\xi/\hbar}}{\sqrt{E/c+mc}}\begin{pmatrix}0\\E/c+mc\\p_x-ip_y\\-p_z\end{pmatrix}}
\\[1ex]
\psi_3&=\underset{\text{wavefunction for antifermion spin up}}
{\frac{e^{i\xi/\hbar}}{\sqrt{E/c+mc}}\begin{pmatrix}p_z\\p_x+ip_y\\E/c+mc\\0\end{pmatrix}}
&
\psi_4&=\underset{\text{wavefunction for antifermion spin down}}
{\frac{e^{i\xi/\hbar}}{\sqrt{E/c+mc}}\begin{pmatrix}p_x-ip_y\\-p_z\\0\\E/c+mc\end{pmatrix}}
\end{align*}

where
\begin{equation*}
\xi=p_\mu x^\mu=Et-p_xx-p_yy-p_zz
\end{equation*}

and
\begin{equation*}
E=\sqrt{p_x^2c^2+p_y^2c^2+p_z^2c^2+m^2c^4}
\end{equation*}

Spinors are $\psi$ without the exponentials.
\begin{align*}
u_1&=\underset{\text{spinor for fermion spin up}}
{\frac{1}{\sqrt{E/c+mc}}\begin{pmatrix}E/c+mc\\0\\p_z\\p_x+ip_y\end{pmatrix}}
&
u_2&=\underset{\text{spinor for fermion spin down}}
{\frac{1}{\sqrt{E/c+mc}}\begin{pmatrix}0\\E/c+mc\\p_x-ip_y\\-p_z\end{pmatrix}}
\\[1ex]
v_1&=\underset{\text{spinor for antifermion spin up}}
{\frac{1}{\sqrt{E/c+mc}}\begin{pmatrix}p_z\\p_x+ip_y\\E/c+mc\\0\end{pmatrix}}
&
v_2&=\underset{\text{spinor for antifermion spin down}}
{\frac{1}{\sqrt{E/c+mc}}\begin{pmatrix}p_x-ip_y\\-p_z\\0\\E/c+mc\end{pmatrix}}
\end{align*}

Spinors are solutions to the momentum-space Dirac equations
\begin{align*}
\slashed pu&=mcu
\\
\slashed pv&=-mcv
\end{align*}

where
\begin{equation*}
\slashed p=p^\mu g_{\mu\nu}\gamma^\nu
\end{equation*}

and
\begin{equation*}
p^\mu=\begin{pmatrix}E/c\\p_x\\p_y\\p_z\end{pmatrix}\quad
\end{equation*}

Spinors have the following ``completeness property.''
\begin{align*}
u_1\bar u_1+u_2\bar u_2&=\slashed p+mc
\\
v_1\bar v_1+v_2\bar v_2&=\slashed p-mc
\end{align*}

Adjoints of spinors are formed as
\begin{equation*}
\bar u=u^\dag\gamma^0,
\quad
\bar v=v^\dag\gamma^0
\end{equation*}

hence $u\bar u$ and $v\bar v$ are outer products that form $4\times4$ matrices.

\bigskip
\href{https://georgeweigt.github.io/examples/dirac-equation-3-demo.html}{Eigenmath script}

\end{document}
