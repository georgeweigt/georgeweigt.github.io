\input{preamble}

\section*{Canonical commutation relation}

This is the canonical commutation relation in one dimension.
\begin{equation*}
XP-PX=i\hbar
\end{equation*}

Let
\begin{equation*}
X=x,\quad P=-i\hbar\frac{\partial}{\partial x}
\end{equation*}

Show that
\begin{equation*}
(XP-PX)\psi(x,t)=i\hbar\psi(x,t)
\end{equation*}

\iffalse
Then
\begin{align*}
(XP-PX)\psi(x,t)&=XP\psi(x,t)-PX\psi(x,t)
\\
&=x\left(-i\hbar\frac{\partial}{\partial x}\psi(x,t)\right)
+i\hbar\frac{\partial}{\partial x}\bigl(x\psi(x,t)\bigr)
\\
&=-i\hbar x\frac{\partial}{\partial x}\psi(x,t)
+i\hbar\left(\frac{\partial}{\partial x}x\right)\psi(x,t)
+i\hbar x\frac{\partial}{\partial x}\psi(x,t)
\\
&=i\hbar\psi(x,t)
\end{align*}
\fi

Eigenmath code:
{\color{blue}
\begin{verbatim}
X(f) = x f
P(f) = -i hbar d(f,x)
X(P(psi(x,t))) - P(X(psi(x,t)))
\end{verbatim}}

Result:

\bigskip
$i\hbar\psi(x,t)$

\bigskip
Another example: Show that
\begin{equation*}
[X^2,P^2]=2i\hbar(XP+PX)
\end{equation*}

Eigenmath code:
{\color{blue}
\begin{verbatim}
X2(f) = X(X(f))
P2(f) = P(P(f))
A = X2(P2(psi(x,t))) - P2(X2(psi(x,t)))
B = 2 i hbar (X(P(psi(x,t))) + P(X(psi(x,t))))
check(A == B) -- continue if A equals B
"ok"
\end{verbatim}}

Result:

\bigskip
ok

\iffalse
\bigskip
Notes:
\begin{enumerate}
\item
Fedak  and Prentis write, ``The theory of Fourier and the correspondence principle of Bohr played a vital role in Heisenberg's development of quantum mechanics.''
\item
Aitchison, MacManus, and Snyder write, ``This `difficulty' clearly unsettled Heisenberg: but it very quickly became clear that the non-commutativity (in general) of kinematical quantities in quantum theory was the really essential new technical idea in the paper.''
\end{enumerate}
\fi

\end{document}
