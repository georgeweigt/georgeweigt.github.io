\documentclass[12pt]{article}
\usepackage{amsmath}
\parindent=0pt
\pdfinfoomitdate=1
\pdftrailerid{}
\begin{document}

\iffalse
Fedak  and Prentis write
\begin{quote}
The theory of Fourier and the correspondence principle of Bohr
played a vital role in Heisenberg's development of quantum mechanics.
\end{quote}

Aitchison, MacManus, and Snyder write
\begin{quote}
This `difficulty’ clearly unsettled Heisenberg: but it very
quickly became clear that the non-commutativity (in general) of kinematical
quantities in quantum theory was the really essential new technical idea in the
paper.
\end{quote}
\fi

Canonical commutation relation in one dimension:
\begin{equation*}
XP-PX=i\hbar
\end{equation*}

Let
\begin{equation*}
X=x,\quad P=-i\hbar\frac{d}{dx}
\end{equation*}

Then
\begin{align*}
(XP-PX)\psi(x)&=XP\psi(x)-PX\psi(x)
\\
&=x\left(-i\hbar\frac{d}{dx}\psi(x)\right)+i\hbar\frac{d}{dx}\bigl(x\psi(x)\bigr)
\\
&=-i\hbar x\frac{d}{dx}\psi(x)+i\hbar\left(\frac{d}{dx}x\right)\psi(x)+i\hbar x\frac{d}{dx}\psi(x)
\\
&=i\hbar\psi(x)
\end{align*}

Eigenmath code:
\begin{verbatim}
X(f) = x f
P(f) = -i hbar d(f,x)
X(P(psi(x))) - P(X(psi(x)))
\end{verbatim}

Result:
\begin{equation*}
i\hbar\psi(x)
\end{equation*}

\iffalse
Commutation relation in three dimensions.
\begin{equation*}
X_1P_1-P_1X_1=i\hbar,\quad
X_2P_2-P_2X_2=i\hbar,\quad
X_3P_3-P_3X_3=i\hbar
\end{equation*}

and
\begin{equation*}
X_jP_k-P_kX_j=0,\quad j\ne k
\end{equation*}
\fi

\end{document}
