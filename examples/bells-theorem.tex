\input{preamble}

\section*{Bell's theorem}

The following theorem of independent random variables
is the key to understanding Bell's theorem.
If two random variables $A$ and $B$ are independent (uncorrelated) then
\begin{equation*}
\langle A\rangle\langle B\rangle=\langle AB\rangle
\end{equation*}

Consider two machines $A$ and $B$ that measure spin.
Each machine can be set in one of two orientations labeled 0 and 1.
Assuming the measurements are uncorrelated we have the following table
of expectation values and a clever formula.
\begin{equation*}
\begin{matrix}
\phantom{-}\langle A_0\rangle & \phantom{-}\langle A_1\rangle
& \phantom{-}\langle B_0\rangle & \phantom{-}\langle B_1\rangle
& \phantom{-}\langle A_0B_0\rangle+\langle A_0B_1\rangle+\langle A_1B_0\rangle-\langle A_1B_1\rangle
\\
\phantom{-}1 & \phantom{-}1 & \phantom{-}1 & \phantom{-}1 & \phantom{-}2
\\
\phantom{-}1 & \phantom{-}1 & \phantom{-}1 &           -1 & \phantom{-}2
\\
\phantom{-}1 & \phantom{-}1 &           -1 & \phantom{-}1 &           -2
\\
\phantom{-}1 & \phantom{-}1 &           -1 &           -1 &           -2
\\
\phantom{-}1 &          -1 & \phantom{-}1 & \phantom{-}1 & \phantom{-}2
\\
\phantom{-}1 &          -1 & \phantom{-}1 &           -1 &           -2
\\
\phantom{-}1 &          -1 &           -1 & \phantom{-}1 & \phantom{-}2
\\
\phantom{-}1 &          -1 &           -1 &           -1 &           -2
\\
          -1 & \phantom{-}1 & \phantom{-}1 & \phantom{-}1 &           -2
\\
          -1 & \phantom{-}1 & \phantom{-}1 &           -1 & \phantom{-}2
\\
          -1 & \phantom{-}1 &           -1 & \phantom{-}1 &           -2
\\
          -1 & \phantom{-}1 &           -1 &           -1 & \phantom{-}2
\\
          -1 &           -1 & \phantom{-}1 & \phantom{-}1 &           -2
\\
          -1 &           -1 & \phantom{-}1 &           -1 &           -2
\\
          -1 &           -1 &           -1 & \phantom{-}1 & \phantom{-}2
\\
          -1 &           -1 &           -1 &           -1 & \phantom{-}2
\end{matrix}
\end{equation*}

Since the table is for all minimum and maximum values we have by inspection
\begin{equation*}
-2\le \langle A_0B_0\rangle+\langle A_0B_1\rangle+\langle A_1B_0\rangle-\langle A_1B_1\rangle\le2
\tag{1}
\end{equation*}

Now suppose a third machine generates two spins in the following entangled state.
\begin{equation*}
|s\rangle=\frac{1}{\sqrt2}\begin{pmatrix}0\\1\\-1\\0\end{pmatrix}
\end{equation*}

One spin is sent to $A$ and the other is sent to $B$.

\bigskip
Let
\begin{equation*}
A_0=\sigma_z,\quad
A_1=\sigma_x,\quad
B_0=-\frac{\sigma_x+\sigma_z}{\sqrt2},\quad
B_1=\frac{\sigma_x-\sigma_z}{\sqrt2}
\end{equation*}

Then for the entangled state $|s\rangle$ we have
\begin{equation*}
\langle A_0B_0\rangle=\frac{1}{\sqrt2},\quad
\langle A_0B_1\rangle=\frac{1}{\sqrt2},\quad
\langle A_1B_0\rangle=\frac{1}{\sqrt2},\quad
\langle A_1B_1\rangle=-\frac{1}{\sqrt2}
\end{equation*}

Hence
\begin{equation*}
\langle A_0B_0\rangle+
\langle A_0B_1\rangle+
\langle A_1B_0\rangle-
\langle A_1B_1\rangle=2\sqrt2
\tag{2}
\end{equation*}

The result in (2) conflicts with (1) because for an entangled state
the random variables are not independent.
Any theory that asserts $A$ and $B$ are independent for all states
is constrained by (1) and falsified by (2).
Hence no hidden variable theory can explain entanglement.

\subsubsection*{Exercises}

1. Verify equation (2).

\bigskip
2. Verify that for the singlet state $|s\rangle$ given above we have
\begin{equation*}
\langle A_0\rangle=0,\quad\langle A_1\rangle=0,\quad\langle B_0\rangle=0,\quad\langle B_1\rangle=0.
\end{equation*}
Hence $\langle A\rangle\langle B\rangle\ne\langle AB\rangle$ for the singlet state.

\bigskip
3. There are three additional entangled states.
\begin{equation*}
|s_1\rangle=\frac{1}{\sqrt2}\begin{pmatrix}1\\0\\0\\1\end{pmatrix},\quad
|s_2\rangle=\frac{1}{\sqrt2}\begin{pmatrix}1\\0\\0\\-1\end{pmatrix},\quad
|s_3\rangle=\frac{1}{\sqrt2}\begin{pmatrix}0\\1\\1\\0\end{pmatrix}
\end{equation*}

Verify that $A$ and $B$ are correlated for all entangled states.

\end{document}
