\input{preamble}

\section*{Bell's theorem}

Consider two observers $A$ and $B$.
Each observer has an apparatus for measuring spin.
Each apparatus can be set in one of two orientations labeled 0 and 1.
For independent random variables $A$ and $B$ we have
\begin{equation*}
\langle A\rangle\langle B\rangle=\langle AB\rangle
\end{equation*}

Now consider all minimum and maximum expectation values along with a special formula.
\begin{equation*}
\begin{matrix}
\phantom{-}\langle A_0\rangle & \phantom{-}\langle A_1\rangle
& \phantom{-}\langle B_0\rangle & \phantom{-}\langle B_1\rangle
& \phantom{-}\langle A_0B_0\rangle+\langle A_0B_1\rangle+\langle A_1B_0\rangle-\langle A_1B_1\rangle
\\
\phantom{-}1 & \phantom{-}1 & \phantom{-}1 & \phantom{-}1 & \phantom{-}2
\\
\phantom{-}1 & \phantom{-}1 & \phantom{-}1 &           -1 & \phantom{-}2
\\
\phantom{-}1 & \phantom{-}1 &           -1 & \phantom{-}1 &           -2
\\
\phantom{-}1 & \phantom{-}1 &           -1 &           -1 &           -2
\\
\phantom{-}1 &          -1 & \phantom{-}1 & \phantom{-}1 & \phantom{-}2
\\
\phantom{-}1 &          -1 & \phantom{-}1 &           -1 &           -2
\\
\phantom{-}1 &          -1 &           -1 & \phantom{-}1 & \phantom{-}2
\\
\phantom{-}1 &          -1 &           -1 &           -1 &           -2
\\
          -1 & \phantom{-}1 & \phantom{-}1 & \phantom{-}1 &           -2
\\
          -1 & \phantom{-}1 & \phantom{-}1 &           -1 & \phantom{-}2
\\
          -1 & \phantom{-}1 &           -1 & \phantom{-}1 &           -2
\\
          -1 & \phantom{-}1 &           -1 &           -1 & \phantom{-}2
\\
          -1 &           -1 & \phantom{-}1 & \phantom{-}1 &           -2
\\
          -1 &           -1 & \phantom{-}1 &           -1 &           -2
\\
          -1 &           -1 &           -1 & \phantom{-}1 & \phantom{-}2
\\
          -1 &           -1 &           -1 &           -1 & \phantom{-}2
\end{matrix}
\end{equation*}

Since the table is for all minimum and maximum values we have by inspection the range
\begin{equation*}
-2\le \langle A_0B_0\rangle+\langle A_0B_1\rangle+\langle A_1B_0\rangle-\langle A_1B_1\rangle\le2
\tag{1}
\end{equation*}

Now suppose a third apparatus generates two spins in the following singlet state.
\begin{equation*}
|s\rangle=\frac{1}{\sqrt2}\begin{pmatrix}0\\1\\-1\\0\end{pmatrix}
\end{equation*}

One spin is sent to $A$ and the other is sent to $B$.

\bigskip
Let
\begin{equation*}
A_0=\sigma_z,\quad
A_1=\sigma_x,\quad
B_0=-\frac{\sigma_x+\sigma_z}{\sqrt2},\quad
B_1=\frac{\sigma_x-\sigma_z}{\sqrt2}
\end{equation*}

Then for the singlet state we have
\begin{equation*}
\langle A_0B_0\rangle=\frac{1}{\sqrt2},\quad
\langle A_0B_1\rangle=\frac{1}{\sqrt2},\quad
\langle A_1B_0\rangle=\frac{1}{\sqrt2},\quad
\langle A_1B_1\rangle=-\frac{1}{\sqrt2}
\end{equation*}

Hence
\begin{equation*}
\langle A_0B_0\rangle+
\langle A_0B_1\rangle+
\langle A_1B_0\rangle-
\langle A_1B_1\rangle=2\sqrt2
\tag{2}
\end{equation*}

The result in (2) conflicts with (1) because for the singlet state
the random variables $A$ and $B$ are not independent.
Any theory that asserts independence of $A$ and $B$ (for example, a hidden variable theory)
is constrained by (1) and falsified by (2).

\subsubsection*{Exercises}

1. Verify equation (2).

\bigskip
2. Verify that for the singlet state
\begin{equation*}
\langle A_0\rangle=0,\quad\langle A_1\rangle=0,\quad\langle B_0\rangle=0,\quad\langle B_1\rangle=0
\end{equation*}
Hence $\langle A\rangle\langle B\rangle\ne\langle AB\rangle$ for the singlet state.
\end{document}
