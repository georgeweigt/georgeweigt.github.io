\input{preamble}

\section*{Klein-Nishina formula}

The Klein-Nishina formula is the differential cross section for photon-electron scattering.

\begin{center}
\begin{tikzpicture}
\draw[dashed] (0,0) circle (0.5cm);
\draw[thick,->] (2,0) node[anchor=west] {$e^-$} -- (0.6,0);
\draw[thick,->] (-2,0) node[anchor=east] {$\gamma$} -- (-0.6,0);
\draw[thick,->] (0.40,0.40) -- (1.3,1.3) node[anchor=south west] {$\gamma$};
\draw[thick,->] (-0.4,-0.4) -- (-1.3,-1.3) node[anchor=north east] {$e^-$};
\draw (1,0.5) node {$\theta$};
\end{tikzpicture}
\end{center}

It is easy to derive the Klein-Nishina formula from Dirac's equation by starting
out in the center-of-mass frame and then boosting to the lab frame.
In the center-of-mass frame we have the following momentum vectors where $E=\sqrt{\omega^2+m^2}$.
\begin{equation*}
p_1=\underset{\substack{\text{inbound}\\ \text{photon}}}
{\begin{pmatrix}\omega\\0\\0\\ \omega\end{pmatrix}}
\qquad
p_2=\underset{\substack{\text{inbound}\\ \text{electron}}}
{\begin{pmatrix}E\\0\\0\\-\omega\end{pmatrix}}
\qquad
p_3=\underset{\substack{\text{outbound}\\ \text{photon}}}
{\begin{pmatrix}
\omega\\
\omega\sin\theta\cos\phi\\
\omega\sin\theta\sin\phi\\
\omega\cos\theta
\end{pmatrix}}
\qquad
p_4=\underset{\substack{\text{outbound}\\ \text{electron}}}
{\begin{pmatrix}
E\\
-\omega\sin\theta\cos\phi\\
-\omega\sin\theta\sin\phi\\
-\omega\cos\theta
\end{pmatrix}}
\end{equation*}

Spinors for the inbound electron.
\begin{equation*}
u_{21}=\frac{1}{\sqrt{E+m}}
\underset{\substack{\text{inbound electron}\\ \text{spin up}}}
{\begin{pmatrix}
E+m\\
0\\
-\omega\\
0
\end{pmatrix}}
\qquad
u_{22}=\frac{1}{\sqrt{E+m}}
\underset{\substack{\text{inbound electron}\\ \text{spin down}}}
{\begin{pmatrix}
0\\
E+m\\
0\\
\omega
\end{pmatrix}}
\end{equation*}

Spinors for the outbound electron.
\begin{equation*}
u_{41}=\frac{1}{\sqrt{E+m}}
\underset{\substack{\text{outbound electron}\\ \text{spin up}}}
{\begin{pmatrix}
E+m\\
0\\
p_{4z}\\
p_{4x}+ip_{4y}
\end{pmatrix}}
\qquad
u_{42}=\frac{1}{\sqrt{E+m}}
\underset{\substack{\text{outbound electron}\\ \text{spin down}}}
{\begin{pmatrix}
0\\
E+m\\
p_{4x}-ip_{4y}\\
-p_{4z}
\end{pmatrix}}
\end{equation*}

The scattering amplitude $\mathcal M_{ab}{}^{\mu\nu}$
for spin $ab$ and polarization $\mu\nu$ is
\begin{equation*}
\mathcal M_{ab}{}^{\mu\nu}=\mathcal M_{1ab}{}^{\mu\nu}+\mathcal M_{2ab}{}^{\nu\mu}
\end{equation*}

where
\begin{align*}
\mathcal M_{1ab}{}^{\mu\nu}
&=\frac{\bar{u}_{4b}(-ie\gamma^\mu)(\slashed{q}_1+m)(-ie\gamma^\nu)u_{2a}}{s-m^2}
\\
\mathcal M_{2ab}{}^{\nu\mu}
&=\frac{\bar{u}_{4b}(-ie\gamma^\nu)(\slashed{q}_2+m)(-ie\gamma^\mu)u_{2a}}{u-m^2}
\end{align*}

Matrices $\slashed q_1$ and $\slashed q_2$ represent momentum transfer.
\begin{align*}
\slashed q_1&=(p_1+p_2)^\alpha g_{\alpha\beta}\gamma^\beta
\\
\slashed q_2&=(p_4-p_1)^\alpha g_{\alpha\beta}\gamma^\beta
\end{align*}

Scalars $s$ and $u$ are Mandelstam variables.
\begin{align*}
s&=(p_1+p_2)^2
\\
u&=(p_1-p_4)^2
\end{align*}

In component form (note that indices $\mu$ and $\nu$ are interchanged for $\mathcal M_{2ab}$)
\begin{align*}
\mathcal M_{1ab}{}^{\mu\nu}=\frac{
(\bar{u}_{4b})_\alpha
(-ie\gamma^{\mu\alpha}{}_\beta)
(\slashed{q}_1+m)^\beta{}_\rho
(-ie\gamma^{\nu\rho}{}_\sigma)
(u_{2a})^\sigma}{s-m^2}
\\
\mathcal M_{2ab}{}^{\nu\mu}=\frac{
(\bar{u}_{4b})_\alpha
(-ie\gamma^{\nu\alpha}{}_\beta)
(\slashed{q}_2+m)^\beta{}_\rho
(-ie\gamma^{\mu\rho}{}_\sigma)
(u_{2a})^\sigma}{u-m^2}
\end{align*}

The expected probability density $\langle|\mathcal M|^2\rangle$
is the average over spin and polarization states.
\begin{equation*}
\langle|\mathcal M|^2\rangle=\frac{1}{4}\sum_{a,b}\sum_{\mu,\nu}
\bigl|\mathcal M_{ab}{}^{\mu\nu}\bigr|^2
\end{equation*}

Summing over $\mu$ and $\nu$ requires $g_{\mu\nu}$ to lower indices.
\begin{equation*}
\langle|\mathcal M|^2\rangle
=\frac{1}{4}\sum_{a,b}\mathcal M_{ab}{}^{\mu\nu}
\left(g_{\mu\alpha}\mathcal M_{ab}{}^{\alpha\beta}g_{\beta\nu}\right)^*
\end{equation*}

Substitute $\mathcal M_{1ab}+\mathcal M_{2ab}$ for $\mathcal M_{ab}$.
(Note that $P_{12}=P_{21}^*$ hence by the property that probabilities are real we have $P_{12}=P_{21}$.)
\begin{multline*}
\langle|\mathcal{M}|^2\rangle
=\frac{1}{4}\sum_{a,b}\biggl[
\underbrace{
\mathcal M_{1ab}{}^{\mu\nu}\left(g_{\mu\alpha}\mathcal M_{1ab}{}^{\alpha\beta}g_{\beta\nu}\right)^*
}_{P_{11ab}}
+
\underbrace{
\mathcal M_{1ab}{}^{\mu\nu}\left(g_{\nu\alpha}\mathcal M_{2ab}{}^{\alpha\beta}g_{\beta\mu}\right)^*
}_{P_{12ab}}
\\
+
\underbrace{
\mathcal M_{2ab}{}^{\nu\mu}\left(g_{\mu\alpha}\mathcal M_{1ab}{}^{\alpha\beta}g_{\beta\nu}\right)^*
}_{P_{21ab}}
+
\underbrace{
\mathcal M_{2ab}{}^{\nu\mu}\left(g_{\nu\alpha}\mathcal M_{2ab}{}^{\alpha\beta}g_{\beta\mu}\right)^*
}_{P_{22ab}}
\biggr]
\end{multline*}

The Casimir trick uses matrix arithmetic to sum over spin and polarization states:
\begin{align*}
\sum_{a,b}P_{11ab}&=\frac{e^4}{(s-m^2)^2}\operatorname{Tr}
\left[
(\slashed p_2+m)\gamma^\mu(\slashed q_1+m)\gamma^\nu(\slashed p_4+m)\gamma_\nu(\slashed q_1+m)\gamma_\mu
\right]
\\
\sum_{a,b}P_{12ab}&=\frac{e^4}{(s-m^2)(u-m^2)}\operatorname{Tr}
\left[
(\slashed p_2+m)\gamma^\mu(\slashed q_2+m)\gamma^\nu(\slashed p_4+m)\gamma_\mu(\slashed q_1+m)\gamma_\nu
\right]
\\
\sum_{a,b}P_{22ab}&=\frac{e^4}{(u-m^2)^2}\operatorname{Tr}
\left[
(\slashed p_2+m)\gamma^\mu(\slashed q_2+m)\gamma^\nu(\slashed p_4+m)\gamma_\nu(\slashed q_2+m)\gamma_\mu
\right]
\end{align*}

Let
\begin{align*}
f_{11}&=\operatorname{Tr}
\left[
(\slashed p_2+m)\gamma^\mu(\slashed q_1+m)\gamma^\nu(\slashed p_4+m)\gamma_\nu(\slashed q_1+m)\gamma_\mu
\right]
\\
f_{12}&=\operatorname{Tr}
\left[
(\slashed p_2+m)\gamma^\mu(\slashed q_2+m)\gamma^\nu(\slashed p_4+m)\gamma_\mu(\slashed q_1+m)\gamma_\nu
\right]
\\
f_{22}&=\operatorname{Tr}
\left[
(\slashed p_2+m)\gamma^\mu(\slashed q_2+m)\gamma^\nu(\slashed p_4+m)\gamma_\nu(\slashed q_2+m)\gamma_\mu
\right]
\end{align*}

so that
\begin{equation*}
\langle|\mathcal{M}|^2\rangle=\frac{e^4}{4}
\left[
\frac{f_{11}}{(s-m^2)^2}+\frac{2f_{12}}{(s-m^2)(u-m^2)}+\frac{f_{22}}{(u-m^2)^2}
\right]
\end{equation*}

The following formulas are equivalent to the Casimir trick.
(Recall that $a\cdot b=a^\mu g_{\mu\nu}b^\nu$.)
\begin{equation*}
\begin{aligned}
f_{11}&=32(p_1\cdot p_2)(p_1\cdot p_4)+32(p_1\cdot p_2)m^2+32m^4
\\
f_{12}&=16(p_1\cdot p_2)m^2-16(p_1\cdot p_4)m^2+32m^4
\\
f_{22}&=32(p_1\cdot p_2)(p_1\cdot p_4)-32(p_1\cdot p_4)m^2+32m^4
\end{aligned}
\tag{2}
\end{equation*}

In Mandelstam variables
\begin{equation*}
\begin{aligned}
f_{11}&=-8 s u + 24 s m^2 + 8 u m^2 + 8 m^4
\\
f_{12}&=8 s m^2 + 8 u m^2 + 16 m^4
\\
f_{22}&=-8 s u + 8 s m^2 + 24 u m^2 + 8 m^4
\end{aligned}
\tag{3}
\end{equation*}

Scattering experiments are typically done in the lab frame.
Define Lorentz boost $\Lambda$ for transforming momentum vectors to the lab frame.
\begin{equation*}
\Lambda=
\begin{pmatrix}
E/m & 0 & 0 & \omega/m\\
0 & 1 & 0 & 0\\
0 & 0 & 1 & 0\\
\omega/m & 0 & 0 & E/m
\end{pmatrix}
\end{equation*}

The electron is at rest in the lab frame.
\begin{equation*}
\Lambda p_2=\begin{pmatrix}m\\0\\0\\0\end{pmatrix}
\end{equation*}

Mandelstam variables are invariant under a boost.
\begin{equation*}
\begin{aligned}
s&=(p_1+p_2)^2=(\Lambda p_1+\Lambda p_2)^2
\\
t&=(p_1-p_3)^2=(\Lambda p_1-\Lambda p_3)^2
\\
u&=(p_1-p_4)^2=(\Lambda p_1-\Lambda p_4)^2
\end{aligned}
\tag{4}
\end{equation*}

In the lab frame, let $\omega_L$ be the angular frequency of the incident photon
and let $\omega_L'$ be the angular frequency of the scattered photon.
\begin{equation*}
\begin{aligned}
\omega_L&=\Lambda p_1\cdot
\begin{pmatrix}1\\0\\0\\0\end{pmatrix}
=\frac{\omega^2}{m}+\frac{\omega E}{m}
\\[1ex]
\omega_L'&=\Lambda p_3\cdot
\begin{pmatrix}1\\0\\0\\0\end{pmatrix}
=\frac{\omega^2\cos\theta}{m}+\frac{\omega E}{m}
\end{aligned}
\end{equation*}

It can be shown that
\begin{equation*}
\begin{aligned}
s&=m^2+2m\omega_L
\\
t&=2m(\omega_L' - \omega_L)
\\
u&=m^2-2 m \omega_L'
\end{aligned}
\tag{5}
\end{equation*}

Then by (1), (3), and (5) we have
\begin{equation*}
\langle|\mathcal{M}|^2\rangle=
2e^4\left(
\frac{\omega_L}{\omega_L'}+\frac{\omega_L'}{\omega_L}
+\left(\frac{m}{\omega_L}-\frac{m}{\omega_L'}+1\right)^2-1
\right)
\tag{6}
\end{equation*}

Lab scattering angle $\theta_L$ is given by the Compton equation
\begin{equation*}
\cos\theta_L=\frac{m}{\omega_L}-\frac{m}{\omega_L'}+1
\end{equation*}

Hence
\begin{align*}
\langle|\mathcal{M}|^2\rangle
&=2e^4\left(
\frac{\omega_L}{\omega_L'}+\frac{\omega_L'}{\omega_L}+\cos^2\theta_L-1
\right)
\\
&=2e^4\left(
\frac{\omega_L}{\omega_L'}+\frac{\omega_L'}{\omega_L}-\sin^2\theta_L
\right)
\end{align*}

Now that we have derived $\langle|\mathcal{M}|^2\rangle$
we can investigate the angular distribution of scattered photons.
For simplicity let us drop the $L$ subscript from lab variables.
From now on the symbols $\omega$, $\omega'$, and $\theta$ will be lab frame variables.

\bigskip
The differential cross section is
\begin{equation*}
\frac{d\sigma}{d\Omega}=\frac{1}{4(4\pi\varepsilon_0)^2s}
\left(\frac{\omega'}{\omega}\right)^2\langle|\mathcal{M}|^2\rangle
\end{equation*}

where
\begin{equation*}
s=m^2+2m\omega=(mc^2)^2+2(mc^2)(\hbar\omega)
\end{equation*}

and $\omega'$ is given by the Compton equation
\begin{equation*}
\omega'=\frac{\omega}{1+\frac{\hbar\omega}{mc^2}(1-\cos\theta)}
\end{equation*}

For the lab frame we have
\begin{equation*}
\langle|\mathcal{M}|^2\rangle
=2e^4\left(
\frac{\omega}{\omega'}+\frac{\omega'}{\omega}-\sin^2\theta
\right)
\end{equation*}

Hence in the lab frame
\begin{equation*}
\frac{d\sigma}{d\Omega}
=\frac{e^4}{2(4\pi\varepsilon_0)^2s}
\left(\frac{\omega'}{\omega}\right)^2
\left(
\frac{\omega}{\omega'}+\frac{\omega'}{\omega}-\sin^2\theta
\right)
\end{equation*}

Substituting
\begin{equation*}
e^2=4\pi\varepsilon_0\alpha\hbar c
\end{equation*}

we have
\begin{equation*}
\frac{d\sigma}{d\Omega}
=\frac{\alpha^2(\hbar c)^2}{2s}
\left(\frac{\omega'}{\omega}\right)^2
\left(
\frac{\omega}{\omega'}+\frac{\omega'}{\omega}-\sin^2\theta
\right)
\end{equation*}

which is the Klein-Nishina formula.

\iffalse

Noting that
\begin{equation*}
d\Omega=\sin\theta\,d\theta\,d\phi
\end{equation*}

we also have
\begin{equation*}
d\sigma
=\frac{\alpha^2(\hbar c)^2}{2s}
\left(\frac{\omega'}{\omega}\right)^2
\left(
\frac{\omega}{\omega'}+\frac{\omega'}{\omega}-\sin^2\theta
\right)
\sin\theta\,d\theta\,d\phi
\end{equation*}

Let $S(\theta_1,\theta_2)$ be the following integral of $d\sigma$.
\begin{equation*}
S(\theta_1,\theta_2)=\int_0^{2\pi}\int_{\theta_1}^{\theta_2}d\sigma
\end{equation*}

The solution is
\begin{equation*}
S(\theta_1,\theta_2)=\frac{\pi\alpha^2(\hbar c)^2}{s}[I(\theta_2)-I(\theta_1)]
\end{equation*}

where
\begin{multline*}
I(\theta)=-\frac{\cos\theta}{R^2}
+\log\bigl(1+R(1-\cos\theta)\bigr)\left(\frac{1}{R}-\frac{2}{R^2}-\frac{2}{R^3}\right)
\\
{}-\frac{1}{2R\bigl(1+R(1-\cos\theta)\bigr)^2}
+\frac{1}{1+R(1-\cos\theta)}\left(-\frac{2}{R^2}-\frac{1}{R^3}\right)
\end{multline*}

and
\begin{equation*}
R=\frac{\hbar\omega}{mc^2}
\end{equation*}

The cumulative distribution function is
\begin{equation*}
F(\theta)
=\frac{S(0,\theta)}{S(0,\pi)}
=\frac{I(\theta)-I(0)}{I(\pi)-I(0)},
\quad
0\le\theta\le\pi
\end{equation*}

The probability of observing scattering events in the interval $\theta_1$ to $\theta_2$ is
\begin{equation*}
P(\theta_1<\theta\le\theta_2)=F(\theta_2)-F(\theta_1)
\end{equation*}

The probability density function is
\begin{equation*}
f(\theta)=\frac{dF(\theta)}{d\theta}
=\frac{1}{I(\pi)-I(0)}
\left(\frac{\omega'}{\omega}\right)^2
\left(\frac{\omega}{\omega'}+\frac{\omega'}{\omega}-\sin^2\theta\right)
\sin\theta
\end{equation*}

\subsubsection*{Thomson scattering}

For $\hbar\omega\ll mc^2$ we have
\begin{equation*}
\omega'=\frac{\omega}{1+\frac{\hbar\omega}{mc^2}\,(1-\cos\theta)}\approx\omega
\end{equation*}

Hence we can use the approximations
\begin{equation*}
\omega=\omega'\quad\text{and}\quad s=(mc^2)^2
\end{equation*}

to obtain
\begin{equation*}
\frac{d\sigma}{d\Omega}=\frac{\alpha^2\hbar^2}{2m^2c^2}\left(1+\cos^2\theta\right)
\end{equation*}

which is the formula for Thomson scattering.

\subsubsection*{High energy approximation}

For $\omega\gg m$ a useful approximation is to set $m=0$ and obtain
\begin{align*}
f_{11}&=-8su
\\
f_{12}&=0
\\
f_{22}&=-8su
\end{align*}

Hence
\begin{align*}
\langle|\mathcal{M}|^2\rangle
&=\frac{e^4}{4}
\left(\frac{-8su}{s^2}+\frac{-8su}{u^2}\right)
\\
&=2e^4
\left(-\frac{u}{s}-\frac{s}{u}\right)
\end{align*}

The Mandelstam variables for $m=0$ are
\begin{align*}
s&=4\omega^2
\\
u&=-2\omega^2(\cos\theta+1)
\end{align*}

Hence
\begin{equation*}
\langle|\mathcal{M}|^2\rangle
=2e^4\left(
\frac{\cos\theta+1}{2}+\frac{2}{\cos\theta+1}
\right)
\end{equation*}

In the center of mass frame
\begin{equation*}
\frac{d\sigma}{d\Omega}=\frac{\langle|\mathcal{M}|^2\rangle}{4s(4\pi\epsilon_o)^2}
=\frac{e^4}{2s(4\pi\epsilon_o)^2}\left(\frac{\cos\theta+1}{2}+\frac{2}{\cos\theta+1}\right)
\end{equation*}

Substitute $e^4=(4\pi\varepsilon_0\alpha\hbar c)^2$ to obtain
\begin{equation*}
\frac{d\sigma}{d\Omega}=\frac{\alpha^2}{2s}
\left(\frac{\cos\theta+1}{2}+\frac{2}{\cos\theta+1}\right)\times(\hbar c)^2
\end{equation*}

It follows that
\begin{equation*}
\frac{d\sigma}{d\cos\theta}=2\pi\frac{d\sigma}{d\Omega}
=\frac{\pi\alpha^2}{s}
\left(\frac{\cos\theta+1}{2}+\frac{2}{\cos\theta+1}\right)\times(\hbar c)^2
\end{equation*}

Cf.~equation (1) of arxiv.org/pdf/hep-ex/0504012.

\fi

\end{document}
