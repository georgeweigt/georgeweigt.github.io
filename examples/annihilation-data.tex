\input{preamble}

\section*{Annihilation data}

See www.hepdata.net/record/ins191231, Table 2, 14.0 GeV.
\begin{equation*}
\begin{matrix}
x & y\\
0.0502 & 0.09983\\
0.1505 & 0.10791\\
0.2509 & 0.12026\\
0.3512 & 0.13002\\
0.4516 & 0.17681\\
0.5521 & 0.19570\\
0.6526 & 0.27900\\
0.7312 & 0.33204
\end{matrix}
\end{equation*}

Data $x$ and $y$ have the following relationship
with the differential cross section formula.
\begin{equation*}
x=\cos\theta,
\quad
y=\frac{d\sigma}{d\Omega}
\end{equation*}

The cross section formula is
\begin{equation*}
\frac{d\sigma}{d\Omega}
=
\frac{\alpha^2}{2s}
\left(
\frac{1+\cos\theta}{1-\cos\theta}+
\frac{1-\cos\theta}{1+\cos\theta}
\right)\times(\hbar c)^2
\end{equation*}

To compute predicted values $\hat{y}$,
multiply by $10^{37}$ to convert square meters to nanobarns.
\begin{equation*}
\hat{y}
=
\frac{\alpha^2}{2s}
\left(
\frac{1+x}{1-x}+
\frac{1-x}{1+x}
\right)
\times(\hbar c)^2
\times10^{37}
\end{equation*}

The following table shows predicted values $\hat y$ for $s=(14.0\,\text{GeV})^2$.
\begin{equation*}
\begin{matrix}
x & y & \hat y\\
0.0502 & 0.09983 & 0.106325\\
0.1505 & 0.10791 & 0.110694\\
0.2509 & 0.12026 & 0.120005\\
0.3512 & 0.13002 & 0.135559\\
0.4516 & 0.17681 & 0.159996\\
0.5521 & 0.19570 & 0.198562\\
0.6526 & 0.27900 & 0.262745\\
0.7312 & 0.33204 & 0.348884\\
\end{matrix}
\end{equation*}

The coefficient of determination $R^2$ measures how well predicted values fit the data.
\begin{equation*}
R^2=1-\frac{\sum(y-\hat{y})^2}{\sum(y-\bar y)^2}=0.98
\end{equation*}

The result indicates that the model $d\sigma$ explains 98\% of the variance in the data.

\end{document}
