\input{preamble}

\section*{Schrodinger for charged particle}

Derive the Schrodinger equation for a charged particle
\begin{equation*}
i\hbar\frac{\partial\psi}{\partial t}
=\frac{1}{2m}\left(\frac{\hbar}{i}\nabla-\frac{q}{c}\mathbf A\right)^2\psi+q\phi\psi
\end{equation*}

from the Lagrangian
\begin{equation*}
L(\mathbf x,\dot{\mathbf x},t)=\frac{m\dot{\mathbf x}^2}{2}
+\frac{q}{c}\dot{\mathbf x}\cdot\mathbf A(\mathbf x,t)-q\phi(\mathbf x,t)
\end{equation*}

Start with the path integral for an action $S$.
\begin{equation*}
\psi(\mathbf x_b,t_b)
=C\int_{\mathbb R^3}\exp\left(\frac{i}{\hbar}S(b,a)\right)
\psi(\mathbf x_a,t_a)\,d\mathbf x_a,\quad
\int_{\mathbb R^3}\equiv\int\limits_{-\infty}^\infty\int\limits_{-\infty}^\infty\int\limits_{-\infty}^\infty
\end{equation*}

For a small time interval $\epsilon=t_b-t_a$ we can use the approximation
\begin{equation*}
S=\epsilon L
\end{equation*}

and write the path integral as
\begin{equation*}
\psi(\mathbf x_b,t+\epsilon)=C\int_{\mathbb R^3}
\exp\left[\frac{i}{\hbar}
\epsilon L\left(\frac{\mathbf x_b-\mathbf x_a}{\epsilon},\frac{\mathbf x_b+\mathbf x_a}{2},t\right)
\right]
\psi(\mathbf x_a,t)\,d\mathbf x_a
\end{equation*}

Substitute for $L$.
\begin{multline*}
\psi(\mathbf x_b,t+\epsilon)=C\int_{\mathbb R^3}
\exp\biggl[\frac{im(\mathbf x_b-\mathbf x_a)^2}{2\hbar\epsilon}
+\frac{iq}{\hbar c}(\mathbf x_b-\mathbf x_a)
\cdot\mathbf A\left(\frac{\mathbf x_b+\mathbf x_a}{2},t\right)
\\
{}-\frac{iq\epsilon}{\hbar}\phi\left(\frac{\mathbf x_b+\mathbf x_a}{2},t\right)
\biggr]
\psi(\mathbf x_a,t)
\,d\mathbf x_a
\end{multline*}

Let
\begin{equation*}
\mathbf x_a=\mathbf x_b+\boldsymbol\eta,\quad
d\mathbf x_a=d\boldsymbol\eta
\end{equation*}

and write
\begin{equation*}
\psi(\mathbf x_b,t+\epsilon)
=C\int_{\mathbb R^3}
\exp\left[\frac{im\boldsymbol\eta^2}{2\hbar\epsilon}
-\frac{iq}{\hbar c}
\boldsymbol\eta\cdot\mathbf A\left(\mathbf x_b+\frac{\boldsymbol\eta}{2},t\right)
-\frac{iq\epsilon}{\hbar}\phi\left(\mathbf x_b+\frac{\boldsymbol\eta}{2},t\right)
\right]
\psi(\mathbf x_b+\boldsymbol\eta,t)
\,d\boldsymbol\eta
\end{equation*}

Substitute $\mathbf x$ for $\mathbf x_b$.
\begin{equation*}
\psi(\mathbf x,t+\epsilon)
=C\int_{\mathbb R^3}
\exp\left[\frac{im\boldsymbol\eta^2}{2\hbar\epsilon}
-\frac{iq}{\hbar c}
\boldsymbol\eta\cdot\mathbf A\left(\mathbf x+\frac{\boldsymbol\eta}{2},t\right)
-\frac{iq\epsilon}{\hbar}\phi\left(\mathbf x+\frac{\boldsymbol\eta}{2},t\right)
\right]
\psi(\mathbf x+\boldsymbol\eta,t)
\,d\boldsymbol\eta
\end{equation*}

Because the exponential is highly oscillatory for large $|\boldsymbol\eta|$,
most of the contribution to the integral is from small $|\boldsymbol\eta|$.
Hence use the approximation $\mathbf x+\tfrac{1}{2}\boldsymbol\eta\approx\mathbf x$
for small $|\boldsymbol\eta|$.
\begin{equation*}
\psi(\mathbf x,t+\epsilon)
=C\int_{\mathbb R^3}
\exp\left(\frac{im\boldsymbol\eta^2}{2\hbar\epsilon}
-\frac{iq}{\hbar c}
\boldsymbol\eta\cdot\mathbf A(\mathbf x,t)
-\frac{iq\epsilon}{\hbar}\phi\left(\mathbf x,t\right)
\right)
\psi(\mathbf x+\boldsymbol\eta,t)
\,d\boldsymbol\eta
\end{equation*}

Use the approximation $\exp(y)\approx1+y$ for the exponential of $\phi$.
\begin{equation*}
\psi(\mathbf x,t+\epsilon)
=C\left(1-\frac{iq\epsilon}{\hbar}\phi(\mathbf x,t)\right)\int_{\mathbb R^3}
\exp\left(\frac{im\boldsymbol\eta^2}{2\hbar\epsilon}
-\frac{iq}{\hbar c}
\boldsymbol\eta\cdot\mathbf A(\mathbf x,t)
\right)
\psi(\mathbf x+\boldsymbol\eta,t)
\,d\boldsymbol\eta
\end{equation*}

Expand $\psi(\mathbf x+\boldsymbol\eta,t)$ as the power series
\begin{equation*}
\psi(\mathbf x+\boldsymbol\eta,t)\approx
\psi(\mathbf x,t)
+\boldsymbol\eta\cdot\nabla\psi
+\frac{1}{2}\boldsymbol\eta\cdot\nabla(\boldsymbol\eta\cdot\nabla\psi)
\end{equation*}

to obtain
\begin{multline*}
\psi(\mathbf x,t+\epsilon)
=C\left(1-\frac{iq\epsilon}{\hbar}\phi(\mathbf x,t)\right)\int_{\mathbb R^3}
\exp\left(\frac{im\boldsymbol\eta^2}{2\hbar\epsilon}
-\frac{iq}{\hbar c}\boldsymbol\eta\cdot\mathbf A(\mathbf x,t)\right)
\\
{}\times
\left(
\psi(\mathbf x,t)
+\boldsymbol\eta\cdot\nabla\psi
+\frac{1}{2}\boldsymbol\eta\cdot\nabla(\boldsymbol\eta\cdot\nabla\psi)
\right)
\,d\boldsymbol\eta
\end{multline*}

Rewrite as
\begin{equation*}
\psi(\mathbf x,t+\epsilon)
=C\left(1-\frac{iq\epsilon}{\hbar}\phi(\mathbf x,t)\right)\sum_kI_k
\end{equation*}

where
\begin{align*}
I_1&=\int_{\mathbb R^3}
\exp\left(\frac{im\boldsymbol\eta^2}{2\hbar\epsilon}
-\frac{iq}{\hbar c}\boldsymbol\eta\cdot\mathbf A(\mathbf x,t)\right)
\psi\,d\boldsymbol\eta
\\
I_2&=\int_{\mathbb R^3}
\exp\left(\frac{im\boldsymbol\eta^2}{2\hbar\epsilon}
-\frac{iq}{\hbar c}\boldsymbol\eta\cdot\mathbf A(\mathbf x,t)\right)
\boldsymbol\eta\cdot\nabla\psi
\,d\boldsymbol\eta
\\
I_3&=\int_{\mathbb R^3}
\exp\left(\frac{im\boldsymbol\eta^2}{2\hbar\epsilon}
-\frac{iq}{\hbar c}\boldsymbol\eta\cdot\mathbf A(\mathbf x,t)\right)
\frac{1}{2}\boldsymbol\eta\cdot\nabla(\boldsymbol\eta\cdot\nabla\psi)
\,d\boldsymbol\eta
\end{align*}

The solutions are
\begin{align*}
I_1&=\left(\frac{2\pi i\hbar\epsilon}{m}\right)^\frac{3}{2}
\exp\left(-\frac{iq^2\epsilon}{2\hbar mc^2}\mathbf A^2\right)\psi
\\
I_2&=\left(\frac{2\pi i\hbar\epsilon}{m}\right)^\frac{3}{2}
\exp\left(-\frac{iq^2\epsilon}{2\hbar mc^2}\mathbf A^2\right)
\frac{q\epsilon}{mc}
\mathbf A\cdot\nabla\psi
\\
I_3&=\left(\frac{2\pi i\hbar\epsilon}{m}\right)^\frac{3}{2}
\exp\left(-\frac{iq^2\epsilon}{2\hbar mc^2}\mathbf A^2\right)
\left(
\frac{i\hbar\epsilon}{2m}\nabla^2\psi
+\frac{q\epsilon}{2mc}\mathbf A\cdot\nabla\psi+O(\epsilon^2)
\right)
\end{align*}

Use the approximation $\exp(y)\approx1+y$ to write the integrals this way.
\begin{align*}
I_1&=\left(\frac{2\pi i\hbar\epsilon}{m}\right)^\frac{3}{2}
\left(1-\frac{iq^2\epsilon}{2\hbar mc^2}\mathbf A^2\right)\psi
\\
I_2&=\left(\frac{2\pi i\hbar\epsilon}{m}\right)^\frac{3}{2}
\left(1-\frac{iq^2\epsilon}{2\hbar mc^2}\mathbf A^2\right)
\frac{q\epsilon}{mc}
\mathbf A\cdot\nabla\psi
\\
I_3&=\left(\frac{2\pi i\hbar\epsilon}{m}\right)^\frac{3}{2}
\left(1-\frac{iq^2\epsilon}{2\hbar mc^2}\mathbf A^2\right)
\left(
\frac{i\hbar\epsilon}{2m}\nabla^2\psi
+\frac{q\epsilon}{2mc}\mathbf A\cdot\nabla\psi+O(\epsilon^2)
\right)
\end{align*}

Discarding terms of order $\epsilon^2$ we have
\begin{equation*}
\sum_kI_k=\left(\frac{2\pi i\hbar\epsilon}{m}\right)^\frac{3}{2}
\left(\psi-\frac{iq^2\epsilon}{2\hbar mc^2}\mathbf A^2\psi
+\frac{q\epsilon}{mc}\mathbf A\cdot\nabla\psi
+\frac{i\hbar\epsilon}{2m}\nabla^2\psi
+\frac{q\epsilon}{2mc}\mathbf A\cdot\nabla\psi
\right)
\end{equation*}

\end{document}
