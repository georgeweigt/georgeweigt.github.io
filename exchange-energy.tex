\documentclass[12pt]{article}
\usepackage[margin=2cm]{geometry}
\usepackage{amsmath}

\begin{document}

\noindent
Let $\psi(x,y)$ be the antisymmetrized wave function for two electrons in a box of length $L$.
\begin{align*}
\psi(x,y)&=\frac{1}{\sqrt{2}}
\big(\phi_1(x)\phi_2(y)-\phi_1(y)\phi_2(x)\big)
\\[2ex]
\phi_n(x)&=\sqrt{\frac{2}{L}}\sin\left(\frac{n\pi x}{L}\right)
\end{align*}

\noindent
For $L=10^{-9}$ meter the expected potential energy is
\begin{equation*}
V=\frac{e^2}{4\pi\epsilon_0}\int_0^L\int_0^L\frac{\psi^*(x,y)\psi(x,y)}{|x-y|}\,dx\,dy
=4.67\,\text{eV}
\end{equation*}

\noindent
Next calculate the potential energy for a wave function that is not antisymmetrized.
\begin{equation*}
V_0=\frac{e^2}{4\pi\epsilon_0}
\int_0^L\int_0^L\frac{\phi_1^*(x)\phi_2^*(y)\phi_1(x)\phi_2(y)}{|x-y|}\,dx\,dy
=12.80\,\text{eV}
\end{equation*}

\noindent
The difference is the exchange energy.
\begin{equation*}
V_{ex}=V-V_0=-8.13\,\text{eV}
\end{equation*}

\noindent
Note that the formula for $V_0$ has a singularity at $x=y$.
The computed value shown above is the result of an arbitrary cutoff in numerical integration.
The actual value of $V_0$ goes to infinity.

\bigskip
\noindent
Note also that there is a singularity at $x=y$ in the formula for $V$.
However, due to antisymmetry we have $\psi(x,x)=0$ and hence the integral converges.

\bigskip
\noindent
We are left to ponder the reality of exchange energy since it cannot be computed.

\end{document}
