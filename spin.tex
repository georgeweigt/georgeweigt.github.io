\documentclass[12pt]{article}
\usepackage{amsmath}
\usepackage{amssymb} % \mathbb
\parindent=0pt
\begin{document}

Spin state $s$ is a unit vector in $\mathbb{C}^2$.
\begin{equation*}
|s\rangle=\begin{pmatrix}c_1\\c_2\end{pmatrix},
\quad
c_1^*c_1+c_2^*c_2=1
\end{equation*}

Here is $|s\rangle$ as a linear combination of basis states ``up'' and ``down.''
\begin{equation*}
|s\rangle=c_1|u\rangle+c_2|d\rangle,
\quad
|u\rangle=\begin{pmatrix}1\\0\end{pmatrix},
\quad
|d\rangle=\begin{pmatrix}0\\1\end{pmatrix}
\end{equation*}

These are the spin operators.
\begin{equation*}
\sigma_x=\begin{pmatrix}0&1\\1&0\end{pmatrix},
\quad
\sigma_y=\begin{pmatrix}0&-i\\i&0\end{pmatrix},
\quad
\sigma_z=\begin{pmatrix}1&0\\0&-1\end{pmatrix}
\end{equation*}

Expectation of spin operators is a projection of $s$ onto Euclidean space.
\begin{equation*}
\langle x\rangle=\langle s|\sigma_x|s\rangle,
\quad
\langle y\rangle=\langle s|\sigma_y|s\rangle,
\quad
\langle z\rangle=\langle s|\sigma_z|s\rangle
\end{equation*}

Let $\mathbf S$ be spin polarization vector
\begin{equation*}
\mathbf S=\begin{pmatrix}\langle x\rangle\\\langle y\rangle\\\langle z\rangle\end{pmatrix}
=\langle s|\sigma|s\rangle,
\quad
\sigma=\begin{pmatrix}\sigma_x\\\sigma_y\\\sigma_z\end{pmatrix}
\end{equation*}

Let $\theta$ and $\phi$ be polar and azimuth angles such that
\begin{equation*}
\mathbf S=\begin{pmatrix}\sin\theta\cos\phi\\\sin\theta\sin\phi\\\cos\theta\end{pmatrix}
\end{equation*}

Then
\begin{equation*}
|s\rangle=\begin{pmatrix}\cos(\theta/2)\\\sin(\theta/2)\exp(i\phi)\end{pmatrix}
\end{equation*}

In component notation $\sigma=\sigma^{\alpha\beta}{}_\gamma$ hence
\begin{equation*}
S^\alpha=s_\beta^*\sigma^{\alpha\beta}{}_\gamma s^\gamma
\end{equation*}

A transpose swaps $\alpha$ and $\beta$ so that summed-over indices are adjacent.
\begin{equation*}
S^\alpha=s_\beta^*\sigma^{\beta\alpha}{}_\gamma s^\gamma
\end{equation*}

Hence the Eigenmath code is
\begin{equation*}
\text{\tt S = dot(conj(s),transpose(sigma),s)}
\end{equation*}

\end{document}
