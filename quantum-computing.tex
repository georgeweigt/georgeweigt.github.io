\documentclass[12pt]{article}
\usepackage[margin=2cm]{geometry}
\usepackage{amsmath}

\begin{document}

\noindent
The file q4.txt defines kets, operators, and a measurement function
for simulating a four qbit quantum computer.
See eigenmath.org/quantum-computer.c for the program that generates q4.txt

\bigskip
\noindent
Ket vectors have 16 elements, one element for each of the 16 states represented by four qbits.
Qbit order is $|q_3 q_2 q_1 q_0\rangle$.
The following basis kets are defined in q4.txt.
\begin{align*}
&|0\rangle=|0000_2\rangle=(1,0,0,0,0,0,0,0,0,0,0,0,0,0,0,0)
\\
&|1\rangle=|0001_2\rangle=(0,1,0,0,0,0,0,0,0,0,0,0,0,0,0,0)
\\
&|2\rangle=|0010_2\rangle=(0,0,1,0,0,0,0,0,0,0,0,0,0,0,0,0)
\\
&|3\rangle=|0011_2\rangle=(0,0,0,1,0,0,0,0,0,0,0,0,0,0,0,0)
\\
&\vdots
\\
&|15\rangle=|1111_2\rangle=(0,0,0,0,0,0,0,0,0,0,0,0,0,0,0,1)
\end{align*}

\noindent
Operators are $16\times16$ matrices that rotate ket vectors.
(A ket always has unit length.)
The following operators are defined in q4.txt.

\bigskip
\begin{tabular}{l l}
$Cmn$ & Controlled not (CNOT) operator, $m$ is the control qbit, $n$ is the target qbit.
\\
\\
$Hn$ & Hadamard operator on qbit $n$.
\\
\\
$Xn$ & Pauli X (NOT) operator on qbit $n$.
\\
\\
$Yn$ & Pauli Y operator on qbit $n$.
\\
\\
$Zn$ & Pauli Z operator on qbit $n$.
\end{tabular}

\bigskip
\noindent
Function $M$ measures the final state by drawing a graph of the probability
for each of 16 states.
\begin{equation*}
M(\psi)
\end{equation*}

\noindent
Quantum algorithms are expressed as sequences of operators applied
to the initial state $|0\rangle$.
The operator sequence should be read backwards, from right to left,
although the direction makes no difference mathematically.

\subsection*{Deutsch-Jozsa algorithm}
Let $f$ be the oracle function.
Then the Deutsch-Jozsa algorithm is
\begin{equation*}
\psi = H_2 \; H_1 \; H_0 \; f \; H_3 \; X_3 \; H_2 \; H_1 \; H_0 \; |0\rangle
\end{equation*}

\subsection*{Bernstein-Vazirani algorithm}
Let $f$ be the oracle function.
Then the Bernstein-Vazirani algorithm is
\begin{equation*}
\psi = H_2 \; H_1 \; H_0 \; f \; Z_3 \; H_3 \; H_2 \; H_1 \; H_0 \; |0\rangle
\end{equation*}

\end{document}
