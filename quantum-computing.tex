\documentclass[12pt]{article}
\usepackage[margin=2cm]{geometry}
\usepackage{amsmath}

\begin{document}

\noindent
The file ``q4.txt'' defines kets, operators, and a measurement function
for simulating a four qbit quantum computer.

\bigskip
\noindent
Each of the following operators defined in ``q4.txt'' is a $16\times16$ matrix.

\bigskip
\begin{tabular}{l l}
$Cmn$ & Controlled not (CNOT) operator, $m$ is the control qbit, $n$ is the target qbit.
\\
\\
$Hn$ & Hadamard operator on qbit $n$.
\\
\\
$Xn$ & Pauli X (NOT) operator on qbit $n$.
\\
\\
$Yn$ & Pauli Y operator on qbit $n$.
\\
\\
$Zn$ & Pauli Z operator on qbit $n$.
\end{tabular}

\bigskip
\noindent
The initial state of the quantum computer is $|0000\rangle$,
i.e., the state in which all qbits are zero.
Ket vectors have 16 elements, one element for each of the 16 states represented by four qbits.
\begin{align*}
&|0000\rangle=(1,0,0,0,0,0,0,0,0,0,0,0,0,0,0,0)
\\
&|1000\rangle=(0,1,0,0,0,0,0,0,0,0,0,0,0,0,0,0)
\\
&|0100\rangle=(0,0,1,0,0,0,0,0,0,0,0,0,0,0,0,0)
\\
&|1100\rangle=(0,0,0,1,0,0,0,0,0,0,0,0,0,0,0,0)
\\
&\vdots
\\
&|1111\rangle=(0,0,0,0,0,0,0,0,0,0,0,0,0,0,0,1)
\end{align*}

\bigskip
\noindent
Function $M$ measures the final state by drawing a graph of the probability
for each of 16 states.
\begin{equation*}
M(\psi)
\end{equation*}

\noindent
Quantum algorithms are expressed as sequences of operators applied
to the initial state $|0000\rangle$.
The operator sequence should be read backwards, from right to left,
although the direction makes no difference mathematically.

\subsection*{Deutsch-Jozsa algorithm}
Let $f$ be the oracle function.
Then the Deutsch-Jozsa algorithm is
\begin{equation*}
\psi = H_2 \; H_1 \; H_0 \; f \; H_3 \; X_3 \; H_2 \; H_1 \; H_0 \; |0000\rangle
\end{equation*}

\subsection*{Bernstein-Vazirani algorithm}
Let $f$ be the oracle function.
Then the Bernstein-Vazirani algorithm is
\begin{equation*}
\psi = H_2 \; H_1 \; H_0 \; f \; Z_3 \; H_3 \; H_2 \; H_1 \; H_0 \; |0000\rangle
\end{equation*}

\end{document}
