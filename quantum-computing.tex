\documentclass[12pt]{article}
\usepackage[margin=2cm]{geometry}
\usepackage{amsmath}
\usepackage{amssymb} % mathbb

\begin{document}

\noindent
The file q4.txt defines kets, operators, and a measurement function
for simulating a four bit quantum computer.
See eigenmath.org/q.c for the program that generates q4.txt.

\bigskip
\noindent
Kets are unit vectors in $\mathbb{C}^{16}$.
The dimension is 16 because four quantum bits have $2^4=16$ basis states.
Quantum bit numbering is $|q_3 q_2 q_1 q_0\rangle$.
The following basis kets are defined in q4.txt.
\begin{align*}
&|0\rangle=|0000_2\rangle=(1,0,0,0,0,0,0,0,0,0,0,0,0,0,0,0)
\\
&|1\rangle=|0001_2\rangle=(0,1,0,0,0,0,0,0,0,0,0,0,0,0,0,0)
\\
&|2\rangle=|0010_2\rangle=(0,0,1,0,0,0,0,0,0,0,0,0,0,0,0,0)
\\
&|3\rangle=|0011_2\rangle=(0,0,0,1,0,0,0,0,0,0,0,0,0,0,0,0)
\\
&\vdots
\\
&|15\rangle=|1111_2\rangle=(0,0,0,0,0,0,0,0,0,0,0,0,0,0,0,1)
\end{align*}

\noindent
Operators are $16\times16$ matrices that rotate ket vectors.
(A ket always has unit length.)
The following operators are defined in q4.txt.

\bigskip
\begin{tabular}{l l}
$H_n$ & Hadamard operator on bit $n$.
\\
\\
$I$ & Identity matrix.
\\
\\
$P_{mn}(\phi)$ & Controlled phase shift, $m$ is the control bit, $n$ is the target bit, $\phi$ is the phase.
\\
\\
$Q$ & Quantum Fourier transform.
\\
\\
$R$ & Inverse quantum Fourier transform.
\\
\\
$S_{mn}$ & Swap bits $m$ and $n$.
\\
\\
$X_n$ & Pauli X (NOT) operator on bit $n$.
\\
\\
$X_{mn}$ & Controlled not (CNOT) operator, $m$ is the control bit, $n$ is the target bit.
\\
\\
$Y_n$ & Pauli Y operator on bit $n$.
\\
\\
$Z_n$ & Pauli Z operator on bit $n$.
\end{tabular}

\bigskip
\noindent
Function $M$ measures the final state by drawing a graph of the probability
for each of 16 states.
\begin{equation*}
M(\psi)
\end{equation*}

\noindent
Quantum algorithms are expressed as sequences of operators applied
to the initial state $|0\rangle$.
The operator sequence should be read backwards, from right to left,
although the direction makes no difference mathematically.

\subsection*{Deutsch-Jozsa algorithm}
Let $f(q_0,q_1,q_2)$ be an operator ($16\times16$ matrix) that operates on $q_3$
in a manner consistent with a constant or balanced oracle.
Then the Deutsch-Jozsa algorithm for identifying $f$ is
\begin{equation*}
\psi = H_2 \; H_1 \; H_0 \; f(q_0,q_1,q_2) \; H_3 \; X_3 \; H_2 \; H_1 \; H_0 \; |0\rangle
\end{equation*}

\subsection*{Bernstein-Vazirani algorithm}
Let $f(q_0,q_1,q_2)$ be an operator ($16\times16$ matrix) that operates on $q_3$.
Then the Bernstein-Vazirani algorithm for identifying $f$ is
\begin{equation*}
\psi = H_2 \; H_1 \; H_0 \; f(q_0,q_1,q_2) \; Z_3 \; H_3 \; H_2 \; H_1 \; H_0 \; |0\rangle
\end{equation*}

\end{document}
