\documentclass[12pt]{article}
\usepackage{amsmath}
\usepackage{amssymb}

\parindent=0pt

\begin{document}

8-3.
Show that $Q_\alpha^c$, $Q_\alpha^s$ are normal coordinates
corresponding to standing wave normal modes
$\cos(2\pi\alpha j/N)$ and $\sin(2\pi\alpha j/N)$,
in the sense that (for $N$ odd)
\begin{equation*}
q_j(t)=
\sqrt{\frac{2}{N}}\left(
\frac{1}{2}Q_0^c(t)
+\sum_{\alpha=1}^{(N-1)/2}
\left(
Q_\alpha^c(t)\cos\frac{2\pi\alpha j}{N}
+Q_\alpha^s(t)\sin\frac{2\pi\alpha j}{N}
\right)
\right)
\tag{8.82}
\end{equation*}

\bigskip
\hrule

\bigskip
Consider the following equations.
\begin{gather*}
Q_\alpha(t)=\frac{1}{\sqrt N}\sum_{k=1}^Nq_k(t)
\left(
\cos\frac{2\pi\alpha k}{N}-i\sin\frac{2\pi\alpha k}{N}
\right)
\tag{8.77}
\\
Q_\alpha^c=\frac{1}{\sqrt2}(Q_\alpha+Q_\alpha^*)
\tag{8.79}
\\
Q_\alpha^s=\frac{i}{\sqrt2}(Q_\alpha-Q_\alpha^*)
\tag{8.80}
\end{gather*}

Substitute (8.77) into (8.82).
\begin{equation*}
q_j=
\frac{1}{\sqrt{2N}}Q_0^c
+
\frac{1}{N}
\sum_{\alpha=1}^{(N-1)/2}
\sum_{k=1}^N
q_k(T_1+T_2+T_3+T_4)
\tag{1}
\end{equation*}
where
\begin{align*}
T_1&=\cos\frac{2\pi\alpha k}{N}\cos\frac{2\pi\alpha j}{N}
-i\sin\frac{2\pi\alpha k}{N}\cos\frac{2\pi\alpha j}{N}
\\
T_2&=\cos\frac{2\pi\alpha k}{N}\cos\frac{2\pi\alpha j}{N}
+i\sin\frac{2\pi\alpha k}{N}\cos\frac{2\pi\alpha j}{N}
\\
T_3&=i\cos\frac{2\pi\alpha k}{N}\sin\frac{2\pi\alpha j}{N}
+\sin\frac{2\pi\alpha k}{N}\sin\frac{2\pi\alpha j}{N}
\\
T_4&=-i\cos\frac{2\pi\alpha k}{N}\sin\frac{2\pi\alpha j}{N}
+\sin\frac{2\pi\alpha k}{N}\sin\frac{2\pi\alpha j}{N}
\end{align*}

It follows that
\begin{equation*}
T_1+T_2+T_3+T_4=
2\cos\frac{2\pi\alpha k}{N}\cos\frac{2\pi\alpha j}{N}
+2\sin\frac{2\pi\alpha k}{N}\sin\frac{2\pi\alpha j}{N}
\end{equation*}

By trigonometric identities
\begin{equation*}
T_1+T_2+T_3+T_4=2\cos\left(\frac{2\pi\alpha}{N}(j-k)\right)
\tag{2}
\end{equation*}

Substitute (2) into (1) to obtain
\begin{equation*}
q_j=
\frac{1}{\sqrt{2N}}Q_0^c
+
\frac{2}{N}
\sum_{\alpha=1}^{(N-1)/2}
\sum_{k=1}^N
q_k\cos\left(\frac{2\pi\alpha}{N}(j-k)\right)
\tag{3}
\end{equation*}

By equations (8.77) and (8.79) with $\alpha=0$ we have
\begin{equation*}
Q_0^c
=\sqrt{\frac{2}{N}}\sum_{k=1}^Nq_k
\tag{4}
\end{equation*}

Substitute (4) into (3).
\begin{equation*}
q_j=
\frac{1}{N}\sum_{k=1}^N q_k
+
\frac{2}{N}
\sum_{\alpha=1}^{(N-1)/2}
\sum_{k=1}^N
q_k\cos\left(\frac{2\pi\alpha}{N}(j-k)\right)
\end{equation*}

Rewrite as
\begin{equation*}
q_j=\sum_{k=1}^N q_k \left(
\frac{1}{N}+\frac{2}{N}\sum_{\alpha=1}^{(N-1)/2}
\cos\left(\frac{2\pi\alpha}{N}(j-k)\right)
\right)
\tag{5}
\end{equation*}

For the sum over $\alpha$ in (5) we have
\begin{equation*}
\sum_{\alpha=1}^{(N-1)/2}
\cos\left(\frac{2\pi\alpha}{N}(j-k)\right)
=\begin{cases}
(N-1)/2 & j=k
\\
-1/2 & j\ne k\quad\text{(see proof below)}
\end{cases}
\end{equation*}

Hence for $j=k$
\begin{equation*}
\frac{1}{N}+\frac{2}{N}\sum_{\alpha=1}^{(N-1)/2}
\cos\left(\frac{2\pi\alpha}{N}(j-k)\right)
=1
\end{equation*}

and for $j\ne k$
\begin{equation*}
\frac{1}{N}+\frac{2}{N}\sum_{\alpha=1}^{(N-1)/2}
\cos\left(\frac{2\pi\alpha}{N}(j-k)\right)
=0
\end{equation*}

It follows that (5) reduces to the following tautology.
\begin{equation*}
q_j=\sum_{k=1}^N q_k\delta(j-k)=q_j
\end{equation*}

Hence (8.82) is proven to be correct.

\bigskip
We will now show that for $N$ odd and $j\ne k$,
\begin{equation*}
\sum_{\alpha=1}^{(N-1)/2}
\cos\left(\frac{2\pi\alpha}{N}(j-k)\right)
=-\frac{1}{2}
\end{equation*}

Consider the following geometric series formula.
\begin{equation*}
\sum_{k=0}^{N-1}z^k=\frac{1-z^N}{1-z},\qquad|z|<1
\tag{6}
\end{equation*}

Let
\begin{equation*}
z=\exp\left(\frac{2\pi in}{N}\right)
\end{equation*}

where $n$ is an integer such that $2n<N$ so that $|z|<1$.
Then
\begin{equation*}
z^N=\exp(2\pi in)=1
\end{equation*}

Hence by (6)
\begin{equation*}
\sum_{k=0}^{N-1}z^k=0
\tag{7}
\end{equation*}

Noting that
\begin{equation*}
z^k=\exp\left(\frac{2\pi ink}{N}\right)=\cos\left(\frac{2\pi nk}{N}\right)+i\sin\left(\frac{2\pi nk}{N}\right)
\end{equation*}

we have by (7)
\begin{equation*}
\sum_{k=0}^{N-1}\cos\left(\frac{2\pi nk}{N}\right)+i\sum_{k=0}^{N-1}\sin\left(\frac{2\pi nk}{N}\right)=0
\end{equation*}

Hence
\begin{equation*}
\sum_{k=0}^{N-1}\cos\left(\frac{2\pi nk}{N}\right)=0
\tag{8}
\end{equation*}

We need to remove the restriction $2n<N$.
Consider
\begin{equation*}
\cos\left(\frac{2\pi(N-n)k}{N}\right)
\tag{9}
\end{equation*}

By trigonometric identities (9) is equivalent to
\begin{equation*}
\cos\left(\frac{2\pi nk}{N}\right)\cos(2\pi k)
+\sin\left(\frac{2\pi nk}{N}\right)\sin(2\pi k)
\end{equation*}

By $\cos(2\pi k)=1$ and $\sin(2\pi k)=0$ we have
\begin{equation*}
\cos\left(\frac{2\pi(N-n)k}{N}\right)=\cos\left(\frac{2\pi nk}{N}\right)
\tag{10}
\end{equation*}

Hence (8) is valid for $0<n<N$.

\bigskip
Returning to (8) and starting the summation at $k=1$, we have
\begin{equation*}
\sum_{k=1}^{N-1}\cos\left(\frac{2\pi nk}{N}\right)=-1
\end{equation*}

By (10) we have for $N$ odd
\begin{equation*}
\sum_{k=1}^{(N-1)/2}\cos\left(\frac{2\pi nk}{N}\right)=-\frac{1}{2}
\end{equation*}

Finally, replace $k$ with $\alpha$ and $n$ with $j-k$.
(The sign of $j-k$ doesn't matter.)
\begin{equation*}
\sum_{\alpha=1}^{(N-1)/2}\cos\left(\frac{2\pi\alpha}{N}(j-k)\right)=-\frac{1}{2}
\end{equation*}

\end{document}
