\documentclass[12pt]{article}
\usepackage{amsmath}

\begin{document}

\noindent
Use problem 3-2 and equation 3.42 to show that
\begin{equation*}
\frac{\partial\psi}{\partial t}=-\frac{i}{\hbar}
\left(-\frac{\hbar^2}{2m}\frac{\partial^2\psi}{\partial x^2}\right)\eqno{(1)}
\end{equation*}

\noindent
From 3.42 we have
\begin{equation*}
\psi(x,t)=\int_{-\infty}^\infty K_0(x,t;x_c,t_c)\psi(x_c,t_c)\,dx_c\eqno{(2)}
\end{equation*}

\noindent
Let
\begin{equation*}
\psi(x_c,t_c)=K_0(x_c,t_c;x_a,t_a)
\end{equation*}

\noindent
and let $I$ be the integrand
\begin{equation*}
I=K_0(x,t;x_c,t_c)\psi(x_c,t_c)
\end{equation*}

\noindent
Since $\psi(x_c,t_c)$ does not depend on $x$ or $t$ we have by problem 3-2
\begin{equation*}
\frac{\partial I}{\partial t}=-\frac{i}{\hbar}
\left(-\frac{\hbar^2}{2m}\frac{\partial^2 I}{\partial x^2}\right)
\end{equation*}

\noindent
Then (1) follows from (2) by linearity of differentiation.

\end{document}
