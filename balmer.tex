\documentclass[12pt]{article}
\usepackage[margin=1in]{geometry}
\usepackage{amsmath}
\usepackage{xcolor}
\parindent=0pt
\begin{document}

The following table is from
``Note on the spectral lines of hydrogen''
by J.~J.~Balmer dated 1885.
Numerical values are wavelengths in units of $10^{-10}$ meter.
%(Data for $H_I$ in the original table is not included here because $H_I$ is not a hydrogen line.
%The $H_I$ data is for Fraunhofer line $H$ which is ionized calcium.)
\begin{center}
\footnotesize
\begin{tabular}{|l|l|l|l|l|l|l|l|l|l|}
\hline
Investigator & $H_\alpha$ & $H_\beta$ & $H_\gamma$ & $H_\delta$ & $H_\epsilon$ &
$H_\zeta$ & $H_\eta$ & $H_\vartheta$ & $H_\iota$
\\
\hline
Van der Willigen & 6565.6 & 4863.94 & 4342.80 & 4103.8 & -- & -- & -- & -- & --
\\
Angstrom & 6562.10 & 4860.74 & 4340.10 & 4101.2 & -- & -- & -- & -- & --
\\
Mendenhall & 6561.62 & 4860.16 & -- & -- & -- & -- & -- & -- & --
\\
Mascart & 6560.7 & 4859.8 & -- & -- & -- & -- & -- & -- & --
\\
Ditscheiner & 6559.5 & 4859.74 & 4338.60 & 4100.0 & -- & -- & -- & -- & --
\\
Huggins & -- & -- & -- & -- & -- & 3887.5 & 3834 & 3795 & 3767.5
\\
Vogel & -- & -- & -- & -- & 3969 & 3887 & 3834 & 3795 & 3769${}^\dag$
\\
\hline
\end{tabular}
\\
{\footnotesize(${}^\dag$The value given in the paper is 6769 which is an obvious typo.)}
\end{center}

Balmer discovered the following formula for fitting the data.
\begin{equation*}
\lambda=\frac{m^2}{m^2-2^2}\times3645.6\times10^{-10}\,\text{meter}
\end{equation*}

Symbol $\lambda$ is spectral line wavelength and parameter $m$ is from the following table.
\begin{center}
\begin{tabular}{cccccccccccc}
& $H_\alpha$ & $H_\beta$ & $H_\gamma$ & $H_\delta$ & $H_\epsilon$ &
$H_\zeta$ & $H_\eta$ & $H_\vartheta$ & $H_\iota$
\\
$m=$ & 3 & 4 & 5 & 6 & 7 & 8 & 9 & 10 & 11
\end{tabular}
\end{center}

Let $\beta$ be the model coefficient for $\lambda$.
Using linear regression and the above data we obtain
\begin{equation*}
\beta=3645.3\times10^{-10}\,\text{meter}
\end{equation*}

%Use linear regression in R to compute the model coefficient.
%{\footnotesize\color{blue}
%\begin{verbatim}
%m = c(3,3,3,3,3,4,4,4,4,4,5,5,5,6,6,6,7,8,8,9,9,10,10,11,11)
%
%x = m^2 / (m^2 - 4)
%
%y = c(
%6565.60, 6562.10, 6561.62, 6560.70, 6559.50,
%4863.94, 4860.74, 4860.16, 4859.80, 4859.74,
%4342.80, 4340.10, 4338.60, 4103.80, 4101.20,
%4100.00, 3969.00, 3887.50, 3887.00, 3834.00,
%3834.00, 3795.00, 3795.00, 3767.50, 3769.00)
%
%coef(lm(y ~ 0 + x))
%\end{verbatim}
%}
%
%\medskip
%The result is
%{\footnotesize
%\begin{verbatim}
%3645.296
%\end{verbatim}
%}
%
%which is a little bit smaller than Balmer's value.
%It should be noted that Balmer did not use the entire data set as we have done here.
%His value 3645.6 was computed using a subset of the data.

The currently accepted value is
\begin{equation*}
\beta=\frac{4}{R_H}=3647.1\times10^{-10}\,\text{meter}
\end{equation*}
where $R_H$ is the Rydberg constant for hydrogen
\begin{equation*}
R_H=1.09677576\times10^7\,\text{meter}^{-1}
\end{equation*}

Balmer's coefficient is within 0.04\% of the modern value.
\begin{equation*}
100\times\frac{4/R_H-3.6456}{4/R_H}=0.04
\end{equation*}

\end{document}
