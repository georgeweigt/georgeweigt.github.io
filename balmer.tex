\documentclass[12pt]{article}
\usepackage[margin=2cm]{geometry}
\usepackage{amsmath}
\usepackage{xcolor}

\begin{document}

\noindent
The following data is from
``Note on the spectral lines of hydrogen''
by J.~J.~Balmer dated 1885.
The numerical values are wavelengths in units of $10^{-10}$ meter.

\begin{center}
\footnotesize
\begin{tabular}{|l|l|l|l|l|l|l|l|l|l|}
\hline
 & $H_\alpha$ & $H_\beta$ & $H_\gamma$ & $H_\delta$ & $H_\epsilon$ &
$H_\zeta$ & $H_\eta$ & $H_\vartheta$ & $H_\iota$
\\
\hline
Van der Willigen & 6565.6 & 4863.94 & 4342.80 & 4103.8 & -- & -- & -- & -- & --
\\
Angstrom & 6562.10 & 4860.74 & 4340.10 & 4101.2 & -- & -- & -- & -- & --
\\
Mendenhall & 6561.2 & 4860.16 & -- & -- & -- & -- & -- & -- & --
\\
Mascart & 6560.7 & 4859.8 & -- & -- & -- & -- & -- & -- & --
\\
Ditscheiner & 6559.5 & 4859.74 & 4338.60 & 4100.0 & -- & -- & -- & -- & --
\\
Huggins & -- & -- & -- & -- & -- & 3887.5 & 3834 & 3795 & 3767.5
\\
Vogel & -- & -- & -- & -- & 3969 & 3887 & 3834 & 3795 & 3769${}^\dag$
\\
\hline
\end{tabular}
\\
{\footnotesize(${}^\dag$The value given in the paper is 6769 which is an obvious typo.)}
\end{center}

\noindent
From this data, Balmer determined that
\begin{equation*}
\hat{y}=\frac{m^2}{m^2-2^2}\times3645.6\times10^{-10}\,\text{meter}
\end{equation*}
where $\hat{y}$ is the predicted wavelength and $m$ is determined by the hydrogen line
according to the following table.
\begin{center}
\begin{tabular}{cccccccccccc}
& $H_\alpha$ & $H_\beta$ & $H_\gamma$ & $H_\delta$ & $H_\epsilon$ &
$H_\zeta$ & $H_\eta$ & $H_\vartheta$ & $H_\iota$
\\
$m=$ & 3 & 4 & 5 & 6 & 7 & 8 & 9 & 10 & 11
\end{tabular}
\end{center}

\noindent
Just for the fun of it, use linear modeling in R to verify the model coefficient.
{\footnotesize\color{blue}
\begin{verbatim}
m = c(3,3,3,3,3,4,4,4,4,4,5,5,5,6,6,6,7,8,8,9,9,10,10,11,11)

x = m^2 / (m^2 - 4)

y = c(6565.6,6562.1,6561.62,6560.7,6559.5,
4863.94,4860.74,4860.16,4859.8,4859.74,
4342.8,4340.1,4338.6,
4103.8,4101.2,4100,
3969,
3887.5,3887,
3834,3834,
3795,3795,
3767.5,3769)

lm(y ~ 0 + x)
\end{verbatim}
}

\bigskip
\noindent
The result is
{\footnotesize
\begin{verbatim}
Call:
lm(formula = y ~ 0 + x)

Coefficients:
   x  
3645
\end{verbatim}
}

\noindent
The actual value is now known from theory to be
\begin{equation*}
3645.07\times10^{-10}\,\text{meter}
\end{equation*}

\end{document}
