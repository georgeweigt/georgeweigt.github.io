\documentclass[12pt]{article}
\usepackage[margin=1in]{geometry}
\usepackage{amsmath}
\usepackage{slashed}
\usepackage{tikz}
\parindent=0pt
\begin{document}

\section*{Bhabha scattering}
Bhabha scattering is the interaction $e^-+e^+\rightarrow e^-+e^+$.

\begin{center}
\begin{tikzpicture}
\draw[dashed] (0,0) circle (0.5cm);
\draw[thick,->] (2,0) node[anchor=west] {$e^-$} -- (0.6,0);
\draw[thick,->] (-2,0) node[anchor=east] {$e^+$} -- (-0.6,0);
\draw[thick,->] (0.40,0.40) -- (1.3,1.3) node[anchor=south west] {$e^+$};
\draw[thick,->] (-0.4,-0.4) -- (-1.3,-1.3) node[anchor=north east] {$e^-$};
\draw (1,0.5) node {$\theta$};
\end{tikzpicture}
\end{center}

Define the following momentum vectors and spinors.
Symbol $p$ is incident momentum.
Symbol $E$ is total energy $E=\sqrt{p^2+m^2}$ where $m$ is electron mass.
Polar angle $\theta$ is the observed scattering angle.
Azimuth angle $\phi$ cancels out in scattering calculations.
\begin{align*}
p_1&=
\underset{\text{inbound positron}}
{
\begin{pmatrix}E\\0\\0\\p\end{pmatrix}
}
&
p_2&=
\underset{\text{inbound electron}}
{
\begin{pmatrix}E\\0\\0\\-p\end{pmatrix}
}
&
p_3&=
\underset{\text{outbound positron}}
{
\begin{pmatrix}
E\\
p\sin\theta\cos\phi\\
p\sin\theta\sin\phi\\
p\cos\theta
\end{pmatrix}
}
&
p_4&=
\underset{\text{outbound electron}}
{
\begin{pmatrix}
E\\
-p\sin\theta\cos\phi\\
-p\sin\theta\sin\phi\\
-p\cos\theta
\end{pmatrix}
}
\\[1ex]
v_{11}&=
\underset{\substack{\text{inbound positron}\\\text{spin up}}}
{
\begin{pmatrix}p\\0\\E+m\\0\end{pmatrix}
}
&
u_{21}&=
\underset{\substack{\text{inbound electron}\\\text{spin up}}}
{
\begin{pmatrix}E+m\\0\\-p\\0\end{pmatrix}
}
&
v_{31}&=
\underset{\substack{\text{outbound positron}\\\text{spin up}}}
{
\begin{pmatrix}p_3^z\\p_3^x+ip_3^y\\E+m\\0\end{pmatrix}
}
&
u_{41}&=
\underset{\substack{\text{outbound electron}\\\text{spin up}}}
{
\begin{pmatrix}E+m\\0\\p_4^z\\p_4^x+ip_4^y\end{pmatrix}
}
\\[1ex]
v_{12}&=
\underset{\substack{\text{inbound positron}\\\text{spin down}}}
{
\begin{pmatrix}0\\-p\\0\\E+m\end{pmatrix}
}
&
u_{22}&=
\underset{\substack{\text{inbound electron}\\\text{spin down}}}
{
\begin{pmatrix}0\\E+m\\0\\p\end{pmatrix}
}
&
v_{32}&=
\underset{\substack{\text{outbound positron}\\\text{spin down}}}
{
\begin{pmatrix}p_3^x-ip_3^y\\-p_3^z\\0\\E+m\end{pmatrix}
}
&
u_{42}&=
\underset{\substack{\text{outbound electron}\\\text{spin down}}}
{
\begin{pmatrix}0\\E+m\\p_4^x-ip_4^y\\-p_4^z\end{pmatrix}
}
\end{align*}

The spinors are not individually normalized.
Instead, a combined spinor normalization constant $N=(E+m)^4$ will be used.

\bigskip
This is the probability density for spin state $abcd$.
The formula is derived from Feynman diagrams for Bhabha scattering.
\begin{equation*}
|\mathcal{M}_{abcd}|^2=\frac{e^4}{N}
\left|
-\frac{1}{t}(\bar{v}_{1a}\gamma^\mu v_{3c})(\bar{u}_{4d}\gamma_\mu u_{2b})
+\frac{1}{s}(\bar{v}_{1a}\gamma^\nu u_{2b})(\bar{u}_{4d}\gamma_\nu v_{3c})
\right|^2
\end{equation*}

Symbol $e$ is electron charge and
\begin{align*}
s&=(p_1+p_2)^2=4E^2
\\
t&=(p_1-p_3)^2=(p_1-p_3)^\mu g_{\mu\nu}(p_1-p_3)^\nu
\end{align*}

Let
\begin{equation*}
a_1=(\bar{v}_{1a}\gamma^\mu v_{3c})(\bar{u}_{4d}\gamma_\mu u_{2b}),
\quad
a_2=(\bar{v}_{1a}\gamma^\nu u_{2b})(\bar{u}_{4d}\gamma_\nu v_{3c})
\end{equation*}

Then
\begin{align*}
|\mathcal{M}_{abcd}|^2
&=
\frac{e^4}{N}\left|{-\frac{a_1}{t}} + \frac{a_2}{s}\right|^2\\
&=
\frac{e^4}{N}\left(-\frac{a_1}{t} + \frac{a_2}{s}\right)\left(-\frac{a_1}{t} + \frac{a_2}{s}\right)^*\\
&=
\frac{e^4}{N}
\left(
\frac{a_1a_1^*}{t^2} - \frac{a_1a_2^*}{st} -
\frac{a_1^*a_2}{st} + \frac{a_2a_2^*}{s^2}
\right)
\end{align*}

The expected probability density $\langle|\mathcal{M}|^2\rangle$ is computed
by summing $|\mathcal{M}_{abcd}|^2$ over all spin states and then dividing by the number of inbound states.
There are four inbound states.
\begin{align*}
\langle|\mathcal{M}|^2\rangle
&=
\frac{1}{4}\sum_{a=1}^2\sum_{b=1}^2\sum_{c=1}^2\sum_{d=1}^2
|\mathcal{M}_{abcd}|^2\\
&=
\frac{e^4}{4N}\sum_{a=1}^2\sum_{b=1}^2\sum_{c=1}^2\sum_{d=1}^2
\left(
\frac{a_1a_1^*}{t^2} - \frac{a_1a_2^*}{st} -
\frac{a_1^*a_2}{st} + \frac{a_2a_2^*}{s^2}
\right)
\end{align*}

The Casimir trick uses matrix arithmetic to compute sums.
\begin{align*}
f_{11}&=\frac{1}{N}\sum_{abcd}a_1a_1^*=
\mathop{\rm Tr}\left(
(\slashed{p}_1-m)\gamma^\mu(\slashed{p}_3-m)\gamma^\nu
\right)
\mathop{\rm Tr}\left(
(\slashed{p}_4+m)\gamma_\mu(\slashed{p}_2+m)\gamma_\nu
\right)
\\
f_{12}&=\frac{1}{N}\sum_{abcd}a_1a_2^*=
\mathop{\rm Tr}\left(
(\slashed{p}_1-m)\gamma^\mu(\slashed{p}_2+m)\gamma^\nu
(\slashed{p}_4+m)\gamma_\mu(\slashed{p}_3-m)\gamma_\nu
\right)
\\
f_{22}&=\frac{1}{N}\sum_{abcd}a_2a_2^*=
\mathop{\rm Tr}\left(
(\slashed{p}_1-m)\gamma^\mu(\slashed{p}_2+m)\gamma^\nu
\right)
\mathop{\rm Tr}\left(
(\slashed{p}_4+m)\gamma_\mu(\slashed{p}_3-m)\gamma_\nu
\right)
\end{align*}

Hence
\begin{equation*}
\langle|\mathcal{M}|^2\rangle
=\frac{e^4}{4}
\left(
\frac{f_{11}}{t^2} - \frac{f_{12}}{st} -
\frac{f_{12}^*}{st} + \frac{f_{22}}{s^2}
\right)
\end{equation*}

The following formulas are equivalent to the Casimir trick.
(Recall that $a\cdot b=a^\mu g_{\mu\nu}b^\nu$)
\begin{align*}
f_{11}&=
32(p_1\cdot p_2)^2
+32(p_1\cdot p_4)^2
-64 m^2(p_1\cdot p_3)
+64 m^4
\\
f_{12}&=
-32 (p_1\cdot p_4)^2
-32 m^2 (p_1\cdot p_2)
+32 m^2 (p_1\cdot p_3)
-32 m^2 (p_1\cdot p_4)
-32 m^4
\\
f_{22}&=
32(p_1\cdot p_3)^2
+32(p_1\cdot p_4)^2
+64 m^2(p_1\cdot p_2)
+64 m^4
\end{align*}

For Mandelstam variables
\begin{align*}
s&=(p_1+p_2)^2
\\
t&=(p_1-p_3)^2
\\
u&=(p_1-p_4)^2
\end{align*}
the formulas are
\begin{align*}
f_{11} &= 8 s^2 + 8 u^2 - 64 s m^2 - 64 u m^2 + 192 m^4
\\
f_{12} &= -8 u^2 + 64 u m^2 - 96 m^4
\\
f_{22} &= 8 t^2 + 8 u^2 - 64 t m^2 - 64 u m^2 + 192 m^4
\end{align*}

\subsection*{High energy approximation}
For high energy experiments $E\gg m$ a useful approximation is to set $m=0$ and obtain
\begin{align*}
f_{11}&= 8 s^2 + 8 u^2\\
f_{12}&= -8 u^2\\
f_{22}&= 8 t^2 + 8 u^2
\end{align*}

Hence
\begin{align*}
\langle|\mathcal{M}|^2\rangle
&=\frac{e^4}{4}
\left(
\frac{f_{11}}{t^2} - \frac{f_{12}}{st} -
\frac{f_{12}^*}{st} + \frac{f_{22}}{s^2}
\right)
\\
&=\frac{e^4}{4}
\left(
\frac{8s^2+8u^2}{t^2} - \frac{-8u^2}{st} - \frac{-8u^2}{st} + \frac{8t^2+8u^2}{s^2}
\right)
\\
&=2e^4
\left(
\frac{s^2+u^2}{t^2} + \frac{2u^2}{st} + \frac{t^2+u^2}{s^2}
\right)
\end{align*}

Combine terms so $\langle|\mathcal{M}|^2\rangle$ has a common denominator.
\begin{equation*}
\langle|\mathcal{M}|^2\rangle
=2e^4
\left(\frac{s^2\left(s^2+u^2\right)+2stu^2+t^2\left(t^2+u^2\right)}{s^2t^2}\right)
\end{equation*}

For $m=0$ the Mandelstam variables are
\begin{align*}
s&=4E^2
\\
t&=2E^2(\cos\theta-1)
\\
u&=-2E^2(\cos\theta+1)
\end{align*}

Hence
\begin{align*}
\langle|\mathcal{M}|^2\rangle
&=2e^4
\left(
\frac{32E^8\cos^4\theta+192E^8\cos^2\theta+288E^8}{64E^8(\cos\theta-1)^2}
\right)
\\
&=e^4
\left(
\frac{\cos^4\theta+6\cos^2\theta+9}{(\cos\theta-1)^2}
\right)
\\
&=e^4
\left(
\frac{\cos^2\theta+3}{\cos\theta-1}
\right)^2
\end{align*}

\subsection*{Cross section}
The differential cross section is
\begin{equation*}
\frac{d\sigma}{d\Omega}
=\frac{\langle|\mathcal{M}|^2\rangle}{4(4\pi\varepsilon_0)^2s},\quad s=(p_1+p_2)^2=4E^2
\end{equation*}

For the high energy experiments we have
\begin{equation*}
\langle|\mathcal{M}|^2\rangle=e^4\left(\frac{\cos^2\theta+3}{\cos\theta-1}\right)^2
\end{equation*}

Substitute for $\langle|\mathcal{M}|^2\rangle$.
\begin{equation*}
\frac{d\sigma}{d\Omega}=\frac{e^4}{4(4\pi\varepsilon_0)^2s}
\left(\frac{\cos^2\theta+3}{\cos\theta-1}\right)^2
\end{equation*}

Noting that
\begin{equation*}
e^2=4\pi\varepsilon_0\alpha\hbar c
\end{equation*}
we can also write
\begin{equation*}
\frac{d\sigma}{d\Omega}
=\frac{\alpha^2(\hbar c)^2}{4s}
\left(\frac{\cos^2\theta+3}{\cos\theta-1}\right)^2
\end{equation*}

We can integrate $d\sigma$ to obtain a cumulative distribution function.
Let $I(\theta)$ be the following integral of $d\sigma$.
(The $\sin\theta$ is from $d\Omega=\sin\theta\,d\theta\,d\phi$.)
\begin{equation*}
I(\theta)=\int
\left(
\frac{\cos^2\theta+3}{\cos\theta-1}
\right)^2
\sin\theta\,d\theta
\end{equation*}

% 16 / (cos(x) - 1) - 1/3 cos(x)^3 - cos(x)^2 - 9 cos(x) - 16 log(-cos(x) + 1)

The result is
\begin{equation*}
I(\theta)=\frac{16}{\cos\theta-1}-\frac{\cos^3\theta}{3}-\cos^2\theta-9\cos\theta-16\log(1-\cos\theta)
\end{equation*}

The cumulative distribution function is
\begin{equation*}
F(\theta)=\frac{I(\theta)-I(a)}{I(\pi)-I(a)},
\quad
a\le\theta\le\pi
\end{equation*}

Angular support is reduced by an arbitrary angle $a>0$ because $I(0)$ is undefined.

\bigskip
The probability of observing scattering events in the interval $\theta_1$ to $\theta_2$ is
\begin{equation*}
P(\theta_1\le\theta\le\theta_2)=F(\theta_2)-F(\theta_1)
\end{equation*}

Let $N$ be the number of scattering events from an experiment.
Then the number of scattering events in the interval $\theta_1$
to $\theta_2$ is predicted to be
$$
N\,\big(F(\theta_2)-F(\theta_1)\big)
$$

The probability density function is
$$
f(\theta)=\frac{dF(\theta)}{d\theta}
=\frac{1}{I(\pi)-I(a)}
\left(\frac{\cos^2\theta+3}{\cos\theta-1}\right)^2
\sin\theta
$$

Note that if we had carried through the $\alpha^2(\hbar c)^2/4s$ in $I(\theta)$,
it would have cancelled out in $F(\theta)$.



\subsection*{Data from SLAC SPEAR experiment}
The following Bhabha scattering data is adapted from SLAC-PUB-1501.

\begin{center}
\begin{tabular}{|r|c|c|}
\hline
$k$ & $x_k$, $x_{k+1}$ & $y$\\
\hline
1 & $0.6, 0.5$ & 4432\\
2 & $0.5, 0.4$ & 2841\\
3 & $0.4, 0.3$ & 2045\\
4 & $0.3, 0.2$ & 1420\\
5 & $0.2, 0.1$ & 1136\\
6 & $0.1, 0.0$ & 852\\
7 & $0.0, -0.1$ & 656\\
8 & $-0.1, -0.2$ & 625\\
9 & $-0.2, -0.3$ & 511\\
10 & $-0.3, -0.4$ & 455\\
11 & $-0.4, -0.5$ & 402\\
12 & $-0.5, -0.6$ & 398\\
\hline
\end{tabular}
\end{center}

Column $k$ is the bin number, column $y$ is the number of scattering events, and
\begin{equation*}
x_k=\cos\theta_k
\end{equation*}

The cumulative distribution function for this experiment is
\begin{equation*}
F(\theta)=\frac{I(\theta)-I(\theta_1)}
{I(\theta_{13})-I(\theta_1)}
\end{equation*}
where
\begin{equation*}
\theta_{13}=\arccos(-0.6),
\quad
\theta_1=\arccos(0.6)
\end{equation*}

The scattering probability $P_k$ is
\begin{equation*}
P_k=F\left(\arccos(x_{k+1})\right)-F\left(\arccos(x_k)\right)
\end{equation*}

Multiply $P_k$ by total scattering events to obtain predicted number of events $\hat{y}_k$.
\begin{equation*}
\sum y_k=15773,\quad \hat{y}_k=15773\,P_k
\end{equation*}

\begin{center}
\begin{tabular}{|r|c|c|c|}
\hline
Bin & $x_k$, $x_{k+1}$ & $y$ & $\hat{y}$ \\
\hline
1 & $0.6, 0.5$ & 4432 & 4598\\
2 & $0.5, 0.4$ & 2841 & 2880\\
3 & $0.4, 0.3$ & 2045 & 1955\\
4 & $0.3, 0.2$ & 1420 & 1410\\
5 & $0.2, 0.1$ & 1136 & 1068\\
6 & $0.1, 0.0$ & 852 & 843\\
7 & $0.0, -0.1$ & 656 & 689\\
8 & $-0.1, -0.2$ & 625 & 582\\
9 & $-0.2, -0.3$ & 511 & 505\\
10 & $-0.3, -0.4$ & 455 & 450\\
11 & $-0.4, -0.5$ & 402 & 411\\
12 & $-0.5, -0.6$ & 398 & 382\\
\hline
\end{tabular}
\end{center}

The coefficient of determination $R^2$ measures how well predicted values fit the data.
\begin{equation*}
R^2=1-\frac{\sum(y-\hat{y})^2}{\sum(y-\bar{y})^2}=0.997
\end{equation*}

The result indicates that $F(\theta)$ explains
99.7\% of the variance in the data.

\subsection*{Data from DESY PETRA experiment}
See www.hepdata.net/record/ins191231, Table 3, 14.0 GeV.

\begin{center}
\begin{tabular}{|c|c|}
\hline
$x$ & $y$\\
\hline
$-0.73\phantom{00}$ & 0.10115\\
$-0.6495$ & 0.12235\\
$-0.5495$ & 0.11258\\
$-0.4494$ & 0.09968\\
$-0.3493$ & 0.14749\\
$-0.2491$ & 0.14017\\
$-0.149\phantom{0}$ & 0.1819\phantom{0}\\
$-0.0488$ & 0.22964\\
$\phantom{+}0.0514$ & 0.25312\\
$\phantom{+}0.1516$ & 0.30998\\
$\phantom{+}0.252\phantom{0}$ & 0.40898\\
$\phantom{+}0.3524$ & 0.62695\\
$\phantom{+}0.4529$ & 0.91803\\
$\phantom{+}0.5537$ & 1.51743\\
$\phantom{+}0.6548$ & 2.56714\\
$\phantom{+}0.7323$ & 4.30279\\
\hline
\end{tabular}
\end{center}

Data $x$ and $y$ have the following relationship with the differential cross section formula.
\begin{equation*}
x=\cos\theta,
\quad
y=\frac{d\sigma}{d\Omega}
\end{equation*}

The cross section formula is
\begin{equation*}
\frac{d\sigma}{d\Omega}
=\frac{\alpha^2}{4s}
\left(\frac{\cos^2\theta+3}{\cos\theta-1}\right)^2\times(\hbar c)^2
\end{equation*}

To compute predicted values $\hat{y}$, multiply by $10^{37}$ to convert square meters to nanobarns.
\begin{equation*}
\hat{y}
=\frac{\alpha^2}{4s}
\left(\frac{x^2+3}{x-1}\right)^2
\times(\hbar c)^2
\times10^{37}
\end{equation*}

The following table shows predicted values $\hat{y}$ for $s=(14.0\,\text{GeV})^2$.

\begin{center}
\begin{tabular}{|c|c|c|}
\hline
$x$ & $y$ & $\hat{y}$\\
\hline
$-0.73\phantom{00}$ & 0.10115 & 0.110296\\
$-0.6495$ & 0.12235 & 0.113816\\
$-0.5495$ & 0.11258 & 0.120101\\
$-0.4494$ & 0.09968 & 0.129075\\
$-0.3493$ & 0.14749 & 0.141592\\
$-0.2491$ & 0.14017 & 0.158934\\
$-0.149\phantom{0}$ & 0.1819\phantom{0} & 0.182976\\
$-0.0488$ & 0.22964 & 0.216737\\
$\phantom{+}0.0514$ & 0.25312 & 0.264989\\
$\phantom{+}0.1516$ & 0.30998 & 0.335782\\
$\phantom{+}0.252\phantom{0}$ & 0.40898 & 0.44363\phantom{0}\\
$\phantom{+}0.3524$ & 0.62695 & 0.615528\\
$\phantom{+}0.4529$ & 0.91803 & 0.9077\phantom{00}\\
$\phantom{+}0.5537$ & 1.51743 & 1.45175\phantom{0}\\
$\phantom{+}0.6548$ & 2.56714 & 2.60928\phantom{0}\\
$\phantom{+}0.7323$ & 4.30279 & 4.61509\phantom{0}\\
\hline
\end{tabular}
\end{center}

The coefficient of determination $R^2$ measures how well predicted values fit the data.
\begin{equation*}
R^2=1-\frac{\sum(y-\hat{y})^2}{\sum(y-\bar{y})^2}=0.995
\end{equation*}

The result indicates that the model $d\sigma$ explains 99.5\% of the variance in the data.

\subsection*{Notes}
Here are a few notes about how the Eigenmath scripts work.
In component notation the trace operators of the Casimir trick become sums over the repeated index $\alpha$.
\begin{align*}
f_{11}&=
\left(
(\slashed{p}_1-m)^\alpha{}_\beta
\gamma^{\mu\beta}{}_\rho
(\slashed{p}_3-m)^\rho{}_\sigma
\gamma^{\nu\sigma}{}_\alpha
\right)
\left(
(\slashed{p}_4+m)^\alpha{}_\beta
\gamma_\mu{}^\beta{}_\rho
(\slashed{p}_2+m)^\rho{}_\sigma
\gamma_\nu{}^\sigma{}_\alpha
\right)
\\
f_{12}&=
(\slashed{p}_1-m)^\alpha{}_\beta
\gamma^{\mu\beta}{}_\rho
(\slashed{p}_2+m)^\rho{}_\sigma
\gamma^{\nu\sigma}{}_\tau
(\slashed{p}_4+m)^\tau{}_\delta
\gamma_\mu{}^\delta{}_\eta
(\slashed{p}_3-m)^\eta{}_\xi
\gamma_\nu{}^\xi{}_\alpha
\\
f_{22}&=
\left(
(\slashed{p}_1-m)^\alpha{}_\beta
\gamma^{\mu\beta}{}_\rho
(\slashed{p}_2+m)^\rho{}_\sigma
\gamma^{\nu\sigma}{}_\alpha
\right)
\left(
(\slashed{p}_4+m)^\alpha{}_\beta
\gamma_\mu{}^\beta{}_\rho
(\slashed{p}_3-m)^\rho{}_\sigma
\gamma_\nu{}^\sigma{}_\alpha
\right)
\end{align*}

To convert the above formulas to Eigenmath code,
the $\gamma$ tensors need to be transposed
so that repeated indices are adjacent to each other.
Also, multiply $\gamma^\mu$ by the metric tensor to lower the index.
\begin{align*}
\gamma^{\beta\mu}{}_\rho\quad&\rightarrow\quad
\text{\tt gammaT = transpose(gamma)}\\
\gamma^\beta{}_{\mu\rho}\quad&\rightarrow\quad
\text{\tt gammaL = transpose(dot(gmunu,gamma))}
\end{align*}

Define the following $4\times4$ matrices.
\begin{align*}
(\slashed{p}_1-m)\quad&\rightarrow\quad\text{\tt X1 = pslash1 - m I}\\
(\slashed{p}_2+m)\quad&\rightarrow\quad\text{\tt X2 = pslash2 + m I}\\
(\slashed{p}_3-m)\quad&\rightarrow\quad\text{\tt X3 = pslash3 - m I}\\
(\slashed{p}_4+m)\quad&\rightarrow\quad\text{\tt X4 = pslash4 + m I}
\end{align*}

Then for $f_{11}$ we have the following Eigenmath code.
The contract function sums over $\alpha$.
\begin{align*}
(\slashed{p}_1-m)^\alpha{}_\beta
\gamma^{\mu\beta}{}_\rho
(\slashed{p}_3-m)^\rho{}_\sigma
\gamma^{\nu\sigma}{}_\alpha
\quad&\rightarrow\quad
\text{\tt T1 = contract(dot(X1,gammaT,X3,gammaT),1,4)}\\
(\slashed{p}_4+m)^\alpha{}_\beta
\gamma_\mu{}^\beta{}_\rho
(\slashed{p}_2+m)^\rho{}_\sigma
\gamma_\nu{}^\sigma{}_\alpha
\quad&\rightarrow\quad
\text{\tt T2 = contract(dot(X4,gammaL,X2,gammaL),1,4)}
\end{align*}

Next, multiply then sum over repeated indices.
The dot function sums over $\nu$ then the contract function
sums over $\mu$. The transpose makes the $\nu$ indices adjacent
as required by the dot function.
$$
f_{11}=
\mathop{\rm Tr}(\cdots\gamma^\mu\cdots\gamma^\nu)
\mathop{\rm Tr}(\cdots\gamma_\mu\cdots\gamma_\nu)
\quad\rightarrow\quad
\text{\tt f11 = contract(dot(T1,transpose(T2)))}
$$

Follow suit for $f_{22}$.
\begin{align*}
(\slashed{p}_1-m)^\alpha{}_\beta
\gamma^{\mu\beta}{}_\rho
(\slashed{p}_2+m)^\rho{}_\sigma
\gamma^{\nu\sigma}{}_\alpha
\quad&\rightarrow\quad
\text{\tt T1 = contract(dot(X1,gammaT,X2,gammaT),1,4)}
\\
(\slashed{p}_4+m)^\alpha{}_\beta
\gamma_\mu{}^\beta{}_\rho
(\slashed{p}_3-m)^\rho{}_\sigma
\gamma_\nu{}^\sigma{}_\alpha
\quad&\rightarrow\quad
\text{\tt T2 = contract(dot(X4,gammaL,X3,gammaL),1,4)}
\end{align*}

Hence
$$
f_{22}=
\mathop{\rm Tr}(\cdots\gamma^\mu\cdots\gamma^\nu)
\mathop{\rm Tr}(\cdots\gamma_\mu\cdots\gamma_\nu)
\quad\rightarrow\quad
\text{\tt f22 = contract(dot(T1,transpose(T2)))}
$$

The calculation of $f_{12}$ begins with
\begin{multline*}
(\slashed{p}_1-m)^\alpha{}_\beta
\gamma^{\mu\beta}{}_\rho
(\slashed{p}_2+m)^\rho{}_\sigma
\gamma^{\nu\sigma}{}_\tau
(\slashed{p}_4+m)^\tau{}_\delta
\gamma_\mu{}^\delta{}_\eta
(\slashed{p}_3-m)^\eta{}_\xi
\gamma_\nu{}^\xi{}_\alpha
\\
\rightarrow\quad
\text{\tt T = contract(dot(X1,gammaT,X2,gammaT,X4,gammaL,X3,gammaL),1,6)}
\end{multline*}

Then sum over repeated indices $\mu$ and $\nu$.
$$
f_{12}=\mathop{\rm Tr}(\cdots\gamma^\mu\cdots\gamma^\nu\cdots\gamma_\mu\cdots\gamma_\nu)
\quad\rightarrow\quad
\text{\tt f12 = contract(contract(T,1,3))}
$$

%F(theta) = -37/8 cos(theta) - 1/4 cos(2 theta) - 1/24 cos(3 theta) - 4 / sin(theta/2)^2 - 16 log(sin(theta/2))

%\subsection*{Epilogue}
%This is the closed form of $I(\theta)$.
%\begin{equation*}
%I(\theta)=-\tfrac{37}{8}\cos\theta - \tfrac{1}{4}\cos(2\theta)
%- \tfrac{1}{24}\cos(3\theta) - \frac{4}{\sin(\theta/2)^2} - 16\log(\sin(\theta/2))
%\end{equation*}

\end{document}
