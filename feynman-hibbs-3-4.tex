\documentclass[12pt]{article}
\usepackage{amsmath}

\begin{document}

\noindent
We are given
\begin{equation*}
\psi(x,0)=C\exp\left(\frac{ipx}{\hbar}\right)
\end{equation*}

\noindent
For a free particle we have
\begin{equation*}
K=\left(\frac{m}{2\pi i\hbar(t_b-t_a)}\right)^{1/2}
\exp\left(\frac{im(x_b-x_a)^2}{2\hbar(t_b-t_a)}\right)
\eqno{(3.3)}
\end{equation*}

\noindent
From
\begin{equation*}
p=\frac{m(x_b-x_a)}{t_b-t_a}
\end{equation*}

\noindent
we have
\begin{equation*}
\frac{m(x_b-x_a)^2}{t_b-t_a}=\frac{p^2(t_b-t_a)}{m}
\end{equation*}

\noindent
Hence
\begin{equation*}
K=\left(\frac{m}{2\pi i\hbar(t_b-t_a)}\right)^{1/2}
\exp\left(\frac{ip^2(t_b-t_a)}{2m\hbar}+\frac{ipx}{\hbar}\right)
\end{equation*}

\noindent
By equation (3.42) we have
\begin{align*}
\psi(x,t_b)
&=K\psi(x,0)
\\
&=C\left(\frac{m}{2\pi i\hbar(t_b-t_a)}\right)^{1/2}
\exp\left(\frac{ip^2(t_b-t_a)}{2m\hbar}+\frac{ipx}{\hbar}\right)
\end{align*}

\noindent
Finally, set $t_a=0$ and change $t_b$ to $t$ to obtain
\begin{equation*}
\psi(x,t)=C\left(\frac{m}{2\pi i\hbar t}\right)^{1/2}
\exp\left(\frac{ip^2t}{2m\hbar}+\frac{ipx}{\hbar}\right)
\end{equation*}





\end{document}
