\documentclass[12pt]{article}
\usepackage{amsmath,amssymb}
\usepackage[margin=2cm]{geometry}
%\hyphenpenalty=10000
\begin{document}

\noindent
{\bf Quantum harmonic oscillator}

\bigskip
\noindent
This is the Schrodinger equation for a quantum harmonic oscillator.
The system is an ``harmonic oscillator'' because the potential energy is proportional to distance squared.
The system is a ``quantum'' harmonic oscillator because $n\in\mathbb{N}_0$ hence the total energy is quantized.
\begin{equation*}
\underset{\substack{\phantom{X} \\ \text{kinetic energy}}}
  {-\frac{\hbar^2}{2m}\frac{\partial^2\psi_n}{\partial x^2}}
+
\underset{\substack{\phantom{X} \\ \text{potential energy}}}
  {\frac{1}{2}m\omega^2 x^2\psi_n}
=
\underset{\substack{\phantom{X} \\ \text{total energy}}}
  {\hbar\omega\left(n+\frac{1}{2}\right)\psi_n}
\end{equation*}

\noindent
The equation can also be written as
\begin{equation*}
\hat{H}\psi_n=E_n\psi_n
\end{equation*}

\noindent
where $\hat{H}$ is the Hamiltonian operator
\begin{equation*}
\hat{H}=-\frac{\hbar^2}{2m}\frac{\partial^2}{\partial x^2}
+\frac{1}{2}m\omega^2 x^2
\end{equation*}

\noindent
and $E_n$ is the observed energy
\begin{equation*}
E_n=\hbar\omega\left(n+\frac{1}{2}\right)
\end{equation*}

\noindent
Wave functions $\psi_n$ are composed of an exponential times a Hermite polynomial $H_n$.
\begin{equation*}
\psi_n=\left(\frac{m\omega}{\pi\hbar}\right)^\frac{1}{4}
\exp\left(-\frac{m\omega x^2}{2\hbar}\right)
\;
\frac{1}{\sqrt{2^n\,n!}}\,
H_n\left(\sqrt{\frac{m\omega}{\hbar}} x\right)
\end{equation*}

\noindent
Hermite polynomials can be computed using the formula
\begin{equation*}
H_n(z)=(-1)^n\exp(z^2)\frac{d^n}{dz^n}\exp(-z^2)
\end{equation*}

\noindent
or the recurrence relation
\begin{equation*}
H_{n+1}(x)=2xH_n(x)-2nH_{n-1}(x)
\end{equation*}

\bigskip
\noindent
{\bf Ladder operators}

\bigskip
\noindent
It can be shown that
\begin{equation*}
\frac{d}{dx}H_n(x)=2nH_{n-1}(x)
\end{equation*}

\noindent
Hence the recurrence relation for Hermite polynomials can be written as
\begin{equation*}
H_{n+1}(x)=2xH_n(x)-\frac{d}{dx}H_n(x)
\end{equation*}

\noindent
This suggests that there is an operator $\hat{O}$ of the form
\begin{equation*}
\hat{O}=ax-\frac{d}{dx}
\end{equation*}

\noindent
such that
\begin{equation*}
\psi_{n+1}\propto\hat{O}\psi_n
\end{equation*}

\noindent
As a convenience of notation, factor $\psi_n=AB$ as follows.
\begin{align*}
A&=\left(\frac{m\omega}{\pi\hbar}\right)^\frac{1}{4}
\exp\left(-\frac{m\omega x^2}{2\hbar}\right)\,\frac{1}{\sqrt{2^n\,n!}}
\\
B&=H_n(z)
\\
z&=\sqrt{\frac{m\omega}{\hbar}} x
\end{align*}

\noindent
Apply provisional operator $\hat{O}$ to wave function $\psi_n$.
\begin{align*}
\hat{O}\psi_n&=\left(ax-\frac{d}{dx}\right)AB
\\
&=axAB-B\frac{dA}{dx}-A\frac{dB}{dz}\frac{dz}{dx}
\\
&=axAB+\frac{m\omega x}{\hbar}AB-A\frac{dB}{dz}\sqrt{\frac{m\omega}{\hbar}}
\\
&=A\left(axB+\frac{m\omega x}{\hbar}B-\frac{dB}{dz}\sqrt{\frac{m\omega}{\hbar}}\right)
\end{align*}

\noindent
Let
\begin{equation*}
a=\frac{m\omega}{\hbar}
\end{equation*}

\noindent
so that
\begin{equation*}
ax+\frac{m\omega x}{\hbar}=\frac{2m\omega x}{\hbar}=2z\sqrt{\frac{m\omega}{\hbar}}
\end{equation*}

\noindent
Returning to the expansion of $\hat{O}\psi_n$ we now have
\begin{equation*}
\hat{O}\psi_n=\sqrt\frac{m\omega}{\hbar}A\left(2zB-\frac{dB}{dz}\right)
\end{equation*}

\noindent
From the recurrence relation for Hermite polynomials we have
\begin{equation*}
2zB-\frac{dB}{dz}=H_{n+1}(z)
\end{equation*}

\noindent
Expanding $A$ and substituting $H_{n+1}$ for the expression in $B$ we have
\begin{equation*}
\hat{O}\psi_n=\sqrt\frac{m\omega}{\hbar}
\left(\frac{m\omega}{\pi\hbar}\right)^\frac{1}{4}
\exp\left(-\frac{m\omega x^2}{2\hbar}\right)\,\frac{1}{\sqrt{2^n\,n!}}
\,H_{n+1}\left(\sqrt{\frac{m\omega}{\hbar}} x\right)
\end{equation*}

\noindent
Noting that
\begin{equation*}
\psi_{n+1}=
\left(\frac{m\omega}{\pi\hbar}\right)^\frac{1}{4}
\exp\left(-\frac{m\omega x^2}{2\hbar}\right)\,\frac{1}{\sqrt{2^{n+1}\,(n+1)!}}
\,H_{n+1}\left(\sqrt{\frac{m\omega}{\hbar}} x\right)
\end{equation*}

\noindent
we conclude that
\begin{equation*}
\psi_{n+1}=\frac{1}{\sqrt{2(n+1)}}
\sqrt\frac{\hbar}{m\omega}\,\hat{O}\psi_n
\end{equation*}

\noindent
The standard notation for a raising operator is $\hat{a}^\dagger$.
Define $\hat{a}^\dagger$ as
\begin{equation*}
\hat{a}^\dagger
=\sqrt\frac{\hbar}{2m\omega}\,\hat{O}
=\sqrt\frac{\hbar}{2m\omega}\left(\frac{m\omega x}{\hbar}-\frac{d}{dx}\right)
\end{equation*}

\noindent
It follows that
\begin{equation*}
\psi_{n+1}=\frac{1}{\sqrt{n+1}}\,\hat{a}^\dagger\psi_n
\end{equation*}

\noindent
A lowering operator follows directly from the derivative of $\psi_n$.
\begin{align*}
\frac{d\psi_n}{dx}
&=B\frac{dA}{dx}+A\frac{dB}{dz}\frac{dz}{dx}
\\
&=-\frac{m\omega x}{\hbar}\psi_n+2nAH_{n-1}(z)\sqrt\frac{m\omega}{\hbar}
\\
&=-\frac{m\omega x}{\hbar}\psi_n+\sqrt{2n}\,\psi_{n-1}
\,\sqrt\frac{m\omega}{\hbar}
\end{align*}

\noindent
Hence
\begin{equation*}
\psi_{n-1}=\frac{1}{\sqrt{n}}\sqrt{\frac{\hbar}{2m\omega}}\left(\frac{m\omega x}{\hbar}+\frac{d}{dx}\right)\psi_n
\end{equation*}

\noindent
The standard notation for a lowering operator is $\hat{a}$.
Define $\hat{a}$ as
\begin{equation*}
\hat{a}=\sqrt\frac{\hbar}{2m\omega}\left(\frac{m\omega x}{\hbar}+\frac{d}{dx}\right)
\end{equation*}

\noindent
It follows that
\begin{equation*}
\psi_{n-1}=\frac{1}{\sqrt n}\,\hat{a}\psi_n
\end{equation*}

\bigskip
\noindent
{\bf Probability density}

\bigskip
\noindent
A wave function squared is a probability density hence
\begin{equation*}
\int_{-\infty}^\infty (\psi_n)^2\,dx=1
\end{equation*}

\noindent
The expectation value for the Hamiltonian of the $n$th energy state is
\begin{equation*}
\langle\hat{H}\rangle_n=\int_{-\infty}^\infty(\hat{H}\psi_n)\psi_n\,dx
=\int_{-\infty}^\infty E_n(\psi_n)^2\,dx=E_n
\end{equation*}

\end{document}
