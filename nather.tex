\documentclass[12pt]{article}
\usepackage{parskip}
\pdfinfoomitdate=1
\pdftrailerid{}
\begin{document}

\begin{center}
Real Programmers Don't Use Fortran, Either!
\end{center}

A recent article devoted to the {\it macho} side of programming
(``Real Programmers Don't Use Pascal,'' by ucbvax!G:tut)
made the bald and unvarnished statement

\begin{quote}
Real Programmers write in Fortran.
\end{quote}

Maybe they do now, in this decadent era of Lite beer, hand
calculators and ``user-friendly'' software, but back in the
Good Old Days, when the term ``software'' sounded funny and
Real Computers were made out of drums and vacuum tubes, Real
Programmers wrote in machine code. Not Fortran. Not RATFOR.
Not, even, assembly language. Machine Code. Raw,
unadorned, inscrutable hexadecimal numbers. Directly.

Lest a whole new generation of programmers grow up in
ignorance of this glorious past, I feel duty-bound to
describe, as best I can through the generation gap, how a
Real Programmer wrote code. I'll call him Mel, because that
was his name.

I first met Mel when I went to work for Royal McBee Computer
Corp., a now-defunct subsidiary of the typewriter company.
The firm manufactured the LGP-30, a small, cheap (by the
standards of the day) drum-memory computer, and had just
started to manufacture the RPC-4000, a much-improved,
bigger, better, faster --- drum-memory computer. Cores cost
too much, and weren't here to stay, anyway. (That's why you
haven't heard of the company, or the computer.)

I had been hired to write a Fortran compiler for this new
marvel and Mel was my guide to its wonders. Mel didn't
approve of compilers.

``If a program can't rewrite its own code,'' he asked, ``what
good is it?''

Mel had written, in hexadecimal, the most popular computer
program the company owned. It ran on the LGP-30 and played
blackjack with potential customers at computer shows. Its
effect was always dramatic. The LGP-30 booth was packed at
every show, and the IBM salesmen stood around talking to
each other. Whether or not this actually sold computers was
a question we never discussed.

Mel's job was to re-write the blackjack program for the
RPC-4000. (Port? What does that mean?) The new computer
had a one-plus-one addressing scheme, in which each machine
instruction, in addition to the operation code and the
address of the needed operand, had a second address that
indicated where, on the revolving drum, the next instruction
was located. In modern parlance, every single instruction
was followed by a GO TO! Put {\it that} in Pascal's pipe and
smoke it.

Mel loved the RPC-4000 because he could optimize his code:
that is, locate instructions on the drum so that just as one
finished its job, the next would be just arriving at the
``read head'' and available for immediate execution. There
was a program to do that job, an ``optimizing assembler,'' but
Mel refused to use it.

``You never know where it's going to put things,'' he
explained, ``so you'd have to use separate constants.''

It was a long time before I understood that remark. Since
Mel knew the numerical value of every operation code, and
assigned his own drum addresses, every instruction he wrote
could also be considered a numerical constant. He could
pick up an earlier ``add'' instruction, say, and multiply by
it, if it had the right numeric value. His code was not
easy for someone else to modify.

I compared Mel's hand-optimized programs with the same code
massaged by the optimizing assembly program, and Mel's
always ran faster. That was because the ``top-down'' method
of program design hadn't been invented yet, and Mel wouldn't
have used it anyway. He wrote the innermost parts of his
program loops first, so they would get first choice of the
optimum address locations on the drum. The optimizing
assembler wasn't smart enough to do it that way.

Mel never wrote time-delay loops, either, even when the
balky Flexowriter required a delay between output characters
to work right. He just located instructions on the drum so
each successive one was just {\it past} the read head when it
was needed; the drum had to execute another complete
revolution to find the next instruction. He coined an
unforgettable term for this procedure. Although ``optimum''
is an absolute term, like ``unique,'' it became common verbal
practice to make it relative: ``not quite optimum'' or ``less
optimum'' or ``not very optimum.'' Mel called the maximum
time-delay locations the ``most pessimum.''

After he finished the blackjack program and got it to run,
(``Even the initializer is optimized,'' he said proudly) he
got a Change Request from the sales department. The program
used an elegant (optimized) random number generator to
shuffle the ``cards'' and deal from the ``deck,'' and some of
the salesmen felt it was too fair, since sometimes the
customers lost. They wanted Mel to modify the program so,
at the setting of a sense switch on the console, they could
change the odds and let the customer win.

Mel balked. He felt this was patently dishonest, which it
was, and that it impinged on his personal integrity as a
programmer, which it did, so he refused to do it. The Head
Salesman talked to Mel, as did the Big Boss and, at the
boss's urging, a few Fellow Programmers. Mel finally gave
in and wrote the code, but he got the test backwards and,
when the sense switch was turned on, the program would
cheat, winning every time. Mel was delighted with this,
claiming his subconscious was uncontrollably ethical, and
adamantly refused to fix it.

After Mel had left the company for greener pa\$ture\$, the Big
Boss asked me to look at the code and see if I could find
the test and reverse it. Somewhat reluctantly, I agreed to
look. Tracking Mel's code was a real adventure.

I have often felt that programming is an art form, whose
real value can only be appreciated by another versed in the
same arcane art; there are lovely gems and brilliant coups
hidden from human view and admiration, sometimes forever, by
the very nature of the process. You can learn a lot about
an individual just by reading through his code, even in
hexadecimal. Mel was, I think, an unsung genius.

Perhaps my greatest shock came when I found an innocent loop
that had no test in it. No test. {\it None.} Common sense said
it had to be a closed loop, where the program would circle,
forever, endlessly. Program control passed right through
it, however, and safely out the other side. It took me two
weeks to figure it out.

The RPC-4000 computer had a really modern facility called an
index register. It allowed the programmer to write a
program loop that used an indexed instruction inside; each
time through, the number in the index register was added to
the address of that instruction, so it would refer to the
next datum in a series. He had only to increment the index
register each time through. Mel never used it.

Instead, he would pull the instruction into a machine
register, add one to its address, and store it back. He
would then execute the modified instruction right from the
register. The loop was written so this additional execution
time was taken into account --- just as this instruction
finished, the next one was right under the drum's read head,
ready to go. But the loop had no test in it.

The vital clue came when I noticed the index register bit,
the bit that lay between the address and the operation code
in the instruction word, was turned on --- yet Mel never used
the index register, leaving it zero all the time. When the
light went on it nearly blinded me.

He had located the data he was working on near the top of
memory --- the largest locations the instructions could
address --- so, after the last datum was handled,
incrementing the instruction address would make it overflow.
The carry would add one to the operation code, changing it
to the next one in the instruction set: a jump instruction.
Sure enough, the next program instruction was in address
location zero, and the program went happily on its way.

I haven't kept in touch with Mel, so I don't know if he ever
gave in to the flood of change that has washed over
programming techniques since those long-gone days. I like
to think he didn't. In any event, I was impressed enough
that I quit looking for the offending test, telling the Big
Boss I couldn't find it. He didn't seem surprised. When I
left the company, the blackjack program would still cheat if
you turned on the right sense switch, and I think that's how
it should be. I didn't feel comfortable hacking up the code
of a Real Programmer.

\end{document}
