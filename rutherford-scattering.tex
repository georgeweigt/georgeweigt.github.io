\documentclass[12pt]{article}
\usepackage[margin=1in]{geometry}
\usepackage{amsmath}
\usepackage{slashed}
\usepackage{tikz}
\parindent=0pt
\begin{document}

\subsection*{Rutherford scattering}
Rutherford scattering is the interaction $e^-+Z\rightarrow e^-+Z$ where $Z$ is a nucleon.

\begin{center}
\begin{tikzpicture}
\draw[dashed] (0,0) circle (0.5cm);
\draw (0,0) node {$Z$};
\draw[thick,->] (-2,0) node[anchor=east] {$e^-$} -- (-0.6,0);
\draw[thick,->] (0.40,0.40) -- (1.3,1.3) node[anchor=south west] {$e^-$};
\draw (1,0.5) node {$\theta$};
\end{tikzpicture}
\end{center}

Define the following momentum vectors and spinors.
Symbol $p$ is incident momentum.
Symbol $E$ is total energy $E=\sqrt{p^2+m^2}$ where $m$ is electron mass.
Polar angle $\theta$ is the observed scattering angle.
Azimuth angle $\phi$ cancels out in scattering calculations.
\begin{align*}
p_1&=
\underset{\substack{\text{inbound}\\\text{electron}}}
{
\begin{pmatrix}E\\0\\0\\p\end{pmatrix}
}
&
p_2&=
\underset{\substack{\text{outbound}\\\text{electron}}}
{
\begin{pmatrix}
E\\
p\sin\theta\cos\phi\\
p\sin\theta\sin\phi\\
p\cos\theta
\end{pmatrix}
}
\end{align*}
\begin{align*}
u_{11}&=
\underset{\substack{\text{inbound electron}\\\text{spin up}}}
{
\begin{pmatrix}E+m\\0\\p\\0\end{pmatrix}
}
&
u_{21}&=
\underset{\substack{\text{outbound electron}\\\text{spin up}}}
{
\begin{pmatrix}E+m\\0\\p_{2z}\\p_{2x}+ip_{2y}\end{pmatrix}
}
\\[1ex]
u_{12}&=
\underset{\substack{\text{inbound electron}\\\text{spin down}}}
{
\begin{pmatrix}0\\E+m\\0\\-p\end{pmatrix}
}
&
u_{22}&=
\underset{\substack{\text{outbound electron}\\\text{spin down}}}
{
\begin{pmatrix}0\\E+m\\p_{2x}-ip_{2y}\\-p_{2z}\end{pmatrix}
}
\end{align*}

The spinors are not individually normalized.
Instead, a combined spinor normalization constant $N=(E+m)^2$ will be used.

\bigskip
This is the probability density for spin state $ab$.
The formula is derived from Feynman diagrams for Rutherford scattering.
\begin{equation*}
|\mathcal{M}_{ab}|^2=\frac{Z^2e^4}{q^4}\frac{1}{N}\left|\bar{u}_{2b}\gamma^0 u_{1a}\right|^2
\end{equation*}

Symbol $Z$ is the atomic number of the nucleon, $e$ is electron charge,
$q$ is momentum transfer, and
\begin{equation*}
q^4=\big((p_1-p_2)^\mu g_{\mu\nu}(p_1-p_2)^\nu\big)^2=4p^4(\cos\theta-1)^2
\end{equation*}

The expected probability density
$\langle\vert\mathcal{M}\vert^2\rangle$
is computed by summing $|\mathcal{M}_{ab}|^2$
over all spin states and then dividing by the number of inbound states.
There are two inbound states.
\begin{equation*}
\langle\vert\mathcal{M}\vert^2\rangle
=\frac{1}{2}\sum_{a=1}^2\sum_{b=1}^2\left|\mathcal{M}_{ab}\right|^2
\end{equation*}

The Casimir trick uses matrix arithmetic to compute the sum.
\begin{equation*}
\langle\vert\mathcal{M}\vert^2\rangle
=\frac{Z^2e^4}{2q^4}\mathop{\rm Tr}\left((\slashed{p}_1+m)\gamma^0(\slashed{p}_2+m)\gamma^0\right)
\end{equation*}

The result is
\begin{equation*}
\langle\vert\mathcal{M}\vert^2\rangle
=\frac{2Z^2e^4}{q^4}\left(E^2+m^2+p^2\cos\theta\right)
\end{equation*}

For low energy experiments $p\ll m$ we can use the approximation
\begin{equation*}
E^2+m^2+p^2\cos\theta\approx2m^2
\end{equation*}

Hence
\begin{equation*}
\langle|\mathcal{M}|^2\rangle=\frac{4Z^2e^4m^2}{q^4}
\end{equation*}

Substituting $e^4=16\pi^2\alpha^2$ and $q^4=4p^4(\cos\theta-1)^2$ we have
\begin{equation*}
\langle|\mathcal{M}|^2\rangle=\frac{16\pi^2Z^2\alpha^2m^2}{p^4(\cos\theta-1)^2}
\end{equation*}

\subsection*{Cross section}
The differential cross section for Rutherford scattering is
\begin{equation*}
\frac{d\sigma}{d\Omega}=\frac{\langle|\mathcal{M}|^2\rangle}{16\pi^2}
\end{equation*}

For low energy experiments we have
\begin{equation*}
\langle|\mathcal{M}|^2\rangle=\frac{16\pi^2Z^2\alpha^2m^2}{p^4(\cos\theta-1)^2}
\end{equation*}

Substitute for $\langle|\mathcal{M}|^2\rangle$.
\begin{equation*}
\frac{d\sigma}{d\Omega}=\frac{Z^2\alpha^2m^2}{p^4(\cos\theta-1)^2}
\end{equation*}

We can integrate $d\sigma$ to obtain a cumulative distribution function.
Let $I(\theta)$ be the following integral of $d\sigma$.
(The $\sin\theta$ is from $d\Omega=\sin\theta\,d\theta\,d\phi$.)
\begin{equation*}
I(\theta)
=\int\frac{1}{(\cos\theta-1)^2}\sin\theta\,d\theta
\end{equation*}

The result is
\begin{equation*}
I(\theta)=\frac{1}{\cos\theta-1}
\end{equation*}

The cumulative distribution function is
\begin{equation*}
F(\theta)=\frac{I(\theta)-I(a)}{I(\pi)-I(a)}
=\frac{2(\cos a-\cos\theta)}{(1+\cos a)(1-\cos\theta)},
\quad
a\le\theta\le\pi
\end{equation*}

Angular support is reduced by an arbitrary angle $a>0$ because $I(0)$ is undefined.

\bigskip
The probability of observing scattering events
in the interval $\theta_1$ to $\theta_2$ is
\begin{equation*}
P(\theta_1\le\theta\le\theta_2)=F(\theta_2)-F(\theta_1)
\end{equation*}

Let $N$ be the number of scattering events from an experiment.
Then the number of scattering events in the interval $\theta_1$ to $\theta_2$ is predicted to be
$$
N\,\big(F(\theta_2)-F(\theta_1)\big)
$$

The probability density function is
$$
f(\theta)=\frac{dF(\theta)}{d\theta}=\frac{1}{I(\pi)-I(a)}\frac{1}{(\cos\theta-1)^2}\sin\theta
$$

Note that if we had carried through the $Z^2\alpha^2m^2/p^4$ in $I(\theta)$,
it would have cancelled out in $F(\theta)$.

\subsection*{Notes}
1. The original Rutherford scattering experiment in 1911 used alpha particles, not electrons.
However, scattering of any charged particle by Coulomb interaction
is now known as Rutherford scattering.
The first Rutherford scattering experiment using electrons appears to have
been done by F.~L.~Arnot, then a student of Rutherford, in 1929.

\bigskip
2. Lancaster and Blundell page 356 has
\begin{equation*}
\frac{d\sigma}{d\Omega}
=\frac{Z^2\alpha^2}{4m^2\mathbf v^4\sin^4(\theta/2)}
\end{equation*}

Noting that
\begin{equation*}
\frac{1}{m^2\mathbf v^4}=\frac{m^2}{m^4\mathbf v^4}=\frac{m^2}{p^4}
\end{equation*}

and
\begin{equation*}
4\sin^4(\theta/2)=(\cos\theta-1)^2
\end{equation*}

we have
\begin{equation*}
\frac{Z^2\alpha^2}{4m^2\mathbf v^4\sin^4(\theta/2)}
=\frac{Z^2\alpha^2m^2}{p^4(\cos\theta-1)^2}
\end{equation*}

\end{document}
