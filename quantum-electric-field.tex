\documentclass[12pt]{article}
\usepackage{amsmath}
\parindent=0pt
\begin{document}

Let $|\Psi\rangle$ be a coherent state where $\bar n$ is the expected number of photons.
\begin{equation*}
|\Psi\rangle=\sum_{n=0}^\infty
\sqrt{\frac{\bar n^n\exp(-\bar n)}{n!}}
\exp\left(-i\left(n+\tfrac{1}{2}\right)\omega t\right)
|n\rangle
\end{equation*}

It can be shown that\footnote{See {\it Quantum Mechanics for Scientists and Engineers} problem 15.6.1.}
\begin{equation*}
\hat a|\Psi\rangle=\sqrt{\bar n}\exp(-i\omega t)|\Psi\rangle
\end{equation*}

It follows that
\begin{equation*}
\langle\Psi|\hat a^\dag=\left(\hat a|\Psi\rangle\right)^\dag=\sqrt{\bar n}\exp(i\omega t)\langle\Psi|
\end{equation*}

Let $\hat E$ be the electric field operator
\begin{equation*}
\hat E=i\sqrt{\frac{\hbar\omega}{2\epsilon_0}}
(\hat a-\hat a^\dag)
\end{equation*}

The expected electric field for the coherent state is
\begin{equation*}
\langle\hat E\rangle
=\langle\Psi|\hat E|\Psi\rangle
=i\sqrt{\frac{\hbar\omega}{2\epsilon_0}}
\langle\Psi|(\hat a-\hat a^\dag)|\Psi\rangle
\end{equation*}

Hence
\begin{equation*}
\langle\hat E\rangle
=i\sqrt{\frac{\hbar\omega}{2\epsilon_0}}
\left(\sqrt{\bar n}\exp(-i\omega t)\langle\Psi|\Psi\rangle-\sqrt{\bar n}\exp(i\omega t)\langle\Psi|\Psi\rangle\right)
\end{equation*}

By $\langle\Psi|\Psi\rangle=1$ we have
\begin{equation*}
\langle\hat E\rangle
=i\sqrt{\frac{\hbar\omega}{2\epsilon_0}}
\left(\sqrt{\bar n}\exp(-i\omega t)-\sqrt{\bar n}\exp(i\omega t)\right)
\end{equation*}

Recalling that
\begin{equation*}
2\sin(\omega t)=i\exp(-i\omega t)-i\exp(i\omega t)
\end{equation*}
we have
\begin{equation*}
\langle\hat E\rangle=\sqrt{\frac{2\hbar\omega\bar n}{\epsilon_0}}\sin(\omega t)
\end{equation*}

Hence the peak amplitude is proportional to $\sqrt{\bar n}$.

\end{document}
