\documentclass[12pt]{article}
\usepackage{amsmath}
\parindent=0pt
\begin{document}

Let $|\Psi\rangle$ be a coherent state where $\bar n$ is the expected number of photons.
\begin{equation*}
|\Psi\rangle=\sum_{n=0}^\infty
\sqrt{\frac{\bar n^n\exp(-\bar n)}{n!}}
\exp\left(-i\left(n+\tfrac{1}{2}\right)\omega t\right)
|n\rangle
\end{equation*}

It can be shown that\footnote{See {\it Quantum Mechanics for Scientists and Engineers} problem 15.6.1.}
\begin{equation*}
\hat a|\Psi\rangle=\sqrt{\bar n}\exp(-i\omega t)|\Psi\rangle
\end{equation*}

It follows that
\begin{equation*}
\langle\Psi|\hat a^\dag=\left(\hat a|\Psi\rangle\right)^\dag=\sqrt{\bar n}\exp(i\omega t)\langle\Psi|
\end{equation*}

Let $\hat E$ be the electric field operator
\begin{equation*}
\hat E=i\sqrt{\frac{\hbar\omega}{2\epsilon_0}}
(\hat a-\hat a^\dag)
\end{equation*}

The expected electric field for the coherent state is
\begin{equation*}
\langle\hat E\rangle
=\langle\Psi|\hat E|\Psi\rangle
=i\sqrt{\frac{\hbar\omega}{2\epsilon_0}}
\langle\Psi|(\hat a-\hat a^\dag)|\Psi\rangle
\end{equation*}

Hence
\begin{equation*}
\langle\hat E\rangle
=i\sqrt{\frac{\hbar\omega}{2\epsilon_0}}
\left(\sqrt{\bar n}\exp(-i\omega t)\langle\Psi|\Psi\rangle-\sqrt{\bar n}\exp(i\omega t)\langle\Psi|\Psi\rangle\right)
\end{equation*}

By $\langle\Psi|\Psi\rangle=1$ we have
\begin{equation*}
\langle\hat E\rangle
=i\sqrt{\frac{\hbar\omega}{2\epsilon_0}}
\left(\sqrt{\bar n}\exp(-i\omega t)-\sqrt{\bar n}\exp(i\omega t)\right)
\end{equation*}

Recalling that
\begin{equation*}
2\sin(\omega t)=i\exp(-i\omega t)-i\exp(i\omega t)
\end{equation*}
we have
\begin{equation*}
\langle\hat E\rangle=\sqrt{\frac{2\hbar\omega\bar n}{\epsilon_0}}\sin(\omega t)
\end{equation*}

Hence the peak amplitude is proportional to $\sqrt{\bar n}$.

\bigskip
The total energy of the electromagnetic field per unit volume is
\begin{equation*}
U=\frac{\epsilon_0}{2}|\mathbf E|^2+\frac{1}{2\mu_0}|\mathbf B|^2
\end{equation*}

For linearly polarized light and a suitable rotation matrix $R$ we have
\begin{equation*}
R\mathbf E=\begin{pmatrix}E_x\\0\\0\end{pmatrix},
\quad
R\mathbf B=\begin{pmatrix}0\\B_y\\0\end{pmatrix}
\end{equation*}

Hence in the rotated frame
\begin{equation*}
U=\frac{\epsilon_0}{2}E_x^2+\frac{1}{2\mu_0}B_y^2
\end{equation*}

For the quantum field we have
\begin{equation*}
U=\frac{\epsilon_0}{2}\langle\hat E_x^2\rangle+\frac{1}{2\mu_0}\langle\hat B_y^2\rangle
\end{equation*}
where
\begin{equation*}
\hat E_x=i\sqrt{\frac{\hbar\omega}{2\epsilon_0}}(\hat a-\hat a^\dag),
\quad
\hat B_y=\sqrt{\frac{\hbar\omega\mu_0}{2}}(\hat a+\hat a^\dag)
\end{equation*}

For the coherent state we have
\begin{align*}
\langle\Psi|\hat a\hat a|\Psi\rangle&=\left(\sqrt{\bar n}\exp(-i\omega t)\right)^2
& &=\bar n\exp(-2i\omega t)
\\
\langle\Psi|\hat a\hat a^\dag|\Psi\rangle&=\langle\Psi|(\hat a^\dag\hat a+1)|\Psi\rangle
& &=\bar n+1
\\
\langle\Psi|\hat a^\dag\hat a|\Psi\rangle
&=\left(\sqrt{\bar n}\exp(i\omega t)\right)\left(\sqrt{\bar n}\exp(-i\omega t)\right)
& &=\bar n
\\
\langle\Psi|\hat a^\dag\hat a^\dag|\Psi\rangle&=\left(\sqrt{\bar n}\exp(i\omega t)\right)^2
& &=\bar n\exp(2i\omega t)
\end{align*}

Hence
\begin{align*}
\langle\hat E_x^2\rangle
=\langle\Psi|\hat E_x\hat E_x|\Psi\rangle
&=-\frac{\hbar\omega}{2\epsilon_0}
\langle\Psi|(\hat a-\hat a^\dag)(\hat a-\hat a^\dag)|\Psi\rangle
\\
&=-\frac{\hbar\omega}{2\epsilon_0}
\left(\bar n\exp(-2i\omega t)+\bar n\exp(2i\omega t)-2\bar n-1\right)
\end{align*}
and
\begin{align*}
\langle\hat B_y^2\rangle
=\langle\Psi|\hat B_y\hat B_y|\Psi\rangle
&=\frac{\hbar\omega\mu_0}{2}
\langle\Psi|(\hat a+\hat a^\dag)(\hat a+\hat a^\dag)|\Psi\rangle
\\
&=\frac{\hbar\omega\mu_0}{2}
\left(\bar n\exp(-2i\omega t)+\bar n\exp(2i\omega t)+2\bar n+1\right)
\end{align*}

Noting that
\begin{align*}
4\sin(\omega t)^2&=-\exp(-2i\omega t)-\exp(2i\omega t)+2\\
4\cos(\omega t)^2&=\exp(-2i\omega t)+\exp(2i\omega t)+2
\end{align*}
we have
\begin{align*}
\langle\hat E_x^2\rangle
&=\frac{2\bar n\hbar\omega}{\epsilon_0}\sin(\omega t)^2+\frac{\hbar\omega}{2\epsilon_0}
\\
\langle\hat B_y^2\rangle
&=2\bar n\hbar\omega\mu_0\cos(\omega t)^2+\frac{\hbar\omega\mu_0}{2}
\end{align*}

Rewrite as
\begin{align*}
\frac{\epsilon_0}{2}\langle\hat E_x^2\rangle&=\bar n\hbar\omega\sin(\omega t)^2+\frac{\hbar\omega}{4}
\\
\frac{1}{2\mu_0}\langle\hat B_y^2\rangle&=\bar n\hbar\omega\cos(\omega t)^2+\frac{\hbar\omega}{4}
\end{align*}

Hence the total energy per unit volume is
\begin{equation*}
U=\frac{\epsilon_0}{2}\langle\hat E_x^2\rangle
+\frac{1}{2\mu_0}\langle\hat B_y^2\rangle
=\left(\bar n+\tfrac{1}{2}\right)\hbar\omega
\end{equation*}

\end{document}
