\documentclass[12pt]{article}
\usepackage{amsmath}
\usepackage{amssymb}
\usepackage{mathrsfs} % \mathscr

\parindent=0pt

\begin{document}

3.29
For each family in Exercise 3.28, describe the natural parameter
space.

\bigskip
\noindent
(b) gamma family
\begin{eqnarray*}
\eta_1&=&\alpha-1\\
\eta_2&=&1/\beta\\
\bar c(\eta_1,\eta_2)&=&{\eta_2^{\eta_1+1}\over\Gamma(\eta_1+1)}
={1\over\Gamma(\alpha)\beta^\alpha}
\end{eqnarray*}
For
$$\int_0^\infty x^{\eta_1}\exp(-\eta_2x)\,dx<\infty$$
we must have $\eta_1>-1$ and $\eta_2>0$.
Therefore the natural parameter space is
$${\mathscr H}=\{(\eta_1,\eta_2):\eta_1\in(-1,\infty),\eta_2\in(0,\infty)\}$$

\bigskip
\noindent
(c) beta familiy
\begin{eqnarray*}
\eta_1&=&\alpha-1\\
\eta_2&=&\beta-1\\
\bar c(\eta_1,\eta_2)&=&
{\Gamma(\eta_1+\eta_2+2)\over\Gamma(\eta_1+1)\Gamma(\eta_2+1)}
={\Gamma(\alpha+\beta)\over\Gamma(\alpha)\Gamma(\beta)}
\end{eqnarray*}
In order to have
$$\int_0^1 x^{\eta_1}(1-x)^{\eta_2}\,dx<\infty$$
we must have $\eta_1,\eta_2>-1$.
Therefore the natural parameter space is
$$\mathscr H=\{(\eta_1,\eta_2):\eta_1,\eta_2\in(-1,\infty)\}$$

\bigskip
\noindent
(e) negative binomial
\begin{eqnarray*}
\eta&=&\log(1-p)\\
\bar c(\eta)&=&(1-e^\eta)^r=p^r
\end{eqnarray*}
In order to have
$$\sum_{x=0}^\infty {r+x-1\choose x}\exp(\eta x)<\infty$$
we must have $\eta<0$.
Therefore the natural parameter space is
$$\mathscr H=\{\eta:\eta\in(-\infty,0)\}$$

\end{document}
