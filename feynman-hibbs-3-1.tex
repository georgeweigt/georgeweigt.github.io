\documentclass[12pt]{article}
\usepackage{amsmath}

\parindent=0pt

\begin{document}

3-1.
The probability that a particle arrives at the point $b$
is by definition proportional to the absolute square of the kernel
$K(b,a)$.
For the free-particle kernel of equation (3.3) this is
\begin{equation*}
P(b)\,dx=\frac{m}{2\pi\hbar(t_b-t_a)}\,dx
\tag{3.6}
\end{equation*}

Clearly this is a relative probability, since the integral over the
complete range of $x$ diverges.
What does the particular normalization mean?
Show that this corresponds to a classical picture in which a particle
starts from the point $a$ with all momenta equally likely.
Show that the corresponding relative probability that the momentum
of the particle lies in the range $dp$ is $dp/2\pi\hbar$.

\bigskip
\hrule

\bigskip
Calculate the normalization constant $C$.
\begin{align*}
C&=\int_{x_a}^{x_b}\frac{m}{2\pi\hbar(t_b-t_a)}\,dx
\\[1ex]
&=\frac{m}{2\pi\hbar(t_b-t_a)}\,x\bigg|_{x_a}^{x_b}
\\[1ex]
&=\frac{m}{2\pi\hbar}\left(\frac{x_b-x_a}{t_b-t_a}\right)
\end{align*}

From $v=(x_b-x_a)/(t_b-t_a)$ and $p=mv$ we have
\begin{equation*}
C=\frac{p}{2\pi\hbar}
\end{equation*}

\noindent
Hence diverging normalization corresponds to unrestricted momentum $p$.

\bigskip
\noindent
From
\begin{equation*}
x+dx=\frac{p+dp}{m}(t_b-t_a)
\end{equation*}

\noindent
we have
\begin{equation*}
dx=\frac{dp}{m}(t_b-t_a)
\end{equation*}

\noindent
It follows that
\begin{equation*}
\frac{m}{2\pi\hbar(t_b-t_a)}\,dx=\frac{dp}{2\pi\hbar}
\end{equation*}

\end{document}
