\documentclass[12pt]{article}
\usepackage{amsmath}

\parindent=0pt

\begin{document}

Feynman and Hibbs problem 3-9

\bigskip
Find the kernel for a particle in a constant field $f$
where the Lagrangian is
\begin{equation*}
L=\frac{m}{2}\dot x^2+fx
\end{equation*}

From problem 2-3 we have
\begin{equation*}
S(b,a)=\frac{m(x_b-x_a)^2}{2T}+\frac{fT(x_b+x_a)}{2}-\frac{f^2T^3}{24m}
\end{equation*}
where $T=t_b-t_a$.

\bigskip
By equation (3.51) which is
\begin{equation*}
K(b,a)=F(T)\exp\left(\frac{iS(b,a)}{\hbar}\right)
\end{equation*}
we have
\begin{equation*}
K(b,a)=
F(T)\exp\left(
\frac{im(x_b-x_a)^2}{2\hbar T}
+\frac{ifT(x_b+x_a)}{2\hbar}
-\frac{if^2T^3}{24\hbar m}
\right)
\tag{1}
\end{equation*}

We now proceed to compute $F$. By equation (2.31) which is
\begin{equation*}
K(b,a)=\int_{-\infty}^\infty K(b,c)K(c,a)\,dx_c
\end{equation*}
we have
\begin{equation*}
K(b,a)=F(t_b-t_c)F(t_c-t_a)\int_{-\infty}^\infty
\exp\left(\frac{iS(b,c)}{\hbar}+\frac{iS(c,a)}{\hbar}\right)
\,dx_c
\end{equation*}

Reorganize as powers of $x_c$.
\begin{multline*}
K(b,a)=F(t_b-t_c)F(t_c-t_a)
\exp\left(-\frac{if^2(t_b-t_c)^3}{24\hbar m}-\frac{if^2(t_c-t_a)^3}{24\hbar m}\right)
\\[1ex]
{}\times
\int_{-\infty}^\infty
\exp\left(Ax_c^2+Bx_c+C\right)
\,dx_c
\tag{2}
\end{multline*}
where
\begin{align*}
A&=\frac{im}{2\hbar}\left(\frac{1}{t_b-t_c}+\frac{1}{t_c-t_a}\right)
\tag{3}
\\
B&=\frac{ifT}{2\hbar}-\frac{im}{\hbar}\left(\frac{x_b}{t_b-t_c}+\frac{x_a}{t_c-t_a}\right)
\tag{4}
\\
C&=\frac{if}{2\hbar}\big(x_b(t_b-t_c)+x_a(t_c-t_a)\big)+\frac{im}{2\hbar}
\left(\frac{x_b^2}{t_b-t_c}+\frac{x_a^2}{t_c-t_a}\right)
\tag{5}
\end{align*}

\bigskip
Note that the exponential involving $f^2$ is independent of $x_c$
and is factored out of the integrand in (2) by the distributive law.

\bigskip
Solve the integral in (2).
\begin{align*}
&\int_{-\infty}^{\infty}\exp(Ax_c^2+Bx_c+C)\,dx_c
=\left(-\frac{\pi}{A}\right)^{1/2}
\exp\left(-\frac{B^2}{4A}+C\right)
\\
&\quad{}=\left(-\frac{2\pi\hbar(t_b-t_c)(t_c-t_a)}{imT}\right)^{1/2}
\\
&\quad{}\times\exp\left(
\frac{im(x_b-x_a)^2}{2\hbar T}
+\frac{ifT(x_b+x_a)}{2\hbar}
-\frac{if^2T(t_b-t_c)(t_c-t_a)}{8\hbar m}
\right)
\tag{6}
\end{align*}

Substitute (6) into (2) to obtain
\begin{align*}
&K(b,a)=F(t_b-t_c)F(t_c-t_a)
\exp\left(-\frac{if^2(t_b-t_c)^3}{24\hbar m}-\frac{if^2(t_c-t_a)^3}{24\hbar m}\right)
\\
&\quad{}\times
\left(-\frac{2\pi\hbar(t_b-t_c)(t_c-t_a)}{imT}\right)^{1/2}
\\
&\quad{}\times
\exp\left(
\frac{im(x_b-x_a)^2}{2\hbar T}
+\frac{ifT(x_b+x_a)}{2\hbar}
-\frac{if^2T(t_b-t_c)(t_c-t_a)}{8\hbar m}
\right)
\end{align*}

Note that
\begin{equation*}
T^3=(t_b-t_c)^3+(t_c-t_a)^3+3T(t_b-t_c)(t_c-t_a)
\tag{7}
\end{equation*}

Use (7) to combine exponentials involving $f^2$.
\begin{align*}
&K(b,a)=F(t_b-t_c)F(t_c-t_a)
\\
&\quad{}\times
\left(-\frac{2\pi\hbar(t_b-t_c)(t_c-t_a)}{imT}\right)^{1/2}
\\
&\quad{}\times
\exp\left(
\frac{im(x_b-x_a)^2}{2\hbar T}
+\frac{ifT(x_b+x_a)}{2\hbar}
-\frac{if^2T^3}{24\hbar m}
\right)
\tag{8}
\end{align*}

Equating (1) with (8) cancels the exponentials and leaves
\begin{equation*}
F(T)=F(t_b-t_c)F(t_c-t_a)
\left(-\frac{2\pi\hbar(t_b-t_c)(t_c-t_a)}{imT}\right)^{1/2}
\tag{9}
\end{equation*}

From problem 3-7, let
\begin{equation*}
F(t)=\left(\frac{m}{2\pi i\hbar t}\right)^{1/2} g(t)
\tag{10}
\end{equation*}

Substitute (10) into (9) to obtain
\begin{multline*}
\left(\frac{m}{2\pi i\hbar T}\right)^{1/2} g(T)
=\left(\frac{m}{2\pi i\hbar(t_b-t_c)}\right)^{1/2} g(t_b-t_c)
\\
{}\times
\left(\frac{m}{2\pi i\hbar(t_c-t_a)}\right)^{1/2} g(t_c-t_a)
\left(-\frac{2\pi\hbar(t_b-t_c)(t_c-t_a)}{imT}\right)^{1/2}
\end{multline*}

The coefficients cancel leaving
\begin{equation*}
g(T)=g(t_b-t_c)g(t_c-t_a)
\tag{11}
\end{equation*}

Hence
\begin{equation*}
g(t)=1
\end{equation*}
and
\begin{equation*}
F(T)=\left(\frac{m}{2\pi i\hbar T}\right)^{1/2}
\tag{12}
\end{equation*}

Substitute (12) into (1).
\begin{equation*}
K(b,a)=\left(\frac{m}{2\pi i\hbar T}\right)^{1/2}
\exp\left(
\frac{im(x_b-x_a)^2}{2\hbar T}
+\frac{ifT(x_b+x_a)}{2\hbar}
-\frac{if^2T^3}{24\hbar m}
\right)
\end{equation*}

\end{document}
