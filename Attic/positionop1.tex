\documentclass[12pt]{article}
\usepackage[margin=2cm]{geometry}
\usepackage{amsmath}

\begin{document}

\noindent
Consider the following eigenstates of a hypothetical quantum system.
\begin{align*}
|00\rangle&=(\text{1 0 0 0})^\dag\qquad\text{no fermions}\\
|10\rangle&=(\text{0 1 0 0})^\dag\qquad\text{one fermion in state $\phi_1$}\\
|01\rangle&=(\text{0 0 1 0})^\dag\qquad\text{one fermion in state $\phi_2$}\\
|11\rangle&=(\text{0 0 0 1})^\dag\qquad\text{two fermions, one in state $\phi_1$, one in state $\phi_2$}
\end{align*}

\noindent
Let fermion states $\phi_n$ be modeled by a one dimensional box of length $L$.
\begin{equation*}
\phi_n(x)=\sqrt{\frac{2}{L}}\sin\left(\frac{n\pi x}{L}\right)
\end{equation*}

\noindent
Creation and annihilation operators are formed from outer products of state vectors.
Sign changes make the operators antisymmetric.
\begin{align*}
\hat{b}_1^\dag&=|10\rangle\langle00|-|11\rangle\langle01| \qquad\text{Create one fermion in state $\phi_1$}
\\
\hat{b}_1&=|00\rangle\langle10|-|01\rangle\langle11| \qquad\text{Annihilate one fermion in state $\phi_1$}
\\
\hat{b}_2^\dag&=|01\rangle\langle00|+|11\rangle\langle10| \qquad\text{Create one fermion in state $\phi_2$}
\\
\hat{b}_2&=|00\rangle\langle01|+|10\rangle\langle11| \qquad\text{Annihilate one fermion in state $\phi_2$}
\end{align*}

\noindent
Let $\hat{r}$ be the position operator
\begin{equation*}
\hat{r}=\sum_{n,m}r_{nm}\hat{b}_n^\dag\hat{b}_m
\end{equation*}

\noindent
where
\begin{equation*}
r_{nm}=\int_0^L\phi_n^*(x)x\phi_m(x)\,dx
\end{equation*}

\noindent
Note that for a one dimensional box
\begin{equation*}
r_{nn}=\langle x\rangle=\tfrac{1}{2}L
\end{equation*}

\noindent
Verify that
\begin{align*}
\langle10|\hat{r}|10\rangle&=r_{11}\\
\langle10|\hat{r}|01\rangle&=r_{12}\\
\langle01|\hat{r}|10\rangle&=r_{21}\\
\langle01|\hat{r}|01\rangle&=r_{22}
\end{align*}

\end{document}
