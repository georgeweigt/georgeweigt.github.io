\documentclass[12pt]{article}
\usepackage[margin=2cm]{geometry}
\usepackage{amsmath}

\begin{document}

\noindent
Show that
\begin{equation*}
\int\frac{\psi_1^2\psi_2^2}{r_{12}}\,dV_1\,dV_2=\tfrac{5}{8}\alpha
\end{equation*}
where
\begin{equation*}
\psi_j=\sqrt{\frac{\alpha^3}{\pi}}\exp\left(-\alpha r_j\right)
\end{equation*}
Symbol $r_{12}$ is the following distance function where $\theta_{12}$ is angular separation.
\begin{equation*}
r_{12}=\sqrt{r_1^2+r_2^2-r_1r_2\cos\theta_{12}}
\end{equation*}

\noindent
Let $I(r_1)$ be the following integral over $V_2$.
\begin{equation*}
I(r_1)=\int\frac{\psi_2^2}{r_{12}}\,dV_2
\end{equation*}
The measure $dV_2$ is a volume element in spherical coordinates.
\begin{equation*}
dV_2=r_2^2\sin\theta_2\,dr_2\,d\theta_2\,d\phi_2
\end{equation*}

\noindent
Write out the full integral and make $\theta_2=\theta_{12}$ by independence of the coordinate system.
\begin{equation*}
I(r_1)=\frac{\alpha^3}{\pi}
\int\limits_0^{2\pi}\int\limits_0^\pi\int\limits_0^\infty
\frac{\exp(-2\alpha r_2)}{\sqrt{r_1^2+r_2^2-r_1r_2\cos\theta_2}}
\,r_2^2\sin\theta_2\,dr_2\,d\theta_2\,d\phi_2
\end{equation*}

\noindent
Integrate over $\phi_2$.
\begin{equation*}
I(r_1)=
2\alpha^3\int\limits_0^\pi\int\limits_0^\infty
\frac{\exp(-2\alpha r_2)}{\sqrt{r_1^2+r_2^2-r_1r_2\cos\theta_2}}
\,r_2^2\sin\theta_2\,dr_2\,d\theta_2
\end{equation*}

\noindent
Expand $1/r_{12}$ in Legendre polynomials.
The first integral is over $r_2<r_1$ and the second integral is over $r_2>r_1$.
\begin{multline*}
I(r_1)=
2\alpha^3\int\limits_0^\pi\int\limits_0^{r_1}
\exp(-2\alpha r_2)
\left(\sum_{k=0}^\infty\frac{r_2^k}{r_1^{k+1}}P_k(\cos\theta_2)\right)
r_2^2\sin\theta_2\,dr_2\,d\theta_2
\\
+2\alpha^3\int\limits_0^\pi\int\limits_{r_1}^\infty
\exp(-2\alpha r_2)
\left(\sum_{k=0}^\infty\frac{r_1^k}{r_2^{k+1}}P_k(\cos\theta_2)\right)
r_2^2\sin\theta_2\,dr_2\,d\theta_2
\end{multline*}

\noindent
It turns out that, after integrating over $\theta_2$, all summands vanish except for $k=0$.
\begin{equation*}
\int\limits_0^\pi P_k(\cos\theta_2)\sin\theta_2\,d\theta_2=
\left\{
\begin{aligned}
&2, & k=0
\\
&0, & k>0
\end{aligned}\right.
\end{equation*}

\noindent
Hence
\begin{equation*}
I(r_1)=
\frac{4\alpha^3}{r_1}\int\limits_0^{r_1}\exp(-2\alpha r_2)\,r_2^2\,dr_2
+4\alpha^3\int\limits_{r_1}^\infty\exp(-2\alpha r_2)\,r_2\,dr_2
\end{equation*}

\noindent
Solve the integrals.
\begin{equation*}
I(r_1)=
\frac{4\alpha^3}{r_1}
\left.
\exp(-2\alpha r_2)\left(-\frac{r_2^2}{2\alpha}-\frac{r_2}{2\alpha^2}-\frac{1}{4\alpha^3}
\right)\right|_0^{r_1}
+4\alpha^3\left.\exp(-2\alpha r_2)\left(-\frac{r_2}{2\alpha}-\frac{1}{4\alpha^2}\right)\right|_{r_1}^\infty
\end{equation*}

\noindent
Evaluate per limits.
\begin{equation*}
I(r_1)=\frac{1}{r_1}-\frac{1}{r_1}\exp(-2\alpha r_1)-\alpha\exp(-2\alpha r_1)
\end{equation*}

\noindent
Having obtained $I(r_1)$ we can now evaluate the integral over $V_1$.
\begin{equation*}
I=\frac{\alpha^3}{\pi}\int\limits_0^{2\pi}\int\limits_0^\pi\int\limits_0^\infty
\exp(-2\alpha r_1)I(r_1)\,r_1^2\sin\theta_1\,dr_1\,d\theta_1\,d\phi_1
\end{equation*}

\noindent
Integrate over $\theta_1$ and $\phi_1$.
\begin{equation*}
I=4\alpha^3\int\limits_0^\infty
\exp(-2\alpha r_1)I(r_1)\,r_1^2\,dr_1
\end{equation*}

\noindent
Expand the integrand.
\begin{equation*}
I=4\alpha^3\int\limits_0^\infty\exp(-2\alpha r_1)\,r_1\,dr_1
-4\alpha^3\int\limits_0^\infty\exp(-4\alpha r_1)\,r_1\,dr_1
-4\alpha^4\int\limits_0^\infty\exp(-4\alpha r_1)\,r_1^2\,dr_1
\end{equation*}

\noindent
Solve the integrals.
\begin{multline*}
I=\exp(-2\alpha r_1)\left(-2\alpha^2r_1-\alpha\right)\bigg|_0^\infty
\\
{}-\exp(-4\alpha r_1)\left(-\alpha^2r_1-\tfrac{1}{4}\alpha\right)\bigg|_0^\infty
{}-\exp(-4\alpha r_1)\left(-\alpha^3r_1^2-\tfrac{1}{2}\alpha^2r_1-\tfrac{1}{8}\alpha\right)\bigg|_0^\infty
\end{multline*}

\noindent
The result vanishes for $r_1=\infty$ hence
\begin{equation*}
I=0-\left(-\alpha+\tfrac{1}{4}\alpha+\tfrac{1}{8}\alpha\right)=\tfrac{5}{8}\alpha
\end{equation*}

\end{document}
