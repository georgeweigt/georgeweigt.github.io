\documentclass[12pt]{article}
\usepackage[margin=2cm]{geometry}
\usepackage{amsmath}

\begin{document}

\noindent
Let $\hat{\psi}$ be a wavefunction operator such that
\begin{equation*}
\hat{\psi}|a\rangle=\psi(x,y)|0\rangle
\end{equation*}

\noindent
Note that
\begin{equation*}
\langle0|\hat{\psi}|a\rangle=\psi(x,y)
\end{equation*}
and consequently
\begin{equation*}
\langle a|\hat{\psi}^\dag|0\rangle=\psi^*(x,y)
\end{equation*}

\noindent
It follows that
\begin{equation*}
\langle E\rangle=\int
\langle a|\hat{\psi}^\dag|0\rangle
\hat{H}
\langle0|\hat{\psi}|a\rangle
\,dx\,dy
\end{equation*}

\noindent
Let $\hat{E}$ be the operator
\begin{equation*}
\hat{E}=\int\hat{\psi}^\dag|0\rangle\hat{H}\langle0|\hat{\psi}\,dx\,dy
\end{equation*}

\noindent
Then
\begin{equation*}
\langle E\rangle=\langle a|\hat{E}|a\rangle
\end{equation*}

\noindent
Let $|\xi\rangle$ be a linear combination of state vectors.
\begin{equation*}
|\xi\rangle=\sum_kc_k|k\rangle,\qquad\langle\xi|\xi\rangle=1
\end{equation*}

\noindent
The expected energy can be determined by matrix multiplication.
\begin{equation*}
\langle E\rangle=\langle\xi|\hat{E}|\xi\rangle=\sum_k|c_k|^2\langle E_k\rangle
\end{equation*}

\end{document}
