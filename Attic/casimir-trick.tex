\documentclass[12pt]{article}
\usepackage{fancyvrb,amsmath,amsfonts,amssymb,graphicx,listings}
\usepackage[usenames,dvipsnames,svgnames,table]{xcolor}
%\usepackage{parskip}
\usepackage{slashed}
\usepackage{mathrsfs}

% change margins
\addtolength{\oddsidemargin}{-.875in}
\addtolength{\evensidemargin}{-.875in}
\addtolength{\textwidth}{1.75in}
\addtolength{\topmargin}{-.875in}
\addtolength{\textheight}{1.75in}

\hyphenpenalty=10000

\begin{document}

\begin{center}
{\sc casimir trick}
\end{center}

\noindent
The Casimir trick is an efficient way to compute probability densities summed over spin states.
The trick is to replace sums of products with matrix products.
In the following example, it is faster to compute the probability density $\langle|\mathcal{M}|^2\rangle$
by evaluating products of matrices $\slashed{p}+m$.
\begin{align*}
\langle|\mathcal{M}|^2\rangle
&=
\frac{e^4}{4}\sum_{s_1=1}^2\sum_{s_2=1}^2\sum_{s_3=1}^2\sum_{s_4=1}^2
\big|(\bar{u}_3\gamma^\mu v_4)(\bar{v}_2\gamma_\mu u_1)\big|^2\\
&=
\frac{e^4}{4}
\mathop{\rm Tr}\left[(\slashed{p}_3+m_3)\gamma^\mu(\slashed{p}_4-m_4)\gamma^\nu\right]
\mathop{\rm Tr}\left[(\slashed{p}_2-m_2)\gamma_\mu(\slashed{p}_1+m_1)\gamma_\nu\right]
\end{align*}
Index $s_j$ selects the spin of $u_j$ or $v_j$.
The spinors are
\begin{gather*}
u_{11}=\begin{pmatrix}E_1+m_1\\0\\p_{1z}\\p_{1x}+ip_{1y}\end{pmatrix}\quad
v_{21}=\begin{pmatrix}p_{2z}\\p_{2x}+ip_{2y}\\E_2+m_2\\0\end{pmatrix}\quad
u_{31}=\begin{pmatrix}E_3+m_3\\0\\p_{3z}\\p_{3x}+ip_{3y}\end{pmatrix}\quad
v_{41}=\begin{pmatrix}p_{4z}\\p_{4x}+ip_{4y}\\E_4+m_4\\0\end{pmatrix}\\
u_{12}=\begin{pmatrix}0\\E_1+m_1\\p_{1x}-ip_{1y}\\-p_{1z}\end{pmatrix}\quad
v_{22}=\begin{pmatrix}p_{2x}-ip_{2y}\\-p_{2z}\\0\\E_2+m_2\end{pmatrix}\quad
u_{32}=\begin{pmatrix}0\\E_3+m_3\\p_{3x}-ip_{3y}\\-p_{3z}\end{pmatrix}\quad
v_{42}=\begin{pmatrix}p_{4x}-ip_{4y}\\-p_{4z}\\0\\E_4+m_4\end{pmatrix}
\end{gather*}
where the second digit of the subscript is the spin state (1 up, 2 down).
The momentum vectors are
$$
p_1=\begin{pmatrix}E_1\\p_{1x}\\p_{1y}\\p_{1z}\end{pmatrix}\quad
p_2=\begin{pmatrix}E_2\\p_{2x}\\p_{2y}\\p_{2z}\end{pmatrix}\quad
p_3=\begin{pmatrix}E_3\\p_{3x}\\p_{3y}\\p_{3z}\end{pmatrix}\quad
p_4=\begin{pmatrix}E_4\\p_{4x}\\p_{4y}\\p_{4z}\end{pmatrix}
$$
%
Run ``casimir-trick.txt'' to verify the Casimir trick for the process shown above.
Here are a few details about how the script works.
In component notation the spin state amplitude is
$$
\mathcal{M}
=
(\bar{u}_3\gamma^\mu v_4)(\bar{v}_2\gamma_\mu u_1)
=
\big(\bar{u}_3{}_\alpha\gamma^{\mu\alpha}{}_\beta v_4{}^\beta\big)
\big(\bar{v}_2{}_\rho\gamma_\mu{}^\rho{}_\sigma u_1{}^\sigma\big)
$$
%
To convert this to Eigenmath code,
the $\gamma$ tensors need to be transposed
so that repeated indices are adjacent to each other.
Also, multiply $\gamma^\mu$ by the metric tensor to lower the index.
\begin{align*}
\gamma^{\alpha\mu}{}_\beta\quad&\rightarrow\quad
\text{\tt gammaT = transpose(gamma)}\\
\gamma^\rho{}_\mu{}_\sigma\quad&\rightarrow\quad
\text{\tt gammaL = transpose(dot(gmunu,gamma))}
\end{align*}
%
Then
\begin{align*}
\bar{u}_3{}_\alpha\gamma^{\mu\alpha}{}_\beta v_4{}^\beta
\quad&\rightarrow\quad
\text{\tt X34 = dot(u3bar[s3],gammaT,v4[s4])}
\\
\bar{v}_2{}_\rho\gamma_\mu{}^\rho{}_\sigma u_1{}^\sigma
\quad&\rightarrow\quad
\text{\tt X21 = dot(v2bar[s2],gammaL,u1[s1])}
\end{align*}
Hence
$$
\mathcal{M}=(\cdots\gamma^\mu\cdots)(\cdots\gamma_\mu\cdots)
\quad\rightarrow\quad
\text{\tt dot(X34,X21)}
$$
%
In component notation the traces become sums over the repeated index $\alpha$.
\begin{align*}
\mathop{\rm Tr}\left[(\slashed{p}_3+m_3)\gamma^\mu(\slashed{p}_4-m_4)\gamma^\nu\right]
&=
(\slashed{p}_3+m_3)^\alpha{}_\beta
\gamma^{\mu\beta}{}_\rho
(\slashed{p}_4-m_4)^\rho{}_\sigma
\gamma^{\nu\sigma}{}_\alpha
\\
\mathop{\rm Tr}\left[(\slashed{p}_2-m_2)\gamma_\mu(\slashed{p}_1+m_1)\gamma_\nu\right]
&=
(\slashed{p}_2-m_2)^\alpha{}_\beta
\gamma_\mu{}^\beta{}_\rho
(\slashed{p}_1+m_1)^\rho{}_\sigma
\gamma_\nu{}^\sigma{}_\alpha
\end{align*}
%
Define the following $4\times4$ matrices.
\begin{align*}
(\slashed{p}_1+m_1)\quad&\rightarrow\quad\text{\tt X1 = pslash1 + m1 I}\\
(\slashed{p}_2-m_2)\quad&\rightarrow\quad\text{\tt X2 = pslash2 - m2 I}\\
(\slashed{p}_3+m_3)\quad&\rightarrow\quad\text{\tt X3 = pslash3 + m3 I}\\
(\slashed{p}_4-m_4)\quad&\rightarrow\quad\text{\tt X4 = pslash4 - m4 I}
\end{align*}
%
Then
\begin{align*}
(\slashed{p}_3+m_3)^\alpha{}_\beta
\gamma^{\mu\beta}{}_\rho
(\slashed{p}_4-m_4)^\rho{}_\sigma
\gamma^{\nu\sigma}{}_\alpha
\quad&\rightarrow\quad
\text{\tt T1 = contract(dot(X3,gammaT,X4,gammaT),1,4)}
\\
(\slashed{p}_2-m_2)^\alpha{}_\beta
\gamma_\mu{}^\beta{}_\rho
(\slashed{p}_1+m_1)^\rho{}_\sigma
\gamma_\nu{}^\sigma{}_\alpha
\quad&\rightarrow\quad
\text{\tt T2 = contract(dot(X2,gammaL,X1,gammaL),1,4)}
\end{align*}
Next, multiply the matrices and sum over repeated indices.
The dot function sums over $\nu$ then the contract function
sums over $\mu$. The transpose makes the $\nu$ indices adjacent
as required by the dot function.
$$
\mathop{\rm Tr}[\cdots\gamma^\mu\cdots\gamma^\nu]\mathop{\rm Tr}[\cdots\gamma_\mu\cdots\gamma_\nu]
\quad\rightarrow\quad
\text{\tt contract(dot(T1,transpose(T2)))}
$$

\end{document}
