\documentclass[12pt]{article}
\usepackage[margin=2cm]{geometry}
\usepackage{amsmath}

\begin{document}

\noindent
Consider the following eigenstates of a hypothetical quantum system.
\begin{align*}
|000\rangle&=(\text{1 0 0 0 0 0 0})^\dag\qquad\text{no fermions}\\
|100\rangle&=(\text{0 1 0 0 0 0 0})^\dag\qquad\text{one fermion in state $\phi_1$}\\
|010\rangle&=(\text{0 0 1 0 0 0 0})^\dag\qquad\text{one fermion in state $\phi_2$}\\
|001\rangle&=(\text{0 0 0 1 0 0 0})^\dag\qquad\text{one fermion in state $\phi_3$}\\
|110\rangle&=(\text{0 0 0 0 1 0 0})^\dag\qquad\text{two fermions, one in state $\phi_1$, one in state $\phi_2$}\\
|101\rangle&=(\text{0 0 0 0 0 1 0})^\dag\qquad\text{two fermions, one in state $\phi_1$, one in state $\phi_3$}\\
|011\rangle&=(\text{0 0 0 0 0 0 1})^\dag\qquad\text{two fermions, one in state $\phi_2$, one in state $\phi_3$}
\end{align*}

\noindent
Let fermion states $\phi_n$ be modeled by a one dimensional box of length $L$.
\begin{equation*}
\phi_n(x)=\sqrt{\frac{2}{L}}\sin\left(\frac{n\pi x}{L}\right)
\end{equation*}

\noindent
Fermion creation operators are formed from outer products of state vectors.
Sign changes make the operators antisymmetric.
\begin{align*}
\hat{b}_1^\dag&=|100\rangle\langle000|-|110\rangle\langle010|-|101\rangle\langle001|
\qquad\text{Create one fermion in state $\phi_1$}\\
\hat{b}_2^\dag&=|010\rangle\langle000|+|110\rangle\langle100|-|011\rangle\langle001|
\qquad\text{Create one fermion in state $\phi_2$}\\
\hat{b}_3^\dag&=|001\rangle\langle000|+|101\rangle\langle100|+|011\rangle\langle010|
\qquad\text{Create one fermion in state $\phi_3$}
\end{align*}

\noindent
Fermion annihilation operators are the adjoint of creation operators.
\begin{align*}
\hat{b}_1&=(\hat{b}_1^\dag)^\dag\qquad\text{Annihilate one fermion in state $\phi_1$}\\
\hat{b}_2&=(\hat{b}_2^\dag)^\dag\qquad\text{Annihilate one fermion in state $\phi_2$}\\
\hat{b}_3&=(\hat{b}_3^\dag)^\dag\qquad\text{Annihilate one fermion in state $\phi_3$}
\end{align*}

\noindent
Given the wavefunction operator
\begin{equation*}
\hat{\psi}=\frac{1}{\sqrt{2}}\sum_{n,m}\phi_n(x)\phi_m(y)\hat{b}_n\hat{b}_m
\end{equation*}

\noindent
show that
\begin{align*}
\hat{\psi}|110\rangle=\frac{1}{\sqrt{2}}\big(\phi_1(x)\phi_2(y)-\phi_1(y)\phi_2(x)\big)|000\rangle
\\[2ex]
\hat{\psi}|101\rangle=\frac{1}{\sqrt{2}}\big(\phi_1(x)\phi_3(y)-\phi_1(y)\phi_3(x)\big)|000\rangle
\\[2ex]
\hat{\psi}|011\rangle=\frac{1}{\sqrt{2}}\big(\phi_2(x)\phi_3(y)-\phi_2(y)\phi_3(x)\big)|000\rangle
\end{align*}

\end{document}
