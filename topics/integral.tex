\documentclass[12pt]{article}
\usepackage[margin=1in]{geometry}
\usepackage{amsmath}
\usepackage{amssymb} % mathbb
\usepackage{graphicx}
\usepackage{xcolor}
\parindent=0pt
\begin{document}

\section*{Integral}

\verb$integral(f,x)$ returns the integral of $f$ with respect to $x$.

{\color{blue}
\begin{verbatim}
integral(x^2,x)
\end{verbatim}
}

$\displaystyle \tfrac{1}{3}x^3$

\bigskip
Extend the argument list for multiple integrals.

{\color{blue}
\begin{verbatim}
f = x y
integral(f,x,y)
\end{verbatim}
}

$\displaystyle \tfrac{1}{4}x^2y^2$

\bigskip
\verb$defint(f,x,a,b)$
computes the definite integral of $f$ with respect to $x$ evaluated from
$a$ to $b$.
The argument list can be extended for multiple integrals.
The following example computes the integral of $f=x^2$
over the domain of a semicircle.
For each $x$ along the abscissa, $y$ ranges from 0 to $\sqrt{1-x^2}$.

{\color{blue}
\begin{verbatim}
defint(x^2, y, 0, sqrt(1 - x^2), x, -1, 1)
\end{verbatim}
}

$\displaystyle \tfrac{1}{8}\pi$

\bigskip
Alternatively, \verb$eval$ can be used to compute a definite integral step by step.

{\color{blue}
\begin{verbatim}
I = integral(x^2,y)
I = eval(I,y,sqrt(1 - x^2)) - eval(I,y,0)
I = integral(I,x)
eval(I,x,1) - eval(I,x,-1)
\end{verbatim}
}

$\displaystyle \tfrac{1}{8}\pi$

\bigskip
Here is a useful trick.
Integrals involving sine and cosine
can often be solved using exponentials.
For example, the definite integral
\begin{equation*}
\int_0^{2\pi}\left(\sin^4t-2\cos^3(t/2)\sin t\right)dt
\end{equation*}

can be solved as follows.

{\color{blue}
\begin{verbatim}
f = sin(t)^4 - 2 cos(t/2)^3 sin(t)
f = circexp(f)
defint(f, t, 0, 2 pi)
\end{verbatim}
}

$\displaystyle \tfrac{3}{4}\pi-\tfrac{16}{5}$

\end{document}
