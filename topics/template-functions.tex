\documentclass[12pt]{article}
\usepackage[margin=1in]{geometry}
\usepackage{amsmath}
\usepackage{amssymb} % mathbb
\usepackage{graphicx}
\usepackage{xcolor}
\parindent=0pt
\begin{document}

\section*{Template functions}

Function $f$ in \verb$d(f,x)$ does not have to be defined,
it can be a template function with just a name and an argument list.
The argument list determines the result.
For example, \verb$d(f(x),x)$ evaluates to itself because $f$ depends on $x$.
However, \verb$d(f(x),y)$ evaluates to zero because $f$ does not depend on $y$.

\bigskip
Example 1. $f(x)$ depends on $x$.

{\color{blue}
\begin{verbatim}
d(f(x),x)
\end{verbatim}}

$\operatorname{d}(f(x),x)$

\bigskip
Example 2. $f(x)$ does not depend on $y$.

{\color{blue}
\begin{verbatim}
d(f(x),y)
\end{verbatim}}

$0$

\bigskip
Example 3. $f(x,y)$ depends on both $x$ and $y$.

{\color{blue}
\begin{verbatim}
d(f(x,y),y)
\end{verbatim}}

$\operatorname{d}(f(x,y),y)$

\bigskip
Example 4. $f()$ is a wildcard that matches any symbol.

{\color{blue}
\begin{verbatim}
d(f(),t)
\end{verbatim}}

$\operatorname{d}(f(),t)$

\bigskip
Template functions are useful for experimenting with differential forms.
For example, verify the identity
\begin{equation*}
\nabla\cdot(\nabla\times\mathbf F)=0
\end{equation*}

{\color{blue}
\begin{verbatim}
F = (Fx(),Fy(),Fz())
div(curl(F))
\end{verbatim}}

$0$

\end{document}
