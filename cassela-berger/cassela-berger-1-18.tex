\documentclass[12pt]{article}
\usepackage{amsmath}
\usepackage{amssymb}

\parindent=0pt

\begin{document}

1.18
If $n$ balls are placed at random into $n$ cells, find the probability that
exactly one cell remains empty.

\bigskip
\noindent
The key to this problem is to visualize the end result.
If exactly one cell is empty, then one cell has two balls and the
remaining cells have one ball.
There are $\binom{n}{2}$ ways of choosing the two special cells.
For the cell with two balls, there are $n$ ways of choosing the
first ball and $n-1$ ways of choosing the second ball.
For the remaining $n-2$ balls, there are $(n-2)!$ ways of placing
them in the remaining $n-2$ cells.
(Alternatively, there are $n!$ ways that imaginary
ball labels can be interchanged after the balls are placed in cells.)
Next we need to compute the total number of ways $n$ balls can be
placed in $n$ cells.
For each ball, there are $n$ ways of choosing the cell.
Therefore there are $n\times n\times\cdots=n^n$
possible configurations.
Hence the probability that exactly one cell is empty is
\[
\binom{n}{2}\times n\times (n-1)\times(n-2)!\times\frac{1}{n^n}
=
\binom{n}{2}\frac{n!}{n^n}
\]

\end{document}