\documentclass[12pt]{article}
\usepackage{amsmath}
\usepackage{amssymb} % \therefore
\parindent=0pt
\begin{document}

{\it The Scattering of $\alpha$ and $\beta$ Particles by Matter and the Structure of the Atom.
By Professor E. Rutherford, F.R.S.,
University of Manchester.}

\bigskip
\S 1. It is well known that the $\alpha$ and the $\beta$ particles suffer
deflexions from their rectilinear paths by encounters with atoms of matter.
This scattering is far more marked for the $\beta$ than for the $\alpha$
particle on account of the much smaller momentum and energy of the former
particle. There seems to be no doubt that such swiftly moving particles pass
through the atoms in their path, and that the deflexions observed are due to
the strong electric field traversed within the atomic system.
It has generally been supposed that the scattering of a pencil of $\alpha$
or $\beta$ rays in passing through a thin plate of matter is the result of a
multitude of small scatterings by the atoms of matter traversed.
The observations, however, of Geiger and Marsden\footnote{Proc. Roy. Soc. lxxxii, p. 495 (1909).}
on the scattering of
$\alpha$ rays indicate that some of the $\alpha$ particles,
about 1 in 20,000 were turned through an average angle of 90 degrees in
passing though a layer of gold-foil about 0.00004 cm.~thick,
which was equivalent in stopping-power of the $\alpha$ particle to
1.6 millimetres of air.
Geiger\footnote{Proc. Roy. Soc. lxxxiii, p. 492 (1910).}
showed later that the most probable angle of deflexion for a
pencil of $\alpha$ particles being deflected through 90 degrees is
vanishingly small.
In addition, it will be seen later that the distribution of the $\alpha$
particles for various angles of large deflexion does not follow the
probability law to be expected if such large deflexion are made up of a
large number of small deviations.
It seems reasonable to suppose that the deflexion through a large angle is
due to a single atomic encounter, for the chance of a second encounter of a
kind to produce a large deflexion must in most cases be exceedingly small.
A simple calculation shows that the atom must be a seat of an intense electric
field in order to produce such a large deflexion at a single encounter.

\bigskip
Recently Sir J. J. Thomson\footnote{Camb.~Lit.~\& Phil Soc.~xv pt.~5 (1910).}
has put forward a theory to
explain the scattering of electrified particles in passing through small thicknesses of matter.
The atom is supposed to consist of a number $N$ of negatively charged corpuscles, accompanied
by an equal quantity of positive electricity uniformly distributed throughout a sphere.
The deflexion of a negatively electrified particle in passing through the atom is ascribed to two causes
(1) the repulsion of the corpuscles distributed through the atom,
and (2) the attraction of the positive electricity in the atom.
The deflexion of the particle in passing through the atom is supposed to be small,
while the average deflexion after a large number $m$ of encounters was taken as $\sqrt m\cdot\theta$,
where $\theta$ is the average deflexion due to a single atom. It was shown that the number $N$ of the electrons within the atom could be deduced from observations of the scattering was examined experimentally by
Crowther\footnote{Crowther, Proc.~Roy.~Soc.~lxxxiv.~p.~226 (1910).}
in a later paper.
His results apparently confirmed the main conclusions of the theory, and he deduced, on the assumption that the positive electricity was continuous, that the number of electrons in an atom was about three times its atomic weight.

\bigskip
The theory of Sir J.~J.~Thomson is based on the assumption that the scattering due to a single atomic encounter is small,
and the particular structure assumed for the atom does not admit of a very large deflexion of an
$\alpha$ particle in traversing a single atom, unless it be supposed that the diameter of the sphere of positive
electricity is minute compared with the diameter of the sphere of influence of the atom.

\bigskip
Since the $\alpha$ and $\beta$ particles traverse the atom, it should be possible from a close study of the nature
of the deflexion to form some idea of the constitution of the atom to produce the effects observed. In fact,
the scattering of high-speed charged particles by the atoms of matter is one of the most promising methods of attack
of this problem.
The development of the scintillation method of counting single $\alpha$ particles affords unusual advantages of
investigation, and the researches of H.~Geiger by this method have already added much to our knowledge of the
scattering of $\alpha$ rays by matter.

\bigskip
\S 2.
We shall first examine theoretically the single encounters\footnote{The deviation of a particle throughout a considerable angle from an encounter with a single atom will in this paper be called `single' scattering. The deviation of a particle resulting from a multitude of small deviations will be termed `compound' scattering.}
with an atom of simple structure, which is able to
produce large deflections of an $\alpha$ particle, and then compare the deductions from the theory with the
experimental data available.

\bigskip
Consider an atom which contains a charge $\pm Ne$ at its centre surrounded by a sphere of electrification containing
a charge $\pm Ne$
[N.B. in the original publication, the second plus/minus sign is inverted to be a minus/plus sign]
supposed uniformly distributed throughout a sphere of radius $R$.
$e$ is the fundamental unit of charge, which in this paper is taken as $4.65\times10^{-10}$ E.S.~unit.
We shall suppose that for distances less than $10^{-12}$ cm.~the central charge and also the charge on the alpha
particle may be supposed to be concentrated at a point.
It will be shown that the main deductions from the theory are independent of whether the central charge is supposed to be positive or negative.
For convenience, the sign will be assumed to be positive.
The question of the stability of the atom proposed need not be considered at this stage,
for this will obviously depend upon the minute structure of the atom,
and on the motion of the constituent charged parts.

\bigskip
In order to form some idea of the forces required to deflect an alpha particle through a large angle,
consider an atom containing a positive charge $Ne$ at its centre, and surrounded by a distribution of negative
electricity $Ne$ uniformly distributed within a sphere of radius $R$.
The electric force $X$ and the potential $V$ at a distance $r$ from the centre of an atom for a point inside the atom,
are given by
\begin{align*}
X&=Ne\left(\frac{1}{r^2}-\frac{r}{R^3}\right)
\\
V&=Ne\left(\frac{1}{r}-\frac{3}{2R}+\frac{r^2}{2R^3}\right)
\end{align*}

Suppose an $\alpha$ particle of mass $m$ and velocity $u$ and charge $E$ shot directly towards the centre of the atom.
It will be brought to rest at a distance $b$ from the centre given by
\begin{equation*}
\tfrac{1}{2}mu^2=NeE\left(\frac{1}{b}-\frac{3}{2R}+\frac{b^2}{2R^3}\right)
\end{equation*}

It will be seen that $b$ is an important quantity in later calculations.
Assuming that the central charge is $100e$, it can be calculated that the value of $b$ for an $\alpha$
particle of velocity $2.09\times10^9$ cms.~per second is about
$3.4\times10^{-12}$ cm.
In this calculation $b$ is supposed to be very small compared with $R$.
Since $R$ is supposed to be of the order of the radius of the atom, viz.~$10^{-8}$ cm.,
it is obvious that the $\alpha$ particle before being turned back penetrates so close to
the central charge, that the field due to the uniform distribution of negative electricity may be neglected.
In general, a simple calculation shows that for all deflexions greater than a degree,
we may without sensible error suppose the deflexion due to the field of the central charge alone.
Possible single deviations due to the negative electricity,
if distributed in the form of corpuscles, are not taken into account at this stage of the theory.
It will be shown later that its effect is in general small compared with that due to the central field.

\bigskip
Consider the passage of a positive electrified particle close to the centre of an atom.
Supposing that the velocity of the particle is not appreciably changed by its passage through the atom,
the path of the particle under the influence of a repulsive force varying inversely as the square of the
distance will be an hyperbola with the centre of the atom $S$ as the external focus.
Suppose the particle to enter the atom in the direction $PO$ (fig. 1),
and that the direction of motion
on escaping the atom is $OP'$.
$OP$ and $OP'$ make equal angles with the line $SA$,
where $A$ is the apse of the hyperbola.
$p=|S-N|=\text{perpendicular}$ distance from centre on direction of initial motion of particle.

\bigskip
Let angle $POA=\theta$.

\bigskip
Let $V=\text{velocity}$ of particle on entering the atom, $v$ its velocity at $A$,
then from consideration of angular momentum
\begin{equation*}
pV=|S-A|v
\end{equation*}

From conservation of energy
\begin{align*}
\tfrac{1}{2}mV^2&=\tfrac{1}{2}mv^2-\frac{NeE}{|S-A|}
\\
v^2&=V^2\left(1-\frac{b}{|S-A|}\right)
\end{align*}

Since the eccentricity is $\sec\theta$,
\begin{align*}
|S-A|=|S-O|+|O-A|&=p\csc\theta\,(1+\cos\theta)
\\
&=p\cot\theta/2
\\
p^2=|S-A|(|S-A|-b)&=p\cot\theta/2\,(p\cot\theta/2-b)
\\
\therefore\,b&=2p\cot\theta
\end{align*}

The angle of deviation $\phi$ of the particles is $\pi-2\theta$ and
\begin{equation*}
\cot\phi/2=\frac{2p}{b}
\tag{1}
\end{equation*}

This gives the angle of deviation of the particle in terms of $b$,
and the perpendicular distance of the direction of projection from the centre of the atom.
(A simple consideration shows that the deflexion is unaltered if the forces are attractive instead of repulsive.)

\bigskip
For illustration, the angle of deviation $\phi$ for different values of $p/b$ are shown in the following table:

\begin{center}
\begin{tabular}{lccccccc}
$p/b\cdots$ & 10 & 5 & 2 & 1 & 0.5 & 0.25 & 0.125
\\
$\phi\cdots$ & $5^\circ\cdot7$ & $11^\circ\cdot4$ & $28^\circ$ & $53^\circ$ & $90^\circ$ & $127^\circ$ & $152^\circ$
\end{tabular}
\end{center}

\S 3. {\it Probability of single deflexion through any angle.}

\bigskip
Suppose a pencil of electrified particles to fall normally on a thin screen of matter of thickness $t$.
With the exception of the few particles which are scattered through a large angle,
the particles are supposed to pass nearly normally through the plate with only a small change of velocity.
Let $n=\text{number}$ of atoms in unit volume of material.
Then the number of collisions of the particle with the atom of radius $R$ is $\pi R^2nt$
in the thickness $t$.

\bigskip
The probability $m$ of entering an atom within a distance $p$ of its center is given by
\begin{equation*}
m=\pi p^2nt
\end{equation*}

Chance $dm$ of striking within radii $p$ and $p+dp$ is given by
\begin{equation*}
dm=2\pi pnt\,dp=\frac{\pi}{4}ntb^2\cot\phi/2\csc^2\phi/2\,d\phi
\tag{2}
\end{equation*}
since
\begin{equation*}
\cot\phi/2=2p/b
\end{equation*}

The value of $dm$ gives the {\it fraction} of the total number of particles
which are deviated between the angles $\phi$ and $\phi+d\phi$.

\bigskip
The fraction $\rho$ of the total number of particles which are deflected through an angle greater than $\phi$ is given by
\begin{equation*}
\rho=\frac{\pi}{4}ntb^2\cot^2\phi/2
\tag{3}
\end{equation*}

The fraction $\rho$ which is deflected between the angles $\phi_1$ and $\phi_2$ is given by
\begin{equation*}
\rho=\frac{\pi}{4}ntb^2
\left(\cot^2\frac{\phi_1}{2}-\cot^2\frac{\phi_2}{2}\right)
\tag{4}
\end{equation*}

It is convenient to express the equation (2) in another form for comparison with experiment.
In the case of the $\alpha$ rays, the number of scintillations appearing on the constant area of the zinc
sulphide screen are counted for different angles with the direction of incidence of the particles.
Let $r=\text{distance}$ from point of incidence of $\alpha$ rays on scattering material,
then if $Q$ be the total number of particles falling on the scattering material,
the number $y$ of $\alpha$ particles falling on unit area which are deflected through an angle $\phi$ is given by
\begin{equation*}
y=\frac{Q\,dm}{2\pi r^2\sin\phi\,d\phi}
=\frac{ntb^2Q\csc^4\phi/2}{16r^2}
\tag{5}
\end{equation*}

Since $b=2NeE/mu^2$,
we see from this equation that the number of $\alpha$ particles (scintillations) per unit area of zinc sulphide
screen at a given distance $r$ from the point of incidence of the rays is proportional to

\begin{quote}
(1) $\csc^4\phi/2$ or $1/\phi^4$ if $\phi$ be small;\newline
(2) thickness of scattering material $t$ provided this is small;\newline
(3) magnitude of central charge $Ne$;\newline
(4) and is inversely proportional to $(mu^2)^2$, or to the fourth power
of the velocity if $m$ be constant.
\end{quote}

In these calculations, it is assumed that the $\alpha$ particles scattered through a large angle suffer
only one large deflexion.
For this to hold, it is essential that the thickness of the scattering material should be so small that
the chance of a second encounter involving another large deflexion is very small.
If, for example, the probability of a single deflexion $\phi$ in passing through a thickness $t$ is 1/1000,
the probability of two successive deflexions each of value $\phi$ is $1/10^6$, and is negligibly small.

\bigskip
The angular distribution of the $\alpha$ particles scattered from a thin metal sheet affords one of the
simplest methods of testing the general correctness of this theory of single scattering.
This has been done recently for $\alpha$ rays by Dr.~Geiger,\footnote{Manch.~Lit.~\& Phil.~Soc.~1910.}
who found that the distribution for particles deflected between $30^\circ$ and $150^\circ$
from a thin gold-foil was in substantial agreement with the theory.
A more detailed account of these and other experiments to test the validity of the theory will be published later.

\bigskip
\S 4. {\it Alteration of velocity in an atomic encounter.}

\bigskip
It has so far been assumed that an $\alpha$ or $\beta$ particle does not suffer an appreciable change of velocity
as the result of a single atomic encounter resulting in a large deflexion of the particle.
The effect of such an encounter in altering the velocity of the particle can be calculated on certain assumptions.
It is supposed that only two systems are involved, viz., the swiftly moving particle and the atom which
it traverses supposed initially at rest.
It is supposed that the principle of conservation of momentum and of energy applies,
and that there is no appreciable loss of energy or momentum by radiation.

\end{document}
