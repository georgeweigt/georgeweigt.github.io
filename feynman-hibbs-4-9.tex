\documentclass[12pt]{article}
\usepackage{amsmath}
\usepackage{amssymb}

\parindent=0pt

\newcommand\INT{\int_{\mathbb R^3}}

\begin{document}

4-9.
Show from the fact that $H$ is hermitian that equation (4.46) holds.
Hint: Choose $f=\phi_2$, $g=\phi_1$ in equation (4.30).

\bigskip
\hrule

\bigskip
From equation (4.42)
\begin{align*}
H\phi_1&=E_1\phi_1
\\
H\phi_2&=E_2\phi_2
\end{align*}

Since $H$ is hermitian we have from equation (4.30)
\begin{equation*}
\int_{-\infty}^\infty(Hg)^*f\,dx=\int_{-\infty}^\infty g^*(Hf)\,dx
\tag{4.30}
\end{equation*}

Substitute $\phi_1$ into $g$ and $\phi_2$ into $f$.
\begin{equation*}
\int_{-\infty}^\infty(H\phi_1)^*\phi_2\,dx=\int_{-\infty}^\infty \phi_1^*(H\phi_2)\,dx
\end{equation*}

Replace $H$ with the corresponding eigenvalue.
(Eigenvalues of $H$ are real.)
\begin{equation*}
E_1\int_{-\infty}^\infty\phi_1^*\phi_2\,dx=E_2\int_{-\infty}^\infty \phi_1^*\phi_2\,dx
\end{equation*}

The integrals are identical hence for $E_1\ne E_2$ we must have
\begin{equation*}
\int_{-\infty}^\infty\phi_1^*\phi_2\,dx=0
\end{equation*}

\end{document}
