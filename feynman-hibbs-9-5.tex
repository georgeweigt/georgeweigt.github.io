\documentclass[12pt]{article}
\usepackage{amsmath}
\usepackage{amssymb}

\parindent=0pt

\begin{document}

9-5.
The momentum in the field is given by
\begin{equation*}
\frac{1}{4\pi c}\int
\mathbf E\times\mathbf B\,d^3\mathbf r
\end{equation*}

In the absence of matter (so $\phi_{\mathbf k}=0$),
show that this is
\begin{equation*}
i\int\mathbf k\left(\mathbf a_{\mathbf k}^*\cdot\dot{\mathbf a}_{\mathbf k}\right)
\frac{d^3\mathbf k}{(2\pi)^3}
\end{equation*}

Note that
\begin{equation*}
\mathbf E\times\mathbf B
=(E_yB_z-E_zB_y)\mathbf i
+(E_zB_x-E_xB_z)\mathbf j
+(E_xB_y-E_yB_x)\mathbf k
\tag{1}
\end{equation*}
where
\begin{equation*}
\mathbf i=\begin{pmatrix}1\\0\\0\end{pmatrix},
\quad
\mathbf j=\begin{pmatrix}0\\1\\0\end{pmatrix},
\quad
\mathbf k=\begin{pmatrix}0\\0\\1\end{pmatrix}
\end{equation*}

Consider equation (9.9).
\begin{equation*}
\mathbf E=-\nabla\phi-\frac{1}{c}\frac{\partial\mathbf A}{\partial t}
\tag{9.9}
\end{equation*}

By hypothesis the scalar potential vanishes leaving
\begin{equation*}
\mathbf E=-\frac{1}{c}\frac{\partial\mathbf A}{\partial t}
=-\frac{1}{c}\frac{\partial A_x}{\partial t}\mathbf i
-\frac{1}{c}\frac{\partial A_y}{\partial t}\mathbf j
-\frac{1}{c}\frac{\partial A_z}{\partial t}\mathbf k
\tag{2}
\end{equation*}

Consider equation (9.7).
\begin{equation*}
\mathbf B=\nabla\times\mathbf A
\tag{9.7}
\end{equation*}

It follows that
\begin{equation*}
\mathbf B
=\left(\frac{\partial A_z}{\partial y}-\frac{\partial A_y}{\partial z}\right)\mathbf i
+\left(\frac{\partial A_x}{\partial z}-\frac{\partial A_z}{\partial x}\right)\mathbf j
+\left(\frac{\partial A_y}{\partial x}-\frac{\partial A_x}{\partial y}\right)\mathbf k
\tag{3}
\end{equation*}

Substitute (2) and (3) into (1) to obtain
\begin{equation*}
\mathbf E\times\mathbf B=F_x\mathbf i+F_y\mathbf j+F_z\mathbf k
\end{equation*}
where

\end{document}
