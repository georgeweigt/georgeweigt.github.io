\documentclass[12pt]{article}
\usepackage{amsmath}
\usepackage{amssymb}

\parindent=0pt

\begin{document}

9-3.
Prove that the relation $\phi_{\mathbf k}=4\pi\rho_{\mathbf k}/k^2$
simply means that $\phi_{\mathbf k}$ at any instant is the Coulomb
potential from the charges at that instant, so that, for example,
if $\rho$ comes from a number of charges $q_i$ at distances $R_i$
from a point, the potential at the point is
$\phi=\sum_iq_i/R_i$.

\bigskip
Integrate using polar coordinates.
\begin{equation*}
\int_0^{2\pi}\int_0^\pi\int_0^\infty
\frac{4\pi\rho_{\mathbf k}(t)}{k^2}\exp(ikR)\,k^2\sin\theta\,dk\,d\theta\,d\phi
=\frac{16\pi^2i\rho_{\mathbf k}(t)}{R}
\tag{1}
\end{equation*}

The result is a Coulomb potential for
$\rho_{\mathbf k}(t)\propto -iq$.

\bigskip
The following integrals show how (1) is obtained.
\begin{equation*}
\int_0^\infty\exp(-a x)\,dx=\frac{1}{a}
\qquad
\int_0^\pi\sin\theta\,d\theta=2
\qquad
\int_0^{2\pi}d\phi=2\pi
\end{equation*}

Note: For multiple charges $q_i$ we have
\begin{align*}
\rho(\mathbf r,t)&=\sum_iq_i\delta(R_i)
\\
\phi(\mathbf r,t)&=\sum_i\frac{q_i}{R_i}
\end{align*}

\end{document}
