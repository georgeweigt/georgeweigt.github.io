\documentclass[12pt]{article}
\usepackage[margin=2cm]{geometry}
\usepackage{amsmath}
\usepackage{amssymb} % \mathbb

\begin{document}

\noindent
Electron spin state $s$ is a unit vector in $\mathbb{C}^2$.
\begin{equation*}
s=\begin{pmatrix}c_1\\c_2\end{pmatrix}
\end{equation*}

\begin{equation*}
s^\dag s=\bar{c}_1c_1+\bar{c}_2c_2=1
\end{equation*}

\noindent
These are the spin operators.
\begin{equation*}
\sigma_x=\begin{pmatrix}0&1\\1&0\end{pmatrix}
\qquad
\sigma_y=\begin{pmatrix}0&-i\\i&0\end{pmatrix}
\qquad
\sigma_z=\begin{pmatrix}1&0\\0&-1\end{pmatrix}
\end{equation*}

\noindent
The expectation value of a spin operator is a projection of $s$ onto Euclidean space.
\begin{equation*}
\langle x\rangle=s^\dag\sigma_xs
\qquad
\langle y\rangle=s^\dag\sigma_ys
\qquad
\langle z\rangle=s^\dag\sigma_zs
\end{equation*}

\noindent
Let
\begin{equation*}
\sigma=\begin{pmatrix}\sigma_x\\\sigma_y\\\sigma_z\end{pmatrix}
\end{equation*}

\noindent
Then the spin polarization vector $P$ is
\begin{equation*}
P=\begin{pmatrix}\langle x\rangle\\\langle y\rangle\\\langle z\rangle\end{pmatrix}
=s^\dag\sigma s
\end{equation*}

\noindent
In component notation $\sigma=\sigma^{\alpha\beta}{}_\gamma$ and
\begin{equation*}
P^\alpha=\bar s_\beta\sigma^{\alpha\beta}{}_\gamma s^\gamma
\end{equation*}

\noindent
In Eigenmath a transpose is needed to swap $\alpha$ and $\beta$ so that summed-over indices are adjacent.
\begin{equation*}
P^\alpha=\bar s_\beta\sigma^{\beta\alpha}{}_\gamma s^\gamma
\end{equation*}

\noindent
Hence the code is
\begin{equation*}
\text{\tt P = dot(conj(s),transpose(sigma),s)}
\end{equation*}

\noindent
Let $\theta$ and $\phi$ be polar and azimuth angles such that
\begin{equation*}
P=\begin{pmatrix}\sin\theta\cos\phi\\\sin\theta\sin\phi\\\cos\theta\end{pmatrix}
\end{equation*}

\noindent
The corresponding spin state is
\begin{equation*}
s=\begin{pmatrix}\cos(\theta/2)\\\sin(\theta/2)\exp(i\phi)\end{pmatrix}
\end{equation*}

\end{document}
