\documentclass[12pt]{article}
\usepackage[margin=1in]{geometry}
\usepackage{amsmath}
\usepackage{parskip}
\parindent=0pt
\begin{document}

\section*{Fermi's golden rule}

Let $\Psi(x,t)$ be the following linear combination of the first two eigenstates
of an infinite square well potential.
\begin{equation*}
\Psi(x,t)=c_1(t)\psi_1(x)\exp(-i\omega_1t)+c_2(t)\psi_2(x)\exp(-i\omega_2t)
\end{equation*}

Given the perturbing potential
\begin{equation*}
V(x,t)=\left(x-\frac{L}{4}\right)^2\sin(\omega t)
\end{equation*}
we want to solve for $c_2(t)$ and find the transition rate
\begin{equation*}
\frac{d}{dt}|c_2(t)|^2
\end{equation*}

From the time-dependent Schrodinger equation we have
\begin{equation*}
i\hbar\frac{\partial\Psi(x,t)}{\partial t}=\hat H(x)\Psi(x,t)+V(x,t)\Psi(x,t)
\end{equation*}

The left-hand side works out to be
\begin{multline*}
LHS=\left(i\hbar\frac{\partial c_1(t)}{\partial t}+\hbar\omega_1c_1(t)\right)
\psi_1(x)\exp(-i\omega_1t)
\\
{}+\left(i\hbar\frac{\partial c_2(t)}{\partial t}+\hbar\omega_2c_2(t)\right)
\psi_2(x)\exp(-i\omega_2t)
\end{multline*}

Using the time-independent Schrodinger equation the right-hand side is
\begin{equation*}
RHS=c_1(t)E_1\psi_1(x)\exp(-i\omega_1t)
+c_2(t)E_2\psi_2(x)\exp(-i\omega_2t)
+V(x,t)\Psi(x,t)
\end{equation*}

The energy terms cancel by the substitutions
$E_1=\hbar\omega_1$ and $E_2=\hbar\omega_2$ leaving
\begin{align*}
LHS&=i\hbar\frac{\partial c_1(t)}{\partial t}\psi_1(x)\exp(-i\omega_1t)
+i\hbar\frac{\partial c_2(t)}{\partial t}\psi_2(x)\exp(-i\omega_2t)
\\
RHS&=V(x,t)\Psi(x,t)
\end{align*}

Hence
\begin{equation*}
i\hbar\frac{\partial c_1(t)}{\partial t}\psi_1(x)\exp(-i\omega_1t)
+i\hbar\frac{\partial c_2(t)}{\partial t}\psi_2(x)\exp(-i\omega_2t)
=V(x,t)\Psi(x,t)
\end{equation*}

Take the inner product of $\psi_2^*(x)$ with the above equation to obtain
\begin{equation*}
i\hbar\frac{\partial c_2(t)}{\partial t}\exp(-i\omega_2t)
=c_1(t)M_{21}\sin(\omega t)\exp(-i\omega_1t)+c_2(t)M_{22}\sin(\omega t)\exp(-i\omega_2t)
\end{equation*}
where $M_{21}$ and $M_{22}$ are the matrix elements
\begin{align*}
M_{21}&=\int\psi_2^*(x)\left(x-\frac{L}{4}\right)^2\psi_1(x)\,dx
\\
M_{22}&=\int\psi_2^*(x)\left(x-\frac{L}{4}\right)^2\psi_2(x)\,dx
\end{align*}

Cancel exponentials.
\begin{equation*}
i\hbar\frac{\partial c_2(t)}{\partial t}
=c_1(t)M_{21}\sin(\omega t)\exp\bigl(i(\omega_2-\omega_1)t\bigr)+c_2(t)M_{22}\sin(\omega t)
\end{equation*}

The initial state is $\Psi(x,0)=\psi_1(x)$ hence the initial conditions are
\begin{equation*}
c_1(0)=1,\quad c_2(0)=0
\end{equation*}

Then for time $t$ near the origin we can use the approximation
\begin{equation*}
i\hbar\frac{\partial c_2(t)}{\partial t}
=M_{21}\sin(\omega t)\exp\bigl(i(\omega_2-\omega_1)t\bigr)
\end{equation*}

Solve for $c_2(t)$ by integrating.
\begin{equation*}
c_2(t)=\frac{M_{21}}{i\hbar}\int_0^t\sin(\omega t')\exp\bigl(i(\omega_2-\omega_1)t'\bigr)\,dt'
\end{equation*}

The solution is
\begin{equation*}
c_2(t)
=\frac{M_{21}}{2i\hbar}
\left(
\frac{\exp\bigl(i(\omega_2-\omega_1-\omega) t\bigr)-1}{\omega_2-\omega_1-\omega}
-
\frac{\exp\bigl(i(\omega_2-\omega_1+\omega) t\bigr)-1}{\omega_2-\omega_1+\omega}
\right)\tag{1}
\end{equation*}

For perturbations such that $\omega\approx\omega_2-\omega_1$ the first term
dominates so discard the second term and write
\begin{equation*}
c_2(t)=\frac{M_{21}}{2i\hbar}
\left(
\frac{\exp\bigl(i(\omega_2-\omega_1-\omega) t\bigr)-1}{\omega_2-\omega_1-\omega}
\right)
\end{equation*}

Rewrite as
\begin{equation*}
c_2(t)=\frac{M_{21}}{2\hbar}\,t
\exp\left(i\,\frac{\omega_2-\omega_1-\omega}{2}\,t\right)
\operatorname{sinc}\left(\frac{\omega_2-\omega_1-\omega}{2}\,t\right)
\tag{2}
\end{equation*}

The probability density is
\begin{equation*}
|c_2(t)|^2=\frac{|M_{21}|^2}{4\hbar^2}\,t^2
\operatorname{sinc}^2\left(\frac{\omega_2-\omega_1-\omega}{2}\,t\right)
\end{equation*}

Use the following approximation for $\operatorname{sinc}^2$
\begin{equation*}
\operatorname{sinc}^2\left(\frac{\omega_2-\omega_1-\omega}{2}\,t\right)
\approx\frac{2\pi}{t}\delta(\omega_2-\omega_1-\omega),
\quad
t>0
\end{equation*}
to obtain
\begin{equation*}
|c_2(t)|^2=\frac{\pi}{2\hbar^2}|M_{21}|^2t\delta(\omega_2-\omega_1-\omega)
\tag{3}
\end{equation*}

Differentiate $|c_2(t)|^2$ with respect to $t$ to obtain the transition rate.
\begin{equation*}
\frac{d}{dt}|c_2(t)|^2=\frac{\pi}{2\hbar^2}|M_{21}|^2\delta(\omega_2-\omega_1-\omega)
\end{equation*}

For the infinite square well it can be shown that
\begin{equation*}
M_{21}=\int_0^L\psi_2^*(x)\left(x-\frac{L}{4}\right)^2\psi_1(x)\,dx
=-\frac{8L^2}{9\pi^2}
\tag{4}
\end{equation*}

Hence the transition rate is
\begin{equation*}
\frac{d}{dt}|c_2(t)|^2=\frac{32L^4}{81\pi^3\hbar^2}
\,\delta(\omega_2-\omega_1-\omega)
\end{equation*}

\end{document}
