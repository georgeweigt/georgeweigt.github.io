\documentclass[12pt]{article}
\usepackage{amsmath}
\parindent=0pt
\begin{document}

This is the Maxwell-Boltzmann velocity distribution.
Technically, it's a joint probability density function of $v_x$, $v_y$, and $v_z$.
\begin{equation*}
f(v_x,v_y,v_z)=
\bigg(\frac{m}{2\pi kT}\bigg)^{3/2}
\exp\bigg({-}\frac{m(v_x^2+v_y^2+v_z^2)}{2kT}\bigg)
\end{equation*}

In spherical coordinates with $v=\sqrt{v_x^2+v_y^2+v_z^2}$ we have
\begin{equation*}
f(v,\theta,\phi)=
\bigg(\frac{m}{2\pi kT}\bigg)^{3/2}
\exp\bigg({-}\frac{mv^2}{2kT}\bigg)
\end{equation*}

By integrating over $\theta$ and $\phi$
we obtain Maxwell's speed distribution,
also a probability density function.
\begin{equation*}
f(v)=4\pi\bigg(\frac{m}{2\pi kT}\bigg)^{3/2}
v^2\exp\bigg({-}\frac{mv^2}{2kT}\bigg)
\end{equation*}

Noting that
\begin{equation*}
dv_x\,dv_y\,dv_z=v^2\sin\theta\,dv\,d\theta\,d\phi
\end{equation*}
and
\begin{equation*}
\int_0^\pi\sin\theta\,d\theta=\cos(0)-\cos(\pi)=2
\end{equation*}
the integral can be done by inspection.
\begin{equation*}
f(v)=\int_0^{2\pi}\int_0^\pi f(v,\theta,\phi)\,v^2\sin\theta\,d\theta\,d\phi
=4\pi v^2f(v,\theta,\phi)
\end{equation*}

Historically, the speed distribution came first.\footnote{\tt https://mathshistory.st-andrews.ac.uk/Projects/Johnson/chapter-6/}
Maxwell derived it in 1867.
Boltzmann extended Maxwell's work a year later in 1868.

\end{document}
