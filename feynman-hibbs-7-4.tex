\documentclass[12pt]{article}
\usepackage{amsmath}
\usepackage{amssymb}

\parindent=0pt

\newcommand\INT{\int_{\mathbb R^3}}

\begin{document}

7-4.
Discuss why the integrated part vanishes.

\bigskip
\hrule

\bigskip
The authors are referring to integration by parts in equation (7.32).
\begin{equation*}
\int\frac{\partial F}{\partial x_k}
\exp\left(\frac{i}{\hbar}S\big(x(t)\big)\right)\,\mathcal Dx(t)
=
-\frac{i}{\hbar}\int F\frac{\partial S}{\partial x_k}
\exp\left(\frac{i}{\hbar}S\big(x(t)\big)\right)\,\mathcal Dx(t)
\tag{7.32}
\end{equation*}
where
\begin{equation*}
x_k=x(t_k)
\end{equation*}

Note that integration by parts in (7.32) uses $x_k$, not $x(t)$.

\bigskip
Let
\begin{align*}
u&=\exp\left(\frac{i}{\hbar}S\big(x(t)\big)\right)\,\mathcal Dx(t)
\\
dv&=\frac{\partial F}{\partial x_k}
\end{align*}

Then
\begin{align*}
du&=\frac{i}{\hbar}
\frac{\partial S}{\partial x_k}
\exp\left(\frac{i}{\hbar}S\big(x(t)\big)\right)
\,\mathcal Dx(t)
\\
v&=F
\end{align*}

Integrate by parts.
\begin{align*}
\int u\,dv&=uv-\int v\,du
\\
&=F\exp\left(\frac{i}{\hbar}S\big(x(t)\big)\right)\,\mathcal Dx(t)
-\frac{i}{\hbar}\int F
\frac{\partial S}{\partial x_k}
\exp\left(\frac{i}{\hbar}S\big(x(t)\big)\right)
\,\mathcal Dx(t)
\end{align*}

The term $uv$ vanishes because there is no integral over $\mathcal Dx(t)$.
There has to be an integration interval to obtain a nonzero transition amplitude.
The concept is similar to a probability density function.
A probability density function must be integrated over an interval to obtain a nonzero probability.

\end{document}
