\input{preamble}

\FBOX{
11.1
Why isn't it {\it trivial} to solve the time-dependent Schr\"odinger equation (11.1),
in its dependence on $t$?
After all, it's a first-order differential equation.

\bigskip
(a) How would you solve the equation
\begin{equation*}
\frac{df}{dt}=kf
\end{equation*}

(for $f(t)$), if $k$ were a constant?

\bigskip
(b) What if $k$ itself is a function of $t$?
(Here $k(t)$ and $f(t)$ might also depend on other variables, such as $\mathbf r$---it doesn't matter.)

\bigskip
(c) Why not do the same thing for the Schr\"odinger equation (with a time-dependent Hamiltonian)?
To see that this doesn't work, consider the simple case
\begin{equation*}
\hat H(t)=\begin{cases}
\hat H_1, & (0<t<\tau),
\\
\hat H_2, & (t>\tau),
\end{cases}
\end{equation*}

where $\hat H_1$ and $\hat H_2$ are themselves time-independent.
If the solution in part (b) held for the Schr\"odinger equation, the wave function
at time $t>\tau$ would be
\begin{equation*}
\Psi(t)=e^{-i\left[\hat H_1\tau+\hat H_2(t-\tau)\right]/\hbar}\Psi(0),
\end{equation*}

but of course we could also write
\begin{equation*}
\Psi(t)=e^{-i\hat H_2(t-\tau)/\hbar}\Psi(\tau)
=e^{-i\hat H_2(t-\tau)/\hbar}e^{-i\hat H_1t/\hbar}\Psi(0).
\end{equation*}

Why are these generally {\it not} the same?
[This is a subtle matter; if you want to pursue it further, see Problem 11.23.]
}

(a)
\begin{equation*}
f(t)=C\exp(kt)
\end{equation*}

(b)
\begin{equation*}
f(t)=C\exp\left(\int k(t)\,dt\right)
\end{equation*}




\end{document}
