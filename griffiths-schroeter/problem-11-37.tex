\input{preamble}

\FBOX{
11.37
Quantum Zeno Paradox.
Suppose a system starts out in an excited state $\psi_b$,
which has a natural lifetime $\tau$ for transition to the ground state $\psi_a$.
Ordinarily, for times substantially less that $\tau$, the probability of a transition
is proportional to $\tau$ (Equation 11.49):
\begin{equation*}
P_{b\rightarrow a}=\frac{t}{\tau}.
\tag{11.145}
\end{equation*}

If we make a measurement after a time $t$, then, the probability that the system
is still in the {\it upper} state is
\begin{equation*}
P_b(t)=1-\frac{t}{\tau}.
\tag{11.146}
\end{equation*}

Suppose we {\it do} find it to be in the upper state.
In that case the wave function collapses back to $\psi_b$, and the process
starts all over again.
If we make a {\it second} measurement, at $2t$, the probability that the system
is {\it still} in the upper state is
\begin{equation*}
\left(1-\frac{t}{\tau}\right)^2\approx1-\frac{2t}{\tau},
\tag{11.147}
\end{equation*}

which is the same as it would have been had we never made the first
measurement at $t$ (as one would naively expect).

\bigskip
However, for {\it extremely} short times, the probability of a transition is
{\it not} proportional to $t$, but rather to $t^2$ (Equation 11.46):
\begin{equation*}
P_{b\rightarrow a}=\alpha t^2.
\tag{11.148}
\end{equation*}

(a) In this case what is the probability that the system is still in the upper
state after the two measurements?
What {\it would} it have been (after the same elapsed time) if we had never
made the first measurement?

\bigskip
(b) Suppose we examine the system at $n$ regular (extremely short) intervals,
from $t=0$ out to $t=T$ (that is, we make measurements at $T/n$, $2T/n$, $3T/n,\ldots,T$).
What is the probability that the system is still in the upper state at time $T$?
What is its limit as $n\rightarrow\infty$?
{\it Moral of the story:}
Because of the collapse of the wave function at every measurement, a {\it continuously}
observed system never decays at all!
}

(a) Probability of upper state after two measurements:
\begin{equation*}
(1-\alpha t^2)^2\approx1-2\alpha t^2
\end{equation*}

Probability of upper state after $2t$:
\begin{equation*}
1-4\alpha t^2
\end{equation*}

(b) The probability that the system is still in the upper state is
\begin{equation*}
\left(1-\frac{\alpha T^2}{n^2}\right)^n
\end{equation*}

Let $y$ be the limit.
\begin{equation*}
y=\lim_{n\rightarrow\infty}\left(1-\frac{\alpha T^2}{n^2}\right)^n
=\lim_{n\rightarrow\infty}e^{n\log\left(1-\frac{\alpha T^2}{n^2}\right)}
=\lim_{n\rightarrow\infty}\exp\left[\frac{\log\left(1-\frac{\alpha T^2}{n^2}\right)}{\frac{1}{n}}\right]
\end{equation*}

By the composition limit law
\begin{equation*}
y=\exp\left[
\lim_{n\rightarrow\infty}
\frac{\log\left(1-\frac{\alpha T^2}{n^2}\right)}
{\frac{1}{n}}
\right]
\end{equation*}

Apply l'H\^opital's rule.
\begin{equation*}
y=\exp\left[
\lim_{n\rightarrow\infty}
\frac{\frac{d}{dn}\log\left(1-\frac{\alpha T^2}{n^2}\right)}
{\frac{d}{dn}\frac{1}{n}}
\right]
=\exp\left(
\lim_{n\rightarrow\infty}
\frac
{\frac{2\alpha T^2}{n^3-\alpha T^2n}}
{-\frac{1}{n^2}}
\right)
\end{equation*}

Hence
\begin{equation*}
y=\exp\left(
\lim_{n\rightarrow\infty}
\frac{2\alpha T^2}{\frac{\alpha T^2}{n}-n}
\right)
=\exp(0)=1
\end{equation*}

\end{document}
