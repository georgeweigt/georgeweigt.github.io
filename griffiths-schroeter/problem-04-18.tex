\input{preamble}

\FBOX{
4.18
A hydrogen atom starts out in the following linear combination of the
stationary states
$n=2$, $\ell=1$, $m=1$ and $n=2$, $\ell=1$, $m=-1$:
\begin{equation*}
\Psi(\mathbf r,0)=\frac{1}{\sqrt2}(\psi_{211}+\psi_{21-1}).
\end{equation*}

(a) Construct $\Psi(\mathbf r,t)$. Simplify it as much as you can.

\bigskip
(b) Find the expectation value of the potential energy, $\langle V\rangle$.
(Does it depend on $t$?)
Give both the formula and the actual number, in electron volts.
}

(a) For the stationary states we have
\begin{align*}
\psi_{211}&=-\frac{r\sin\theta}{8\sqrt{\pi a_0^5}}\exp\left(-\frac{r}{2a_0}+i\phi\right)
\\
\psi_{21-1}&=\frac{r\sin\theta}{8\sqrt{\pi a_0^5}}\exp\left(-\frac{r}{2a_0}-i\phi\right)=-\psi_{211}^*
\end{align*}

The general solution to the Schr\"odinger equation is
\begin{equation*}
\sum c_n\psi_n(\mathbf r)\exp\left(-\frac{iE_nt}{\hbar}\right)\tag{4.9}
\end{equation*}

Hence
\begin{equation*}
\Psi(\mathbf r,t)=\frac{r\sin\theta}{8\sqrt{2\pi a_0^5}}
\exp\left(-\frac{r}{2a_0}-\frac{iE_2t}{\hbar}\right)
\bigl(\exp(-i\phi)-\exp(i\phi)\bigr)
\end{equation*}

(b) For the potential energy we have
\begin{equation*}
V(r)=-\frac{e^2}{4\pi\epsilon_0}\frac{1}{r}\tag{4.52}
\end{equation*}

The expectation value is
\begin{align*}
\langle V\rangle&=\langle\Psi|V|\Psi\rangle
\\
&=\int_0^\infty\int_0^\pi\int_0^{2\pi}
\Psi^*V\Psi r^2\sin\theta\,dr\,d\theta\,d\phi
\\
&=-\frac{e^2}{16\pi\epsilon_0a_0}\tag{1}
\end{align*}

Convert to electron volts.
\begin{equation*}
\langle V\rangle=-6.8\,\text{eV}
\end{equation*}

The expectation value $\langle V\rangle$ does not depend on $t$.

\end{document}
