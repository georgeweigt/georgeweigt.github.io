\input{preamble}

\FBOX{
11.23
In Problem 11.1 you showed that the solution to
\begin{equation*}
\frac{df}{dt}=k(t)f(t)
\end{equation*}

(where $k(t)$ is a function of $t$) is
\begin{equation*}
f(t)=e^{K(t)}f(0),\quad\text{where}\quad K(t)\equiv\int_0^tk(t')\,dt'.
\end{equation*}

This suggests that the solution to the Schr\"odinger equation (11.1) might be
\begin{equation*}
\Psi(t)=e^{\hat G(t)}\Psi(0),\quad\text{where}\quad
\hat G(t)\equiv-\frac{i}{\hbar}\int_0^t\hat H(t')\,dt'.
\tag{11.108}
\end{equation*}

It doesn't work, because $\hat H(t)$ is an {\it operator}, not a function,
and $\hat H(t_1)$ does not (in general) commute with $\hat H(t_2)$.

\bigskip
(a) Try calculating $i\hbar\partial\Psi/\partial t$, using Equation 11.108.
{\it Note:} as always, the exponentiated operator is to be interpreted as a power series:
\begin{equation*}
e^{\hat G}=1+\hat G+\frac{1}{2}\hat G\hat G+\frac{1}{3!}\hat G\hat G\hat G+\cdots.
\end{equation*}

Show that {\it if} $[\hat G,\hat H]=0$, then $\Psi$ satisfies the Schr\"odinger equation.
}

(a)
\begin{align*}
i\hbar\frac{\partial}{\partial t}\Psi(t)
&=i\hbar\left[\frac{\partial}{\partial t}\hat G
+\frac{1}{2}\frac{\partial}{\partial t}(\hat G\hat G)
+\frac{1}{3!}\frac{\partial}{\partial t}(\hat G\hat G\hat G)+\cdots\right]\Psi(0)
\\
&=\left[\hat H
+\frac{1}{2}(\hat H\hat G+\hat G\hat H)
+\frac{1}{3!}(\hat H\hat G\hat G+\hat G\hat H\hat G+\hat G\hat G\hat H)
+\cdots\right]\Psi(0)
\end{align*}

If $\hat G$ and $\hat H$ commute then $\hat G\hat H=\hat H\hat G$ and
for the general case of $n$ operators
\begin{equation*}
\frac{\partial}{\partial t}\hat G^{\,n}
=n\hat H\hat G^{\,n-1}
\end{equation*}

Hence
\begin{align*}
i\hbar\frac{\partial}{\partial t}\Psi(t)
&=\left[\hat H+\hat H\hat G+\frac{1}{2}\hat H\hat G\hat G+\ldots\right]\Psi(0)
\\
&=\hat H\left[1+\hat G+\frac{1}{2}\hat G\hat G+\ldots\right]\Psi(0)
\\
&=\hat H\Psi(t)
\end{align*}

Hence $\Psi(t)$ satisfies the Schr\"odinger equation.

\bigskip
\FBOX{
(b) Check that the correct solution in the general case ($[\hat G,\hat H]\ne0$) is
\begin{multline*}
\Psi(t)=\bigg\{1+\left(-\frac{i}{\hbar}\right)\int_0^t\hat H(t_1)\,dt_1
+\left(-\frac{i}{\hbar}\right)^2\int_0^t\hat H(t_1)\left[\int_0^{t_1}\hat H(t_2)\,dt_2\right]\,dt_1
\\
+\left(-\frac{i}{\hbar}\right)^3\int_0^t\hat H(t_1)
\left[\int_0^{t_1}\hat H(t_2)\left(\int_0^{t_2}\hat H(t_3)\,dt_3\right)\,dt_2\right]
\,dt_1+\cdots
\bigg\}\Psi(0)
\tag{11.109}
\end{multline*}
}

(b) For the single integral
\begin{equation*}
\frac{\partial}{\partial t}
\int_0^t\hat H(t_1)\,dt_1
=\hat H(t)
\end{equation*}

For the double integral we have
\begin{multline*}
\frac{\partial}{\partial t}
\int_0^t\hat H(t_1)\left[\int_0^{t_1}\hat H(t_2)\,dt_2\right]\,dt_1
\\
=\int_0^t\frac{\partial\hat H(t_1)}{\partial t}
\left[\int_0^{t_1}\hat H(t_2)\,dt_2\right]\,dt_1
+\int_0^t\hat H(t_1)
\underbrace{\frac{\partial}{\partial t}\left[\int_0^{t_1}\hat H(t_2)\,dt_2\right]}
_{\substack{\text{vanishes, integral is}\\ \text{not a function of $t$}}}
\,dt_1
\\
=\hat H(t)\int_0^t\hat H(t_1)\,dt_1
\end{multline*}

and similarly for the other integrals. Hence
\begin{align*}
i\hbar\frac{\partial}{\partial t}\Psi(t)
&=i\hbar\left\{\left(-\frac{i}{\hbar}\right)\hat H(t)+\left(-\frac{i}{\hbar}\right)^2\hat H(t)\int_0^t\hat H(t_1)\,dt_1+\cdots\right\}\Psi(0)
\\
&=\left\{\hat H(t)+\left(-\frac{i}{\hbar}\right)\hat H(t)\int_0^t\hat H(t_1)\,dt_1+\cdots\right\}\Psi(0)
\\
&=\hat H(t)\left\{1+\left(-\frac{i}{\hbar}\right)\int_0^t\hat H(t_1)\,dt_1+\cdots\right\}\Psi(0)
\\
&=\hat H(t)\Psi(t)
\end{align*}

\end{document}
