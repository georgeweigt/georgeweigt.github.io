\input{preamble}

\FBOX{
9.4
Calculate the lifetimes for $\mathrm U^{238}$ and $\mathrm{Po}^{121}$,
using Equations 9.26 and 9.29.
{\it Hint:} The density of nuclear matter is relativeley constant
(i.e.~the same for all nuclei), so $(r_1)^3$ is proportional to $A$
(the number of neutrons plus protons).
Empirically,
\begin{equation*}
r_1\approx(1.07\,\text{fm})A^{1/3}.\tag{9.30}
\end{equation*}

The energy of the emitted alpha particle can be deduced by using Einstein's
formula $\left(E=mc^2\right)$:
\begin{equation*}
E=m_pc^2-m_dc^2-m_\alpha c^2,\tag{9.31}
\end{equation*}

where $m_p$ is the mass of the parent nucleus, $m_d$ is the mass of the
daughter nucleus, and $m_\alpha$ is the mass of the alpha particle
(which is to say, $\mathrm{He}^4$ nucleus).
To figure out what the daughter nucleus is, note that the alpha particle
carries off two protons and two neutrons,
so $Z$ decreases by 2 and $A$ by 4.
Look up the relevant nuclear masses.
To estimate $v$, use $E=(1/2)m_\alpha v^2$; this ignores the (negative)
potential energy inside the nucleus, and surely {\it underestimates} $v$,
but it's about the best we can do at this stage.
Incidentally, the experimental lifetimes are
$6\times10^9\,\text{yrs}$ and $0.5\,\mu\mathrm s$, respectively.
}

\begin{equation*}
\gamma=K_1\frac{Z}{\sqrt E}-K_2\sqrt{Zr_1}\tag{9.26}
\end{equation*}

where
\begin{equation*}
K_1=1.980\,\mathrm{MeV}^{1/2},\quad K_2=1.485\,\mathrm{fm}^{-1/2}
\end{equation*}

\begin{equation*}
\tau=\frac{2r_1}{v}e^{2\gamma}\tag{9.29}
\end{equation*}

For $\mathrm U^{238}$ we have $\mathrm U^{238}\rightarrow\mathrm{Th}^{234}$ and
\begin{align*}
A&=238
\\
Z&=92
\\
E&=5.29\,\mathrm{MeV}
\\
v&=1.60\times10^7\,\text{m/s}
\\
\tau&=2.11\times10^8\,\text{years}
\end{align*}

Note: For $E=5.1\,\text{MeV}$ the result is $\tau=4.28\times10^9\,\text{years}$.

\bigskip
Per Wikipedia $E=4.267\,\mathrm{MeV}$ and $\tau=4.468\times10^9\,\mathrm{years}$.

\bigskip
FIXME $\mathrm{Po}^{121}$

\end{document}
