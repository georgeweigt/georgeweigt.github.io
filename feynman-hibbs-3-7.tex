\documentclass[12pt]{article}
\usepackage{amsmath}

\begin{document}

\noindent
Using equations (3.52) and (2.31),
express $F(t+s)$ in terms of $F(t)$ and $F(s)$,
where $t=t_b-t_c$ and $s=t_c-t_a$.

\bigskip
\noindent
From equation (3.52)
\begin{equation*}
K(b,a)=F(t_b-t_a)\exp\left(\frac{im(x_b-x_a)^2}{2\hbar (t_b-t_a)}\right)
\end{equation*}

\noindent
It follows that
\begin{equation*}
K(b,a)=F(t+s)\exp\left(\frac{im(x_b-x_a)^2}{2\hbar (t+s)}\right)
\eqno{(1)}
\end{equation*}

\noindent
From equation (2.31)
\begin{equation*}
K(b,a)=\int_{-\infty}^\infty K(b,c)K(c,a)\,dx_c
\end{equation*}

\noindent
Hence
\begin{equation*}
K(b,a)=F(t)F(s)\int_{-\infty}^\infty
\exp\left(\frac{im(x_b-x_c)^2}{2\hbar t}\right)
\exp\left(\frac{im(x_c-x_a)^2}{2\hbar s}\right)
\,dx_c
\end{equation*}

\noindent
Solve the integral.
\begin{equation*}
K(b,a)=F(t)F(s)
\left(\frac{2\pi i\hbar ts}{m(t+s)}\right)^{1/2}
\exp\left(\frac{im(x_b-x_a)^2}{2\hbar(t+s)}\right)
\eqno{(2)}
\end{equation*}

\noindent
Equating (1) with (2) causes the exponentials to cancel leaving
\begin{equation*}
F(t+s)=F(t)F(s)\left(\frac{2\pi i\hbar ts}{m(t+s)}\right)^{1/2}\eqno{(3)}
\end{equation*}

\noindent
Show that if
\begin{equation*}
F(t)=\left(\frac{m}{2\pi i\hbar t}\right)^{1/2} f(t)
\end{equation*}
then the new function $f(t)$ must satisfy
\begin{equation*}
f(t+s)=f(t)f(s)
\end{equation*}

\noindent
By substitution
\begin{equation*}
F(t+s)=\left(\frac{m}{2\pi i\hbar (t+s)}\right)^{1/2} f(t+s)
\end{equation*}
and
\begin{equation*}
F(t)F(s)=
\left(\frac{m}{2\pi i\hbar t}\right)^{1/2}
\left(\frac{m}{2\pi i\hbar s}\right)^{1/2}
f(t)f(s)
\end{equation*}

\noindent
Then by (3) we have
\begin{multline*}
\left(\frac{m}{2\pi i\hbar(t+s)}\right)^{1/2}
f(t+s)=
\\
\left(\frac{m}{2\pi i\hbar t}\right)^{1/2}
\left(\frac{m}{2\pi i\hbar s}\right)^{1/2}
f(t)f(s)
\left(\frac{2\pi i\hbar ts}{m(t+s)}\right)^{1/2}
\end{multline*}

\noindent
The coefficients cancel leaving
\begin{equation*}
f(t+s)=f(t)f(s)
\end{equation*}

\end{document}
