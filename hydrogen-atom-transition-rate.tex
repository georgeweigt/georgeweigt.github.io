\documentclass[12pt]{article}
\usepackage[margin=2cm]{geometry}
\usepackage{amsmath}

\begin{document}

\noindent
This is the transition rate $A_{nm}$
for a spontaneous emission process $\psi_n\rightarrow\psi_m$.
\begin{equation*}
A_{nm}=\frac{e^2}{3\pi\varepsilon_0\hbar c^3}\,\omega_{nm}^3\,|\langle r_{nm}\rangle|^2
\end{equation*}
The transition frequency is
\begin{equation*}
\omega_{nm}=\frac{1}{\hbar}(E_n-E_m)
\end{equation*}

\noindent
For a hydrogen atom we have
\begin{equation*}
E_n=-\frac{\mu}{2n^2}\left(\frac{e^2}{4\pi\varepsilon_0\hbar}\right)^2
\end{equation*}
where $\mu$ is reduced electron mass.

\bigskip
\noindent
The radial density is
\begin{equation*}
|\langle r_{nm}\rangle|^2
=|\langle x_{nm}\rangle|^2
+|\langle y_{nm}\rangle|^2
+|\langle z_{nm}\rangle|^2
\end{equation*}
where
\begin{align*}
\langle x_{nm}\rangle&=\int\psi_m^*\,(r\sin\theta\cos\phi)\,\psi_n\,dV
\\
\langle y_{nm}\rangle&=\int\psi_m^*\,(r\sin\theta\sin\phi)\,\psi_n\,dV
\\
\langle z_{nm}\rangle&=\int\psi_m^*\,(r\cos\theta)\,\psi_n\,dV
\end{align*}

\noindent
Let us compute $A_{21}$ for a hydrogen atom.
For $n=2$ there are four possible states.
\begin{center}
\begin{tabular}{rrr}
$n$ & $\ell$ & $m$\\
2 & 1 & 1 \\
2 & 1 & $-1$ \\
2 & 1 & 0 \\
2 & 0 & 0
\end{tabular}
\end{center}

\noindent
The following table shows the radial density for every possible transition.
\begin{center}
\begin{tabular}{rcccc}
& $\psi_{2,1,1}\rightarrow\psi_{1,0,0}$
& $\psi_{2,1,-1}\rightarrow\psi_{1,0,0}$
& $\psi_{2,1,0}\rightarrow\psi_{1,0,0}$
& $\psi_{2,0,0}\rightarrow\psi_{1,0,0}$
\\[2ex]
$\langle x_{21}\rangle=$ & $-\frac{128}{243}\,a_0$ & $\frac{128}{243}\,a_0$ & 0 & 0
\\[2ex]
$\langle y_{21}\rangle=$ & $-\frac{128}{243}i\,a_0$ & $-\frac{128}{243}i\,a_0$ & 0 & 0
\\[2ex]
$\langle z_{21}\rangle=$ & 0 & 0 & $\frac{128}{243}\sqrt{2}\,a_0$ & 0
\\[2ex]
$|\langle r_{21}\rangle|^2=$ & $\frac{32768}{59049}\,a_0^2$ & $\frac{32768}{59049}\,a_0^2$ & $\frac{32768}{59049}\,a_0^2$ & 0
\end{tabular}
\end{center}

\medskip
\noindent
Note that the transition rate of $\psi_{2,0,0}\rightarrow\psi_{1,0,0}$ is zero.
For the allowed transitions, the radial density
$|\langle r_{21}\rangle|^2$ is independent of $\ell$ and $m$.

\bigskip
\noindent
Symbol $a_0$ is the Bohr radius
\begin{equation*}
a_0=\frac{4\pi\varepsilon_0\hbar^2}{e^2 m_e}
=5.29\times10^{-11}\,\text{meter}
\end{equation*}

\bigskip
\noindent
For the transition frequency we have
\begin{equation*}
\omega_{21}=\frac{1}{\hbar}(E_2-E_1)=\frac{3e^4\mu}{128\pi^2\varepsilon_0^2\hbar^3}
=1.55\times10^{16}\,\text{second}^{-1}
\end{equation*}

\noindent
Hence
\begin{equation*}
A_{21}=\frac{e^2}{3\pi\varepsilon_0\hbar c^3}
\underset{\substack{\\[1ex]\omega_{21}}}
{\left(\frac{3e^4\mu}{128\pi^2\varepsilon_0^2\hbar^3}\right)^3}
\,
\underset{\substack{\\[1ex]|\langle r_{21}\rangle|^2}}
{\frac{32768}{59049}\left(\frac{4\pi\varepsilon_0\hbar^2}{e^2 m_e}\right)^2}
=
\frac{e^{10}\mu^3}{26244\pi^5\varepsilon_0^5\hbar^6 c^3 m_e^2}
=6.26\times10^8\,\text{second}^{-1}
\end{equation*}

\noindent
Let us analyze the units involved.
For the coefficient of $A_{nm}$ we have
\begin{equation*}
\frac{e^2}{3\pi\varepsilon_0\hbar c^3}\propto
\frac{
\underset{e^2}
{\text{ampere}^2\,\text{second}^2}
}
{
\underset{\varepsilon_0}
{\left(\frac{\text{ampere}^2\,\text{second}^4}{\text{kilogram}\,\text{meter}^3}\right)}
\underset{\hbar}
{\left(\frac{\text{kilogram}\,\text{meter}^2}{\text{second}}\right)}
\underset{c^3}
{\left(\frac{\text{meter}^3}{\text{second}^3}\right)}
}
=\frac{\text{second}^2}{\text{meter}^2}
\end{equation*}

\noindent
For the transition frequency we have
\begin{equation*}
\omega_{21}=
\frac{3e^4\mu}{128\pi^2\varepsilon_0^2\hbar^3}
\propto
\frac{
\underset{e^4}
{\left(\text{ampere}^4\,\text{second}^4\right)}
\,
\underset{\mu}
{\text{kilogram}}
}
{
\underset{\varepsilon_0^2}
{\left(\frac{\text{ampere}^4\,\text{second}^8}{\text{kilogram}^2\,\text{meter}^6}\right)}
\underset{\hbar^3}
{\left(\frac{\text{kilogram}^3\,\text{meter}^6}{\text{second}^3}\right)}
}
=\text{second}^{-1}
\end{equation*}

\noindent
For the Bohr radius we have
\begin{equation*}
a_0=\frac{4\pi\varepsilon_0\hbar^2}{e^2 m_e}
\propto
\frac
{
\underset{\varepsilon_0}
{\left(\frac{\text{ampere}^2\,\text{second}^4}{\text{kilogram}\,\text{meter}^3}\right)}
\underset{\hbar^2}
{\left(\frac{\text{kilogram}^2\,\text{meter}^4}{\text{second}^2}\right)}
}
{
\underset{e^2}
{\left(\text{ampere}^2\,\text{second}^2\right)}
\,
\underset{m_e}
{\text{kilogram}}
}
=\text{meter}
\end{equation*}

\noindent
Hence
\begin{equation*}
A_{nm}\propto
\frac{\text{second}^2}{\text{meter}^2}
\times
\underset{\substack{\\[1ex]\omega_{nm}^3}}{\text{second}^{-3}}
\times
\underset{\substack{\\[1ex]a_0^2}}{\text{meter}^2}
=\text{second}^{-1}
\end{equation*}

\end{document}
