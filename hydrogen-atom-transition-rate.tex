\documentclass[12pt]{article}
\usepackage[margin=2cm]{geometry}
\usepackage{amsmath}

\begin{document}

\noindent
The transition rate from state $n$ to $m$ is
\begin{equation*}
A_{nm}=\frac{e^2}{3\pi\varepsilon_0\hbar c^3}\,\omega_{nm}^3\,|\langle r_{nm}\rangle|^2
\end{equation*}
where
\begin{equation*}
\omega_{nm}=\frac{1}{\hbar}(E_n-E_m),\qquad
E_n=-\frac{\mu}{2n^2}\left(\frac{e^2}{4\pi\varepsilon_0\hbar}\right)^2
\end{equation*}
Symbol $\mu$ is the reduced electron mass.

\bigskip
\noindent
The radial density is
\begin{equation*}
|\langle r_{nm}\rangle|^2
=|\langle x_{nm}\rangle|^2
+|\langle y_{nm}\rangle|^2
+|\langle z_{nm}\rangle|^2
\end{equation*}
where
\begin{align*}
\langle x_{nm}\rangle&=\int\psi_m^*\,(r\sin\theta\cos\phi)\,\psi_n\,dV
\\
\langle y_{nm}\rangle&=\int\psi_m^*\,(r\sin\theta\sin\phi)\,\psi_n\,dV
\\
\langle z_{nm}\rangle&=\int\psi_m^*\,(r\cos\theta)\,\psi_n\,dV
\end{align*}

\noindent
Let us compute $A_{21}$ for the hydrogen atom.
For $n=2$ there are four possible states.
\begin{center}
\begin{tabular}{rrr}
$n$ & $\ell$ & $m$\\
2 & 1 & 1 \\
2 & 1 & $-1$ \\
2 & 1 & 0 \\
2 & 0 & 0
\end{tabular}
\end{center}

\noindent
The following table shows calculations for every possible transition.
\begin{center}
\begin{tabular}{rcccc}
& {\footnotesize$(2,1,1)\rightarrow(1,0,0)$}
& {\footnotesize$(2,1,-1)\rightarrow(1,0,0)$}
& {\footnotesize$(2,1,0)\rightarrow(1,0,0)$}
& {\footnotesize$(2,0,0)\rightarrow(1,0,0)$}
\\[2ex]
$\langle x_{nm}\rangle=$ & $-\frac{128}{243}\,a_0$ & $\frac{128}{243}\,a_0$ & 0 & 0
\\[1ex]
$\langle y_{nm}\rangle=$ & $-\frac{128}{243}i\,a_0$ & $-\frac{128}{243}i\,a_0$ & 0 & 0
\\[1ex]
$\langle z_{nm}\rangle=$ & 0 & 0 & $\frac{128}{243}\sqrt{2}\,a_0$ & 0
\\[1ex]
$|\langle r_{nm}\rangle|^2=$ & $\frac{32768}{59049}\,a_0^2$ & $\frac{32768}{59049}\,a_0^2$ & $\frac{32768}{59049}\,a_0^2$ & 0
\end{tabular}
\end{center}

\bigskip
\noindent
Note that the transition $(2,0,0)\rightarrow(1,0,0)$ is not allowed.
For the allowed transitions, the result is independent of $\ell$ and $m$.

\bigskip
\noindent
Symbol $a_0$ is the Bohr radius
\begin{equation*}
a_0=\frac{4\pi\varepsilon_0\hbar^2}{e^2\mu}
\end{equation*}

\bigskip
\noindent
We have
\begin{equation*}
\omega_{12}=\frac{1}{\hbar}(E_2-E_1)=\frac{3e^4\mu}{128\pi^2\varepsilon_0^2\hbar^3}
\end{equation*}

\noindent
Hence
\begin{equation*}
A_{12}=\frac{e^2}{3\pi\varepsilon_0\hbar c^3}
\underset{\substack{\\[1ex]\omega_{12}}}
{\left(\frac{3e^4\mu}{128\pi^2\varepsilon_0^2\hbar^3}\right)^3}
\,
\underset{\substack{\\[1ex]|\langle r_{12}\rangle|^2}}
{\frac{32768}{59049}\left(\frac{4\pi\varepsilon_0\hbar^2}{e^2\mu}\right)^2}
=
\frac{e^{10}\mu}{26244\pi^5\varepsilon_0^5\hbar^6 c^3}
=6.27\times10^8\,\text{second}^{-1}
\end{equation*}

\end{document}
