\documentclass[12pt]{article}
\usepackage{amsmath}
\usepackage{amssymb}

\parindent=0pt

\newcommand\U{\vert\Phi_0\vert^2}

\begin{document}

9-7.
Show, for the vacuum state, the expectation value
of $\bar a_{1,\mathbf k}^*\bar a_{1,\mathbf q}$
is $(\hbar/2kc)\delta_{\mathbf k,\mathbf q}$ and that
of $\bar a_{1,\mathbf k}\bar a_{1,\mathbf q}$
is $(\hbar/2kc)\delta_{-\mathbf k,\mathbf q}$.
Develop a formula for the expectation of
$(\bar a_{1,\mathbf k}^*\bar a_{1,\mathbf k})^r$
for integral $r$ and explain thereby how the expectation of such
quantities as
$(\bar a_{1,\mathbf k}^*\bar a_{1,\mathbf k})^r
(\bar a_{1,\mathbf q}^*\bar a_{1,\mathbf q})^s$
can be got for $\mathbf q\ne\mathbf k$.
Show that the expectation of 
$(\bar a_{1,\mathbf k})^2$ or
$(\bar a_{1,\mathbf k}^*)^2$ vanishes.
Show that the expectation of 
$(\bar a_{1,\mathbf k})^2$ or
$(\bar a_{1,\mathbf k}^*)^2$ vanishes.
Show that the expectation of the product of any odd number of
$\bar a$'s is zero and that you can compute the expectation value of
any product of $\bar a$'s or $\bar a^*$'s for the vacuum state.

\bigskip
We will use the following table of integrals.
\begin{align*}
&\int_{-\infty}^\infty\exp(-ax^2+b)\,dx=\sqrt{\frac{\pi}{a}}\exp(b)
\tag{1}
\\
&\int_{-\infty}^\infty x\exp(-ax^2+b)\,dx=0
\tag{2}
\\
&\int_{-\infty}^\infty x^2\exp(-ax^2+b)\,dx=\frac{1}{2a}\sqrt{\frac{\pi}{a}}\exp(b)
\tag{3}
\end{align*}

For simplicity of notation, let
\begin{equation*}
A=\bar a_{1,\mathbf k}^c
\qquad
B=\bar a_{1,\mathbf k}^s
\qquad
C=\bar a_{1,\mathbf q}^c
\qquad
D=\bar a_{1,\mathbf q}^s
\end{equation*}

These formulas convert $\bar a$ to sine and cosine modes.
\begin{equation*}
\bar a_{1,\mathbf k}=\frac{1}{\sqrt2}(A-iB)
\qquad
\bar a_{1,\mathbf q}=\frac{1}{\sqrt2}(C-iD)
\tag{4}
\end{equation*}

Adapted from equation (8.84)
\begin{equation*}
\langle\Phi_0|f|\Phi_0\rangle
=\int\limits_{-\infty}^\infty\cdots\int\limits_{-\infty}^\infty
f\U
\,dA\,dB\,dC\,dD
\end{equation*}

The following $\U$ is adapted from equation (9.43).
Symbol $q$ is a mode (physical unit $\text{meter}^{-1}$), not an electric charge.
Note that we {\it could} include other modes in addition to $k$ and $q$.
However, integrals over unused modes are
cancelled by the normalization constant.
\begin{equation*}
\U=\Phi_0^*\Phi_0=\exp\left(
-\frac{kc}{\hbar}A^2
-\frac{kc}{\hbar}B^2
-\frac{qc}{\hbar}C^2
-\frac{qc}{\hbar}D^2
\right)
\end{equation*}

Compute the normalization constant $K$.
\begin{equation*}
K=\langle\Phi_0|1|\Phi_0\rangle
=\int\limits_{-\infty}^\infty\cdots\int\limits_{-\infty}^\infty
\U\,dA\,dB\,dC\,dD
\end{equation*}

By integral (1) for each factor in the measure and with $a=kc/\hbar$
\begin{equation*}
K=
\left(\frac{\pi\hbar}{kc}\right)^{1/2}
\left(\frac{\pi\hbar}{kc}\right)^{1/2}
\left(\frac{\pi\hbar}{qc}\right)^{1/2}
\left(\frac{\pi\hbar}{qc}\right)^{1/2}
\end{equation*}

Compute the expectation of $\bar a_{1,\mathbf k}^*\bar a_{1,\mathbf k}$.
From (4) we have
\begin{equation*}
\bar a_{1,\mathbf k}^*\bar a_{1,\mathbf k}=\frac{A^2+B^2}{2}
\end{equation*}

Hence
\begin{equation*}
\langle\Phi_0|\bar a_{1,\mathbf k}^*\bar a_{1,\mathbf k}|\Phi_0\rangle
=\frac{1}{K}\int\limits_{-\infty}^\infty\cdots\int\limits_{-\infty}^\infty
\left(\frac{A^2+B^2}{2}\right)
\U\,dA\,dB\,dC\,dD
\end{equation*}

Rewrite as
\begin{multline*}
\langle\Phi_0|\bar a_{1,\mathbf k}^*\bar a_{1,\mathbf k}|\Phi_0\rangle
=\frac{1}{2K}
\int\limits_{-\infty}^\infty\cdots\int\limits_{-\infty}^\infty
A^2
\U\,dA\,dB\,dC\,dD
\\{}+
\frac{1}{2K}
\int\limits_{-\infty}^\infty\cdots\int\limits_{-\infty}^\infty
B^2
\U\,dA\,dB\,dC\,dD
\end{multline*}

By integrals (1) and (3) with $a=kc/\hbar$ we have
\begin{equation*}
\langle\Phi_0|\bar a_{1,\mathbf k}^*\bar a_{1,\mathbf k}|\Phi_0\rangle
=\frac{1}{K}\frac{\hbar}{2kc}
\left(\frac{\pi\hbar}{kc}\right)^{1/2}
\left(\frac{\pi\hbar}{kc}\right)^{1/2}
\left(\frac{\pi\hbar}{qc}\right)^{1/2}
\left(\frac{\pi\hbar}{qc}\right)^{1/2}
\end{equation*}

The radicals are cancelled by the normalization constant $K$, hence
\begin{equation*}
\langle\Phi_0|\bar a_{1,\mathbf k}^*\bar a_{1,\mathbf k}|\Phi_0\rangle=\frac{\hbar}{2kc}
\tag{5}
\end{equation*}

Compute the expectation of $\bar a_{1,\mathbf k}^*\bar a_{1,\mathbf q}$.
\begin{equation*}
\bar a_{1,\mathbf k}^*\bar a_{1,\mathbf q}
=\frac{AC+BD-iAD+iBC}{2}
\end{equation*}

Hence
\begin{multline*}
\langle\Phi_0|\bar a_{1,\mathbf k}^*\bar a_{1,\mathbf q}|\Phi_0\rangle=
\\
\frac{1}{K}\int\limits_{-\infty}^\infty\cdots\int\limits_{-\infty}^\infty
\left(\frac{AC+BD-iAD+iBC}{2}\right)
\U\,dA\,dB\,dC\,dD
\end{multline*}

By integral (2) all terms are zero, hence
\begin{equation*}
\langle\Phi_0|\bar a_{1,\mathbf k}^*\bar a_{1,\mathbf q}|\Phi_0\rangle=0
\tag{6}
\end{equation*}

Combine (5) and (6) to obtain
\begin{equation*}
\langle\Phi_0|\bar a_{1,\mathbf k}^*\bar a_{1,\mathbf q}|\Phi_0\rangle=\frac{\hbar}{2kc}\delta_{\mathbf k,\mathbf q}
\end{equation*}

By equation (8.77)
\begin{equation*}
\bar a_{1,\mathbf k}^*=\bar a_{1,-\mathbf k}
\end{equation*}

Hence
\begin{equation*}
\langle\Phi_0|\bar a_{1,-\mathbf k}\bar a_{1,\mathbf q}|\Phi_0\rangle=\frac{\hbar}{2kc}\delta_{\mathbf k,\mathbf q}
\end{equation*}

(9-7 cont'd)
Develop a formula for the expectation of
$(\bar a_{1,\mathbf k}^*\bar a_{1,\mathbf k})^r$
for integral $r$ and explain thereby how the expectation of such
quantities as
$(\bar a_{1,\mathbf k}^*\bar a_{1,\mathbf k})^r
(\bar a_{1,\mathbf q}^*\bar a_{1,\mathbf q})^s$
can be got for $\mathbf q\ne\mathbf k$.

\bigskip
By the binomial theorem
\begin{equation*}
\left(\frac{A^2+B^2}{2}\right)^r=\frac{1}{2^r}\sum_{j=0}^r\binom{r}{j}A^{2j}B^{2(r-j)}
\tag{7}
\end{equation*}

To compute the expectation of (7) we need the following integral.
\begin{align*}
\int_{-\infty}^\infty x^{2n}\exp(-ax^2+b)
&=\frac{1\cdot3\cdot5\cdots(2n-1)}{2^na^n}\sqrt{\frac{\pi}{a}}\exp(b)
\\
&=(2n-1)!!
\left(\frac{1}{2a}\right)^n
\sqrt{\frac{\pi}{a}}\exp(b)
\tag{8}
\end{align*}

From equation (8), define the following function $F$.
(The $\sqrt{\pi/a}$ factor is left out because it gets cancelled by the normalization constant $K$.)
\begin{equation*}
F(n)=(2n-1)!!\left(\frac{\hbar}{2kc}\right)^n
\end{equation*}

Note that
\begin{equation*}
F(j)F(r-j)=
(2j-1)!!\,(2r-2j-1)!!
\left(\frac{\hbar}{2kc}\right)^j
\left(\frac{\hbar}{2kc}\right)^{r-j}
\end{equation*}

It turns out that
\begin{equation*}
\frac{1}{2^r}\sum_{j=0}^r\binom{r}{j}(2j-1)!!\,(2r-2j-1)!!=r!
\end{equation*}

Hence
\begin{equation*}
\langle\Phi_0^*|(\bar a_{1,\mathbf k}^*\bar a_{1,\mathbf k})^r|\Phi_0\rangle
=r!\left(\frac{\hbar}{2kc}\right)^r
\end{equation*}

Regarding the $\mathbf q\ne\mathbf k$ part of the problem, we have
\begin{multline*}
\left(\frac{A^2+B^2}{2}\right)^r\left(\frac{C^2+D^2}{2}\right)^s={}
\\
\left(\frac{1}{2^r}\sum_{j=0}^r\binom{r}{j}A^{2j}B^{2(r-j)}\right)
\left(\frac{1}{2^s}\sum_{k=0}^s\binom{s}{k}C^{2k}D^{2(r-k)}\right)
\end{multline*}

Hence
\begin{equation*}
\langle\Phi_0^*|
(\bar a_{1,\mathbf k}^*\bar a_{1,\mathbf k})^r
(\bar a_{1,\mathbf q}^*\bar a_{1,\mathbf q})^s
|\Phi_0\rangle
=r!\left(\frac{\hbar}{2kc}\right)^r
s!\left(\frac{\hbar}{2qc}\right)^s
\end{equation*}

(9-7 cont'd)
Show that the expectation of 
$(\bar a_{1,\mathbf k})^2$ or
$(\bar a_{1,\mathbf k}^*)^2$ vanishes.

\bigskip
We have
\begin{equation*}
(\bar a_{1,\mathbf k})^2=\frac{A^2-B^2}{2}-iAB
\qquad
(\bar a_{1,\mathbf k}^*)^2=\frac{A^2-B^2}{2}+iAB
\end{equation*}

The integrals of $A^2$ and $-B^2$ cancel each other.
The integral of $AB$ vanishes by integral (2).

\bigskip
(9-7 cont'd)
Show that the expectation of the product of any odd number of
$\bar a$'s is zero and that you can compute the expectation value of
any product of $\bar a$'s or $\bar a^*$'s for the vacuum state.

% FIXME answer last question

\end{document}
