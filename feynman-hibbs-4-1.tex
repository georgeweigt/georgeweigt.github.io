\documentclass[12pt]{article}
\usepackage{amsmath}
\usepackage{amssymb}

\parindent=0pt

\begin{document}

Feynman and Hibbs problem 4-1

https://www2.oberlin.edu/physics/dstyer/FeynmanHibbs/

\bigskip
Show that for a single particle moving in three dimensions
in a potential energy $V(\mathbf x,t)$ the Schrodinger equation is
\begin{equation*}
\frac{\partial\psi(\mathbf x,t)}{\partial t}
=-\frac{i}{\hbar}\left(
-\frac{\hbar^2}{2m}\nabla^2\psi(\mathbf x,t)+V(\mathbf x,t)\psi(\mathbf x,t)\right)
\end{equation*}

This is the Lagrangian.
\begin{equation*}
L(\dot{\mathbf x},\mathbf x)=\frac{m}{2}\dot{\mathbf x}^2-V(\mathbf{x},t)\tag{1}
\end{equation*}

Extend equation (4.3) from one dimension to three dimensions.
\begin{equation*}
\psi(\mathbf{x},t+\epsilon)
=\frac{1}{A}\int_{\mathbb R^3}
\exp\left(
\frac{i\epsilon}{\hbar}L
\left(\frac{\mathbf{x}-\mathbf{y}}{\epsilon},\frac{\mathbf{x}+\mathbf{y}}{2}\right)
\right)\psi(\mathbf{y},t)\,dy_1\,dy_2\,dy_3
\end{equation*}
where
\begin{equation*}
\int_{\mathbb R^3}
\equiv\int_{-\infty}^\infty\int_{-\infty}^\infty\int_{-\infty}^\infty
\end{equation*}

From (1) we have
\begin{equation*}
L\left(\frac{\mathbf{x}-\mathbf{y}}{\epsilon},\frac{\mathbf{x}+\mathbf{y}}{2}\right)
=\frac{m}{2\epsilon^2}(\mathbf{x}-\mathbf{y})^2
-V\left(\frac{\mathbf{x}+\mathbf{y}}{2},t\right)
\end{equation*}

Hence
\begin{multline*}
\psi(\mathbf{x},t+\epsilon)=
\frac{1}{A}\int_{\mathbb R^3}
\exp\left(
\frac{im}{2\hbar\epsilon}(\mathbf{x}-\mathbf{y})^2
-\frac{i\epsilon}{\hbar}V\left(\frac{\mathbf{x}+\mathbf{y}}{2},t\right)
\right)
\\
{}\times\psi(\mathbf{y},t)
\,dy_1\,dy_2\,dy_3
\end{multline*}

Let
\begin{equation*}
\mathbf y=\mathbf x+\boldsymbol\eta
\end{equation*}

Then
\begin{equation*}
(\mathbf x-\mathbf y)^2=\boldsymbol\eta^2,\quad
\frac{\mathbf x+\mathbf y}{2}=\mathbf{x}+\tfrac{1}{2}\boldsymbol\eta,\quad
dy_1\,dy_2\,dy_3=d\eta_1\,d\eta_2\,d\eta_3
\end{equation*}

Hence
\begin{multline*}
\psi(\mathbf{x},t+\epsilon)=
\frac{1}{A}\int_{\mathbb R^3}
\exp\left(
\frac{im}{2\hbar\epsilon}\boldsymbol\eta^2
-\frac{i\epsilon}{\hbar} V\left(\mathbf x+\tfrac{1}{2}\boldsymbol\eta,t\right)
\right)
\\
{}\times\psi(\mathbf{x}+\boldsymbol\eta,t)\,d\eta_1\,d\eta_2\,d\eta_3
\end{multline*}

Factor the exponential.
\begin{multline*}
\psi(\mathbf{x},t+\epsilon)=
\frac{1}{A}\int_{\mathbb R^3}
\exp\left(\frac{im}{2\hbar\epsilon}\boldsymbol\eta^2\right)
\exp\left(-\frac{i\epsilon}{\hbar}V\left(\mathbf x+\tfrac{1}{2}\boldsymbol\eta,t\right)\right)
\\
{}\times\psi(\mathbf{x}+\boldsymbol\eta,t)\,d\eta_1\,d\eta_2\,d\eta_3
\tag{2}
\end{multline*}

Now we are going to use an approximation for the exponential of $V$.
From the identity $\exp(i\theta)=\cos(\theta)+i\sin(\theta)$ we have
\begin{multline*}
\exp\left(-\frac{i\epsilon}{\hbar}V\left(\mathbf x+\tfrac{1}{2}\boldsymbol\eta,t\right)\right)=
\\
\cos\left(-\frac{\epsilon}{\hbar}V\left(\mathbf x+\tfrac{1}{2}\boldsymbol\eta,t\right)\right)
+i\sin\left(-\frac{\epsilon}{\hbar}V\left(\mathbf x+\tfrac{1}{2}\boldsymbol\eta,t\right)\right)
\end{multline*}

For small $\epsilon$ we have the approximation
\begin{equation*}
\exp\left(-\frac{i\epsilon}{\hbar}V\left(\mathbf x+\tfrac{1}{2}\boldsymbol\eta,t\right)\right)\approx
1-\frac{i\epsilon}{\hbar}V\left(\mathbf x+\tfrac{1}{2}\boldsymbol\eta,t\right)
\end{equation*}

The authors write that the $\tfrac{1}{2}\boldsymbol\eta$ term can be dropped
``because the error is of higher order than $\epsilon$.''
Hence
\begin{equation*}
\exp\left(-\frac{i\epsilon}{\hbar}V\left(\mathbf x+\tfrac{1}{2}\boldsymbol\eta,t\right)\right)\approx
1-\frac{i\epsilon}{\hbar}V\left(\mathbf x,t\right)
\tag{3}
\end{equation*}

Substituting (3) into (2) yields
\begin{multline*}
\psi(\mathbf{x},t+\epsilon)=
\frac{1}{A}\int_{\mathbb R^3}
\exp\left(\frac{im}{2\hbar\epsilon}\boldsymbol\eta^2\right)
\left(1-\frac{i\epsilon}{\hbar}V\left(\mathbf x,t\right)\right)
\\
{}\times\psi(\mathbf x+\boldsymbol\eta,t)\,d\eta_1\,d\eta_2\,d\eta_3
\tag{4}
\end{multline*}

Next we will use the following Taylor series approximations.
\begin{equation*}
\begin{aligned}
\psi(\mathbf x,t+\epsilon)&\approx\psi(\mathbf x,t)+\epsilon\frac{\partial\psi}{\partial t}
\\
\psi(\mathbf x+\boldsymbol\eta,t)&\approx\psi(\mathbf x,t)+\boldsymbol\eta\cdot\nabla\psi
+\tfrac{1}{2}\boldsymbol\eta\cdot\nabla(\boldsymbol\eta\cdot\nabla\psi)
\end{aligned}
\tag{5}
\end{equation*}

Note: In component notation
\begin{equation*}
\boldsymbol\eta\cdot\nabla\psi=
\eta_1\frac{\partial\psi}{\partial x_1}+
\eta_2\frac{\partial\psi}{\partial x_2}+
\eta_2\frac{\partial\psi}{\partial x_2}
\end{equation*}
and
\begin{multline*}
\boldsymbol\eta\cdot\nabla(\boldsymbol\eta\cdot\nabla\psi)=
\eta_1^2\frac{\partial^2\psi}{\partial x_1^2}
+\eta_2^2\frac{\partial^2\psi}{\partial x_2^2}
+\eta_3^2\frac{\partial^2\psi}{\partial x_3^2}
\\
{}+2\eta_1\eta_2\frac{\partial^2\psi}{\partial x_1\partial x_2}
+2\eta_1\eta_3\frac{\partial^2\psi}{\partial x_1\partial x_3}
+2\eta_2\eta_3\frac{\partial^2\psi}{\partial x_2\partial x_3}
\end{multline*}

Substitute the approximations (5) into (4).
\begin{multline*}
\psi(\mathbf{x},t)+\epsilon\frac{\partial\psi}{\partial t}=
\frac{1}{A}\int_{\mathbb R^3}
\exp\left(\frac{im}{2\hbar\epsilon}\boldsymbol\eta^2\right)
\left(1-\frac{i\epsilon}{\hbar}V\left(\mathbf x,t\right)\right)
\\
{}\times\left(
\psi(\mathbf x,t)+\boldsymbol\eta\cdot\nabla\psi
+\tfrac{1}{2}\boldsymbol\eta\cdot\nabla(\boldsymbol\eta\cdot\nabla\psi)
\right)\,d\eta_1\,d\eta_2\,d\eta_3
\tag{6}
\end{multline*}

Expand the right-hand side.
\begin{align*}
&\psi(\mathbf{x},t)+\epsilon\frac{\partial\psi}{\partial t}
=\frac{\psi(\mathbf x,t)}{A}\int_{\mathbb R^3}
\exp\left(\frac{im}{2\hbar\epsilon}\boldsymbol\eta^2\right)
\,d\eta_1\,d\eta_2\,d\eta_3
\tag{7}
\\
&\qquad{}+\frac{1}{A}\int_{\mathbb R^3}
\exp\left(\frac{im}{2\hbar\epsilon}\boldsymbol\eta^2\right)
\boldsymbol\eta\cdot\nabla\psi
\,d\eta_1\,d\eta_2\,d\eta_3
\tag{8}
\\
&\qquad{}+\frac{1}{2A}\int_{\mathbb R^3}
\exp\left(\frac{im}{2\hbar\epsilon}\boldsymbol\eta^2\right)
\boldsymbol\eta\cdot\nabla(\boldsymbol\eta\cdot\nabla\psi)
\,d\eta_1\,d\eta_2\,d\eta_3
\tag{9}
\\
&\qquad{}-\frac{i\epsilon}{A\hbar}V\left(\mathbf x,t\right)\psi(\mathbf x,t)\int_{\mathbb R^3}
\exp\left(\frac{im}{2\hbar\epsilon}\boldsymbol\eta^2\right)
\,d\eta_1\,d\eta_2\,d\eta_3
\tag{10}
\\
&\qquad{}-\frac{i\epsilon}{A\hbar}V\left(\mathbf x,t\right)\int_{\mathbb R^3}
\exp\left(\frac{im}{2\hbar\epsilon}\boldsymbol\eta^2\right)
\boldsymbol\eta\cdot\nabla\psi
\,d\eta_1\,d\eta_2\,d\eta_3
\tag{11}
\\
&\qquad{}-\frac{i\epsilon}{2A\hbar}V\left(\mathbf x,t\right)\int_{\mathbb R^3}
\exp\left(\frac{im}{2\hbar\epsilon}\boldsymbol\eta^2\right)
\boldsymbol\eta\cdot\nabla(\boldsymbol\eta\cdot\nabla\psi)
\,d\eta_1\,d\eta_2\,d\eta_3
\tag{12}
\end{align*}

To solve the above integrals, we will use the following formulas provided by the authors.
\begin{align*}
&\int_{-\infty}^\infty\exp\left(\frac{imx^2}{2\hbar\epsilon}\right)\,dx
=\left(\frac{2\pi i\hbar\epsilon}{m}\right)^{1/2}
\tag{13}
\\
&\int_{-\infty}^\infty\exp\left(\frac{imx^2}{2\hbar\epsilon}\right)x\,dx
=0
\tag{14}
\\
&\int_{-\infty}^\infty\exp\left(\frac{imx^2}{2\hbar\epsilon}\right)x^2\,dx
=\frac{i\hbar\epsilon}{m}\left(\frac{2\pi i\hbar\epsilon}{m}\right)^{1/2}
\tag{15}
\end{align*}

Rewrite the integral in (7) and (10) in component notation.
\begin{multline*}
\int_{\mathbb R^3}
\exp\left(\frac{im}{2\hbar\epsilon}\boldsymbol\eta^2\right)
\,d\eta_1\,d\eta_2\,d\eta_3
\\
{}=\int_{\mathbb R^3}
\exp\left(\frac{im\eta_1^2}{2\hbar\epsilon}\right)
\exp\left(\frac{im\eta_2^2}{2\hbar\epsilon}\right)
\exp\left(\frac{im\eta_3^2}{2\hbar\epsilon}\right)
\,d\eta_1\,d\eta_2\,d\eta_3
\end{multline*}

Then by equation (13)
\begin{equation*}
\int_{\mathbb R^3}
\exp\left(\frac{im}{2\hbar\epsilon}\boldsymbol\eta^2\right)
\,d\eta_1\,d\eta_2\,d\eta_3
=\left(\frac{2\pi i\hbar\epsilon}{m}\right)^{3/2}
\tag{16}
\end{equation*}

Rewrite the integral in (8) and (11) in component notation.
\begin{align*}
&\int_{\mathbb R^3}
\exp\left(\frac{im}{2\hbar\epsilon}\boldsymbol\eta^2\right)
\boldsymbol\eta\cdot\nabla\psi
\,d\eta_1\,d\eta_2\,d\eta_3
\\
&\qquad{}=
\int_{\mathbb R^3}
\exp\left(\frac{im\eta_1^2}{2\hbar\epsilon}\right)
\exp\left(\frac{im\eta_2^2}{2\hbar\epsilon}\right)
\exp\left(\frac{im\eta_3^2}{2\hbar\epsilon}\right)
\eta_1\frac{\partial\psi}{\partial x_1}
\,d\eta_1\,d\eta_2\,d\eta_3
\\
&\qquad\qquad{}+\int_{\mathbb R^3}
\exp\left(\frac{im\eta_1^2}{2\hbar\epsilon}\right)
\exp\left(\frac{im\eta_2^2}{2\hbar\epsilon}\right)
\exp\left(\frac{im\eta_3^2}{2\hbar\epsilon}\right)
\eta_2\frac{\partial\psi}{\partial x_2}
\,d\eta_1\,d\eta_2\,d\eta_3
\\
&\qquad\qquad{}+\int_{\mathbb R^3}
\exp\left(\frac{im\eta_1^2}{2\hbar\epsilon}\right)
\exp\left(\frac{im\eta_2^2}{2\hbar\epsilon}\right)
\exp\left(\frac{im\eta_3^2}{2\hbar\epsilon}\right)
\eta_3\frac{\partial\psi}{\partial x_3}
\,d\eta_1\,d\eta_2\,d\eta_3
\end{align*}

Then by equation (14)
\begin{equation*}
\int_{\mathbb R^3}
\exp\left(\frac{im}{2\hbar\epsilon}\boldsymbol\eta^2\right)
\boldsymbol\eta\cdot\nabla\psi
\,d\eta_1\,d\eta_2\,d\eta_3=0
\tag{17}
\end{equation*}

Rewrite the integral in (9) and (12) in component notation.
\begin{align*}
&\int_{\mathbb R^3}
\exp\left(\frac{im}{2\hbar\epsilon}\boldsymbol\eta^2\right)
\boldsymbol\eta\cdot\nabla(\boldsymbol\eta\cdot\nabla\psi)
\,d\eta_1\,d\eta_2\,d\eta_3
\\
&\qquad{}=\int_{\mathbb R^3}
\exp\left(\frac{im\eta_1^2}{2\hbar\epsilon}\right)
\exp\left(\frac{im\eta_2^2}{2\hbar\epsilon}\right)
\exp\left(\frac{im\eta_3^2}{2\hbar\epsilon}\right)
\eta_1^2\frac{\partial^2\psi}{\partial x_1^2}
\,d\eta_1\,d\eta_2\,d\eta_3
\\
&\qquad\qquad{}+\int_{\mathbb R^3}
\exp\left(\frac{im\eta_1^2}{2\hbar\epsilon}\right)
\exp\left(\frac{im\eta_2^2}{2\hbar\epsilon}\right)
\exp\left(\frac{im\eta_3^2}{2\hbar\epsilon}\right)
\eta_2^2\frac{\partial^2\psi}{\partial x_2^2}
\,d\eta_1\,d\eta_2\,d\eta_3
\\
&\qquad\qquad{}+\int_{\mathbb R^3}
\exp\left(\frac{im\eta_1^2}{2\hbar\epsilon}\right)
\exp\left(\frac{im\eta_2^2}{2\hbar\epsilon}\right)
\exp\left(\frac{im\eta_3^2}{2\hbar\epsilon}\right)
\eta_3^2\frac{\partial^2\psi}{\partial x_3^2}
\,d\eta_1\,d\eta_2\,d\eta_3
\\
&\qquad\qquad{}+\int_{\mathbb R^3}
\exp\left(\frac{im\eta_1^2}{2\hbar\epsilon}\right)
\exp\left(\frac{im\eta_2^2}{2\hbar\epsilon}\right)
\exp\left(\frac{im\eta_3^2}{2\hbar\epsilon}\right)
2\eta_1\eta_2\frac{\partial^2\psi}{\partial x_1\partial x_2}
\,d\eta_1\,d\eta_2\,d\eta_3
\\
&\qquad\qquad{}+\int_{\mathbb R^3}
\exp\left(\frac{im\eta_1^2}{2\hbar\epsilon}\right)
\exp\left(\frac{im\eta_2^2}{2\hbar\epsilon}\right)
\exp\left(\frac{im\eta_3^2}{2\hbar\epsilon}\right)
2\eta_1\eta_3\frac{\partial^2\psi}{\partial x_1\partial x_3}
\,d\eta_1\,d\eta_2\,d\eta_3
\\
&\qquad\qquad{}+\int_{\mathbb R^3}
\exp\left(\frac{im\eta_1^2}{2\hbar\epsilon}\right)
\exp\left(\frac{im\eta_2^2}{2\hbar\epsilon}\right)
\exp\left(\frac{im\eta_3^2}{2\hbar\epsilon}\right)
2\eta_2\eta_3\frac{\partial^2\psi}{\partial x_2\partial x_3}
\,d\eta_1\,d\eta_2\,d\eta_3
\end{align*}

By equations (13) and (15)
\begin{align*}
&\int_{\mathbb R^3}
\exp\left(\frac{im\eta_1^2}{2\hbar\epsilon}\right)
\exp\left(\frac{im\eta_2^2}{2\hbar\epsilon}\right)
\exp\left(\frac{im\eta_3^2}{2\hbar\epsilon}\right)
\eta_1^2\frac{\partial^2\psi}{\partial x_1^2}
\,d\eta_1\,d\eta_2\,d\eta_3
\\
&\qquad{}=\frac{i\hbar\epsilon}{m}\left(\frac{2\pi i\hbar\epsilon}{m}\right)^{3/2}
\frac{\partial^2\psi}{\partial x_1^2}
\\
&\int_{\mathbb R^3}
\exp\left(\frac{im\eta_1^2}{2\hbar\epsilon}\right)
\exp\left(\frac{im\eta_2^2}{2\hbar\epsilon}\right)
\exp\left(\frac{im\eta_3^2}{2\hbar\epsilon}\right)
\eta_2^2\frac{\partial^2\psi}{\partial x_2^2}
\,d\eta_1\,d\eta_2\,d\eta_3
\\
&\qquad{}=\frac{i\hbar\epsilon}{m}\left(\frac{2\pi i\hbar\epsilon}{m}\right)^{3/2}
\frac{\partial^2\psi}{\partial x_2^2}
\\
&\int_{\mathbb R^3}
\exp\left(\frac{im\eta_1^2}{2\hbar\epsilon}\right)
\exp\left(\frac{im\eta_2^2}{2\hbar\epsilon}\right)
\exp\left(\frac{im\eta_3^2}{2\hbar\epsilon}\right)
\eta_3^2\frac{\partial^2\psi}{\partial x_3^2}
\,d\eta_1\,d\eta_2\,d\eta_3
\\
&\qquad{}=\frac{i\hbar\epsilon}{m}\left(\frac{2\pi i\hbar\epsilon}{m}\right)^{3/2}
\frac{\partial^2\psi}{\partial x_3^2}
\end{align*}

By equation (14)
\begin{align*}
\int_{\mathbb R^3}
\exp\left(\frac{im\eta_1^2}{2\hbar\epsilon}\right)
\exp\left(\frac{im\eta_2^2}{2\hbar\epsilon}\right)
\exp\left(\frac{im\eta_3^2}{2\hbar\epsilon}\right)
2\eta_1\eta_2\frac{\partial^2\psi}{\partial x_1\partial x_2}
\,d\eta_1\,d\eta_2\,d\eta_3
&=0
\\
\int_{\mathbb R^3}
\exp\left(\frac{im\eta_1^2}{2\hbar\epsilon}\right)
\exp\left(\frac{im\eta_2^2}{2\hbar\epsilon}\right)
\exp\left(\frac{im\eta_3^2}{2\hbar\epsilon}\right)
2\eta_1\eta_3\frac{\partial^2\psi}{\partial x_1\partial x_3}
\,d\eta_1\,d\eta_2\,d\eta_3
&=0
\\
\int_{\mathbb R^3}
\exp\left(\frac{im\eta_1^2}{2\hbar\epsilon}\right)
\exp\left(\frac{im\eta_2^2}{2\hbar\epsilon}\right)
\exp\left(\frac{im\eta_3^2}{2\hbar\epsilon}\right)
2\eta_2\eta_3\frac{\partial^2\psi}{\partial x_2\partial x_3}
\,d\eta_1\,d\eta_2\,d\eta_3
&=0
\end{align*}

Hence
\begin{multline*}
\int_{\mathbb R^3}
\exp\left(\frac{im}{2\hbar\epsilon}\boldsymbol\eta^2\right)
\boldsymbol\eta\cdot\nabla(\boldsymbol\eta\cdot\nabla\psi)
\,d\eta_1\,d\eta_2\,d\eta_3
\\
{}=\frac{i\hbar\epsilon}{m}\left(\frac{2\pi i\hbar\epsilon}{m}\right)^{3/2}
\left(
\frac{\partial^2}{\partial x_1^2}+
\frac{\partial^2}{\partial x_2^2}+
\frac{\partial^2}{\partial x_3^2}
\right)\psi(\mathbf x,t)
\tag{18}
\end{multline*}

Substitute the solved integrals into (6) to obtain
\begin{align*}
&\psi(\mathbf{x},t)+\epsilon\frac{\partial\psi}{\partial t}
\\
&\qquad{}=\frac{1}{A}\left(\frac{2\pi i\hbar\epsilon}{m}\right)^{3/2}\psi(\mathbf x,t)
\tag*{\text{from (7) and (16)}}
\\
&\qquad\qquad{}+\frac{i\hbar\epsilon}{2Am}\left(\frac{2\pi i\hbar\epsilon}{m}\right)^{3/2}\nabla^2\psi(\mathbf x,t)
\tag*{\text{from (9) and (18)}}
\\
&\qquad\qquad{}-\frac{i\epsilon}{A\hbar}\left(\frac{2\pi i\hbar\epsilon}{m}\right)^{3/2}V(\mathbf x,t)\psi(\mathbf x,t)
\tag*{\text{from (10) and (16)}}
\\
&\qquad\qquad{}+\frac{\epsilon^2}{2Am}\left(\frac{2\pi i\hbar\epsilon}{m}\right)^{3/2}
V(\mathbf x,t)\nabla^2\psi(\mathbf x,t)
\tag*{\text{from (12) and (18)}}
\end{align*}

In the limit as $\epsilon\rightarrow0$ we have
\begin{equation*}
\psi(\mathbf{x},t)=\frac{1}{A}\left(\frac{2\pi i\hbar\epsilon}{m}\right)^{3/2}\psi(\mathbf x,t)
\end{equation*}

Hence
\begin{equation*}
A=\left(\frac{2\pi i\hbar\epsilon}{m}\right)^{3/2}
\end{equation*}

Cancel $A$ and drop the order $\epsilon^2$ term.
\begin{equation*}
\psi(\mathbf{x},t)+\epsilon\frac{\partial\psi}{\partial t}
=\psi(\mathbf{x},t)
+\frac{i\hbar\epsilon}{2m}\nabla^2\psi(\mathbf{x},t)
-\frac{i\epsilon}{\hbar}V(\mathbf x,t)\psi(\mathbf{x},t)
\end{equation*}

The $\psi(\mathbf x,t)$ terms cancel.
\begin{equation*}
\epsilon\frac{\partial\psi}{\partial t}
=\frac{i\hbar\epsilon}{2m}\nabla^2\psi(\mathbf{x},t)
-\frac{i\epsilon}{\hbar}V(\mathbf x,t)\psi(\mathbf{x},t)
\end{equation*}

Divide through by $\epsilon$.
\begin{equation*}
\frac{\partial\psi}{\partial t}
=\frac{i\hbar}{2m}\nabla^2\psi(\mathbf{x},t)
-\frac{i}{\hbar}V(\mathbf x,t)\psi(\mathbf{x},t)
\end{equation*}

\end{document}







From the terms involving $\psi(\mathbf x,t)$ we have
\begin{multline*}
\psi(\mathbf{x},t)=
\frac{\psi(\mathbf x,t)}{A}
\int\limits_{-\infty}^\infty
\int\limits_{-\infty}^\infty
\int\limits_{-\infty}^\infty
\exp\left(\frac{im}{2\hbar\epsilon}\boldsymbol\eta^2\right)
\,d\eta_1\,d\eta_2\,d\eta_3
\\
{}-\frac{i\epsilon\psi(\mathbf x,t)}{A\hbar}
\int\limits_{-\infty}^\infty
\int\limits_{-\infty}^\infty
\int\limits_{-\infty}^\infty
\exp\left(\frac{im}{2\hbar\epsilon}\boldsymbol\eta^2\right)
V\left(\mathbf x,t\right)
\,d\eta_1\,d\eta_2\,d\eta_3
\end{multline*}

Solve the first integral.
\begin{multline*}
\psi(\mathbf{x},t)=\frac{\psi(\mathbf x,t)}{A}\left(\frac{2\pi i\hbar\epsilon}{m}\right)^{3/2}
\\
{}-\frac{i\epsilon\psi(\mathbf x,t)}{A\hbar}
\int\limits_{-\infty}^\infty
\int\limits_{-\infty}^\infty
\int\limits_{-\infty}^\infty
\exp\left(\frac{im}{2\hbar\epsilon}\boldsymbol\eta^2\right)
V\left(\mathbf x,t\right)
\,d\eta_1\,d\eta_2\,d\eta_3
\end{multline*}

In the limit $\epsilon\rightarrow0$ the rightmost term drops out leaving
\begin{equation*}
\psi(\mathbf{x},t)=\frac{\psi(\mathbf x,t)}{A}\left(\frac{2\pi i\hbar\epsilon}{m}\right)^{3/2}
\end{equation*}

Hence
\begin{equation*}
A=\left(\frac{2\pi i\hbar\epsilon}{m}\right)^{3/2}
\end{equation*}

It can be shown that
\begin{equation*}
\frac{1}{A}\int\limits_{-\infty}^\infty
\int\limits_{-\infty}^\infty
\int\limits_{-\infty}^\infty
\exp\left(\frac{im}{2\hbar\epsilon}\boldsymbol\eta^2\right)\boldsymbol\eta
\,d\eta_1\,d\eta_2\,d\eta_3
=(0,0,0)
\end{equation*}
and
\begin{equation*}
\frac{1}{A}\int\limits_{-\infty}^\infty
\int\limits_{-\infty}^\infty
\int\limits_{-\infty}^\infty
\exp\left(\frac{im}{2\hbar\epsilon}\boldsymbol\eta^2\right)\boldsymbol\eta^2
\,d\eta_1\,d\eta_2\,d\eta_3
=\left(\frac{i\hbar\epsilon}{m}\right)^3
\end{equation*}

From equation (4.10)
\begin{equation*}
\int_{-\infty}^\infty\exp\left(\frac{imx^2}{2\hbar\epsilon}\right)x^2\,dx
=\left(-\frac{2\pi i\hbar^3\epsilon^3}{m^3}\right)^{1/2}
\end{equation*}


\end{document}
