\documentclass[12pt]{article}
\usepackage{amsmath}
\parindent=0pt
\begin{document}

Exercise 5.1.
Verify this claim.

\bigskip
\hrule

\bigskip
The claim is that any $2\times2$ Hermitian matrix $\mathbf L$ can be written as
\begin{equation*}
\mathbf L=a\sigma_x+b\sigma_y+c\sigma_z+dI
\end{equation*}
where $a$, $b$, $c$, and $d$ are real numbers.

\bigskip
Let $\mathbf L$ be the Hermitian matrix
\begin{equation*}
\mathbf L=\begin{pmatrix}L_{11}&L_{12}\\L_{21}&L_{22}\end{pmatrix}
\tag{1}
\end{equation*}

Recall from page 137 that
\begin{equation*}
\sigma_x=\begin{pmatrix}0&1\\1&0\end{pmatrix},
\quad
\sigma_y=\begin{pmatrix}0&-i\\i&0\end{pmatrix},
\quad
\sigma_z=\begin{pmatrix}1&0\\0&-1\end{pmatrix},
\quad
I=\begin{pmatrix}1&0\\0&1\end{pmatrix}
\end{equation*}

It follows that
\begin{equation*}
\mathbf L=\begin{pmatrix}d+c&a-ib\\a+ib&d-c\end{pmatrix}
\tag{2}
\end{equation*}

Then by equivalence of (1) and (2) we have
\begin{equation*}
a=\frac{L_{12}+L_{21}}{2},
\quad
b=\frac{i(L_{12}-L_{21})}{2},
\quad
c=\frac{L_{11}-L_{22}}{2},
\quad
d=\frac{L_{11}+L_{22}}{2}
\end{equation*}

By Hermiticity we have $\mathbf L=\mathbf L^\dag$ hence
\begin{equation*}
\begin{pmatrix}L_{11}&L_{12}\\L_{21}&L_{22}\end{pmatrix}
=\begin{pmatrix}L_{11}^*&L_{21}^*\\L_{12}^*&L_{22}^*\end{pmatrix}
\end{equation*}

Therefore $L_{11}$ and $L_{22}$ are real, hence $c$ and $d$ are real.

\bigskip
Also by Hermiticity, $L_{21}=L_{12}^*$ hence $a$ and $b$ are real.
\begin{align*}
a&=\frac{L_{12}+L_{21}}{2}=\frac{L_{12}+L_{12}^*}{2}=\text{Re}(L_{12})
\\
b&=\frac{i(L_{12}-L_{21})}{2}=\frac{i(L_{12}-L_{12}^*)}{2}=-\text{Im}(L_{12})
\end{align*}

\end{document}
