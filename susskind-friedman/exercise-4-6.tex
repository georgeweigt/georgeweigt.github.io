\documentclass[12pt]{article}
\usepackage{amsmath}
\parindent=0pt
\begin{document}

Exercise 4.6.
Carry out the Schrodinger Ket recipe for a
single spin.
The Hamiltonian is
$\mathbf H=\frac{\omega\hbar}{2}\sigma_z$ and the final
observable is $\sigma_x$.
The initial state is given as $|u\rangle$
(the state in which $\sigma_z=+1$).

\bigskip
After time $t$, an experiment is done to measure $\sigma_y$.
What are the possible outcomes and what are the probabilities
for those outcomes?

\bigskip
\hrule

\bigskip
Note: Use $|r\rangle$ for the initial state instead of $|u\rangle$.
Otherwise, the result is time-independent.

\bigskip
Step 1 of the Schrodinger Ket recipe is obtain $\mathbf H$,
which we already have by hypothesis.
\begin{equation*}
\mathbf H=\frac{\hbar\omega}{2}\sigma_z
=\frac{\hbar\omega}{2}\begin{pmatrix}1&0\\0&-1\end{pmatrix}
\end{equation*}

Step 2.
Prepare an initial state $|\Psi(0)\rangle$.
By hypothesis the initial state is $|r\rangle$.
\begin{equation*}
|r\rangle=\frac{1}{\sqrt2}|u\rangle+\frac{1}{\sqrt2}|d\rangle
\tag{2.5}
\end{equation*}

Hence
\begin{equation*}
|\Psi(0)\rangle=|r\rangle=\begin{pmatrix}\frac{1}{\sqrt2}\\[1ex]\frac{1}{\sqrt2}\end{pmatrix}
\end{equation*}

Step 3.
Find the eigenvalues and eigenvectors of $\mathbf H$ by solving
the time-independent Schrodinger equation,
\begin{equation*}
\mathbf H|E_j\rangle=E_j|E_j\rangle
\end{equation*}

The eigenvalues are obtained by solving the characteristic equation
$\det(\mathbf H-E_j\mathbf I)=0$.
\begin{align*}
\det(\mathbf H-E_j\mathbf I)
&=\left|
\begin{matrix}\frac{\hbar\omega}{2}-E_j&0\\0&-\frac{\hbar\omega}{2}-E_j\end{matrix}
\right|
\\[1ex]
&=\left(\frac{\hbar\omega}{2}-E_j\right)\left(-\frac{\hbar\omega}{2}-E_j\right)
\\[1ex]
&=E_j^2-\left(\frac{\hbar\omega}{2}\right)^2=0
\end{align*}

Hence the eigenvalues are
\begin{equation*}
E_1=\frac{\hbar\omega}{2},
\quad
E_2=-\frac{\hbar\omega}{2}
\end{equation*}

The eigenvectors of $\mathbf H$ are $|u\rangle$ and $|d\rangle$.
\begin{equation*}
|E_1\rangle=|u\rangle=\begin{pmatrix}1\\0\end{pmatrix},
\quad
|E_2\rangle=|d\rangle=\begin{pmatrix}0\\1\end{pmatrix}
\end{equation*}

Step 4.
Use the initial state vector $|\Psi(0)\rangle$,
along with the eigenvectors $|E_j\rangle$ from step 3,
to calculate the initial coefficients $\alpha_j(0)$:
\begin{equation*}
\alpha_j(0)=\langle E_j|\Psi(0)\rangle
\end{equation*}

We have
\begin{align*}
\alpha_1(0)&=\langle E_1|r\rangle=\frac{1}{\sqrt2}
\\[1ex]
\alpha_2(0)&=\langle E_2|r\rangle=\frac{1}{\sqrt2}
\end{align*}

Step 5.
Rewrite $|\Psi(0)\rangle$ in terms of the eigenvectors $|E_j\rangle$ and
the initial coefficients $\alpha_j(0)$:
\begin{equation*}
|\Psi(0)\rangle=\sum_j\alpha_j(0)|E_j\rangle
\end{equation*}

We have
\begin{equation*}
|\Psi(0)\rangle
=\frac{1}{\sqrt2}\begin{pmatrix}1\\0\end{pmatrix}
+\frac{1}{\sqrt2}\begin{pmatrix}0\\1\end{pmatrix}
\end{equation*}

Step 6.
In the above equation, replace each $\alpha_j(0)$ with $\alpha_j(t)$
to capture its time-independence.
As a result, $|\Psi(0)\rangle$ becomes $|\Psi(t)\rangle$:
\begin{equation*}
|\Psi(t)\rangle=\sum_j\alpha_j(t)|E_j\rangle
\end{equation*}

We have
\begin{equation*}
|\Psi(t)\rangle
=\alpha_1(t)\begin{pmatrix}1\\0\end{pmatrix}
+\alpha_2(t)\begin{pmatrix}0\\1\end{pmatrix}
\end{equation*}

Step 7.
Using Eq.~4.30, replace each $\alpha_j(t)$ with
$\alpha_j(0)\exp(-iE_jt/\hbar)$:
\begin{equation*}
|\Psi(t)\rangle=\sum_j\alpha_j(0)
\exp\left(-\frac{i}{\hbar}E_jt\right)
|E_j\rangle
\tag{4.34}
\end{equation*}

We have
\begin{equation*}
|\Psi(t)\rangle
=\frac{1}{\sqrt2}\exp\left(-\frac{i}{\hbar}E_1t\right)\begin{pmatrix}1\\0\end{pmatrix}
+\frac{1}{\sqrt2}\exp\left(-\frac{i}{\hbar}E_2t\right)\begin{pmatrix}0\\1\end{pmatrix}
\end{equation*}

Hence
\begin{equation*}
|\Psi(t)\rangle
=\frac{1}{\sqrt2}\exp\left(-\frac{i\omega t}{2}\right)\begin{pmatrix}1\\0\end{pmatrix}
+\frac{1}{\sqrt2}\exp\left(\frac{i\omega t}{2}\right)\begin{pmatrix}0\\1\end{pmatrix}
\end{equation*}

This concludes the Schrodinger Ket recipe.

\bigskip
Recall that
\begin{equation*}
|i\rangle=\begin{pmatrix}\frac{1}{\sqrt2}\\[1ex]\frac{i}{\sqrt2}\end{pmatrix},
\qquad
|o\rangle=\begin{pmatrix}\frac{1}{\sqrt2}\\[1ex]-\frac{i}{\sqrt2}\end{pmatrix}
\end{equation*}

The probability of measuring $\sigma_y=1$ is
\begin{equation*}
P(1\mid\sigma_y)=\langle i|\Psi(t)\rangle\langle\Psi(t)|i\rangle
=\frac{1+\sin\omega t}{2}
\end{equation*}

The probability of measuring $\sigma_y=-1$ is
\begin{equation*}
P(-1\mid\sigma_y)=\langle o|\Psi(t)\rangle\langle\Psi(t)|o\rangle
=\frac{1-\sin\omega t}{2}
\end{equation*}

\end{document}
