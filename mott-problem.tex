\documentclass[12pt]{article}
\usepackage[margin=1in]{geometry}
\usepackage{amsmath}
\parindent=0pt
\begin{document}

\section*{Mott problem}

Consider the emission of an $\alpha$ particle in a cloud chamber.
The quantum mechanical model for the $\alpha$ particle is a spherical wave emanating from the origin.
A spherical wave should ionize atoms throughout the cloud chamber.
However, only straight tracks are observed.
Nevill Mott showed that straight tracks are consistent with the Schrodinger equation.

\bigskip

Let $\mathbf R$ be the position of the $\alpha$ particle,
let $\mathbf a_1$ and $\mathbf a_2$ be the positions of two atoms ionized by the $\alpha$ particle,
and let $\mathbf r_1$ and $\mathbf r_2$ be the positions of the free electrons.
The Hamiltonian for the system is
\begin{equation*}
\hat H=\hat K_\alpha+\hat K_1+\hat K_2+U_1+U_2+V_1+V_2
\end{equation*}
where
\begin{align*}
\hat K_\alpha&=-\frac{\hbar^2}{2M}\nabla_\alpha^2 & & \text{kinetic energy of $\alpha$ particle}
\\[1ex]
\hat K_1&=-\frac{\hbar^2}{2m}\nabla_1^2 & & \text{kinetic energy of 1st electron}
\\[1ex]
\hat K_2&=-\frac{\hbar^2}{2m}\nabla_2^2 & & \text{kinetic energy of 2nd electron}
\\[1ex]
U_1&=-\frac{e^2}{|\mathbf r_1-\mathbf a_1|} & & \text{potential energy of 1st electron}
\\[1ex]
U_2&=-\frac{e^2}{|\mathbf r_2-\mathbf a_2|} & & \text{potential energy of 2nd electron}
\\[1ex]
V_1&=-\frac{2e^2}{|\mathbf R-\mathbf r_1|} & & \text{potential energy of $\alpha$ and 1st electron}
\\[1ex]
V_2&=-\frac{2e^2}{|\mathbf R-\mathbf r_2|} & & \text{potential energy of $\alpha$ and 2nd electron}
\end{align*}

Let $\psi_1$ and $\psi_2$ be atomic wavefunctions such that
\begin{equation*}
\hat H\psi_1=E_1\psi_1,
\quad
\hat H\psi_2=E_2\psi_2
\end{equation*}

We want to find a wavefunction $F(\mathbf R,\mathbf r_1,\mathbf r_2)$ such that
\begin{equation*}
\hat HF=EF
\end{equation*}

Let
\begin{equation*}
F=F_0+F_1+F_2+\cdots
\end{equation*}
and let
\begin{equation*}
\hat H_0=\hat K_\alpha+E_1+E_2
\end{equation*}

Start by finding an $F_0$ such that
\begin{equation*}
\hat H_0F_0=EF_0
\end{equation*}

The solution is
\begin{equation*}
F_0(\mathbf R,\mathbf r_1,\mathbf r_2)=f_0(\mathbf R)\psi_1(\mathbf r_1-\mathbf a_1)\psi_2(\mathbf r_2-\mathbf a_2)
\tag{1}
\end{equation*}
where
\begin{equation*}
f_0(\mathbf R)=\frac{1}{|\mathbf R|}\exp\left(\frac{ik|\mathbf R|}{\hbar}\right),\quad k=\sqrt{2M(E-E_1-E_2)}
\end{equation*}

It follows that for the full Hamiltonian
\begin{equation*}
\hat HF_0=EF_0+(V_1+V_2)F_0
\end{equation*}

To cancel $(V_1+V_2)F_0$ from the full Hamiltonian, find an $F_1$ such that
\begin{equation*}
\hat H_0F_1=EF_1-(V_1+V_2)F_0
\end{equation*}

Rewrite as
\begin{equation*}
\left(\hat H_0-E\right)F_1=-(V_1+V_2)F_0
\end{equation*}

Expand $F_1$ and $F_0$.
\begin{equation*}
\left(\hat H_0-E\right)f_1(\mathbf R)\psi_1(\mathbf r_1-\mathbf a_1)\psi_2(\mathbf r_2-\mathbf a_2)
=-(V_1+V_2)f_0(\mathbf R)\psi_1(\mathbf r_1-\mathbf a_1)\psi_2(\mathbf r_2-\mathbf a_2)
\end{equation*}

To solve for $f_1(\mathbf R)$ multiply both sides by
\begin{equation*}
\psi_1^*(\mathbf r_1-\mathbf a_1)\psi_2^*(\mathbf r_2-\mathbf a_2)
\end{equation*}
and integrate over $\mathbf r_1$ and $\mathbf r_2$ to obtain
\begin{equation*}
\left(\hat H_0-E\right)f_1(\mathbf R)
=V_1(\mathbf R)f_0(\mathbf R)+V_2(\mathbf R)f_0(\mathbf R)
\tag{2}
\end{equation*}
where
\begin{equation*}
V_1(\mathbf R)=
2e^2\int\frac{|\psi_1(\mathbf r)|^2}{|\mathbf R-\mathbf a_1-\mathbf r|}\,d\mathbf r,
\quad
V_2(\mathbf R)=
2e^2\int\frac{|\psi_2(\mathbf r)|^2}{|\mathbf R-\mathbf a_2-\mathbf r|}\,d\mathbf r
\end{equation*}

Per Mott the solution to (2) is
\begin{multline*}
f_1(\mathbf R)=
\frac{M}{2\pi\hbar^2}\int\frac{V_1(\mathbf r)f_0(\mathbf r)}{|\mathbf R-\mathbf r|}
\exp\left(\frac{ik|\mathbf R-\mathbf r|}{\hbar}\right)\,d\mathbf r
\\
{}+\frac{M}{2\pi\hbar^2}\int\frac{V_2(\mathbf r)f_0(\mathbf r)}{|\mathbf R-\mathbf r|}
\exp\left(\frac{ik|\mathbf R-\mathbf r|}{\hbar}\right)\,d\mathbf r
\end{multline*}

Let $I_1$ be the first integral.
\begin{equation*}
I_1(\mathbf R)=
\frac{M}{2\pi\hbar^2}\int\frac{V_1(\mathbf r)f_0(\mathbf r)}{|\mathbf R-\mathbf r|}
\exp\left(\frac{ik|\mathbf R-\mathbf r|}{\hbar}\right)\,d\mathbf r
\end{equation*}

Substitute for $f_0(\mathbf r)$ to obtain
\begin{equation*}
I_1(\mathbf R)=
\frac{M}{2\pi\hbar^2}\int\frac{V_1(\mathbf r)}{|\mathbf R-\mathbf r||\mathbf r|}
\exp\left(\frac{ik|\mathbf R-\mathbf r|}{\hbar}+\frac{ik|\mathbf r|}{\hbar}\right)\,d\mathbf r
\end{equation*}

Change of variable $\mathbf r\rightarrow\mathbf r+\mathbf a_1$
\begin{equation*}
I_1(\mathbf R)=
\frac{M}{2\pi\hbar^2}
\int
\frac{V_1(\mathbf r+\mathbf a_1)}{|\mathbf R-\mathbf r-\mathbf a_1||\mathbf r+\mathbf a_1|}
\exp\left(
\frac{ik|\mathbf R-\mathbf r-\mathbf a_1|}{\hbar}+\frac{ik|\mathbf r+\mathbf a_1|}{\hbar}
\right)\,d\mathbf r
\end{equation*}

Per Mott (see also Figari and Teta)
\begin{multline*}
I_1(\mathbf R)\approx\frac{1}{|\mathbf R-\mathbf a_1|}
\exp\left(\frac{ik|\mathbf R-\mathbf a_1|}{\hbar}\right)
\\
{}\times
\frac{M}{2\pi\hbar^2}
\int
\frac{V_1(\mathbf r+\mathbf a_1)}{|\mathbf r+\mathbf a_1|}
\exp\left(-\frac{ik\mathbf u_1(\mathbf R)\cdot\mathbf r}{\hbar}
+\frac{ik|\mathbf r+\mathbf a_1|}{\hbar}\right)
\,d\mathbf r
\end{multline*}
where
\begin{equation*}
\mathbf u_1(\mathbf R)=\frac{\mathbf R-\mathbf a_1}{|\mathbf R-\mathbf a_1|}
\end{equation*}

The condition for stationary phase is
\begin{equation*}
g'=\frac{d}{d\mathbf r}\left(-\mathbf u_1(\mathbf R)\cdot\mathbf r+|\mathbf r+\mathbf a_1|\right)
=-\mathbf u_1(\mathbf R)+\frac{\mathbf r+\mathbf a_1}{|\mathbf r+\mathbf a_1|}=0
\end{equation*}

Note that $V_1(\mathbf r+\mathbf a_1)$ is small except for $\mathbf r\approx0$ so
we only require stationarity at the origin.
Hence for $\mathbf r=0$ the integral is stationary ($g'=0$)
when $\mathbf R$ satisfies the condition
\begin{equation*}
\mathbf u_1(\mathbf R)
=\frac{\mathbf R-\mathbf a_1}{|\mathbf R-\mathbf a_1|}
=\frac{\mathbf a_1}{|\mathbf a_1|}
\end{equation*}

By symmetry of the integrals, $I_2$ is stationary when $\mathbf R$ satisfies the condition
\begin{equation*}
\mathbf u_2(\mathbf R)
=\frac{\mathbf R-\mathbf a_2}{|\mathbf R-\mathbf a_2|}
=\frac{\mathbf a_2}{|\mathbf a_2|}
\end{equation*}

Gentle reminder that
\begin{equation*}
f_1(\mathbf R)=I_1(\mathbf R)+I_2(\mathbf R)
\end{equation*}

Since both atoms are ionized, it follows that both $V_1$ and $V_2$ should contribute to $f_1$.
Hence both integrals should be stationary for the same values of $\mathbf R$.
Simultaneous stationarity implies $\mathbf u_1(\mathbf R)=\mathbf u_2(\mathbf R)$ hence
\begin{equation*}
\frac{\mathbf a_1}{|\mathbf a_1|}=\frac{\mathbf a_2}{|\mathbf a_2|}
\end{equation*}

This condition means that both $\mathbf a_1$ and $\mathbf a_2$
are on the same ray emanating from the origin.
If we keep $\mathbf a_1$ and choose a different $\mathbf a_2$ we get the same result.
It follows that all of the ionized atoms are on the same ray.
Hence straight tracks are consistent with the Schrodinger equation.

\end{document}

By the method of stationary phase
\begin{align*}
I_1(\mathbf R)&=\frac{1}{|\mathbf R-\mathbf a_1|}
\exp\left(\frac{ik|\mathbf R-\mathbf a_1|}{\hbar}\right)
\frac{M}{2\pi\hbar^2}
\left(\frac{2\pi i\hbar|\mathbf a_1|}{k}\right)^{\frac{1}{2}}
\frac{V_1(\mathbf a_1)}{|\mathbf a_1|}
\exp\left(\frac{ik|\mathbf a_1|}{\hbar}\right)
%\frac{\mathbf R}{|\mathbf R|}=\frac{\mathbf a_1}{|\mathbf a_1|}
\\
I_2(\mathbf R)&=\frac{1}{|\mathbf R-\mathbf a_2|}
\exp\left(\frac{ik|\mathbf R-\mathbf a_2|}{\hbar}\right)
\frac{M}{2\pi\hbar^2}
\left(\frac{2\pi i\hbar|\mathbf a_2|}{k}\right)^{\frac{1}{2}}
\frac{V_2(\mathbf a_2)}{|\mathbf a_2|}
\exp\left(\frac{ik|\mathbf a_2|}{\hbar}\right)
%\frac{\mathbf R}{|\mathbf R|}=\frac{\mathbf a_2}{|\mathbf a_2|}
\end{align*}

\subsection*{Note}

To integrate using the method of stationary phase, let
\begin{align*}
f&=\frac{V_1(\mathbf r+\mathbf a_1)}{|\mathbf r+\mathbf a_1|}
\\
g&=-k\mathbf u(\mathbf R)\cdot\mathbf r+k|\mathbf r+\mathbf a_1|
\\
h&=\frac{1}{\hbar}
\end{align*}

The integral is
\begin{equation*}
I\approx\pm\left(\frac{2\pi i}{\lambda g''}\right)^{1/2}f\exp(i\lambda g)
\end{equation*}
evaluated at $\mathbf r=\mathbf r_c$ where $g'(\mathbf r_c)=0$.
For the $g$ given above, $\mathbf r_c=0$ and
\begin{equation*}
g''=\frac{k\mathbf a_1}{(\mathbf r_c+\mathbf a_1)|\mathbf r_c+\mathbf a_1|}
=\frac{k}{|\mathbf a_1|}
\end{equation*}

Hence
\begin{equation*}
I\approx\pm\left(\frac{2\pi i|\mathbf a_1|}{k}\right)^{1/2}
\frac{V_1(\mathbf a_1)}{|\mathbf a_1|}
\exp(\left(\frac{ik|\mathbf a_1|}{\hbar}\right)
\end{equation*}

\end{document}
