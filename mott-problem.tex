\documentclass[12pt]{article}
\usepackage[margin=1in]{geometry}
\usepackage{amsmath}
\parindent=0pt
\begin{document}

\section*{Mott problem}

Consider the emission of an $\alpha$ particle in a cloud chamber.
The quantum mechanical model of the particle is a spherical wave emanating from the origin.
A spherical wave should ionize atoms throughout the cloud chamber.
However, only straight tracks are observed.
Nevill Mott used the Schrodinger equation to explain why straight tracks are observed.

\bigskip

Let $\mathbf R$ be the position of the $\alpha$ particle,
let $\mathbf a_1$ and $\mathbf a_2$ be the positions of two atoms ionized by the $\alpha$ particle,
and let $\mathbf r_1$ and $\mathbf r_2$ be the positions of the free electrons.
The Hamiltonian for the system is
\begin{equation*}
\hat H=\hat K_\alpha+\hat K_1+\hat K_2+V_1+V_2+U_1+U_2
\end{equation*}
where
\begin{align*}
\hat K_\alpha&=-\frac{\hbar^2}{2M}\nabla_\alpha^2 & & \text{kinetic energy of $\alpha$ particle}
\\[1ex]
\hat K_1&=-\frac{\hbar^2}{2m}\nabla_1^2 & & \text{kinetic energy of 1st electron}
\\[1ex]
\hat K_2&=-\frac{\hbar^2}{2m}\nabla_2^2 & & \text{kinetic energy of 2nd electron}
\\[1ex]
V_1&=-\frac{e^2}{|\mathbf r_1-\mathbf a_1|} & & \text{potential energy of 1st electron}
\\[1ex]
V_2&=-\frac{e^2}{|\mathbf r_2-\mathbf a_2|} & & \text{potential energy of 2nd electron}
\\[1ex]
U_1&=-\frac{2e^2}{|\mathbf R-\mathbf r_1|} & & \text{potential energy of $\alpha$ and 1st electron}
\\[1ex]
U_2&=-\frac{2e^2}{|\mathbf R-\mathbf r_2|} & & \text{potential energy of $\alpha$ and 2nd electron}
\end{align*}

Let $\psi_1$ and $\psi_2$ be atomic wavefunctions such that
\begin{equation*}
\hat H\psi_1=E_1\psi_1,
\quad
\hat H\psi_2=E_2\psi_2
\end{equation*}

We want to find a wavefunction $F(\mathbf R,\mathbf r_1,\mathbf r_2)$ such that
\begin{equation*}
\hat HF=EF
\end{equation*}

Let
\begin{equation*}
F=F_0+F_1+F_2+\cdots
\end{equation*}
and let
\begin{equation*}
\hat H_0=\hat K_\alpha+E_1+E_2
\end{equation*}

Start by finding an $F_0$ such that
\begin{equation*}
\hat H_0F_0=EF_0
\end{equation*}

The solution is
\begin{equation*}
F_0(\mathbf R,\mathbf r_1,\mathbf r_2)=f_0(\mathbf R)\psi_1(\mathbf r_1-\mathbf a_1)\psi_2(\mathbf r_2-\mathbf a_2)
\tag{1}
\end{equation*}
where
\begin{equation*}
f_0(\mathbf R)=\frac{1}{|\mathbf R|}\exp\left(\frac{ik|\mathbf R|}{\hbar}\right),\quad k=\sqrt{2M(E-E_1-E_2)}
\end{equation*}

It follows that for the full Hamiltonian
\begin{equation*}
\hat HF_0=EF_0+(U_1+U_2)F_0
\end{equation*}

To cancel $(U_1+U_2)F_0$ from the full Hamiltonian, find an $F_1$ such that
\begin{equation*}
\hat H_0F_1=EF_1-(U_1+U_2)F_0
\end{equation*}

Rewrite as
\begin{equation*}
\left(\hat H_0-E\right)F_1=-(U_1+U_2)F_0
\end{equation*}

Expand $F_1$ and $F_0$.
\begin{equation*}
\left(\hat H_0-E\right)f_1(\mathbf R)\psi_1(\mathbf r_1-\mathbf a_1)\psi_2(\mathbf r_2-\mathbf a_2)
=-(U_1+U_2)f_0(\mathbf R)\psi_1(\mathbf r_1-\mathbf a_1)\psi_2(\mathbf r_2-\mathbf a_2)
\end{equation*}

To solve for $f_1(\mathbf R)$ multiply both sides by
\begin{equation*}
\psi_1^*(\mathbf r_1-\mathbf a_1)\psi_2^*(\mathbf r_2-\mathbf a_2)
\end{equation*}
and integrate over $\mathbf r_1$ and $\mathbf r_2$ to obtain
\begin{equation*}
\left(\hat H_0-E\right)f_1(\mathbf R)
=V(\mathbf R)f_0(\mathbf R)
\tag{2}
\end{equation*}
where
\begin{multline*}
V(\mathbf R)
=2e^2\int\frac{|\psi_1(\mathbf r_1-\mathbf a_1)|^2\,|\psi_2(\mathbf r_2-\mathbf a_2)|^2}{|\mathbf R-\mathbf r_1|}
\,d\mathbf r_1\,d\mathbf r_2
\\
{}+2e^2\int\frac{|\psi_1(\mathbf r_1-\mathbf a_1)|^2\,|\psi_2(\mathbf r_2-\mathbf a_2)|^2}{|\mathbf R-\mathbf r_2|}
\,d\mathbf r_1\,d\mathbf r_2
\end{multline*}

Because $|\psi|^2$ is a normalized probability density function we have
\begin{equation*}
V(\mathbf R)
=2e^2\int\frac{|\psi_1(\mathbf r_1-\mathbf a_1)|^2}{|\mathbf R-\mathbf r_1|}
\,d\mathbf r_1
+2e^2\int\frac{|\psi_2(\mathbf r_2-\mathbf a_2)|^2}{|\mathbf R-\mathbf r_2|}
\,d\mathbf r_2
\end{equation*}

Per Mott the solution to (2) is
\begin{equation*}
f_1(\mathbf R)=
\frac{M}{2\pi\hbar^2}\int\frac{V(\mathbf r)f_0(\mathbf r)}{|\mathbf R-\mathbf r|}
\exp\left(\frac{ik|\mathbf R-\mathbf r|}{\hbar}\right)\,d\mathbf r,
\quad
k=\sqrt{2M(E-E_1-E_2)}
\end{equation*}

Substitute for $f_0(\mathbf r)$ to obtain
\begin{equation*}
f_1(\mathbf R)=
\frac{M}{2\pi\hbar^2}\int\frac{V(\mathbf r)}{|\mathbf R-\mathbf r||\mathbf r|}
\exp\left(\frac{ik|\mathbf R-\mathbf r|}{\hbar}+\frac{ik|\mathbf r|}{\hbar}\right)\,d\mathbf r
\end{equation*}

Change of variable $\mathbf r\rightarrow\mathbf y+\mathbf a_1$.
\begin{equation*}
f_1(\mathbf R)=
\frac{M}{2\pi\hbar^2}\int
\frac{V(\mathbf y+\mathbf a_1)}{|\mathbf R-\mathbf y-\mathbf a_1||\mathbf y+\mathbf a_1|}
\exp\left(\frac{ik|\mathbf R-\mathbf y-\mathbf a_1|}{\hbar}
+\frac{ik|\mathbf y+\mathbf a_1|}{\hbar}\right)\,d\mathbf y
\end{equation*}

Per Mott (see also Figari and Teta)
\begin{equation*}
f_1(\mathbf R)\approx
\frac{\exp(ik|\mathbf R-\mathbf a_1|)}{|\mathbf R-\mathbf a_1|}
\frac{M}{2\pi\hbar^2}\int\frac{V(\mathbf y+\mathbf a_1)}{|\mathbf y+\mathbf a_1|}
\exp\left(-\frac{ik\mathbf u\cdot\mathbf y}{\hbar}+\frac{ik|\mathbf y|}{\hbar}\right)\,d\mathbf y
\tag{3}
\end{equation*}
where
\begin{equation*}
\mathbf u=\frac{\mathbf R-\mathbf a_1}{|\mathbf R-\mathbf a_1|}
\end{equation*}

The method of stationary phase requires that
\begin{equation*}
\frac{d}{d\mathbf y}\left(-\mathbf u\cdot\mathbf y+|\mathbf y+\mathbf a_1|\right)
=-\mathbf u+\frac{\mathbf y+\mathbf a_1}{|\mathbf y+\mathbf a_1|}=0
\tag{4}
\end{equation*}

Note that $V(\mathbf y+\mathbf a_1)$ is small
except for $\mathbf y\approx0$ and $\mathbf y\approx\mathbf a_2-\mathbf a_1$ so we only need to consider
$\mathbf y$ near those values.
To satisfy (4) for both $\mathbf y=0$ and $\mathbf y=\mathbf a_2-\mathbf a_1$ we have
\begin{equation*}
\mathbf u=\frac{\mathbf a_1}{|\mathbf a_1|}=\frac{\mathbf a_2}{|\mathbf a_2|}
\end{equation*}

Hence $\mathbf a_1$ and $\mathbf a_2$ form a line through the origin.

\end{document}
