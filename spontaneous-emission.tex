\documentclass[12pt]{article}
\usepackage[margin=2cm]{geometry}
\usepackage{amsmath}

\begin{document}

\noindent
$A_{nm}$ is the spontaneous emission rate for the transition $E_n\rightarrow E_m$.
\begin{equation*}
A_{nm}=\frac{e^2}{3\pi\varepsilon_0\hbar c^3}\,\omega_{nm}^3\,|r_{nm}|^2
\end{equation*}

\noindent
The transition frequency $\omega_{nm}$ is given by Bohr's frequency condition.
\begin{equation*}
\omega_{nm}=\frac{1}{\hbar}(E_n-E_m)
\end{equation*}

\noindent
The transition probability (multiplied by a physical constant) is
\begin{equation*}
|r_{nm}|^2
=|x_{nm}|^2
+|y_{nm}|^2
+|z_{nm}|^2
\end{equation*}
For wave functions $\psi$ in spherical coordinates we have the following transition amplitudes.
\begin{align*}
x_{nm}&=\int\psi_m^*\,(r\sin\theta\cos\phi)\,\psi_n\,dV
\\
y_{nm}&=\int\psi_m^*\,(r\sin\theta\sin\phi)\,\psi_n\,dV
\\
z_{nm}&=\int\psi_m^*\,(r\cos\theta)\,\psi_n\,dV
\end{align*}

\noindent
Let us compute $A_{21}$ for hydrogen.
The energy levels for hydrogen are
\begin{equation*}
E_n=-\frac{e^2}{8\pi\varepsilon_0 a_0n^2}
\end{equation*}
where $a_0$ is the Bohr radius
\begin{equation*}
a_0=\frac{4\pi\varepsilon_0\hbar^2}{e^2 m_e}
=5.29\times10^{-11}\,\text{meter}
\end{equation*}

\noindent
For the transition frequency we have
\begin{equation*}
\omega_{21}=\frac{1}{\hbar}(E_2-E_1)
=1.55\times10^{16}\,\text{second}^{-1}
\end{equation*}

\noindent
To compute the transition probability $|r_{21}|^2$ we need to
consider all four eigenstates for $n=2$.
\begin{center}
\begin{tabular}{rrr}
$n$ & $\ell$ & $m_\ell$\\
2 & 1 & 1 \\
2 & 1 & $-1$ \\
2 & 1 & 0 \\
2 & 0 & 0
\end{tabular}
\end{center}

\noindent
The following table shows the probability for every possible transition of $\psi_2$ to $\psi_1$.
\begin{center}
\begin{tabular}{rcccc}
& $\psi_{2,1,1}\rightarrow\psi_{1,0,0}$
& $\psi_{2,1,-1}\rightarrow\psi_{1,0,0}$
& $\psi_{2,1,0}\rightarrow\psi_{1,0,0}$
& $\psi_{2,0,0}\rightarrow\psi_{1,0,0}$
\\[2ex]
$x_{21}=$ & $-\frac{128}{243}\,a_0$ & $\frac{128}{243}\,a_0$ & 0 & 0
\\[2ex]
$y_{21}=$ & $-\frac{128}{243}i\,a_0$ & $-\frac{128}{243}i\,a_0$ & 0 & 0
\\[2ex]
$z_{21}=$ & 0 & 0 & $\frac{128}{243}\sqrt{2}\,a_0$ & 0
\\[2ex]
$|r_{21}|^2=$ & $\frac{32768}{59049}\,a_0^2$ & $\frac{32768}{59049}\,a_0^2$ & $\frac{32768}{59049}\,a_0^2$ & 0
\end{tabular}
\end{center}

\noindent
The transition $\psi_{2,0,0}\rightarrow\psi_{1,0,0}$ has zero probability.

\bigskip
\noindent
For the remaining transitions, the probability $|r_{21}|^2$ is independent of $m_\ell$.

\bigskip
\noindent
Now that we have $|r_{21}|^2$ we can compute a numerical value for $A_{21}$.
\begin{equation*}
A_{21}=\frac{e^2}{3\pi\varepsilon_0\hbar c^3}
\times
\omega_{21}^3
\times
\tfrac{32768}{59049}\,a_0^2
=6.27\times10^8\,\text{second}^{-1}
\end{equation*}

\noindent
Here is $A_{21}$ as a product of fundamental constants.
\begin{equation*}
A_{21}=\frac{e^2}{3\pi\varepsilon_0\hbar c^3}
\times
\underset{\substack{\\[1ex]\omega_{21}^3}}
{\left(\frac{3e^4 m_e}{128\pi^2\varepsilon_0^2\hbar^3}\right)^3}
\times
\underset{\substack{\\[1ex]|r_{21}|^2}}
{\frac{32768}{59049}
\left(\frac{4\pi\varepsilon_0\hbar^2}{e^2 m_e}\right)^2}
=\frac{e^{10}m_e}{26244\,\pi^5\varepsilon_0^5\hbar^6 c^3}
\end{equation*}

\noindent
The parameters $n=2$ and $m=1$ contribute the following numerical factor to $A_{21}$.
\begin{equation*}
\underset{\substack{\\[1ex]\text{from $(E_2-E_1)^3$}}}
{\left(-\frac{1}{2^2}+\frac{1}{1^2}\right)^3}
\times
\underset{\substack{\\[1ex]\text{from $|r_{21}|^2$}}}
{\frac{32768}{59049}}
=\frac{512}{2187}=\frac{2^9}{3^7}
\end{equation*}

\noindent
Multiplying out numerical factors yields the numerical factor shown above for $A_{21}$.
\begin{equation*}
\frac{1}{3}
\times
\underset{\substack{\\[1ex]\text{from $(E_n-E_m)^3$}}}
{\left(\frac{1}{32}\right)^3}
\times
\underset{\substack{\\[1ex]\text{from $a_0^2$}}}{4^2}
\times\frac{512}{2187}=\frac{1}{26244}=\frac{1}{2^2 3^8}
\end{equation*}

\noindent
Let us analyze the units involved in computing $A_{nm}$.
For the coefficient of $A_{nm}$ we have
\begin{equation*}
\frac{e^2}{3\pi\varepsilon_0\hbar c^3}\propto
\frac{
\underset{e^2}
{\text{ampere}^2\,\text{second}^2}
}
{
\underset{\varepsilon_0}
{\left(\frac{\text{ampere}^2\,\text{second}^4}{\text{kilogram}\,\text{meter}^3}\right)}
\underset{\hbar}
{\left(\frac{\text{kilogram}\,\text{meter}^2}{\text{second}}\right)}
\underset{c^3}
{\left(\frac{\text{meter}^3}{\text{second}^3}\right)}
}
=\frac{\text{second}^2}{\text{meter}^2}
\end{equation*}

\noindent
For the transition frequency we have
\begin{equation*}
\omega_{21}=
\frac{3e^4 m_e}{128\pi^2\varepsilon_0^2\hbar^3}
\propto
\frac{
\underset{e^4}
{\left(\text{ampere}^4\,\text{second}^4\right)}
\,
\underset{m_e}
{\text{kilogram}}
}
{
\underset{\varepsilon_0^2}
{\left(\frac{\text{ampere}^4\,\text{second}^8}{\text{kilogram}^2\,\text{meter}^6}\right)}
\underset{\hbar^3}
{\left(\frac{\text{kilogram}^3\,\text{meter}^6}{\text{second}^3}\right)}
}
=\text{second}^{-1}
\end{equation*}

\noindent
For the Bohr radius we have
\begin{equation*}
a_0=\frac{4\pi\varepsilon_0\hbar^2}{e^2 m_e}
\propto
\frac
{
\underset{\varepsilon_0}
{\left(\frac{\text{ampere}^2\,\text{second}^4}{\text{kilogram}\,\text{meter}^3}\right)}
\underset{\hbar^2}
{\left(\frac{\text{kilogram}^2\,\text{meter}^4}{\text{second}^2}\right)}
}
{
\underset{e^2}
{\left(\text{ampere}^2\,\text{second}^2\right)}
\,
\underset{m_e}
{\text{kilogram}}
}
=\text{meter}
\end{equation*}

\noindent
Hence
\begin{equation*}
A_{nm}\propto
\frac{\text{second}^2}{\text{meter}^2}
\times
\underset{\substack{\\[1ex]\omega_{nm}^3}}{\text{second}^{-3}}
\times
\underset{\substack{\\[1ex]a_0^2}}{\text{meter}^2}
=\text{second}^{-1}
\end{equation*}

\noindent
The coefficients $B_{12}$ (absorption) and $B_{21}$ (induced emission) can be computed from $A_{21}$.
\begin{gather*}
B_{21}=\frac{c^2}{2h\nu^3}\,A_{21}=
\frac{4.25\times10^{58}}{\nu^3}
\\[2ex]
B_{12}=\frac{g_2}{g_1}\,B_{21}=\frac{6}{2}\,B_{21}
=\frac{1.28\times10^{59}}{\nu^3}
\end{gather*}

\noindent
Symbol $g_n$ is the multiplicity associated with energy level $n$.
\begin{equation*}
g=(2s+1)(2\ell+1)
\end{equation*}

\end{document}
