\documentclass[12pt]{article}
\usepackage[margin=2cm]{geometry}
\usepackage{amsmath}

\begin{document}

\noindent
Matrix elements for position $X$ and momentum $P$ are the following transition amplitudes.
\begin{align*}
X_{kj}&=\int_{-\infty}^\infty \psi_k\,x\,\psi_j\,dx
\\[1ex]
P_{kj}&=\int_{-\infty}^\infty \psi_k\left(-i\hbar\frac{d}{dx}\right)\psi_j\,dx
\end{align*}

\noindent
For $4\times4$ matrices we have
\begin{align*}
X&=\left(\frac{\hbar}{2m\omega}\right)^{1/2}
\begin{pmatrix}
0 & 1 & 0 & 0
\\
1 & 0 & \sqrt{2} & 0
\\
0 & \sqrt{2} & 0 & \sqrt{3}
\\
0 & 0 & \sqrt{3} & 0
\end{pmatrix}
\\[1ex]
P&=i\left(\frac{\hbar m\omega}{2}\right)^{1/2}
\begin{pmatrix}
0 & -1 & 0 & 0
\\
1 & 0 & -\sqrt{2} & 0
\\
0 & \sqrt{2} & 0 & -\sqrt{3}
\\
0 & 0 & \sqrt{3} & 0
\end{pmatrix}
\\[1ex]
H&=\frac{P^2}{2m}+\frac{1}{2}m\omega^2 X^2
=\begin{pmatrix}
\tfrac{1}{2}\hbar\omega & 0 & 0 & 0
\\
0 & \tfrac{3}{2}\hbar\omega & 0 & 0
\\
0 & 0 & \tfrac{5}{2}\hbar\omega & 0
\\
0 & 0 & 0 & \tfrac{7}{2}\hbar\omega
\end{pmatrix}
\end{align*}

\noindent
$H_{33}$ cannot be computed using $4\times4$ matrices.
The value $\tfrac{7}{2}\hbar\omega$ is the corrected eigenvalue.

\bigskip
\noindent
Consider the following eigenfunction.
\begin{equation*}
\Psi=\sum_k c_k\psi_k
\end{equation*}

\noindent
Let us compute the expected value of $x$ for a system in state $\Psi$.
\begin{equation*}
\langle x\rangle=\int_{-\infty}^\infty \Psi^* x\Psi\,dx
\end{equation*}

\noindent
Expand the integrand.
\begin{align*}
\langle x\rangle
&=\int_{-\infty}^\infty
\left(\sum_kc_k^*\psi_k^*\right) x \left(\sum_j c_j\psi_j\right)
\\
&=\sum_k\sum_jc_k^*c_j\int_{-\infty}^\infty\psi_k^*\,x\,\psi_j\,dx
\\
&=\sum_k\sum_jc_k^*c_j X_{kj}
\end{align*}

\noindent
Hence
\begin{equation*}
\langle x\rangle=
\begin{pmatrix}c_0^* & c_1^* & c_2^* & \ldots\end{pmatrix}
X
\begin{pmatrix}c_0\\ c_1\\ c_2\\ \vdots\end{pmatrix}
\end{equation*}

\end{document}

\noindent
Noting that $X_{kj}=0$ for $|k-j|\ne1$ we have
\begin{equation*}
\langle x\rangle=\sum_{k=0}^2\left(c_k^*c_{k+1}X_{k,k+1}+c_{k+1}^*c_kX_{k+1,k}\right)
\end{equation*}

\noindent
Noting that $X_{k,k+1}=X_{k+1,k}$ we have
\begin{align*}
\langle x\rangle&=\sum_{k=0}^2\left(c_k^*c_{k+1}+c_{k+1}^*c_k\right)X_{k,k+1}
\\
&=(c_0^*c_1+c_1^*c_0)X_{01}+(c_1^*c_2+c_2^*c_1)X_{12}+(c_2^*c_3+c_3^*c_2)X_{23}
\end{align*}

\noindent
From equation (1) above, the expected value of $x$ can also be computed this way.
\begin{equation*}
\langle x\rangle=\begin{pmatrix}c_0^* & c_1^* & c_2^* & c_3^*\end{pmatrix}
X\begin{pmatrix}c_0\\ c_1\\ c_2\\ c_3\end{pmatrix}
\end{equation*}

\end{document}
