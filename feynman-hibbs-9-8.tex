\documentclass[12pt]{article}
\usepackage{amsmath}
\usepackage{amssymb}

\parindent=0pt

\begin{document}

9-8.
For the state for which there is just one photon
present in level $1,\mathbf k$, all of the factors in the wave function are
$\phi_0$ except one, which is $\phi_1$.
But for an oscillator $\phi_1(x)=\sqrt2x\phi_0(x)$.
The wave function representing an excited running wave is a linear
superposition of the state with the cosine mode excited and $i$
times the state with the sine wave excited, so show that the
unnormalized wave function for just one photon present in
$1,\mathbf k$ is $\bar a_{1,\mathbf k}^*\Phi_0$.
The normalization is
$\int\Phi_0^*\bar a_{1,\mathbf k}\bar a_{1,\mathbf k}^*\Phi_0\,d\bar a$,
or the expectation of $\bar a_{1,\mathbf k}\bar a_{1,\mathbf k}^*$ for the
vacuum, which we have seen in the preceding problem is $\hbar/2kc$.
Hence the normalized one-photon state is
$\sqrt{2kc/\hbar}\bar a_{1,\mathbf k}^*\Phi_0$.

\bigskip
From problem 9-6, let
\begin{equation*}
\Phi_0=\exp\left(-\frac{kc}{4\hbar}(\bar a_{1,\mathbf k}^c)^2-\frac{kc}{4\hbar}(\bar a_{1,\mathbf k}^s)^2\right)
\end{equation*}

It follows that
\begin{equation*}
\Phi_0^*\Phi_0=\exp\left(-\frac{kc}{2\hbar}(\bar a_{1,\mathbf k}^c)^2-\frac{kc}{2\hbar}(\bar a_{1,\mathbf k}^s)^2\right)
\end{equation*}

The expectation of $\Phi_0$ is
\begin{equation*}
\langle\Phi_0\rangle=
\int\limits_{-\infty}^\infty
\int\limits_{-\infty}^\infty
\Phi_0^*\Phi_0
\,d\bar a_{1,\mathbf k}^c\,d\bar a_{1,\mathbf k}^s=\frac{2\pi\hbar}{kc}
\tag{1}
\end{equation*}

Let
\begin{equation*}
\Phi_1=\bar a_{1,\mathbf k}^*\Phi_0
\end{equation*}

Then
\begin{equation*}
\Phi_1^*\Phi_1=\Phi_0^*\frac{(\bar a_{1,\mathbf k}^c)^2+(\bar a_{1,\mathbf k}^s)^2}{2}\Phi_0
\end{equation*}

The expectation of $\Phi_1$ is
\begin{equation*}
\langle\Phi_1\rangle=
\int\limits_{-\infty}^\infty
\int\limits_{-\infty}^\infty
\Phi_1^*\Phi_1
\,d\bar a_{1,\mathbf k}^c\,d\bar a_{1,\mathbf k}^s
=\frac{\hbar}{kc}\frac{2\pi\hbar}{kc}
\tag{2}
\end{equation*}

The expectation of $\Phi_1$ is $\hbar/kc$ times the vacuum state.

\end{document}
