\documentclass[12pt]{article}
\usepackage{amsmath}
\usepackage{amssymb}

\parindent=0pt

\newcommand\U{\vert\Phi_0\vert^2}

\begin{document}

9-8.
For the state for which there is just one photon
present in level $1,\mathbf k$, all of the factors in the wave function are
$\phi_0$ except one, which is $\phi_1$.
But for an oscillator $\phi_1(x)=\sqrt2x\phi_0(x)$.
The wave function representing an excited running wave is a linear
superposition of the state with the cosine mode excited and $i$
times the state with the sine wave excited, so show that the
unnormalized wave function for just one photon present in
$1,\mathbf k$ is $\bar a_{1,\mathbf k}^*\Phi_0$.
The normalization is
$\int\Phi_0^*\bar a_{1,\mathbf k}\bar a_{1,\mathbf k}^*\Phi_0\,d\bar a$,
or the expectation of $\bar a_{1,\mathbf k}\bar a_{1,\mathbf k}^*$ for the
vacuum, which we have seen in the preceding problem is $\hbar/2kc$.
Hence the normalized one-photon state is
$\sqrt{2kc/\hbar}\bar a_{1,\mathbf k}^*\Phi_0$.

\bigskip
This is a state with the cosine mode excited.
\begin{equation*}
\bar a_{1,\mathbf k}^c\Phi_0
\end{equation*}

This is a state with the sine mode excited.
\begin{equation*}
\bar a_{1,\mathbf k}^s\Phi_0
\end{equation*}

This is a superposition of the cosine state and $i$ times the sine state.
The factor $1/\sqrt2$ is for normalization, i.e., $|1+i|=\sqrt2$.
\begin{equation*}
\frac{1}{\sqrt2}\left(\bar a_{1,\mathbf k}^c\Phi_0+i\bar a_{1,\mathbf k}^s\Phi_0\right)
=\bar a_{1,\mathbf k}^*\Phi_0
\end{equation*}

Here are some additional results.

\bigskip
From equation (9.43)
\begin{equation*}
\U=\Phi_0^*\Phi_0=\exp\left(
-\frac{kc}{\hbar}(\bar a_{1,\mathbf k}^c)^2-\frac{kc}{\hbar}(\bar a_{1,\mathbf k}^s)^2
-\frac{kc}{\hbar}(\bar a_{2,\mathbf k}^c)^2-\frac{kc}{\hbar}(\bar a_{2,\mathbf k}^s)^2
\right)
\end{equation*}

For simplicity of notation, let
\begin{equation*}
d\bar a=
d\bar a_{1,\mathbf k}^c\,d\bar a_{1,\mathbf k}^s
\,d\bar a_{2,\mathbf k}^c\,d\bar a_{2,\mathbf k}^s
\end{equation*}

The expectation of $\Phi_0$ is
\begin{equation*}
\langle\Phi_0\rangle=
\int\limits_{-\infty}^\infty\cdots
\int\limits_{-\infty}^\infty
\U\,d\bar a
=\left(\frac{\pi\hbar}{kc}\right)^2
\tag{1}
\end{equation*}

Let
\begin{equation*}
\Phi_1=\bar a_{1,\mathbf k}^*\Phi_0
\end{equation*}

Then
\begin{equation*}
\vert\Phi_1\vert^2
=\Phi_1^*\Phi_1
=\left(\frac{(\bar a_{1,\mathbf k}^c)^2+(\bar a_{1,\mathbf k}^s)^2}{2}\right)
\U
\end{equation*}

The expectation of $\Phi_1$ is
\begin{equation*}
\langle\Phi_1\rangle=
\int\limits_{-\infty}^\infty\cdots
\int\limits_{-\infty}^\infty
\left(\frac{(\bar a_{1,\mathbf k}^c)^2+(\bar a_{1,\mathbf k}^s)^2}{2}\right)
\U\,d\bar a
=\frac{\hbar}{2kc}
\left(\frac{\pi\hbar}{kc}\right)^2
\tag{2}
\end{equation*}

The expectation for $n$ photons is
\begin{equation*}
\langle\Phi_n\rangle=
\int\limits_{-\infty}^\infty\cdots
\int\limits_{-\infty}^\infty
\left(\frac{(\bar a_{1,\mathbf k}^c)^2+(\bar a_{1,\mathbf k}^s)^2}{2}\right)^n
\U\,d\bar a
\end{equation*}

By a result from problem 9-7
\begin{equation*}
\langle\Phi_n\rangle
=n!\left(\frac{\hbar}{2kc}\right)^n\left(\frac{\pi\hbar}{kc}\right)^2
=n!\left(\frac{\hbar}{2kc}\right)^n\langle\Phi_0\rangle
\tag{3}
\end{equation*}

\end{document}
