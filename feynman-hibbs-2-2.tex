\documentclass[12pt]{article}
\usepackage{amsmath}

\parindent=0pt

\begin{document}

2-2.
This is the Lagrangian for a harmonic oscillator.
\begin{equation*}
L=\frac{m}{2}(\dot{x}^2-\omega^2 x^2)
\end{equation*}

Let $T=t_b-t_a$.
Show that the classical action is
\begin{equation*}
S_{cl}=\frac{m\omega}{2\sin(\omega T)}
\bigg((x_b^2+x_a^2)\cos(\omega T)-2x_b x_a\bigg)
\end{equation*}

From the above Lagrangian we have
\begin{equation*}
\frac{d}{dt}\left(\frac{\partial L}{\partial\dot x}\right)=m\ddot x
\end{equation*}
and
\begin{equation*}
\frac{\partial L}{\partial x}=-m\omega^2x
\end{equation*}

By equation (2.7) which is
\begin{equation*}
\frac{d}{dt}\left(\frac{\partial L}{\partial\dot x}\right)=\frac{\partial L}{\partial x}
\end{equation*}
we have
\begin{equation*}
\ddot x=-\omega^2x
\tag{1}
\end{equation*}

The well-known solution to (1) is
\begin{equation*}
x(t)=A\sin(\omega t)+B\cos(\omega t)
\end{equation*}

We have the following boundary conditions.
\begin{align*}
x(0)&=x_a
\\[1ex]
x(T)&=x_b
\end{align*}

Solve for $B$.
\begin{equation*}
x(0)=B=x_a
\end{equation*}

For $x(T)$ we have
\begin{equation*}
x(T)=A\sin(\omega T)+B\cos(\omega T)
\end{equation*}

Solve for $A$.
\begin{equation*}
A=\frac{x(T)-B\cos(\omega T)}{\sin(\omega T)}=
\frac{x_b-x_a\cos(\omega T)}{\sin(\omega T)}
\end{equation*}

Hence the equation of motion is
\begin{equation*}
x(t)=\frac{x_b-x_a\cos(\omega T)}{\sin(\omega T)}\sin(\omega t)+x_a\cos(\omega t)
\tag{2}
\end{equation*}

Differentiate $x(t)$ to obtain velocity $\dot x(t)$.
\begin{equation*}
\dot x(t)=\frac{d}{dt}x(t)=
\omega\left(
\frac{x_b-x_a\cos(\omega T)}{\sin(\omega T)}\cos(\omega t)-x_a\sin(\omega t)
\right)
\tag{3}
\end{equation*}

Using the action integral
\begin{equation*}
S=\int_0^T L\,dt
\end{equation*}
we have for the classical action
\begin{align*}
S_{cl}&=\frac{m}{2}\int_0^T (\dot{x}^2-\omega^2 x^2)\,dt
\\[1ex]
&=\frac{m}{2}\left(
\int_0^T\dot{x}^2\,dt
-\int_0^T\omega^2x^2\,dt\right)
\end{align*}

Let $u=v=\dot x$ and note that
\begin{align*}
\dot u&=\ddot x
\\[1ex]
\int v\,dt&=x
\end{align*}

Apply integration by parts to the integral of $\dot x^2$.
\begin{align*}
\int_0^T \dot x^2\,dt
&=\int_0^T uv\,dt
\\[1ex]
&=\left(u\int v\,dt\right)_0^T
-\int_0^T\dot u\left(\int v\,dt\right)\,dt
\\[1ex]
&=\dot xx\bigg|_0^T-\int_0^T \ddot xx\,dt
\end{align*}

Hence
\begin{equation*}
S_{cl}=\frac{m}{2}\left(
\dot xx\bigg|_0^T-\int_0^T \ddot xx\,dt
-\int_0^T\omega^2x^2\,dt
\right)
\end{equation*}
The remaining integrals cancel by $\ddot x=-\omega^2x$ from equation (1).

\bigskip
We now have
\begin{align*}
S_{cl}&=\frac{m}{2}\dot xx\bigg|_0^T
\\[1ex]
&=\frac{m}{2}\bigg(\dot x(T)x(T)-\dot x(0)x(0)\bigg)
\tag{4}
\end{align*}

By substitution and simplification
\begin{equation*}
S_{cl}=\frac{m\omega}{2\sin(\omega T)}
\bigg((x_b^2+x_a^2)\cos(\omega T)-2x_b x_a\bigg)
\tag{5}
\end{equation*}

\end{document}
