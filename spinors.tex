\documentclass[12pt]{article}
\usepackage[margin=2cm]{geometry}
\usepackage{amsmath}
\usepackage{slashed}

\begin{document}

\noindent
Let $\phi$ be the field
\begin{equation*}
\phi(x,y,z,t)=p_xx+p_yy+p_zz-Et
\end{equation*}
where
\begin{equation*}
E=\sqrt{p_x^2c^2+p_y^2c^2+p_z^2c^2+m^2c^4}
\end{equation*}

\noindent
Fermion fields are represented by
the following solutions to the Dirac equation.
\begin{equation*}
\psi_1=\underset{\text{fermion spin up}}
{
\begin{pmatrix}E/c+mc\\0\\p_z\\p_x+ip_y\end{pmatrix}
\exp\left(\frac{i\phi}{\hbar}\right)
}
\quad
\psi_2=\underset{\text{fermion spin down}}
{
\begin{pmatrix}0\\E/c+mc\\p_x-ip_y\\-p_z\end{pmatrix}
\exp\left(\frac{i\phi}{\hbar}\right)
}
\end{equation*}
\begin{equation*}
\psi_7=\underset{\text{anti-fermion spin up}}
{
\begin{pmatrix}p_z\\p_x+ip_y\\E/c+mc\\0\end{pmatrix}
\exp\left(-\frac{i\phi}{\hbar}\right)
}
\quad
\psi_8=\underset{\text{anti-fermion spin down}}
{
\begin{pmatrix}p_x-ip_y\\-p_z\\0\\E/c+mc\end{pmatrix}
\exp\left(-\frac{i\phi}{\hbar}\right)
}
\end{equation*}

\noindent
A spinor is the vector part of $\psi$.
The following spinors are used for scattering calculations.
Symbol $u$ indicates a fermion and symbol $v$ indicates an anti-fermion.
\begin{equation*}
u_1=\underset{\text{fermion spin up}}{\begin{pmatrix}E/c+mc\\0\\p_z\\p_x+ip_y\end{pmatrix}}
\quad
u_2=\underset{\text{fermion spin down}}{\begin{pmatrix}0\\E/c+mc\\p_x-ip_y\\-p_z\end{pmatrix}}
\quad
v_1=\underset{\text{anti-fermion spin up}}{\begin{pmatrix}p_z\\p_x+ip_y\\E/c+mc\\0\end{pmatrix}}
\quad
v_2=\underset{\text{anti-fermion spin down}}{\begin{pmatrix}p_x-ip_y\\-p_z\\0\\E/c+mc\end{pmatrix}}
\end{equation*}

\noindent
This is the corresponding space-time momentum vector $p$.
$$
p=\begin{pmatrix}E/c\\p_x\\p_y\\p_z\end{pmatrix}\quad
$$

\noindent
Spinors are solutions to the following momentum-space Dirac equations where
$\slashed{p}=p^\mu g_{\mu\nu}\gamma^\nu$.
$$
\slashed{p}u=mcu\qquad\slashed{p}v=-mcv
$$
Up and down spinors have the following ``completeness property.''
$$
u_1\bar{u}_1+u_2\bar{u}_2=(E/c+mc)(\slashed{p}+mc)\qquad
v_1\bar{v}_1+v_2\bar{v}_2=(E/c+mc)(\slashed{p}-mc)
$$
The spinor adjoints are $\bar{u}=u^\dag\gamma^0$ and $\bar{v}=v^\dag\gamma^0$.
The adjoint is a row vector hence $u\bar{u}$ and $v\bar{v}$ are outer products.

\end{document}
