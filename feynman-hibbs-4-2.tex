\documentclass[12pt]{article}
\usepackage{amsmath}
\usepackage{amssymb}

\parindent=0pt

\newcommand\INT{\int_{\mathbb R^3}}

\begin{document}

4-2.
For a particle of charge $e$ in a magnetic field the Lagrangian is
\begin{equation*}
L(\dot{\mathbf x},\mathbf x)=\frac{m}{2}\dot{\mathbf x}^2
+\frac{e}{c}\dot{\mathbf x}\cdot\mathbf A(\mathbf x,t)-e\phi(\mathbf x,t)
\end{equation*}
where $\dot{\mathbf x}$ is the velocity vector,
$c$ is the velocity of light, and $\mathbf A$ and $\phi$
are the vector and scalar potentials.
Show that the corresponding Schrodinger equation is
\begin{equation*}
\frac{\partial\psi}{\partial t}
=-\frac{i}{\hbar}\left(
\frac{1}{2m}\left(\frac{\hbar}{i}\nabla-\frac{e}{c}\mathbf A\right)
\cdot
\left(\frac{\hbar}{i}\nabla-\frac{e}{c}\mathbf A\right)\psi
+e\phi\psi
\right)
\tag{4.18}
\end{equation*}

From equation (4.3) with a minor correction of $y-x$ instead of $x-y$.
\begin{equation*}
\psi(\mathbf x,t+\epsilon)=\frac{1}{A}\INT\exp\left(
\frac{i\epsilon}{\hbar}L\left(\frac{\mathbf y-\mathbf x}{\epsilon},\frac{\mathbf x+\mathbf y}{2}\right)
\right)\psi(\mathbf y,t)
\,dy_1\,dy_2\,dy_3
\tag{1}
\end{equation*}

This is the Lagrangian with arguments from (1).
\begin{multline*}
L\left(\frac{\mathbf y-\mathbf x}{\epsilon},\frac{\mathbf x+\mathbf y}{2}\right)
\\
=\frac{m}{2\epsilon^2}(\mathbf x-\mathbf y)^2
-\frac{e}{c\epsilon}(\mathbf x-\mathbf y)\cdot\mathbf A\left(\frac{\mathbf x+\mathbf y}{2},t\right)
-e\phi\left(\frac{\mathbf x+\mathbf y}{2},t\right)
\end{multline*}

Hence
\begin{align*}
&\psi(\mathbf x,t+\epsilon)=\frac{1}{A}\INT
\\
&\quad\exp
\left(
\frac{im}{2\hbar\epsilon}(\mathbf x-\mathbf y)^2
-\frac{ie}{\hbar c}(\mathbf x-\mathbf y)\cdot\mathbf A\left(\frac{\mathbf x+\mathbf y}{2},t\right)
-\frac{ie\epsilon}{\hbar}\phi\left(\frac{\mathbf x+\mathbf y}{2},t\right)
\right)
\\
&\quad\quad{}\times\psi(\mathbf y,t)
\,dy_1\,dy_2\,dy_3
\end{align*}

Let
\begin{equation*}
\mathbf y=\mathbf x+\boldsymbol\eta
\end{equation*}

Then
\begin{equation*}
\mathbf x-\mathbf y=\boldsymbol\eta,\quad
\frac{\mathbf x+\mathbf y}{2}=\mathbf{x}+\tfrac{1}{2}\boldsymbol\eta,\quad
dy_1\,dy_2\,dy_3=d\eta_1\,d\eta_2\,d\eta_3
\end{equation*}

Hence
\begin{multline*}
\psi(\mathbf x,t+\epsilon)=
\frac{1}{A}\INT\exp
\left(
\frac{im}{2\hbar\epsilon}\boldsymbol\eta^2
-\frac{ie}{\hbar c}\boldsymbol\eta\cdot\mathbf A\left(\mathbf x+\tfrac{1}{2}\boldsymbol\eta,t\right)
-\frac{ie\epsilon}{\hbar}\phi\left(\mathbf x+\tfrac{1}{2}\boldsymbol\eta,t\right)
\right)
\\
{}\times\psi(\mathbf x+\boldsymbol\eta,t)
\,d\eta_1\,d\eta_2\,d\eta_3
\end{multline*}

Factor the exponential.
\begin{align*}
&\psi(\mathbf x,t+\epsilon)=
\frac{1}{A}\INT
\\
&\quad{}\exp\left(\frac{im}{2\hbar\epsilon}\boldsymbol\eta^2\right)
\exp\left(-\frac{ie}{\hbar c}\boldsymbol\eta\cdot\mathbf A\left(\mathbf x+\tfrac{1}{2}\boldsymbol\eta,t\right)\right)
\exp\left(-\frac{ie\epsilon}{\hbar}\phi\left(\mathbf x+\tfrac{1}{2}\boldsymbol\eta,t\right)\right)
\\
&\quad\quad{}\times\psi(\mathbf x+\boldsymbol\eta,t)
\,d\eta_1\,d\eta_2\,d\eta_3
\tag{2}
\end{align*}

From the identity $\exp(i\theta)=\cos(\theta)+i\sin(\theta)$ we have
\begin{multline*}
\exp\left(-\frac{ie\epsilon}{\hbar}\phi\left(\mathbf x+\tfrac{1}{2}\boldsymbol\eta,t\right)\right)
\\
=\cos\left(-\frac{e\epsilon}{\hbar}\phi\left(\mathbf x+\tfrac{1}{2}\boldsymbol\eta,t\right)\right)
+i\sin\left(-\frac{e\epsilon}{\hbar}\phi\left(\mathbf x+\tfrac{1}{2}\boldsymbol\eta,t\right)\right)
\end{multline*}

Then for small $\epsilon$
\begin{equation*}
\exp\left(-\frac{ie\epsilon}{\hbar}\phi\left(\mathbf x+\tfrac{1}{2}\boldsymbol\eta,t\right)\right)
\approx
1-\frac{ie\epsilon}{\hbar}\phi\left(\mathbf x+\tfrac{1}{2}\boldsymbol\eta,t\right)
\end{equation*}

The $\boldsymbol\eta$ term can be discarded because the integral is Gaussian.
(Contributions to the integral are small for $\boldsymbol\eta^2>2\hbar\epsilon/m$.)
\begin{equation*}
\exp\left(-\frac{ie\epsilon}{\hbar}\phi\left(\mathbf x+\tfrac{1}{2}\boldsymbol\eta,t\right)\right)\approx
1-\frac{ie\epsilon}{\hbar}\phi(\mathbf x,t)
\tag{3}
\end{equation*}

Substitute (3) into (2).
\begin{multline*}
\psi(\mathbf x,t+\epsilon)=
\frac{1}{A}\left(1-\frac{ie\epsilon}{\hbar}\phi\right)\INT
\\
\exp\left(\frac{im}{2\hbar\epsilon}\boldsymbol\eta^2\right)
\exp\left(-\frac{ie}{\hbar c}\boldsymbol\eta\cdot\mathbf A\left(\mathbf x+\tfrac{1}{2}\boldsymbol\eta,t\right)\right)
\psi(\mathbf x+\boldsymbol\eta,t)
\,d\eta_1\,d\eta_2\,d\eta_3
\end{multline*}

Approximate the exponential involving $\mathbf A$ with a Taylor series.
\begin{equation*}
\exp\left(-\frac{ie}{\hbar c}\boldsymbol\eta\cdot\mathbf A\left(\mathbf x+\tfrac{1}{2}\boldsymbol\eta,t\right)\right)
\approx
\exp\left(
-\frac{ie}{\hbar c}\boldsymbol\eta\cdot\mathbf A(\mathbf x)
-\frac{ie}{2\hbar c}\boldsymbol\eta\cdot(\boldsymbol\eta\cdot\nabla\mathbf A(\mathbf x))
\right)
\end{equation*}

Expand the right-hand side as a power series.
\begin{equation*}
\exp\left(-\frac{ie}{\hbar c}\boldsymbol\eta\cdot\mathbf A\left(\mathbf x+\tfrac{1}{2}\boldsymbol\eta,t\right)\right)
\approx
\left(1+T+\tfrac{1}{2}T^2\right)
\end{equation*}
where
\begin{equation*}
T=-\frac{ie}{\hbar c}\boldsymbol\eta\cdot\mathbf A(\mathbf x)
-\frac{ie}{2\hbar c}\boldsymbol\eta\cdot(\boldsymbol\eta\cdot\nabla\mathbf A(\mathbf x))
\end{equation*}

Discard high order terms.
\begin{multline*}
\exp\left(-\frac{ie}{\hbar c}\boldsymbol\eta\cdot\mathbf A\left(\mathbf x+\tfrac{1}{2}\boldsymbol\eta,t\right)\right)
\\
\approx
1
-\frac{ie}{\hbar c}\boldsymbol\eta\cdot\mathbf A(\mathbf x)
-\frac{ie}{2\hbar c}\boldsymbol\eta\cdot(\boldsymbol\eta\cdot\nabla\mathbf A(\mathbf x))
+\frac{1}{2}\left(-\frac{ie}{\hbar c}\boldsymbol\eta\cdot\mathbf A(\mathbf x)\right)^2
\end{multline*}

Hence
\begin{align*}
&\psi(\mathbf x,t+\epsilon)=
\frac{1}{A}
\left(1-\frac{ie\epsilon}{\hbar}\phi\right)
\INT\exp\left(\frac{im}{2\hbar\epsilon}\boldsymbol\eta^2\right)
\\
&{}\times\left(
1
-\frac{ie}{\hbar c}\boldsymbol\eta\cdot\mathbf A
-\frac{ie}{2\hbar c}\boldsymbol\eta\cdot(\boldsymbol\eta\cdot\nabla\mathbf A)
+\frac{1}{2}\left(-\frac{ie}{\hbar c}\boldsymbol\eta\cdot\mathbf A\right)^2
\right)
\\
&{}\times\psi(\mathbf x+\boldsymbol\eta,t)
\,d\eta_1\,d\eta_2\,d\eta_3
\tag{4}
\end{align*}

Next we will use the following Taylor series approximations.
\begin{equation*}
\begin{aligned}
\psi(\mathbf x,t+\epsilon)&\approx\psi+\epsilon\frac{\partial\psi}{\partial t}
\\
\psi(\mathbf x+\boldsymbol\eta,t)&\approx\psi+\boldsymbol\eta\cdot\nabla\psi
+\tfrac{1}{2}\boldsymbol\eta\cdot\nabla(\boldsymbol\eta\cdot\nabla\psi)
\end{aligned}
\tag{5}
\end{equation*}

Substitute the approximations (5) into (4).
\begin{align*}
&\psi+\epsilon\frac{\partial\psi}{\partial t}=
\frac{1}{A}
\left(1-\frac{ie\epsilon}{\hbar}\phi\right)
\INT\exp\left(\frac{im}{2\hbar\epsilon}\boldsymbol\eta^2\right)
\\
&{}\times\left(
1
-\frac{ie}{\hbar c}\boldsymbol\eta\cdot\mathbf A
-\frac{ie}{2\hbar c}\boldsymbol\eta\cdot(\boldsymbol\eta\cdot\nabla\mathbf A)
+\frac{1}{2}\left(-\frac{ie}{\hbar c}\boldsymbol\eta\cdot\mathbf A\right)^2
\right)
\\
&{}\times\left(
\psi+\boldsymbol\eta\cdot\nabla\psi
+\tfrac{1}{2}\boldsymbol\eta\cdot\nabla(\boldsymbol\eta\cdot\nabla\psi)
\right)
\,d\eta_1\,d\eta_2\,d\eta_3
\tag{6}
\end{align*}

To solve the above integral, we will use the following formulas provided by the authors.
\begin{align*}
&\int_{-\infty}^\infty\exp\left(\frac{im\eta_k^2}{2\hbar\epsilon}\right)\,d\eta_k
=\left(\frac{2\pi i\hbar\epsilon}{m}\right)^{1/2}
\tag{4.7}
\\
&\int_{-\infty}^\infty\eta_k\exp\left(\frac{im\eta_k^2}{2\hbar\epsilon}\right)\,d\eta_k
=0
\tag{4.9}
\\
&\int_{-\infty}^\infty\eta_k^2\exp\left(\frac{im\eta_k^2}{2\hbar\epsilon}\right)\,d\eta_k
=\frac{i\hbar\epsilon}{m}\left(\frac{2\pi i\hbar\epsilon}{m}\right)^{1/2}
\tag{4.10}
\end{align*}

The integrand in (6) has twelve terms.
The following table summarizes the integrals $\int uv$ for each factor pair $uv$.
Integrals $I_9$ and $I_{12}$ are order $\epsilon^2$ by (4.10) and are discarded.

\begin{center}
\begin{tabular}{|l|c|c|c|}
\hline
& & & \\ % vertical padding
& $v=\psi$ & $v=\boldsymbol\eta\cdot\nabla\psi$ & $v=\tfrac{1}{2}\boldsymbol\eta\cdot\nabla(\boldsymbol\eta\cdot\nabla\psi)$
\\
& & & \\ % vertical padding
\hline
& & & \\ % vertical padding
$u=1$ & $I_1$ & 0 & $I_3$
\\
& & & \\ % vertical padding
\hline
& & &
\\
$\displaystyle u=-\frac{ie}{\hbar c}\boldsymbol\eta\cdot\mathbf A$ & 0 & $I_5$ & 0
\\
& & &
\\
\hline
& & & \\ % vertical padding
$\displaystyle u=-\frac{ie}{2\hbar c}\boldsymbol\eta\cdot(\boldsymbol\eta\cdot\nabla\mathbf A)$ & $I_7$ & 0 &
$I_9\propto\epsilon^2$
\\
& & & \\ % vertical padding
\hline
& & & \\ % vertical padding
$\displaystyle u=\frac{1}{2}\left(-\frac{ie}{\hbar c}\boldsymbol\eta\cdot\mathbf A\right)^2$ & $I_{10}$ & 0 &
$I_{12}\propto\epsilon^2$
\\
& & & \\ % vertical padding
\hline
\end{tabular}
\end{center}

\bigskip
\begin{align*}
I_1&=\left(\frac{2\pi i\hbar\epsilon}{m}\right)^{3/2}\psi
\\
I_3&=\frac{i\hbar\epsilon}{2m}\left(\frac{2\pi i\hbar\epsilon}{m}\right)^{3/2}\nabla^2\psi
\\
I_5&=\frac{i\hbar\epsilon}{m}\left(\frac{2\pi i\hbar\epsilon}{m}\right)^{3/2}
\frac{-ie}{\hbar c}\mathbf A\cdot\nabla\psi
\\
I_7&=\frac{i\hbar\epsilon}{m}
\left(\frac{2\pi i\hbar\epsilon}{m}\right)^{3/2}
\frac{-ie}{2\hbar c}\nabla\cdot\mathbf A\psi
\\
I_{10}&=\frac{i\hbar\epsilon}{m}
\left(\frac{2\pi i\hbar\epsilon}{m}\right)^{3/2}
\frac{1}{2}
\left(\frac{-ie}{\hbar c}\right)^2
\mathbf A^2\psi
\end{align*}

Substitute the solved integrals into (6) to obtain
\begin{multline*}
\psi+\epsilon\frac{\partial\psi}{\partial t}
=\frac{1}{A}
\left(1-\frac{ie\epsilon}{\hbar}\phi\right)\left(\frac{2\pi i\hbar\epsilon}{m}\right)^{3/2}
\\
\times\left(
\psi
+\frac{i\hbar\epsilon}{2m}\nabla^2\psi
+\frac{e\epsilon}{mc}\mathbf A\cdot\nabla\psi
+\frac{e\epsilon}{2mc}\nabla\cdot\mathbf A\psi
-\frac{ie^2\epsilon}{2m\hbar c^2}\mathbf A^2\psi
\right)
\end{multline*}

In the limit as $\epsilon\rightarrow0$ we have
\begin{equation*}
\psi=\frac{1}{A}\left(\frac{2\pi i\hbar\epsilon}{m}\right)^{3/2}\psi
\end{equation*}
hence
\begin{equation*}
A=\left(\frac{2\pi i\hbar\epsilon}{m}\right)^{3/2}
\end{equation*}

Cancel $A$.
\begin{multline*}
\psi+\epsilon\frac{\partial\psi}{\partial t}
=\left(1-\frac{ie\epsilon}{\hbar}\phi\right)
\\
\times\left(
\psi
+\frac{i\hbar\epsilon}{2m}\nabla^2\psi
+\frac{e\epsilon}{mc}\mathbf A\cdot\nabla\psi
+\frac{e\epsilon}{2mc}\nabla\cdot\mathbf A\psi
-\frac{ie^2\epsilon}{2m\hbar c^2}\mathbf A^2\psi
\right)
\end{multline*}

Expand the product and discard terms of order $\epsilon^2$.
\begin{multline*}
\psi+\epsilon\frac{\partial\psi}{\partial t}
\\
=\psi
+\frac{i\hbar\epsilon}{2m}\nabla^2\psi
+\frac{e\epsilon}{mc}\mathbf A\cdot\nabla\psi
+\frac{e\epsilon}{2mc}\nabla\cdot\mathbf A\psi
-\frac{ie^2\epsilon}{2m\hbar c^2}\mathbf A^2\psi
-\frac{ie\epsilon}{\hbar}\phi\psi
\end{multline*}

Cancel leading terms $\psi$ and divide through by $\epsilon$.
\begin{equation*}
\frac{\partial\psi}{\partial t}
=\frac{i\hbar}{2m}\nabla^2\psi
+\frac{e}{mc}\mathbf A\cdot\nabla\psi
+\frac{e}{2mc}\nabla\cdot\mathbf A\psi
-\frac{ie^2}{2m\hbar c^2}\mathbf A^2\psi
-\frac{ie}{\hbar}\phi\psi
\tag{7}
\end{equation*}

To show that (7) is the same as (4.18), expand (4.18) step by step.
First we have
\begin{equation*}
\left(\frac{\hbar}{i}\nabla-\frac{e}{c}\mathbf A\right)\psi
=-i\hbar\nabla\psi-\frac{e}{c}\mathbf A\psi
\end{equation*}

Next
\begin{multline*}
\left(\frac{\hbar}{i}\nabla-\frac{e}{c}\mathbf A\right)^2\psi
=\left(-i\hbar\nabla-\frac{e}{c}\mathbf A\right)
\left(-i\hbar\nabla\psi-\frac{e}{c}\mathbf A\psi\right)
\\
=-\hbar^2\nabla(\nabla\psi)
+\frac{ie\hbar}{c}\nabla(\mathbf A\psi)
+\frac{ie\hbar}{c}\mathbf A\cdot\nabla\psi
+\frac{e^2}{c^2}\mathbf A\cdot\mathbf A\psi
\end{multline*}

Because $\nabla\psi$ is a vector we have
\begin{equation*}
\nabla(\nabla\psi)=\nabla\cdot\nabla\psi=\nabla^2\psi
\end{equation*}
and because $\mathbf A$ is a vector and $\psi$ is a scalar we have
\begin{equation*}
\nabla(\mathbf A\psi)=\nabla\cdot\mathbf A\psi+\mathbf A\cdot\nabla\psi
\end{equation*}

Hence
\begin{equation*}
\left(\frac{\hbar}{i}\nabla-\frac{e}{c}\mathbf A\right)^2\psi
=-\hbar^2\nabla^2\psi
+\frac{2ie\hbar}{c}\mathbf A\cdot\nabla\psi
+\frac{ie\hbar}{c}\nabla\cdot\mathbf A\psi
+\frac{e^2}{c^2}\mathbf A^2\psi
\end{equation*}

Divide through by $2m$.
\begin{multline*}
\frac{1}{2m}\left(\frac{\hbar}{i}\nabla-\frac{e}{c}\mathbf A\right)^2\psi
\\
=-\frac{\hbar^2}{2m}\nabla^2\psi
+\frac{ie\hbar}{mc}\mathbf A\cdot\nabla\psi
+\frac{ie\hbar}{2mc}\nabla\cdot\mathbf A\psi
+\frac{e^2}{2mc^2}\mathbf A^2\psi
\end{multline*}

Add the scalar potential.
\begin{multline*}
\frac{1}{2m}\left(\frac{\hbar}{i}\nabla-\frac{e}{c}\mathbf A\right)^2\psi+e\phi\psi
\\
=-\frac{\hbar^2}{2m}\nabla^2\psi
+\frac{ie\hbar}{mc}\mathbf A\cdot\nabla\psi
+\frac{ie\hbar}{2mc}\nabla\cdot\mathbf A\psi
+\frac{e^2}{2mc^2}\mathbf A^2\psi
+e\phi\psi
\end{multline*}

Finally, multiply by $-i/\hbar$.
\begin{multline*}
-\frac{i}{\hbar}\left(
\frac{1}{2m}\left(\frac{\hbar}{i}\nabla-\frac{e}{c}\mathbf A\right)^2\psi+e\phi\psi
\right)
\\
=\frac{i\hbar}{2m}\nabla^2\psi
+\frac{e}{mc}\mathbf A\cdot\nabla\psi
+\frac{e}{2mc}\nabla\cdot\mathbf A\psi
-\frac{ie^2}{2m\hbar c^2}\mathbf A^2\psi
-\frac{ie}{\hbar}\phi\psi
\end{multline*}

The result is identical to (7).
\begin{equation*}
\frac{\partial\psi}{\partial t}
=\frac{i\hbar}{2m}\nabla^2\psi
+\frac{e}{mc}\mathbf A\cdot\nabla\psi
+\frac{e}{2mc}\nabla\cdot\mathbf A\psi
-\frac{ie^2}{2m\hbar c^2}\mathbf A^2\psi
-\frac{ie}{\hbar}\phi\psi
\tag{7}
\end{equation*}

\end{document}
