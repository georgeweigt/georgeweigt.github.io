\documentclass[12pt]{article}
\usepackage{amsmath}
\usepackage{amssymb}

\parindent=0pt

\newcommand\INT{\int_{\mathbb R^3}}

\begin{document}

Feynman and Hibbs problem 4-2

\bigskip
For a particle of charge $e$ in a magnetic field the Lagrangian is
\begin{equation*}
L(\dot{\mathbf x},\mathbf x)=\frac{m}{2}\dot{\mathbf x}^2
+\frac{e}{c}\dot{\mathbf x}\cdot\mathbf A(\mathbf x,t)-e\phi(\mathbf x,t)
\end{equation*}
where $\dot{\mathbf x}$ is the velocity vector,
$c$ is the velocity of light, and $\mathbf A$ and $\phi$
are the vector and scalar potentials.
Show that the corresponding Schrodinger equation is
\begin{equation*}
\frac{\partial\psi}{\partial t}
=-\frac{i}{\hbar}\left(
\frac{1}{2m}\left(\frac{\hbar}{i}\nabla-\frac{e}{c}\mathbf A\right)
\cdot
\left(\frac{\hbar}{i}\nabla-\frac{e}{c}\mathbf A\right)\psi
+e\phi\psi
\right)
\end{equation*}

From equation (4.3)
\begin{equation*}
\psi(\mathbf x,t+\epsilon)=\frac{1}{A}\INT\exp\left(
\frac{i\epsilon}{\hbar}L\left(\frac{\mathbf x-\mathbf y}{\epsilon},\frac{\mathbf x+\mathbf y}{2}\right)
\right)\psi(\mathbf y,t)
\,dy_1\,dy_2\,dy_3
\tag{1}
\end{equation*}

By substitution of the given Lagrangian
\begin{multline*}
L\left(\frac{\mathbf x-\mathbf y}{\epsilon},\frac{\mathbf x+\mathbf y}{2}\right)
\\
=\frac{m}{2\epsilon^2}(\mathbf x-\mathbf y)^2
+\frac{e}{c\epsilon}(\mathbf x-\mathbf y)\cdot\mathbf A\left(\frac{\mathbf x+\mathbf y}{2},t\right)
-e\phi\left(\frac{\mathbf x+\mathbf y}{2},t\right)
\end{multline*}

Then from equation (1)
\begin{align*}
&\psi(\mathbf x,t+\epsilon)=\frac{1}{A}\INT
\\
&\quad\exp
\left(
\frac{im}{2\hbar\epsilon}(\mathbf x-\mathbf y)^2
+\frac{ie}{\hbar c}(\mathbf x-\mathbf y)\cdot\mathbf A\left(\frac{\mathbf x+\mathbf y}{2},t\right)
-\frac{ie\epsilon}{\hbar}\phi\left(\frac{\mathbf x+\mathbf y}{2},t\right)
\right)
\\
&\quad\quad{}\times\psi(\mathbf y,t)
\,dy_1\,dy_2\,dy_3
\end{align*}

Let
\begin{equation*}
\mathbf y=\mathbf x+\boldsymbol\eta
\end{equation*}

Then
\begin{equation*}
\mathbf x-\mathbf y=\boldsymbol\eta,\quad
\frac{\mathbf x+\mathbf y}{2}=\mathbf{x}+\tfrac{1}{2}\boldsymbol\eta,\quad
dy_1\,dy_2\,dy_3=d\eta_1\,d\eta_2\,d\eta_3
\end{equation*}

Hence
\begin{multline*}
\psi(\mathbf x,t+\epsilon)=
\frac{1}{A}\INT\exp
\left(
\frac{im}{2\hbar\epsilon}\boldsymbol\eta^2
+\frac{ie}{\hbar c}\boldsymbol\eta\cdot\mathbf A\left(\mathbf x+\tfrac{1}{2}\boldsymbol\eta,t\right)
-\frac{ie\epsilon}{\hbar}\phi\left(\mathbf x+\tfrac{1}{2}\boldsymbol\eta,t\right)
\right)
\\
{}\times\psi(\mathbf x+\boldsymbol\eta,t)
\,d\eta_1\,d\eta_2\,d\eta_3
\end{multline*}

Factor the exponential.
\begin{align*}
&\psi(\mathbf x,t+\epsilon)=
\frac{1}{A}\INT
\\
&\quad{}\exp
\left(
\frac{im}{2\hbar\epsilon}\boldsymbol\eta^2
+\frac{ie}{\hbar c}\boldsymbol\eta\cdot\mathbf A\left(\mathbf x+\tfrac{1}{2}\boldsymbol\eta,t\right)
\right)
\exp\left(-\frac{ie\epsilon}{\hbar}\phi\left(\mathbf x+\tfrac{1}{2}\boldsymbol\eta,t\right)\right)
\\
&\quad\quad{}\times\psi(\mathbf x+\boldsymbol\eta,t)
\,d\eta_1\,d\eta_2\,d\eta_3
\tag{2}
\end{align*}

From the identity $\exp(i\theta)=\cos(\theta)+i\sin(\theta)$ we have
\begin{multline*}
\exp\left(-\frac{ie\epsilon}{\hbar}\phi\left(\mathbf x+\tfrac{1}{2}\boldsymbol\eta,t\right)\right)
\\
=\cos\left(-\frac{e\epsilon}{\hbar}\phi\left(\mathbf x+\tfrac{1}{2}\boldsymbol\eta,t\right)\right)
+i\sin\left(-\frac{e\epsilon}{\hbar}\phi\left(\mathbf x+\tfrac{1}{2}\boldsymbol\eta,t\right)\right)
\end{multline*}

Then for small $\epsilon$
\begin{equation*}
\exp\left(-\frac{ie\epsilon}{\hbar}\phi\left(\mathbf x+\tfrac{1}{2}\boldsymbol\eta,t\right)\right)
\approx
1-\frac{ie\epsilon}{\hbar}\phi\left(\mathbf x+\tfrac{1}{2}\boldsymbol\eta,t\right)
\tag{3}
\end{equation*}

Substitute (3) into (2).
\begin{align*}
&\psi(\mathbf x,t+\epsilon)=
\frac{1}{A}\INT
\\
&\quad{}\exp
\left(
\frac{im}{2\hbar\epsilon}\boldsymbol\eta^2
+\frac{ie}{\hbar c}\boldsymbol\eta\cdot\mathbf A\left(\mathbf x+\tfrac{1}{2}\boldsymbol\eta,t\right)
\right)
\left(1-\frac{ie\epsilon}{\hbar}\phi\left(\mathbf x+\tfrac{1}{2}\boldsymbol\eta,t\right)\right)
\\
&\quad\quad{}\times\psi(\mathbf x+\boldsymbol\eta,t)
\,d\eta_1\,d\eta_2\,d\eta_3
\tag{4}
\end{align*}

Next we will use the following Taylor series approximations.
\begin{equation*}
\begin{aligned}
\psi(\mathbf x,t+\epsilon)&\approx\psi(\mathbf x,t)+\epsilon\frac{\partial\psi}{\partial t}
\\
\psi(\mathbf x+\boldsymbol\eta,t)&\approx\psi(\mathbf x,t)+\boldsymbol\eta\cdot\nabla\psi
+\tfrac{1}{2}\boldsymbol\eta\cdot\nabla(\boldsymbol\eta\cdot\nabla\psi)
\end{aligned}
\tag{5}
\end{equation*}

Note: In component notation
\begin{equation*}
\boldsymbol\eta\cdot\nabla\psi=
\eta_1\frac{\partial\psi}{\partial x_1}+
\eta_2\frac{\partial\psi}{\partial x_2}+
\eta_2\frac{\partial\psi}{\partial x_2}
\end{equation*}
and
\begin{multline*}
\boldsymbol\eta\cdot\nabla(\boldsymbol\eta\cdot\nabla\psi)=
\eta_1^2\frac{\partial^2\psi}{\partial x_1^2}
+\eta_2^2\frac{\partial^2\psi}{\partial x_2^2}
+\eta_3^2\frac{\partial^2\psi}{\partial x_3^2}
\\
{}+2\eta_1\eta_2\frac{\partial^2\psi}{\partial x_1\partial x_2}
+2\eta_1\eta_3\frac{\partial^2\psi}{\partial x_1\partial x_3}
+2\eta_2\eta_3\frac{\partial^2\psi}{\partial x_2\partial x_3}
\end{multline*}

Substitute the approximations (5) into (4).
\begin{align*}
&\psi(\mathbf x,t)+\epsilon\frac{\partial\psi}{\partial t}=
\frac{1}{A}\INT
\\
&\quad{}\exp
\left(
\frac{im}{2\hbar\epsilon}\boldsymbol\eta^2
+\frac{ie}{\hbar c}\boldsymbol\eta\cdot\mathbf A\left(\mathbf x+\tfrac{1}{2}\boldsymbol\eta,t\right)
\right)
\left(1-\frac{ie\epsilon}{\hbar}\phi\left(\mathbf x+\tfrac{1}{2}\boldsymbol\eta,t\right)\right)
\\
&\quad\quad{}\times
\left(
\psi(\mathbf x,t)+\boldsymbol\eta\cdot\nabla\psi
+\tfrac{1}{2}\boldsymbol\eta\cdot\nabla(\boldsymbol\eta\cdot\nabla\psi)
\right)
\,d\eta_1\,d\eta_2\,d\eta_3
\end{align*}

Without any justification, drop the $\boldsymbol\eta$ term in $\mathbf A$ and $\phi$.
\begin{align*}
&\psi(\mathbf x,t)+\epsilon\frac{\partial\psi}{\partial t}=
\frac{1}{A}\INT
\\
&\quad{}\exp
\left(
\frac{im}{2\hbar\epsilon}\boldsymbol\eta^2
+\frac{ie}{\hbar c}\boldsymbol\eta\cdot\mathbf A\left(\mathbf x,t\right)
\right)
\left(1-\frac{ie\epsilon}{\hbar}\phi\left(\mathbf x,t\right)\right)
\\
&\quad\quad{}\times
\left(
\psi(\mathbf x,t)+\boldsymbol\eta\cdot\nabla\psi
+\tfrac{1}{2}\boldsymbol\eta\cdot\nabla(\boldsymbol\eta\cdot\nabla\psi)
\right)
\,d\eta_1\,d\eta_2\,d\eta_3
\tag{6}
\end{align*}

Let $a_k$ be the exponential argument in component notation.
\begin{equation*}
a_k=\frac{im}{2\hbar\epsilon}\eta_k^2+\frac{ie}{\hbar c}\eta_kA_k(\mathbf x,t)
\end{equation*}

Expand the right-hand side of (6) using $a=a_1+a_2+a_3$.
\begin{align*}
&\psi(\mathbf{x},t)+\epsilon\frac{\partial\psi}{\partial t}
=\frac{\psi(\mathbf x,t)}{A}\int_{\mathbb R^3}
\exp(a)\,d\eta_1\,d\eta_2\,d\eta_3
\tag{7}
\\
&\qquad{}+\frac{1}{A}\int_{\mathbb R^3}
(\nabla\psi\cdot\boldsymbol\eta)
\exp(a)\,d\eta_1\,d\eta_2\,d\eta_3
\tag{8}
\\
&\qquad{}+\frac{1}{2A}\int_{\mathbb R^3}
(\nabla(\nabla\psi\cdot\boldsymbol\eta)\cdot\boldsymbol\eta)
\exp(a)\,d\eta_1\,d\eta_2\,d\eta_3
\tag{9}
\\
&\qquad{}-\frac{ie\epsilon}{A\hbar}
\phi\left(\mathbf x,t\right)\psi(\mathbf x,t)\int_{\mathbb R^3}
\exp(a)\,d\eta_1\,d\eta_2\,d\eta_3
\tag{10}
\\
&\qquad{}-\frac{ie\epsilon}{A\hbar}
\phi\left(\mathbf x,t\right)\int_{\mathbb R^3}
(\nabla\psi\cdot\boldsymbol\eta)
\exp(a)\,d\eta_1\,d\eta_2\,d\eta_3
\tag{11}
\\
&\qquad{}-\frac{ie\epsilon}{2A\hbar}
\phi\left(\mathbf x,t\right)\int_{\mathbb R^3}
(\nabla(\nabla\psi\cdot\boldsymbol\eta)\cdot\boldsymbol\eta)
\exp(a)\,d\eta_1\,d\eta_2\,d\eta_3
\tag{12}
\end{align*}

To solve the above integrals, we will use the following formulas obtained from online resources.
\begin{align*}
&\int_{-\infty}^\infty\exp(a_k)\,d\eta_k
\\
&\qquad{}=-\left(\frac{2\pi i\hbar\epsilon}{m}\right)^{1/2}
\exp\left(-\frac{ie^2\epsilon A_k(\mathbf x,t)^2}{2m\hbar c^2}\right)
\tag{13}
\\
&\int_{-\infty}^\infty\eta_k\exp(a_k)\,d\eta_k
\\
&\qquad{}=-\frac{e\epsilon A_k(\mathbf x,t)}{mc}
\left(\frac{2\pi i\hbar}{m}\right)^{1/2}
\exp\left(-\frac{ie^2\epsilon A_k(\mathbf x,t)^2}{2m\hbar c^2}\right)
\tag{14}
\\
&\int_{-\infty}^\infty\eta_k^2\exp(a_k)\,d\eta_k
\\
&\qquad{}=\left(\frac{e^2\epsilon^2 A_k(\mathbf x,t)^2}{m^2c^2}+\frac{i\hbar\epsilon}{m}\right)
\left(\frac{2\pi i\hbar\epsilon}{m}\right)^{1/2}
\exp\left(-\frac{ie^2\epsilon A_k(\mathbf x,t)^2}{2m\hbar c^2}\right)
\tag{15}
\end{align*}

By equation (13)
\begin{equation*}
\int_{\mathbb R^3}
\exp(a)\,d\eta_1\,d\eta_2\,d\eta_3
=-\left(\frac{2\pi i\hbar\epsilon}{m}\right)^{3/2}
\exp\left(-\frac{ie^2\epsilon\mathbf A^2}{2m\hbar c^2}\right)
\tag{16}
\end{equation*}
where
\begin{equation*}
\mathbf A^2=\mathbf A\cdot\mathbf A=A_1(\mathbf x,t)^2+A_2(\mathbf x,t)^2+A_3(\mathbf x,t)^2
\end{equation*}

Rewrite the integral in (8) and (11) in component notation.
\begin{multline*}
\int_{\mathbb R^3}
(\nabla\psi\cdot\boldsymbol\eta)
\exp(a)\,d\eta_1\,d\eta_2\,d\eta_3
=\int_{\mathbb R^3}
\frac{\partial\psi}{\partial x_1}
\eta_1\exp(a)
\,d\eta_1\,d\eta_2\,d\eta_3
\\
+\int_{\mathbb R^3}
\frac{\partial\psi}{\partial x_2}
\eta_2\exp(a)\,d\eta_1\,d\eta_2\,d\eta_3
+\int_{\mathbb R^3}
\frac{\partial\psi}{\partial x_3}
\eta_3\exp(a)\,d\eta_1\,d\eta_2\,d\eta_3
\end{multline*}

Then by equations (13) and (14)
\begin{align*}
&\int_{\mathbb R^3}
\frac{\partial\psi}{\partial x_1}
\eta_1\exp(a)\,d\eta_1\,d\eta_2\,d\eta_3
\\
&\qquad{}=-\frac{e\epsilon A_1(\mathbf x,t)}{mc}
\left(\frac{2\pi i\hbar\epsilon}{m}\right)^{3/2}
\exp\left(-\frac{ie^2\epsilon\mathbf A^2}{2m\hbar c^2}\right)
\frac{\partial\psi}{\partial x_1}
\\
&\int_{\mathbb R^3}
\frac{\partial\psi}{\partial x_2}
\eta_2\exp(a)\,d\eta_1\,d\eta_2\,d\eta_3
\\
&\qquad{}=-\frac{e\epsilon A_2(\mathbf x,t)}{mc}
\left(\frac{2\pi i\hbar\epsilon}{m}\right)^{3/2}
\exp\left(-\frac{ie^2\epsilon\mathbf A^2}{2m\hbar c^2}\right)
\frac{\partial\psi}{\partial x_2}
\\
&\int_{\mathbb R^3}
\frac{\partial\psi}{\partial x_3}
\eta_3\exp(a)\,d\eta_1\,d\eta_2\,d\eta_3
\\
&\qquad{}=-\frac{e\epsilon A_3(\mathbf x,t)}{mc}
\left(\frac{2\pi i\hbar\epsilon}{m}\right)^{3/2}
\exp\left(-\frac{ie^2\epsilon\mathbf A^2}{2m\hbar c^2}\right)
\frac{\partial\psi}{\partial x_3}
\end{align*}

Hence
\begin{multline*}
\int_{\mathbb R^3}
(\nabla\psi\cdot\boldsymbol\eta)
\exp(a)\,d\eta_1\,d\eta_2\,d\eta_3
\\
{}=
-\frac{e\epsilon}{mc}
\left(\frac{2\pi i\hbar\epsilon}{m}\right)^{3/2}
\exp\left(-\frac{ie^2\epsilon\mathbf A^2}{2m\hbar c^2}\right)
\nabla\psi\cdot\mathbf A(\mathbf x,t)
\tag{17}
\end{multline*}

Rewrite the integral in (9) and (12) in component notation.
\begin{align*}
&\int_{\mathbb R^3}
(\nabla(\nabla\psi\cdot\boldsymbol\eta)\cdot\boldsymbol\eta)
\exp(a)\,d\eta_1\,d\eta_2\,d\eta_3
\\
&\qquad{}=\int_{\mathbb R^3}
\eta_1^2\frac{\partial^2\psi}{\partial x_1^2}
\exp(a)\,d\eta_1\,d\eta_2\,d\eta_3
\\
&\qquad\quad{}+\int_{\mathbb R^3}
\eta_2^2\frac{\partial^2\psi}{\partial x_2^2}
\exp(a)\,d\eta_1\,d\eta_2\,d\eta_3
\\
&\qquad\quad{}+\int_{\mathbb R^3}
\eta_3^2\frac{\partial^2\psi}{\partial x_3^2}
\exp(a)\,d\eta_1\,d\eta_2\,d\eta_3
\\
&\qquad\quad{}+\int_{\mathbb R^3}
2\eta_1\eta_2\frac{\partial^2\psi}{\partial x_1\partial x_2}
\exp(a)\,d\eta_1\,d\eta_2\,d\eta_3
\\
&\qquad\quad{}+\int_{\mathbb R^3}
2\eta_1\eta_3\frac{\partial^2\psi}{\partial x_1\partial x_3}
\exp(a)\,d\eta_1\,d\eta_2\,d\eta_3
\\
&\qquad\quad{}+\int_{\mathbb R^3}
2\eta_2\eta_3\frac{\partial^2\psi}{\partial x_2\partial x_3}
\exp(a)\,d\eta_1\,d\eta_2\,d\eta_3
\end{align*}

By equations (13) and (15)
\begin{align*}
&\int_{\mathbb R^3}
\eta_1^2\frac{\partial^2\psi}{\partial x_1^2}
\exp(a)\,d\eta_1\,d\eta_2\,d\eta_3
\\
&\qquad{}=\left(\frac{e^2\epsilon^2 A_1(\mathbf x,t)^2}{m^2c^2}+\frac{i\hbar\epsilon}{m}\right)
\left(\frac{2\pi i\hbar\epsilon}{m}\right)^{3/2}
\exp\left(-\frac{ie^2\epsilon\mathbf A^2}{2m\hbar c^2}\right)
\frac{\partial^2\psi}{\partial x_1^2}
\\
&\int_{\mathbb R^3}
\eta_2^2\frac{\partial^2\psi}{\partial x_2^2}
\exp(a)\,d\eta_1\,d\eta_2\,d\eta_3
\\
&\qquad{}=\left(\frac{e^2\epsilon^2 A_2(\mathbf x,t)^2}{m^2c^2}+\frac{i\hbar\epsilon}{m}\right)
\left(\frac{2\pi i\hbar\epsilon}{m}\right)^{3/2}
\exp\left(-\frac{ie^2\epsilon\mathbf A^2}{2m\hbar c^2}\right)
\frac{\partial^2\psi}{\partial x_2^2}
\\
&\int_{\mathbb R^3}
\eta_3^2\frac{\partial^2\psi}{\partial x_3^2}
\exp(a)\,d\eta_1\,d\eta_2\,d\eta_3
\\
&\qquad{}=\left(\frac{e^2\epsilon^2 A_3(\mathbf x,t)^2}{m^2c^2}+\frac{i\hbar\epsilon}{m}\right)
\left(\frac{2\pi i\hbar\epsilon}{m}\right)^{3/2}
\exp\left(-\frac{ie^2\epsilon\mathbf A^2}{2m\hbar c^2}\right)
\frac{\partial^2\psi}{\partial x_3^2}
\end{align*}

By equations (13) and (14)
\begin{align*}
&\int_{\mathbb R^3}
2\eta_1\eta_2\frac{\partial^2\psi}{\partial x_1\partial x_2}
\exp(a)\,d\eta_1\,d\eta_2\,d\eta_3
\\
&\qquad{}=
\left(\frac{e^2\epsilon^2 A_1(\mathbf x,t)^2}{m^2c^2}+\frac{i\hbar\epsilon}{m}\right)
\left(\frac{e^2\epsilon^2 A_2(\mathbf x,t)^2}{m^2c^2}+\frac{i\hbar\epsilon}{m}\right)
\\
&\qquad\qquad{}\times
\left(\frac{2\pi i\hbar\epsilon}{m}\right)^{3/2}
\exp\left(-\frac{ie^2\epsilon\mathbf A^2}{2m\hbar c^2}\right)
\frac{\partial^2\psi}{\partial x_1 x_2}
\\
&\int_{\mathbb R^3}
2\eta_1\eta_2\frac{\partial^2\psi}{\partial x_1\partial x_3}
\exp(a)\,d\eta_1\,d\eta_2\,d\eta_3
\\
&\qquad{}=
\left(\frac{e^2\epsilon^2 A_1(\mathbf x,t)^2}{m^2c^2}+\frac{i\hbar\epsilon}{m}\right)
\left(\frac{e^2\epsilon^2 A_3(\mathbf x,t)^2}{m^2c^2}+\frac{i\hbar\epsilon}{m}\right)
\\
&\qquad\qquad{}\times
\left(\frac{2\pi i\hbar\epsilon}{m}\right)^{3/2}
\exp\left(-\frac{ie^2\epsilon\mathbf A^2}{2m\hbar c^2}\right)
\frac{\partial^2\psi}{\partial x_1 x_3}
\\
&\int_{\mathbb R^3}
2\eta_1\eta_2\frac{\partial^2\psi}{\partial x_1\partial x_2}
\exp(a)\,d\eta_1\,d\eta_2\,d\eta_3
\\
&\quad{}=
\left(\frac{e^2\epsilon^2 A_2(\mathbf x,t)^2}{m^2c^2}+\frac{i\hbar\epsilon}{m}\right)
\left(\frac{e^2\epsilon^2 A_3(\mathbf x,t)^2}{m^2c^2}+\frac{i\hbar\epsilon}{m}\right)
\left(\frac{2\pi i\hbar\epsilon}{m}\right)^{3/2}
\\
&\qquad{}\times
\exp\left(-\frac{ie^2\epsilon\mathbf A^2}{2m\hbar c^2}\right)
\frac{\partial^2\psi}{\partial x_2 x_3}
\end{align*}

Discard terms involving powers of $\epsilon$ to obtain
\begin{multline*}
\int_{\mathbb R^3}
(\nabla(\nabla\psi\cdot\boldsymbol\eta)\cdot\boldsymbol\eta)
\exp(a)\,d\eta_1\,d\eta_2\,d\eta_3
\\
{}=\frac{i\hbar\epsilon}{m}\left(\frac{2\pi i\hbar\epsilon}{m}\right)^{3/2}
\exp\left(-\frac{ie^2\epsilon\mathbf A^2}{2m\hbar c^2}\right)
\nabla^2\psi(\mathbf x,t)
\tag{18}
\end{multline*}

Substitute the solved integrals into (6) to obtain
\begin{align*}
&\psi(\mathbf{x},t)+\epsilon\frac{\partial\psi}{\partial t}
\\
&\quad{}=\frac{1}{A}\left(\frac{2\pi i\hbar\epsilon}{m}\right)^{3/2}
\exp\left(-\frac{ie^2\epsilon\mathbf A^2}{2m\hbar c^2}\right)
\psi(\mathbf x,t)
\tag*{\text{from (7) and (16)}}
\\
&\qquad-\frac{e\epsilon}{Amc}
\left(\frac{2\pi i\hbar\epsilon}{m}\right)^{3/2}
\exp\left(-\frac{ie^2\epsilon\mathbf A^2}{2m\hbar c^2}\right)
\nabla\psi\cdot\mathbf A(\mathbf x,t)
\tag*{\text{from (8) and (17)}}
\\
&\qquad{}+\frac{i\hbar\epsilon}{2Am}\left(\frac{2\pi i\hbar\epsilon}{m}\right)^{3/2}
\exp\left(-\frac{ie^2\epsilon\mathbf A^2}{2m\hbar c^2}\right)
\nabla^2\psi(\mathbf x,t)
\tag*{\text{from (9) and (18)}}
\\
&\qquad{}-\phi(\mathbf x,t)
\frac{ie\epsilon}{A\hbar}
\left(\frac{2\pi i\hbar\epsilon}{m}\right)^{3/2}
\exp\left(-\frac{ie^2\epsilon\mathbf A^2}{2m\hbar c^2}\right)
\psi(\mathbf x,t)
\tag*{\text{from (10) and (16)}}
\\
&\qquad{}+\phi(\mathbf x,t)
\frac{ie^2\epsilon^2}{Am\hbar c}
\left(\frac{2\pi i\hbar\epsilon}{m}\right)^{3/2}
\exp\left(-\frac{ie^2\epsilon\mathbf A^2}{2m\hbar c^2}\right)
\nabla\psi\cdot\mathbf A(\mathbf x,t)
\tag*{\text{from (11) and (17)}}
\\
&\qquad{}-\phi(\mathbf x,t)
\frac{ie\epsilon^2}{2Am}\left(\frac{2\pi i\hbar\epsilon}{m}\right)^{3/2}
\exp\left(-\frac{ie^2\epsilon\mathbf A^2}{2m\hbar c^2}\right)
\nabla^2\psi(\mathbf x,t)
\tag*{\text{from (12) and (18)}}
\end{align*}

\end{document}
