\documentclass[12pt]{article}
\usepackage{amsmath}
\usepackage{amssymb}

\parindent=0pt

\newcommand\INT{\int_{\mathbb R^3}}

\begin{document}

\begin{center}
Feynman and Hibbs problem 4-2
\end{center}

For a particle of charge $e$ in a magnetic field the Lagrangian is
\begin{equation*}
L(\dot{\mathbf x},\mathbf x)=\frac{m}{2}\dot{\mathbf x}^2
+\frac{e}{c}\dot{\mathbf x}\cdot\mathbf A(\mathbf x,t)-e\phi(\mathbf x,t)
\end{equation*}
where $\dot{\mathbf x}$ is the velocity vector,
$c$ is the velocity of light, and $\mathbf A$ and $\phi$
are the vector and scalar potentials.
Show that the corresponding Schrodinger equation is
\begin{equation*}
\frac{\partial\psi}{\partial t}
=-\frac{i}{\hbar}\left(
\frac{1}{2m}\left(\frac{\hbar}{i}\nabla-\frac{e}{c}\mathbf A\right)
\cdot
\left(\frac{\hbar}{i}\nabla-\frac{e}{c}\mathbf A\right)\psi
+e\phi\psi
\right)
\end{equation*}

From equation (4.3)
\begin{equation*}
\psi(\mathbf x,t+\epsilon)=\frac{1}{A}\INT\exp\left(
\frac{i\epsilon}{\hbar}L\left(\frac{\mathbf x-\mathbf y}{\epsilon},\frac{\mathbf x+\mathbf y}{2}\right)
\right)\psi(\mathbf y,t)
\,dy_1\,dy_2\,dy_3
\tag{1}
\end{equation*}

This is the Lagrangian with arguments from (1).
\begin{multline*}
L\left(\frac{\mathbf x-\mathbf y}{\epsilon},\frac{\mathbf x+\mathbf y}{2}\right)
\\
=\frac{m}{2\epsilon^2}(\mathbf x-\mathbf y)^2
+\frac{e}{c\epsilon}(\mathbf x-\mathbf y)\cdot\mathbf A\left(\frac{\mathbf x+\mathbf y}{2},t\right)
-e\phi\left(\frac{\mathbf x+\mathbf y}{2},t\right)
\end{multline*}

Hence
\begin{align*}
&\psi(\mathbf x,t+\epsilon)=\frac{1}{A}\INT
\\
&\quad\exp
\left(
\frac{im}{2\hbar\epsilon}(\mathbf x-\mathbf y)^2
+\frac{ie}{\hbar c}(\mathbf x-\mathbf y)\cdot\mathbf A\left(\frac{\mathbf x+\mathbf y}{2},t\right)
-\frac{ie\epsilon}{\hbar}\phi\left(\frac{\mathbf x+\mathbf y}{2},t\right)
\right)
\\
&\quad\quad{}\times\psi(\mathbf y,t)
\,dy_1\,dy_2\,dy_3
\end{align*}

Let
\begin{equation*}
\mathbf y=\mathbf x+\boldsymbol\eta
\end{equation*}

Then
\begin{equation*}
\mathbf x-\mathbf y=\boldsymbol\eta,\quad
\frac{\mathbf x+\mathbf y}{2}=\mathbf{x}+\tfrac{1}{2}\boldsymbol\eta,\quad
dy_1\,dy_2\,dy_3=d\eta_1\,d\eta_2\,d\eta_3
\end{equation*}

Hence
\begin{multline*}
\psi(\mathbf x,t+\epsilon)=
\frac{1}{A}\INT\exp
\left(
\frac{im}{2\hbar\epsilon}\boldsymbol\eta^2
+\frac{ie}{\hbar c}\boldsymbol\eta\cdot\mathbf A\left(\mathbf x+\tfrac{1}{2}\boldsymbol\eta,t\right)
-\frac{ie\epsilon}{\hbar}\phi\left(\mathbf x+\tfrac{1}{2}\boldsymbol\eta,t\right)
\right)
\\
{}\times\psi(\mathbf x+\boldsymbol\eta,t)
\,d\eta_1\,d\eta_2\,d\eta_3
\end{multline*}

Factor the exponential.
\begin{align*}
&\psi(\mathbf x,t+\epsilon)=
\frac{1}{A}\INT
\\
&\quad{}\exp\left(\frac{im}{2\hbar\epsilon}\boldsymbol\eta^2\right)
\exp\left(\frac{ie}{\hbar c}\boldsymbol\eta\cdot\mathbf A\left(\mathbf x+\tfrac{1}{2}\boldsymbol\eta,t\right)\right)
\exp\left(-\frac{ie\epsilon}{\hbar}\phi\left(\mathbf x+\tfrac{1}{2}\boldsymbol\eta,t\right)\right)
\\
&\quad\quad{}\times\psi(\mathbf x+\boldsymbol\eta,t)
\,d\eta_1\,d\eta_2\,d\eta_3
\tag{2}
\end{align*}

From the identity $\exp(i\theta)=\cos(\theta)+i\sin(\theta)$ we have
\begin{multline*}
\exp\left(-\frac{ie\epsilon}{\hbar}\phi\left(\mathbf x+\tfrac{1}{2}\boldsymbol\eta,t\right)\right)
\\
=\cos\left(-\frac{e\epsilon}{\hbar}\phi\left(\mathbf x+\tfrac{1}{2}\boldsymbol\eta,t\right)\right)
+i\sin\left(-\frac{e\epsilon}{\hbar}\phi\left(\mathbf x+\tfrac{1}{2}\boldsymbol\eta,t\right)\right)
\end{multline*}

Then for small $\epsilon$
\begin{equation*}
\exp\left(-\frac{ie\epsilon}{\hbar}\phi\left(\mathbf x+\tfrac{1}{2}\boldsymbol\eta,t\right)\right)
\approx
1-\frac{ie\epsilon}{\hbar}\phi\left(\mathbf x+\tfrac{1}{2}\boldsymbol\eta,t\right)
\end{equation*}

The authors write that the $\boldsymbol\eta$ term can be dropped
``because the error is of higher order than $\epsilon$.''
Hence
\begin{equation*}
\exp\left(-\frac{ie\epsilon}{\hbar}\phi\left(\mathbf x+\tfrac{1}{2}\boldsymbol\eta,t\right)\right)\approx
1-\frac{ie\epsilon}{\hbar}\phi\left(\mathbf x,t\right)
\tag{3}
\end{equation*}

Substitute (3) into (2).
\begin{multline*}
\psi(\mathbf x,t+\epsilon)=
\frac{1}{A}\left(1-\frac{ie\epsilon}{\hbar}\phi\left(\mathbf x,t\right)\right)\INT
\\
\exp\left(\frac{im}{2\hbar\epsilon}\boldsymbol\eta^2\right)
\exp\left(\frac{ie}{\hbar c}\boldsymbol\eta\cdot\mathbf A\left(\mathbf x+\tfrac{1}{2}\boldsymbol\eta,t\right)\right)
\psi(\mathbf x+\boldsymbol\eta,t)
\,d\eta_1\,d\eta_2\,d\eta_3
\end{multline*}

%We need to get rid of $\boldsymbol\eta$ in the vector potential.
%The authors write that contributions to the integral
%diminish as $\boldsymbol\eta$ increases.
%Therefore we will use the following approximation.
%\begin{equation*}
%\int_{Gaussian}
%\exp\left(\frac{ie}{\hbar c}\boldsymbol\eta\cdot\mathbf A\left(\mathbf x+\tfrac{1}{2}\boldsymbol\eta,t\right)\right)
%\approx
%\int_{Gaussian}
%\exp\left(-\frac{ie}{\hbar c}\boldsymbol\eta\cdot\mathbf A(\mathbf x,t)\right)
%\end{equation*}

Let
\begin{equation*}
T=-\frac{ie}{\hbar c}\boldsymbol\eta\cdot\mathbf A(\mathbf x,t)
\end{equation*}

Then
\begin{equation*}
\exp\left(\frac{ie}{\hbar c}\boldsymbol\eta\cdot\mathbf A\left(\mathbf x+\tfrac{1}{2}\boldsymbol\eta,t\right)\right)
\approx
\left(1+T+\tfrac{1}{2}T^2\right)
\end{equation*}

Hence
\begin{multline*}
\psi(\mathbf x,t+\epsilon)=
\frac{1}{A}
\left(1-\frac{ie\epsilon}{\hbar}\phi\left(\mathbf x,t\right)\right)
\\
{}\times\INT\exp\left(\frac{im}{2\hbar\epsilon}\boldsymbol\eta^2\right)
\left(1+T+\tfrac{1}{2}T^2\right)
\psi(\mathbf x+\boldsymbol\eta,t)
\,d\eta_1\,d\eta_2\,d\eta_3
\tag{4}
\end{multline*}

Next we will use the following Taylor series approximations.
\begin{equation*}
\begin{aligned}
\psi(\mathbf x,t+\epsilon)&\approx\psi(\mathbf x,t)+\epsilon\frac{\partial\psi}{\partial t}
\\
\psi(\mathbf x+\boldsymbol\eta,t)&\approx\psi(\mathbf x,t)+\boldsymbol\eta\cdot\nabla\psi
+\tfrac{1}{2}\boldsymbol\eta\cdot\nabla(\boldsymbol\eta\cdot\nabla\psi)
\end{aligned}
\tag{5}
\end{equation*}

Note: In component notation
\begin{equation*}
\boldsymbol\eta\cdot\nabla\psi=
\eta_1\frac{\partial\psi}{\partial x_1}+
\eta_2\frac{\partial\psi}{\partial x_2}+
\eta_2\frac{\partial\psi}{\partial x_2}
\end{equation*}
and
\begin{multline*}
\boldsymbol\eta\cdot\nabla(\boldsymbol\eta\cdot\nabla\psi)=
\eta_1^2\frac{\partial^2\psi}{\partial x_1^2}
+\eta_2^2\frac{\partial^2\psi}{\partial x_2^2}
+\eta_3^2\frac{\partial^2\psi}{\partial x_3^2}
\\
{}+2\eta_1\eta_2\frac{\partial^2\psi}{\partial x_1\partial x_2}
+2\eta_1\eta_3\frac{\partial^2\psi}{\partial x_1\partial x_3}
+2\eta_2\eta_3\frac{\partial^2\psi}{\partial x_2\partial x_3}
\end{multline*}

Substitute the approximations (5) into (4).
\begin{multline*}
\psi(\mathbf x,t)+\epsilon\frac{\partial\psi}{\partial t}=
\frac{1}{A}
\left(1-\frac{ie\epsilon}{\hbar}\phi\left(\mathbf x,t\right)\right)
\INT\exp\left(\frac{im}{2\hbar\epsilon}\boldsymbol\eta^2\right)
\\
{}\times
\left(1+T+\tfrac{1}{2}T^2\right)
\left(\psi(\mathbf x,t)+\boldsymbol\eta\cdot\nabla\psi
+\tfrac{1}{2}\boldsymbol\eta\cdot\nabla(\boldsymbol\eta\cdot\nabla\psi)\right)
\,d\eta_1\,d\eta_2\,d\eta_3
\tag{6}
\end{multline*}

To solve the above integral, we will use the following formulas provided by the authors.
\begin{align*}
I_k&=\int_{-\infty}^\infty\exp\left(\frac{im\eta_k^2}{2\hbar\epsilon}\right)\,d\eta_k
=\left(\frac{2\pi i\hbar\epsilon}{m}\right)^{1/2}
\tag*{\text{FH (4.7)}}
\\
J_k&=\int_{-\infty}^\infty\eta_k\exp\left(\frac{im\eta_k^2}{2\hbar\epsilon}\right)\,d\eta_k
=0
\tag*{\text{FH (4.9)}}
\\
K_k&=\int_{-\infty}^\infty\eta_k^2\exp\left(\frac{im\eta_k^2}{2\hbar\epsilon}\right)\,d\eta_k
=\frac{i\hbar\epsilon}{m}\left(\frac{2\pi i\hbar\epsilon}{m}\right)^{1/2}
\tag*{\text{FH (4.10)}}
\end{align*}

Hence
\begin{align*}
&
\INT\exp\left(\frac{im}{2\hbar\epsilon}\boldsymbol\eta^2\right)\psi(\mathbf x,t)
\,d\eta_1\,d\eta_2\,d\eta_3
\\
&{}=I_1 I_2 I_3 \psi(\mathbf x,t)
\\
&{}=\left(\frac{2\pi i\hbar\epsilon}{m}\right)^{3/2}\psi(\mathbf x,t)
\tag{7}
\end{align*}

\begin{align*}
&
\INT\exp\left(\frac{im}{2\hbar\epsilon}\boldsymbol\eta^2\right)
\boldsymbol\eta\cdot\nabla\psi(\mathbf x,t)
\,d\eta_1\,d\eta_2\,d\eta_3
\\
&{}
=J_1 I_2 I_3 \frac{\partial\psi}{\partial x_1}
+I_1 J_2 I_3 \frac{\partial\psi}{\partial x_2}
+I_1 I_2 J_3 \frac{\partial\psi}{\partial x_3}
\\
&{}
=0
\tag{8}
\end{align*}

\begin{align*}
&\INT\exp\left(\frac{im}{2\hbar\epsilon}\boldsymbol\eta^2\right)
\tfrac{1}{2}\boldsymbol\eta\cdot\nabla(\boldsymbol\eta\cdot\nabla\psi)
\,d\eta_1\,d\eta_2\,d\eta_3
\\
&{}
=\tfrac{1}{2}K_1 I_2 I_3 \frac{\partial^2\psi}{\partial x_1^2}
+\tfrac{1}{2}I_1 K_2 I_3 \frac{\partial^2\psi}{\partial x_2^2}
+\tfrac{1}{2}I_1 I_2 K_3 \frac{\partial^2\psi}{\partial x_3^2}
\\
&{}
+J_1 J_2 I_1 \frac{\partial^2\psi}{\partial x_1 x_2}
+J_1 I_2 J_3 \frac{\partial^2\psi}{\partial x_1 x_3}
+I_1 J_2 J_3 \frac{\partial^2\psi}{\partial x_2 x_3}
\\
&{}=\frac{i\hbar\epsilon}{2m}\left(\frac{2\pi i\hbar\epsilon}{m}\right)^{3/2}\nabla^2\psi
\tag{9}
\end{align*}

\begin{align*}
&
\INT
\exp\left(\frac{im}{2\hbar\epsilon}\boldsymbol\eta^2\right)
T % \left(\frac{ie}{\hbar c}\boldsymbol\eta\cdot\mathbf A(\mathbf x,t)\right)
\psi(\mathbf x,t)
\,d\eta_1\,d\eta_2\,d\eta_3
\\
&{}
=J_1 I_2 I_3\frac{-ie}{\hbar c}A_1 \psi
+I_1 J_2 I_3\frac{-ie}{\hbar c}A_2 \psi
+I_1 I_2 J_3\frac{-ie}{\hbar c}A_3 \psi
\\
&{}
=0
\tag{10}
\end{align*}

\begin{align*}
&
\INT\exp\left(\frac{im}{2\hbar\epsilon}\boldsymbol\eta^2\right)
T
\boldsymbol\eta\cdot\nabla\psi
\,d\eta_1\,d\eta_2\,d\eta_3
\\
&{}
=\INT\exp\left(\frac{im}{2\hbar\epsilon}\boldsymbol\eta^2\right)
\\
&{}\times
\frac{-ie}{\hbar c}
(\eta_1A_1+\eta_2A_2+\eta_3A_3)
\left(
 \eta_1\frac{\partial\psi}{\partial x_1}
+\eta_2\frac{\partial\psi}{\partial x_2}
+\eta_3\frac{\partial\psi}{\partial x_3}
\right)
\,d\eta_1\,d\eta_2\,d\eta_3
\\
&{}
=K_1 I_2 I_3 \frac{-ieA_1}{\hbar c} \frac{\partial\psi}{\partial x_1}
+J_1 J_2 I_3 \frac{-ieA_1}{\hbar c} \frac{\partial\psi}{\partial x_2}
+J_1 I_2 J_3 \frac{-ieA_1}{\hbar c} \frac{\partial\psi}{\partial x_3}
\\
&{}
+J_1 J_2 I_3 \frac{-ieA_2}{\hbar c} \frac{\partial\psi}{\partial x_1}
+I_1 K_2 I_3 \frac{-ieA_2}{\hbar c} \frac{\partial\psi}{\partial x_2}
+I_1 J_2 J_3 \frac{-ieA_2}{\hbar c} \frac{\partial\psi}{\partial x_3}
\\
&{}
+J_1 I_2 J_3 \frac{-ieA_3}{\hbar c} \frac{\partial\psi}{\partial x_1}
+I_1 J_2 J_3 \frac{-ieA_3}{\hbar c} \frac{\partial\psi}{\partial x_2}
+I_1 I_2 K_3 \frac{-ieA_3}{\hbar c} \frac{\partial\psi}{\partial x_3}
\\
&{}
=\frac{i\hbar\epsilon}{m}\left(\frac{2\pi i\hbar\epsilon}{m}\right)^{3/2}\frac{-ie}{\hbar c}
\left(
 A_1\frac{\partial\psi}{\partial x_1}
+A_2\frac{\partial\psi}{\partial x_2}
+A_3\frac{\partial\psi}{\partial x_3}
\right)
\\
&{}
=\frac{i\hbar\epsilon}{m}\left(\frac{2\pi i\hbar\epsilon}{m}\right)^{3/2}
\frac{-ie}{\hbar c}\mathbf A\nabla\psi
\tag{11}
\end{align*}

\begin{align*}
&
\INT
\exp\left(\frac{im}{2\hbar\epsilon}\boldsymbol\eta^2\right)
T
\tfrac{1}{2}\boldsymbol\eta\cdot\nabla(\boldsymbol\eta\cdot\nabla\psi)
\,d\eta_1\,d\eta_2\,d\eta_3
\\
&{}=\INT
\exp\left(\frac{im}{2\hbar\epsilon}\boldsymbol\eta^2\right)
\frac{-ie}{\hbar c}(\eta_1A_1+\eta_2A_2+\eta_3A_3)\times{}
\\
&{}\frac{1}{2}\left(
 \eta_1^2\frac{\partial^2\psi}{\partial x_1^2}
+\eta_2^2\frac{\partial^2\psi}{\partial x_2^2}
+\eta_3^2\frac{\partial^2\psi}{\partial x_3^2}
+2\eta_1\eta_2\frac{\partial^2\psi}{\partial x_1x_2}
+2\eta_1\eta_3\frac{\partial^2\psi}{\partial x_1x_3}
+2\eta_2\eta_3\frac{\partial^2\psi}{\partial x_2x_3}
\right)
\\
&{}
=0
\tag{12}
\end{align*}

\begin{align*}
&
\INT
\exp\left(\frac{im}{2\hbar\epsilon}\boldsymbol\eta^2\right)
\tfrac{1}{2} T^2 % \left(\frac{ie}{\hbar c}\boldsymbol\eta\cdot\mathbf A(\mathbf x,t)\right)^2
\psi(\mathbf x,t)
\,d\eta_1\,d\eta_2\,d\eta_3
\\
&{}
=\INT
\exp\left(\frac{im}{2\hbar\epsilon}\boldsymbol\eta^2\right)
\left(-\frac{e^2}{2\hbar^2 c^2}\right)
(\eta_1A_1+\eta_2A_2+\eta_3A_3)^2
\psi(\mathbf x,t)
\,d\eta_1\,d\eta_2\,d\eta_3
\\
&{}=\frac{i\hbar\epsilon}{m}
\left(\frac{2\pi i\hbar\epsilon}{m}\right)^{3/2}
\left(-\frac{e^2}{2\hbar^2 c^2}\right)
(A_1^2+A_2^2+A_3^2)\psi(\mathbf x,t)
\\
&{}=\frac{i\hbar\epsilon}{m}
\left(\frac{2\pi i\hbar\epsilon}{m}\right)^{3/2}
\left(-\frac{e^2}{2\hbar^2 c^2}\right)
\mathbf A^2\psi
\tag{13}
\end{align*}

\begin{align*}
&
\INT\exp\left(\frac{im}{2\hbar\epsilon}\boldsymbol\eta^2\right)
\tfrac{1}{2}T^2
\boldsymbol\eta\cdot\nabla\psi
\,d\eta_1\,d\eta_2\,d\eta_3
\\&
{}=\INT\exp\left(\frac{im}{2\hbar\epsilon}\boldsymbol\eta^2\right)
\left(-\frac{e^2}{2\hbar^2 c^2}\right)
\\
&{}\times
(\eta_1A_1+\eta_2A_2+\eta_3A_3)^2
\left(
 \eta_1\frac{\partial\psi}{\partial x_1}
+\eta_2\frac{\partial\psi}{\partial x_2}
+\eta_3\frac{\partial\psi}{\partial x_3}
\right)
\,d\eta_1\,d\eta_2\,d\eta_3
\\
&{}=0
\tag{14}
\end{align*}

\begin{align*}
&
\INT
\exp\left(\frac{im}{2\hbar\epsilon}\boldsymbol\eta^2\right)
\tfrac{1}{2}T^2
\tfrac{1}{2}\boldsymbol\eta\cdot\nabla(\boldsymbol\eta\cdot\nabla\psi)
\,d\eta_1\,d\eta_2\,d\eta_3
\\
&{}=\INT
\exp\left(\frac{im}{2\hbar\epsilon}\boldsymbol\eta^2\right)
\left(-\frac{e^2}{2\hbar^2c^2}\right)
(\eta_1A_1+\eta_2A_2+\eta_3A_3)^2\times{}
\\
&{}\frac{1}{2}
\left(
 \eta_1^2\frac{\partial^2\psi}{\partial x_1^2}
+\eta_2^2\frac{\partial^2\psi}{\partial x_2^2}
+\eta_3^2\frac{\partial^2\psi}{\partial x_3^2}
+2\eta_1\eta_2\frac{\partial^2\psi}{\partial x_1x_2}
+2\eta_1\eta_3\frac{\partial^2\psi}{\partial x_1x_3}
+2\eta_2\eta_3\frac{\partial^2\psi}{\partial x_2x_3}
\right)
\\
&{}=\frac{e^2\epsilon^2}{2m^2c^2}
\left(\frac{2\pi i\hbar\epsilon}{m}\right)^{3/2}
\\
&{}\times
\left(
\frac{1}{2}\mathbf A^2\nabla^2\psi+\mathbf A\nabla\nabla\psi\mathbf A
-\frac{3}{2}
\left(
 A_1^2\frac{\partial^2}{\partial x_1^2}
+A_2^2\frac{\partial^2}{\partial x_2^2}
+A_3^2\frac{\partial^2}{\partial x_3^2}
\right)
\right)
\tag{15}
\end{align*}

Substitute the solved integrals into (6) to obtain
\begin{equation*}
\psi(\mathbf x,t)+\epsilon\frac{\partial\psi}{\partial t}
=\frac{1}{A}
\left(1-\frac{ie\epsilon}{\hbar}\phi(\mathbf x,t)\right) I
\end{equation*}
where
\begin{align*}
I&=\left(\frac{2\pi i\hbar\epsilon}{m}\right)^{3/2}\psi(\mathbf x,t)
\tag*{\text{from (7)}}
\\
&\quad{}+\frac{i\hbar\epsilon}{2m}\left(\frac{2\pi i\hbar\epsilon}{m}\right)^{3/2}\nabla^2\psi
\tag*{\text{from (9)}}
\\
&\quad{}+\frac{i\hbar\epsilon}{m}\left(\frac{2\pi i\hbar\epsilon}{m}\right)^{3/2}
\frac{-ie}{\hbar c}\mathbf A\nabla\psi
\tag*{\text{from (11)}}
\\
&\quad{}+\frac{i\hbar\epsilon}{m}
\left(\frac{2\pi i\hbar\epsilon}{m}\right)^{3/2}
\left(-\frac{e^2}{2\hbar^2 c^2}\right)
\mathbf A^2\psi
\tag*{\text{from (13)}}
\\
&\quad{}+0
\tag*{\text{from (15)}}
\end{align*}

The result from (15) is discarded because it is proportional to $\epsilon^2$.

\bigskip
Simplify $I$ as follows.
\begin{equation*}
I=\left(\frac{2\pi i\hbar\epsilon}{m}\right)^{3/2}
\left(
\psi(\mathbf x,t)
+\frac{i\hbar\epsilon}{2m}\nabla^2\psi
+\frac{e\epsilon}{mc}\mathbf A\nabla\psi
-\frac{ie^2\epsilon}{2m\hbar c^2}\mathbf A^2\psi
\right)
\tag{16}
\end{equation*}

\bigskip
In the limit as $\epsilon\rightarrow0$ we have
\begin{equation*}
\psi(\mathbf x,t)=\frac{1}{A}\left(\frac{2\pi i\hbar\epsilon}{m}\right)^{3/2}\psi(\mathbf x,t)
\end{equation*}
hence
\begin{equation*}
A=\left(\frac{2\pi i\hbar\epsilon}{m}\right)^{3/2}
\end{equation*}

Cancel $A$ with the coefficient in $I$ to obtain
\begin{multline*}
\psi(\mathbf x,t)+\epsilon\frac{\partial\psi}{\partial t}
=\left(1-\frac{ie\epsilon}{\hbar}\phi(\mathbf x,t)\right)
\\
{}\times
\left(
\psi(\mathbf x,t)
+\frac{i\hbar\epsilon}{2m}\nabla^2\psi
+\frac{e\epsilon}{mc}\mathbf A\nabla\psi
-\frac{ie^2\epsilon}{2m\hbar c^2}\mathbf A^2\psi
\right)
\end{multline*}

Expand the product and discard terms of order $\epsilon^2$.
\begin{multline*}
\psi(\mathbf x,t)+\epsilon\frac{\partial\psi}{\partial t}
\\
{}=
\psi(\mathbf x,t)
+\frac{i\hbar\epsilon}{2m}\nabla^2\psi
+\frac{e\epsilon}{mc}\mathbf A\nabla\psi
-\frac{ie^2\epsilon}{2m\hbar c^2}\mathbf A^2\psi
-\frac{ie\epsilon}{\hbar}\phi(\mathbf x,t)\psi
\end{multline*}

Cancel leading terms $\psi(\mathbf x,t)$ and divide through by $\epsilon$.
\begin{equation*}
\frac{\partial\psi}{\partial t}
\\
{}=
\frac{i\hbar}{2m}\nabla^2\psi
+\frac{e}{mc}\mathbf A\nabla\psi
-\frac{ie^2}{2m\hbar c^2}\mathbf A^2\psi
-\frac{ie}{\hbar}\phi(\mathbf x,t)\psi
\tag{17}
\end{equation*}

\end{document}
