\documentclass[12pt]{article}
\usepackage{amsmath}
\usepackage{amssymb}

\parindent=0pt

\newcommand\INT{\int_{\mathbb R^3}}

\begin{document}

Feynman and Hibbs problem 4-2

\bigskip
For a particle of charge $e$ in a magnetic field the Lagrangian is
\begin{equation*}
L(\dot{\mathbf x},\mathbf x)=\frac{m}{2}\dot{\mathbf x}^2
+\frac{e}{c}\dot{\mathbf x}\cdot\mathbf A(\mathbf x,t)-e\phi(\mathbf x,t)
\end{equation*}
where $\dot{\mathbf x}$ is the velocity vector,
$c$ is the velocity of light, and $\mathbf A$ and $\phi$
are the vector and scalar potentials.
Show that the corresponding Schrodinger equation is
\begin{equation*}
\frac{\partial\psi}{\partial t}
=-\frac{i}{\hbar}\left(
\frac{1}{2m}\left(\frac{\hbar}{i}\nabla-\frac{e}{c}\mathbf A\right)
\cdot
\left(\frac{\hbar}{i}\nabla-\frac{e}{c}\mathbf A\right)\psi
+e\phi\psi
\right)
\tag{1}
\end{equation*}

From equation (4.3)
\begin{equation*}
\psi(\mathbf x,t+\epsilon)=\frac{1}{A}\INT\exp\left(
\frac{i\epsilon}{\hbar}L\left(\frac{\mathbf x-\mathbf y}{\epsilon},\frac{\mathbf x+\mathbf y}{2}\right)
\right)\psi(\mathbf y,t)
\,dy_1\,dy_2\,dy_3
\end{equation*}

By substitution
\begin{multline*}
L\left(\frac{\mathbf x-\mathbf y}{\epsilon},\frac{\mathbf x+\mathbf y}{2}\right)
\\
=\frac{m}{2\epsilon^2}(\mathbf x-\mathbf y)^2
+\frac{e}{c\epsilon}(\mathbf x-\mathbf y)\cdot\mathbf A\left(\frac{\mathbf x+\mathbf y}{2},t\right)
-e\phi\left(\frac{\mathbf x+\mathbf y}{2},t\right)
\end{multline*}

Hence
\begin{align*}
&\psi(\mathbf x,t+\epsilon)=\frac{1}{A}\INT
\\
&\quad\exp
\left(
\frac{im}{2\hbar\epsilon}(\mathbf x-\mathbf y)^2
+\frac{ie}{\hbar c}(\mathbf x-\mathbf y)\cdot\mathbf A\left(\frac{\mathbf x+\mathbf y}{2},t\right)
-\frac{ie\epsilon}{\hbar}\phi\left(\frac{\mathbf x+\mathbf y}{2},t\right)
\right)
\\
&\quad\quad{}\times\psi(\mathbf y,t)
\,dy_1\,dy_2\,dy_3
\end{align*}

Let
\begin{equation*}
\mathbf y=\mathbf x+\boldsymbol\eta
\end{equation*}

Then
\begin{equation*}
\mathbf x-\mathbf y=\boldsymbol\eta,\quad
\frac{\mathbf x+\mathbf y}{2}=\mathbf{x}+\tfrac{1}{2}\boldsymbol\eta,\quad
dy_1\,dy_2\,dy_3=d\eta_1\,d\eta_2\,d\eta_3
\end{equation*}

Hence
\begin{multline*}
\psi(\mathbf x,t+\epsilon)=
\frac{1}{A}\INT\exp
\left(
\frac{im}{2\hbar\epsilon}\boldsymbol\eta^2
+\frac{ie}{\hbar c}\boldsymbol\eta\cdot\mathbf A\left(\mathbf x+\tfrac{1}{2}\boldsymbol\eta,t\right)
-\frac{ie\epsilon}{\hbar}\phi\left(\mathbf x+\tfrac{1}{2}\boldsymbol\eta,t\right)
\right)
\\
{}\times\psi(\mathbf x+\boldsymbol\eta,t)
\,d\eta_1\,d\eta_2\,d\eta_3
\end{multline*}






\end{document}
