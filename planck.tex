\documentclass[12pt]{article}
\usepackage{amsmath}
\parindent=0pt
\begin{document}

Max Planck used two experimental results to calculate $h$ and $k$ in his 1901 paper
``On the Law of Distribution of Energy in the Normal Spectrum.''
Although the quantum of action $h$ is well known as Planck's constant,
the use of $k$ for Boltzmann's constant is also due to Planck.
In addition, Planck was the first to compute a numerical value for $k$.

\bigskip
One of the experimental results Planck used was the difference $S_{100}-S_0$
determined by Ferdinand Kurlbaum in 1898 where
$S_t$ is the power radiated by a black body at $t$ degrees Celsius.
\begin{equation*}
S_{100}-S_0=7.31\times10^5\,\text{erg}\,\text{cm}^{-2}\,\text{s}^{-1}
\end{equation*}

From the radiant power formula $S_t=(t+273)^4\sigma$ we have
\begin{equation*}
S_{100}-S_0=(100+273)^4\sigma-(0+273)^4\sigma=(373^4-273^4)\sigma
\end{equation*}

Hence the Stefan-Boltzmann constant $\sigma$ can be determined from $S_{100}-S_0$.
\begin{equation*}
\sigma=\frac{S_{100}-S_0}{373^4-273^4}%\,\text{cm}^{-2}\,\text{s}^{-1}\text{K}^{-4}
\end{equation*}

The Stefan-Boltzmann law is the relation between energy density and temperature $\theta$.
\begin{equation*}
\text{``energy per unit volume''}=\frac{4\sigma \theta^4}{c}
\end{equation*}
The use of $\theta$ for temperature looks strange but that is what scientists used at the time.

\bigskip
Using the Stefan-Boltzmann law and Kurlbaum's measurement,
Planck calculated energy density for temperature $\theta=1$.
\begin{equation*}
\frac{4}{c}\times
\frac{S_{100}-S_0}{373^4-273^4}
=\frac{4}{3\times10^{10}}\times\frac{7.31\times10^5}{373^4-273^4}
=7.061\times10^{-15}\,\text{erg}\,\text{cm}^{-3}
\end{equation*}

Planck's 1901 paper has the following formula (Equation 12) for energy distribution
$u$ as a function of frequency $\nu$ and temperature $\theta$.
\begin{equation*}
u=\frac{8\pi h\nu^3}{c^3}\,\frac{1}{e^{h\nu/k\theta}-1}
\end{equation*}

The integral of $u$ over all frequencies yields the total energy density $u^*$.
\begin{equation*}
u^{*}=\int_0^\infty u\,d\nu
=\frac{8\pi h}{c^3}\int_0^\infty\frac{\nu^3}{e^{h\nu/k\theta}-1}\,d\nu
\end{equation*}

Planck used a series expansion to solve the integral for $\theta=1$.
However, we will use the following identity.
\begin{equation*}
\int_0^\infty\frac{x^3}{e^x-1}\,dx=\frac{\pi^4}{15}
\end{equation*}

By the change of variable $x=h\nu/k$ we have
\begin{equation*}
u^*=\frac{8\pi h}{c^3}\left(\frac{k}{h}\right)^4\frac{\pi^4}{15}
\end{equation*}

Planck then set $u^*$ equal to the result from the Stefan-Bolztmann law.
\begin{equation*}
\frac{8\pi h}{c^3}\left(\frac{k}{h}\right)^4\frac{\pi^4}{15}=7.061\times10^{-15}
\end{equation*}

Hence
\begin{equation*}
\frac{k^4}{h^3}=7.061\times10^{-15}\times\frac{15c^3}{8\pi^5}=1.1682\times10^{15}
\end{equation*}

The second experimental result Planck used was $\lambda_m\theta=0.294$ obtained in 1900 by
Otto Lummer and Ernst Pringsheim.
Symbol $\lambda_m$ is the wavelength in centimeters
of peak radiant energy for a black body at temperature $\theta$ in Kelvin.

\bigskip
Planck's 1901 paper has the following formula (Equation~13)
for energy distribution $E$ as a function of wavelength $\lambda$ and temperature $\theta$.
\begin{equation*}
E=\frac{8\pi ch}{\lambda^5}\,\frac{1}{e^{ch/k\lambda\theta}-1}
\end{equation*}

Planck solves $dE/d\lambda=0$ to obtain $\lambda_m$ which we will now do step by step.
First, compute $dE/d\lambda$.
\begin{equation*}
\frac{dE}{d\lambda}
=\frac{8\pi c^2h^2}{k\lambda^7\theta}\,\frac{e^{ch/k\lambda\theta}}{(e^{ch/k\lambda\theta}-1)^2}
-\frac{40\pi ch}{\lambda^6}\,\frac{1}{e^{ch/k\lambda\theta}-1}
\end{equation*}

Set $dE/d\lambda=0$ to obtain
\begin{equation*}
\frac{8\pi c^2h^2}{k\lambda^7\theta}\,\frac{e^{ch/k\lambda\theta}}{(e^{ch/k\lambda\theta}-1)^2}
=\frac{40\pi ch}{\lambda^6}\,\frac{1}{e^{ch/k\lambda\theta}-1}
\end{equation*}

Then by cancellation of terms
\begin{equation*}
\frac{ch}{5k\lambda\theta}\,\frac{e^{ch/k\lambda\theta}}{e^{ch/k\lambda\theta}-1}=1
\end{equation*}

Multiply both sides by $e^{ch/k\lambda\theta}-1$.
\begin{equation*}
\frac{ch}{5k\lambda\theta}\, e^{ch/k\lambda\theta}
=e^{ch/k\lambda\theta}-1
\end{equation*}

Subtract $e^{ch/k\lambda\theta}$ from both sides.
\begin{equation*}
\left(\frac{ch}{5k\lambda\theta}-1\right)e^{ch/k\lambda\theta}=-1
\end{equation*}

Multiply both sides by $-1$ to obtain Planck's result.
\begin{equation*}
\left(1-\frac{ch}{5k\lambda\theta}\right)e^{ch/k\lambda\theta}=1
\end{equation*}

Planck then provides the following numerical solution.
\begin{equation*}
\frac{ch}{k\lambda\theta}=4.9651
\end{equation*}

Then using $c=3\times10^{10}$ and $\lambda\theta=0.294$ Planck calculates
\begin{equation*}
\frac{h}{k}=4.9651\times\frac{\lambda\theta}{c}=4.9651\times\frac{0.294}{3\times10^{10}}=4.866\times10^{-11}
\end{equation*}

Planck then solves for $k$.
Plug $h/k=4.866\times10^{-11}$ into the formula for $k^4/h^3$ to obtain
\begin{equation*}
k=1.1682\times10^{15}\times\frac{h^3}{k^3}
=1.1682\times10^{15}\times(4.866\times10^{-11})^3=1.346\times10^{-16}\,\text{erg}\,\text{K}^{-1}
\end{equation*}

Then calculate $h$ directly from $k$.
\begin{equation*}
h=k\times4.866\times10^{-11}=6.55\times10^{-27}\,\text{erg}\,\text{s}
\end{equation*}

\end{document}
