\documentclass[12pt]{article}
\usepackage{amsmath}
\usepackage{amssymb}

\parindent=0pt

\begin{document}

8-4.
Show that the ground-state wave function for the Lagrangian of
equation (8.78) can be written
\begin{equation*}
\Phi_0=A\exp\left(
-\frac{1}{2\hbar}
\sum_{\alpha=1}^{N-1}
\omega_\alpha Q_\alpha^*Q_\alpha
\right)
\tag{8.83}
\end{equation*}
(where $A$ is a constant) by starting with the wave function in terms of
the real variables $Q_\alpha^c$ and $Q_\alpha^s$.

\begin{equation*}
L=\frac{1}{2}\sum_{\alpha=0}^{N-1}
\left(\dot Q_\alpha^*\dot Q_\alpha-\omega_\alpha^2Q_\alpha^*Q_\alpha\right)
\tag{8.78}
\end{equation*}

Consider the following equation from p.~216.
\begin{equation*}
Q_\alpha=\frac{1}{\sqrt2}(Q_\alpha^c-iQ_\alpha^s)
\end{equation*}

It follows that
\begin{equation*}
Q_\alpha^*Q_\alpha
=\frac{1}{2}(Q_\alpha^c)^2+\frac{1}{2}(Q_\alpha^s)^2
\tag{1}
\end{equation*}

Substitute (1) into (8.78).
\begin{equation*}
L=\frac{1}{4}\sum_{\alpha=0}^{N-1}
\left(
(\dot Q_\alpha^c)^2
+(\dot Q_\alpha^s)^2
-\omega_\alpha^2(Q_\alpha^c)^2
-\omega_\alpha^2(Q_\alpha^s)^2
\right)
\tag{2}
\end{equation*}

Consider equation (2.7).
\begin{equation*}
\frac{d}{dt}\frac{\partial L}{\partial\dot Q}=\frac{\partial L}{\partial Q}
\tag{2.7}
\end{equation*}

Substitute (2) into (2.7) to obtain the following equations of motion.
\begin{equation*}
\ddot Q_\alpha^c(t)=-\omega_\alpha^2Q_\alpha^c(t)
\qquad
\ddot Q_\alpha^s(t)=-\omega_\alpha^2Q_\alpha^s(t)
\tag{3}
\end{equation*}

From equation (8.58) and the associated text on p.~210, the ground state eigenfunction related to (3) is
\begin{equation*}
\phi_0(x_\alpha)=\exp\left(-\frac{\omega_\alpha x_\alpha^2}{2\hbar}\right)
\end{equation*}

Then by equation (8.62)
\begin{equation*}
\Phi_0=\prod_{\alpha=0}^{N-1}\phi_0(Q_\alpha^c)\phi_0(Q_\alpha^s)
=\exp\left(-\frac{1}{2\hbar}\sum_{\alpha=0}^{N-1}\omega_\alpha(Q_\alpha^c+Q_\alpha^s)\right)
\end{equation*}

\end{document}

Consider equation (8.77).
\begin{equation*}
Q_\alpha(t)=\frac{1}{\sqrt N}\sum_{j=1}^Nq_j(t)\exp\left(-\frac{2\pi i\alpha j}{N}\right)
\tag{8.77}
\end{equation*}

Note that $Q_0$ is real hence $Q_0^s$, the imaginary part of $Q_0$, is zero.

\end{document}
