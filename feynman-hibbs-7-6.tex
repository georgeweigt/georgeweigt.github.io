\documentclass[12pt]{article}
\usepackage{amsmath}
\usepackage{amssymb}

\parindent=0pt

\newcommand\INT{\int_{\mathbb R^3}}

\begin{document}

7-6.
Show, for a particle moving in three-dimensional space $x$, $y$, $z$,
\begin{align*}
\langle(x_{k+1}-x_k)^2\rangle
=\langle(y_{k+1}-y_k)^2\rangle
&=\langle(z_{k+1}-z_k)^2\rangle
=-\frac{\hbar\epsilon}{im}\langle1\rangle
\tag{7.50}
\\
\langle(x_{k+1}-x_k)(y_{k+1}-y_k)\rangle
&=\langle(x_{k+1}-x_k)(z_{k+1}-z_k)\rangle
\\
&=\langle(y_{k+1}-y_k)(z_{k+1}-z_k)\rangle=0
\tag{7.51}
\end{align*}

To extend (7.33) to three dimensions, change derivative to gradient, hence
\begin{equation*}
\left\langle\nabla_kF\right\rangle=-\frac{i}{\hbar}\left\langle F\nabla_kS\right\rangle
\tag{1}
\end{equation*}
where
\begin{equation*}
\nabla_k
=\mathbf i\frac{\partial}{\partial x}
+\mathbf j\frac{\partial}{\partial y}
+\mathbf k\frac{\partial}{\partial z}
\end{equation*}
and
\begin{equation*}
\mathbf i=\begin{pmatrix}1\\0\\0\end{pmatrix},
\quad
\mathbf j=\begin{pmatrix}0\\1\\0\end{pmatrix},
\quad
\mathbf k=\begin{pmatrix}0\\0\\1\end{pmatrix}
\end{equation*}

Let
\begin{equation*}
F=x_k+y_k+z_k
\end{equation*}

Then
\begin{align*}
\nabla_kF
&=\frac{\partial F}{\partial x_k}\mathbf i
+\frac{\partial F}{\partial y_k}\mathbf j
+\frac{\partial F}{\partial z_k}\mathbf k
\\
&=\mathbf i+\mathbf j+\mathbf k
\end{align*}

Hence
\begin{equation*}
\langle\nabla_kF\rangle
=\langle1\rangle\mathbf i
+\langle1\rangle\mathbf j
+\langle1\rangle\mathbf k
\tag{2}
\end{equation*}

By equation (7.43)
\begin{align*}
\left\langle(x_k+y_k+z_k)\frac{\partial S}{\partial x_k}\right\rangle\mathbf i
&=\left\langle x_k\frac{\partial S}{\partial x_k}\right\rangle\mathbf i
\\
\left\langle(x_k+y_k+z_k)\frac{\partial S}{\partial y_k}\right\rangle\mathbf j
&=\left\langle y_k\frac{\partial S}{\partial y_k}\right\rangle\mathbf j
\\
\left\langle(x_k+y_k+z_k)\frac{\partial S}{\partial z_k}\right\rangle\mathbf k
&=\left\langle z_k\frac{\partial S}{\partial z_k}\right\rangle\mathbf k
\end{align*}

Hence
\begin{equation*}
\langle F\nabla_kS\rangle
=\left\langle x_k\frac{\partial S}{\partial x_k}\right\rangle\mathbf i
+\left\langle y_k\frac{\partial S}{\partial y_k}\right\rangle\mathbf j
+\left\langle z_k\frac{\partial S}{\partial z_k}\right\rangle\mathbf k
\tag{3}
\end{equation*}

By equations (1), (2), and (3)
\begin{align*}
\langle1\rangle\mathbf i
&=-\frac{i}{\hbar}\left\langle x_k\frac{\partial S}{\partial x_k}\right\rangle\mathbf i
\\
\langle1\rangle\mathbf j
&=-\frac{i}{\hbar}\left\langle y_k\frac{\partial S}{\partial y_k}\right\rangle\mathbf j
\\
\langle1\rangle\mathbf k
&=-\frac{i}{\hbar}\left\langle z_k\frac{\partial S}{\partial z_k}\right\rangle\mathbf k
\end{align*}

Then by the same arguments in the book for equation (7.49) we have
\begin{align*}
\left\langle(x_{k+1}-x_k)^2\right\rangle\mathbf i&=-\frac{\hbar\epsilon}{im}\langle1\rangle\mathbf i
\\
\left\langle(y_{k+1}-y_k)^2\right\rangle\mathbf j&=-\frac{\hbar\epsilon}{im}\langle1\rangle\mathbf j
\\
\left\langle(z_{k+1}-z_k)^2\right\rangle\mathbf k&=-\frac{\hbar\epsilon}{im}\langle1\rangle\mathbf k
\end{align*}

Hence (7.50) is shown to be true.

\bigskip
The action for a particle in three dimensions is
\begin{equation*}
S=\int_{t_a}^{t_b}\left(\frac{m}{2}\left(\dot x^2+\dot y^2+\dot z^2\right)-V(x,y,z)\right)\,dt
\end{equation*}

Hence
\begin{equation*}
\nabla_kS
=\frac{\partial S}{\partial x_k}\mathbf i
+\frac{\partial S}{\partial y_k}\mathbf j
+\frac{\partial S}{\partial z_k}\mathbf k
\end{equation*}
where
\begin{align*}
\frac{\partial S}{\partial x_k}
&=-m\left(\frac{x_{k+1}-x_k}{\epsilon}-\frac{x_k-x_{k-1}}{\epsilon}\right)
-\epsilon\frac{\partial V}{\partial x}\bigg|_{x_k,y_k,z_k}
\\
\frac{\partial S}{\partial y_k}
&=-m\left(\frac{y_{k+1}-y_k}{\epsilon}-\frac{y_k-y_{k-1}}{\epsilon}\right)
-\epsilon\frac{\partial V}{\partial y}\bigg|_{x_k,y_k,z_k}
\\
\frac{\partial S}{\partial z_k}
&=-m\left(\frac{z_{k+1}-z_k}{\epsilon}-\frac{z_k-z_{k-1}}{\epsilon}\right)
-\epsilon\frac{\partial V}{\partial z}\bigg|_{x_k,y_k,z_k}
\end{align*}

% FIXME show (7.51)

\end{document}
