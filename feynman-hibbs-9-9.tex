\documentclass[12pt]{article}
\usepackage{amsmath}
\usepackage{amssymb}

\parindent=0pt

\begin{document}

9-9.
For a complicated system moving nonrelativistically
\begin{equation*}
(j_{1,\mathbf k})_{NM}=\sum_i\left(
q_i\mathbf e_1\cdot\dot{\mathbf x}_i
\exp(-i\mathbf k\cdot\mathbf x_i)\right)_{NM}
\end{equation*}

where $\mathbf e_1$ is a unit vector in the direction of the
polarization of the light and $q_i$ and $\mathbf x_i$ are the
charge and position of the $i$th particle.
Assume the wavelength of the light is very large compared with the
size of the atom, i.e., that the absolute square of the wave function
describing the position of the $i$th electron falls to zero over a
distance small compared with $1/k$.
Show that we can then approximate $\exp(-i\mathbf k\cdot\mathbf x_i)$
by unity and write the matrix element as
\begin{equation*}
(j_{1,\mathbf k})_{NM}=i\omega\mathbf e_1\cdot\boldsymbol\mu_{NM}
\tag{9.57}
\end{equation*}

where
\begin{equation*}
\boldsymbol\mu_{NM}=\sum_i(q_i\mathbf x_i)_{NM}
\tag{9.58}
\end{equation*}

The function $\boldsymbol\mu_{NM}$ is called the
{\it matrix element of the electric dipole moment}
of the atom, and the approximation used to derive equation (9.57)
is called the {\it dipole approximation}.
Show that the probability to emit light in any direction per unit time is
\begin{equation*}
\frac{dP}{dt}=\frac{4\omega^3}{3\hbar c^3}\left|\boldsymbol\mu_{NM}\right|^2
\tag{9.59}
\end{equation*}

(Integrate equation (9.54) over all directions, remembering that
$\mathbf e_1$ is perpendicular to $\mathbf k$ and that there are two
possible directions of polarization.)
\begin{equation*}
\frac{dP}{dt}=\frac{\omega}{2\pi\hbar c^3}\left|j_{1,\mathbf q}\right|_{NM}^2
\,d\Omega
\tag{9.54}
\end{equation*}

\bigskip
\hrule

\bigskip
Adapted from problem 7-12,
\begin{equation*}
\dot{\mathbf x}_i=-\frac{i}{\hbar}(\mathbf x_i\hat H-\hat H\mathbf x_i)=i\omega\mathbf x_i
\end{equation*}

The squared magnitude of $i\omega$ in (9.57) is $\omega^2$.
It follows that
\begin{equation*}
\int\frac{\omega}{2\pi\hbar c^3}
\left(\left|j_{1,\mathbf k}\right|_{NM}^2+\left|j_{2,\mathbf k}\right|_{NM}^2\right)
\,d\Omega
=2\left|\boldsymbol\mu_{NM}\right|^2
\int_0^{2\pi}\int_0^\pi
\frac{\omega^3}{2\pi\hbar c^3}
\sin\theta
\,d\theta\,d\phi
\end{equation*}

From the following integrals
\begin{equation*}
\int_0^\pi\sin\theta\,d\theta=2
\qquad
\int_0^{2\pi}d\phi=2\pi
\end{equation*}

the combined multiplier is $4\pi$ hence
\begin{equation*}
\frac{dP}{dt}
=\frac{4\omega^3}{\hbar c^3}
\left|\boldsymbol\mu_{NM}\right|^2
\end{equation*}

\end{document}
