\documentclass[12pt]{article}
\usepackage[margin=2cm]{geometry}
\usepackage{amsmath}

\begin{document}

\noindent
Consider the following eigenstates of a hypothetical quantum system.\footnote{
Adapted from problem 16.1.1 of ``Quantum Mechanics for Scientists and Engineers.''\\
{\tt https://ee.stanford.edu/{\textasciitilde}dabm/QMbook.html}}
\begin{align*}
|00\rangle&=(1,0,0,0)\qquad\text{no fermions}\\
|10\rangle&=(0,1,0,0)\qquad\text{one fermion in state 1}\\
|01\rangle&=(0,0,1,0)\qquad\text{one fermion in state 2}\\
|11\rangle&=(0,0,0,1)\qquad\text{two fermions, one in state 1, one in state 2}
\end{align*}

\noindent
Creation and annihilation operators are formed from outer products of state vectors.
Sign changes make the operators antisymmetric.
\begin{align*}
\hat{b}_1^\dag&=|10\rangle\langle00|-|11\rangle\langle01| \qquad\text{Create one fermion in state 1}
\\
\hat{b}_1&=|00\rangle\langle10|-|01\rangle\langle11| \qquad\text{Annihilate one fermion in state 1}
\\
\hat{b}_2^\dag&=|01\rangle\langle00|+|11\rangle\langle10| \qquad\text{Create one fermion in state 2}
\\
\hat{b}_2&=|00\rangle\langle01|+|10\rangle\langle11| \qquad\text{Annihilate one fermion in state 2}
\end{align*}

\noindent
The operators in matrix form.
\begin{equation*}
\hat{b}_1^\dag=\begin{pmatrix}0&0&0&0\\1&0&0&0\\0&0&0&0\\0&0&-1&0\end{pmatrix}
\quad
\hat{b}_1=\begin{pmatrix}0&1&0&0\\0&0&0&0\\0&0&0&-1\\0&0&0&0\end{pmatrix}
\quad
\hat{b}_2^\dag=\begin{pmatrix}0&0&0&0\\0&0&0&0\\1&0&0&0\\0&1&0&0\end{pmatrix}
\quad
\hat{b}_2=\begin{pmatrix}0&0&1&0\\0&0&0&1\\0&0&0&0\\0&0&0&0\end{pmatrix}
\end{equation*}

\noindent
Verify anticommutation relations of the operators.
\begin{align*}
\hat{b}_j\hat{b}_k+\hat{b}_k\hat{b}_j&=0
\\[2ex]
\hat{b}_j^\dag\hat{b}_k^\dag+\hat{b}_k^\dag\hat{b}_j^\dag&=0
\\[2ex]
\hat{b}_j\hat{b}_k^\dag+\hat{b}_k^\dag\hat{b}_j&=\delta_{jk}
\end{align*}

\end{document}
