\documentclass[12pt]{article}
\usepackage[margin=2cm]{geometry}
\usepackage{amsmath}

\begin{document}

\noindent
Let $\psi(x,y)$ be the wave function of two electrons in a one dimensional box of length $L$.
\begin{align*}
\psi(x,y)&=\frac{1}{\sqrt{2}}
\big(\phi_1(x)\phi_2(y)-\phi_1(y)\phi_2(x)\big)
\\[2ex]
\phi_n(x)&=\sqrt{\frac{2}{L}}\sin\left(\frac{n\pi x}{L}\right)
\end{align*}

\noindent
Wave function $\psi(x,y)$ is antisymmetric with respect to interchange of electron coordinates.
\begin{equation*}
\psi(x,y)=-\psi(y,x)
\end{equation*}

\noindent
For length $L=10^{-9}$ meter the average potential energy $\langle V\rangle$ is
\begin{equation*}
\langle V\rangle=\frac{e^2}{4\pi\epsilon_0}\int_0^L\int_0^L\frac{\psi^*(x,y)\psi(x,y)}{|x-y|}\,dx\,dy
=4.67\,\text{eV}
\end{equation*}

\noindent
Next calculate the average potential energy $\langle V_s\rangle$ for a symmetric wave function.
\begin{equation*}
\langle V_s\rangle=\frac{e^2}{4\pi\epsilon_0}
\int_0^L\int_0^L\frac{\phi_1^*(x)\phi_2^*(y)\phi_1(x)\phi_2(y)}{|x-y|}\,dx\,dy
=12.80\,\text{eV}
\end{equation*}

\noindent
The difference is the exchange energy.
\begin{equation*}
\langle V_{ex}\rangle=\langle V\rangle-\langle V_s\rangle=-8.13\,\text{eV}
\end{equation*}

\noindent
Note that the formula for $\langle V_s\rangle$ has a singularity at $x=y$.
The computed value shown above is the result of an arbitrary cutoff in numerical integration.
The actual value of $\langle V_s\rangle$ goes to infinity.

\bigskip
\noindent
Note also that there is a singularity at $x=y$ in the formula for $\langle V\rangle$.
However, due to antisymmetry we have $\psi(x,x)=0$ and hence the integral converges.

\end{document}
