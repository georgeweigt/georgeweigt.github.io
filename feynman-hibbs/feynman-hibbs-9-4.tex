\documentclass[12pt]{article}
\usepackage{amsmath}
\usepackage{amssymb}

\parindent=0pt

\begin{document}

9-4.
In section 2-1 we discussed the mechanisms for obtaining
the mechanical equations of motion from the form of the action $S$ by
obtaining the extremum $S_{cl}$ under the conditions $\delta S=0$
for variations
of the coordinates, $\delta\mathbf x$. Show how Maxwell's equations can be
derived
from the action $S$ defined in equation (9.23) by requiring $\delta S$
for first-order variations of $\mathbf A$ and $\phi$.
\begin{equation*}
S=S_1+S_2+S_3
\tag{9.23}
\end{equation*}

\bigskip
\hrule

\bigskip
Since $S_1$ does not depend on $\mathbf A$ or $\phi$,
we only need $S_2$ and $S_3$.
\begin{align*}
S_2&=-\sum_i e_i\int\left(
\phi(\mathbf x_i(t),t)
-\frac{1}{c}\dot{\mathbf x}_i(t)\cdot\mathbf A(\mathbf x_i(t),t)
\right)
\,dt
\tag{9.25}
\\
S_3&=\frac{1}{8\pi}\int\int
\left(
\left|-\nabla\phi-\frac{1}{c}\frac{\partial\mathbf A}{\partial t}\right|^2
-\left|\nabla\times\mathbf A\right|^2
\right)
\,d^3\mathbf r\,dt
\tag{9.26}
\end{align*}

Consider equation (2.7), the classical Lagrangian equation of motion.
\begin{equation*}
\frac{d}{dt}\frac{\partial L}{\partial\dot x}=\frac{\partial L}{\partial x}
\tag{2.7}
\end{equation*}

Extend (2.7) to three dimensions.
\begin{equation*}
\frac{d}{dt}\dot\nabla L=\nabla L
\tag{1}
\end{equation*}
where
\begin{equation*}
\dot\nabla
=\mathbf i\frac{\partial}{\partial\dot x}
+\mathbf j\frac{\partial}{\partial\dot y}
+\mathbf k\frac{\partial}{\partial\dot z}
\qquad
\nabla
=\mathbf i\frac{\partial}{\partial x}
+\mathbf j\frac{\partial}{\partial y}
+\mathbf k\frac{\partial}{\partial z}
\end{equation*}
and
\begin{equation*}
\mathbf i=\begin{pmatrix}1\\0\\0\end{pmatrix}
\qquad
\mathbf j=\begin{pmatrix}0\\1\\0\end{pmatrix}
\qquad
\mathbf k=\begin{pmatrix}0\\0\\1\end{pmatrix}
\end{equation*}

From equation (9.25) for a single particle, let
\begin{equation*}
L=\phi-\frac{1}{c}(\dot xA_x+\dot yA_y+\dot zA_z)
\end{equation*}

Then
\begin{align*}
\frac{d}{dt}\dot\nabla L
&=-\frac{1}{c}\frac{d}{dt}(A_x\mathbf i+A_y\mathbf j+A_z\mathbf k)
\\[1ex]
&=-\frac{1}{c}\frac{d}{dt}\mathbf A
\tag{2}
\end{align*}
and
\begin{align*}
\nabla L
&=\nabla\phi-\frac{1}{c}
\left(
\dot x\frac{\partial A_x}{\partial x}\mathbf i
+\dot y\frac{\partial A_y}{\partial y}\mathbf j
+\dot z\frac{\partial A_z}{\partial z}\mathbf k
\right)
\\
&=\nabla\phi-\frac{1}{c}\nabla(\dot{\mathbf x}\cdot\mathbf A)
\tag{3}
\end{align*}

Hence by equations (1), (2), and (3)
\begin{equation*}
-\frac{1}{c}\frac{d}{dt}\mathbf A=\nabla\phi
-\frac{1}{c}\nabla(\dot{\mathbf x}\cdot\mathbf A)
\end{equation*}

% FIXME incomplete

\end{document}
