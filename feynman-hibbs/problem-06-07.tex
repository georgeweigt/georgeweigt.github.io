\input{preamble}

\FBOX{
6-7.
Suppose the potential energy $V(\mathbf r)=-e\phi(\mathbf r)$
is the result of a charge distribution $\rho(\mathbf r)$ so that
\begin{equation*}
\nabla^2\phi(\mathbf r)=-4\pi\rho(\mathbf r)
\tag{6.48}
\end{equation*}

By assuming that $\rho(\mathbf r)$ goes to 0 as $|\mathbf r|\rightarrow\infty$,
multiplying equation (6.48) by $e^{i(\breve{\mathbf p}/\hbar)\cdot\mathbf r}$,
and integrating twice over $\mathbf r$,
show that $v(\breve{\mathbf p})$ can be expressed in terms of $\rho(\mathbf r)$ as
\begin{equation*}
v(\breve{\mathbf p})=-\frac{4\pi\hbar^2e}{\breve p^2}
\int e^{i(\breve{\mathbf p}/\hbar)\cdot\mathbf r}\rho(\mathbf r)\,d^3\mathbf r
\tag{6.49}
\end{equation*}
}

\begin{equation*}
v(\breve{\mathbf p})=-4\pi e\int\int_0^{\mathbf R}
\exp\left(\frac{ipr\cos\theta}{\hbar}\right)\phi(\mathbf R)\,d\mathbf r\,d\mathbf R
\end{equation*}

\begin{equation*}
\int_0^\mathbf R
\exp\left(\frac{ipr\cos\theta}{\hbar}\right)r^2\sin\theta\,dr\,d\theta\,d\phi
\end{equation*}

\bigskip
In polar coordinates
\begin{equation*}
I=\int_0^{2\pi}\int_0^\pi\int_0^\infty
\nabla^2\phi(\mathbf r)
\exp\left(\frac{ipr\cos\theta}{\hbar}\right)
r^2\sin\theta\,dr\,d\theta\,d\phi
\end{equation*}

FIXME

\end{document}
