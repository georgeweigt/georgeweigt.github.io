\input{preamble}

\FBOX{
4-3.
Show that the complex conjugate function $\psi^*$,
defined as the function $\psi$ with every $i$ changed to $-i$,
satisfies
\begin{equation*}
\frac{\partial\psi^*}{\partial t}=+\frac{i}{\hbar}(H\psi)^*
\end{equation*}
}

Start with equation (4.14).
\begin{equation*}
\frac{\partial\psi}{\partial t}=-\frac{i}{\hbar}H\psi
\tag{4.14}
\end{equation*}

Conjugate both sides.
\begin{equation*}
\left(\frac{\partial\psi}{\partial t}\right)^*=+\frac{i}{\hbar}(H\psi)^*
\end{equation*}

It is well known that conjugation and differentiation commute, hence
\begin{equation*}
\left(\frac{\partial\psi}{\partial t}\right)^*=\frac{\partial\psi^*}{\partial t}=+\frac{i}{\hbar}(H\psi)^*
\end{equation*}

However, just for the fun of it, let us complete the proof without using the commutation rule.

\bigskip
Consider equation (2.22).
\begin{equation*}
K(b,a)=\lim_{\epsilon\rightarrow0}\frac{1}{A^N}\int\cdots\int
\exp\left(\frac{i}{\hbar}S(b,a)\right)\,dx_1\cdots dx_{N-1}
\tag{2.22}
\end{equation*}

Differentiate (2.22) with respect to $t_b$.
\begin{equation*}
\frac{\partial}{\partial t_b}K(b,a)
\\
=
\lim_{\epsilon\rightarrow0}\frac{1}{A^N}
\int\cdots\int
\frac{i}{\hbar}\frac{\partial}{\partial t_b}S(b,a)
\exp\left(\frac{i}{\hbar}S(b,a)\right)\,dx_1\cdots dx_{N-1}
\end{equation*}

Then conjugate.
\begin{multline*}
\left(\frac{\partial}{\partial t_b}K(b,a)\right)^*
\\
{}=\lim_{\epsilon\rightarrow0}\frac{1}{A^N}
\int\cdots\int
-\frac{i}{\hbar}\frac{\partial}{\partial t_b}S(b,a)
\exp\left(-\frac{i}{\hbar}S(b,a)\right)\,dx_1\cdots dx_{N-1}
\end{multline*}

Clearly the result is the same for conjugate first then differentiate, hence
\begin{equation*}
\left(\frac{\partial}{\partial t_b}K(b,a)\right)^*
=\frac{\partial}{\partial t_b}K^*(b,a)
\tag{1}
\end{equation*}

Now consider this form of equation (4.2) that has $x,t$ instead of $x_b,t_b$.
\begin{equation*}
\psi(x,t)=\int_{\infty}^\infty K(x,t,x_a,t_a)\psi(x_a,t_a)\,dx_a
\tag{2}
\end{equation*}

Differentiate (2) with respect to $t$ then conjugate.
Note that $\psi(x_a,t_a)$ is a constant with respect to $t$.
\begin{equation*}
\left(\frac{\partial}{\partial t}\psi(x,t)\right)^*
=\int_{\infty}^\infty\left(\frac{\partial}{\partial t}K(x,t,x_a,t_a)\right)^*\psi^*(x_a,t_a)\,dx_a
\end{equation*}

Now do the reverse, conjugate (2) then differentiate.
\begin{equation*}
\frac{\partial}{\partial t}\psi^*(x,t)
=\int_{\infty}^\infty\frac{\partial}{\partial t}K^*(x,t,x_a,t_a)\psi^*(x_a,t_a)\,dx_a
\end{equation*}

By equation (1) the integrals are equivalent, hence
\begin{equation*}
\left(\frac{\partial}{\partial t}\psi(x,t)\right)^*=\frac{\partial}{\partial t}\psi^*(x,t)
\end{equation*}

\end{document}
