\input{preamble}

\FBOX{
6-10.
Consider adiatomic molecule containing two atoms, $A$ and $B$,
arranged with their centers at the points given by the vectors $\mathbf a$ and $\mathbf b$.
Using the Born approximation, show that the amplitude for an electron to be scattered
from such a molecule is
\begin{equation*}
K^{(1)}=e^{i(\breve{\mathbf p}/\hbar)\cdot\mathbf a}f_A(\breve{\mathbf p})
+e^{i(\breve{\mathbf p}/\hbar)\cdot\mathbf b}f_B(\breve{\mathbf p})
\tag{6.57}
\end{equation*}

where $f_A$ and $f_B$ are the amplitudes for scattering by the two atoms
individually when each atom is located at the center of a coordinate
system. (Within the Born approximation, these $f$ values are real for
spherically symmetric potentials.) The atomic binding does not change
the charge distributions around the nuclei very much (except for very
light nuclei such as hydrogen) because the binding forces affect only a
few of the outermost electrons.

\bigskip
Using Eq.~(6.57), show that the probability of scattering at a
particular value of $\breve{\mathbf p}$ is proportional to
$f_A^2+f_B^2+2f_Af_B\cos(\breve{\mathbf p}\cdot\mathbf d/\hbar)$
where $\mathbf d$ is $\mathbf a-\mathbf b$.
}

\begin{align*}
f_A(\breve{\mathbf p})&=C_A\int
\exp\left(\frac{i\breve{\mathbf p}\cdot\mathbf r}{\hbar}\right)V_A(\mathbf r)\,d^3\mathbf r
\\
f_B(\breve{\mathbf p})&=C_B\int
\exp\left(\frac{i\breve{\mathbf p}\cdot\mathbf r}{\hbar}\right)V_B(\mathbf r)\,d^3\mathbf r
\end{align*}

This is the combined potential.
\begin{equation*}
V(\mathbf r)=V_A(\mathbf r-\mathbf a)+V_B(\mathbf r-\mathbf b)
\end{equation*}

Hence
\begin{equation*}
K^{(1)}=C_A\int\exp\left(\frac{i\breve{\mathbf p}\cdot\mathbf r}{\hbar}\right)
V_A(\mathbf r-\mathbf a)\,d^3\mathbf r
+C_B\int\exp\left(\frac{i\breve{\mathbf p}\cdot\mathbf r}{\hbar}\right)
V_B(\mathbf r-\mathbf b)\,d^3\mathbf r
\end{equation*}

Substitute $\mathbf r+\mathbf a$ for $\mathbf r$ in the first integral
and $\mathbf r+\mathbf b$ for $\mathbf r$ in the second integral.
\begin{equation*}
K^{(1)}
=C_A\int\exp\left(\frac{i\breve{\mathbf p}\cdot(\mathbf r+\mathbf a)}{\hbar}\right)
V_A(\mathbf r)\,d^3\mathbf r
+C_B\int\exp\left(\frac{i\breve{\mathbf p}\cdot(\mathbf r+\mathbf b)}{\hbar}\right)
V_B(\mathbf r)\,d^3\mathbf r
\end{equation*}

Factor the exponentials.
\begin{multline*}
K^{(1)}
=C_A\exp\left(\frac{i\breve{\mathbf p}\cdot\mathbf a}{\hbar}\right)
\int\exp\left(\frac{i\breve{\mathbf p}\cdot\mathbf r}{\hbar}\right)
V_A(\mathbf r)\,d^3\mathbf r
\\
{}+C_B\exp\left(\frac{i\breve{\mathbf p}\cdot\mathbf b}{\hbar}\right)
\int\exp\left(\frac{i\breve{\mathbf p}\cdot\mathbf r}{\hbar}\right)
V_B(\mathbf r)\,d^3\mathbf r
\end{multline*}

Hence
\begin{equation*}
K^{(1)}=\exp\left(\frac{i\breve{\mathbf p}\cdot\mathbf a}{\hbar}\right)f_A(\breve{\mathbf p})
+\exp\left(\frac{i\breve{\mathbf p}\cdot\mathbf b}{\hbar}\right)f_B(\breve{\mathbf p})
\end{equation*}

The probability of scattering is
\begin{equation*}
\left|K^{(1)}\right|^2=f_A^2+f_B^2
+f_Af_B\exp\left(\frac{i\breve{\mathbf p}\cdot(\mathbf a-\mathbf b)}{\hbar}\right)
+f_Af_B\exp\left(\frac{i\breve{\mathbf p}\cdot(\mathbf b-\mathbf a)}{\hbar}\right)
\end{equation*}

Substitute $\mathbf d$ for $\mathbf a-\mathbf b$.
\begin{equation*}
\left|K^{(1)}\right|^2=f_A^2+f_B^2
+f_Af_B\exp\left(\frac{i\breve{\mathbf p}\cdot\mathbf d}{\hbar}\right)
+f_Af_B\exp\left(-\frac{i\breve{\mathbf p}\cdot\mathbf d}{\hbar}\right)
\end{equation*}

Change complex exponentials to rectangular form.
\begin{multline*}
\left|K^{(1)}\right|^2=f_A^2+f_B^2
+f_Af_B\left[\cos\left(\frac{\breve{\mathbf p}\cdot\mathbf d}{\hbar}\right)
+i\sin\left(\frac{\breve{\mathbf p}\cdot\mathbf d}{\hbar}\right)\right]
\\
{}+f_Af_B\left[\cos\left(\frac{\breve{\mathbf p}\cdot\mathbf d}{\hbar}\right)
-i\sin\left(\frac{\breve{\mathbf p}\cdot\mathbf d}{\hbar}\right)\right]
\end{multline*}

The sine functions cancel.
\begin{equation*}
\left|K^{(1)}\right|^2=f_A^2+f_B^2
+2f_Af_B\cos\left(\frac{\breve{\mathbf p}\cdot\mathbf d}{\hbar}\right)
\end{equation*}

\end{document}
