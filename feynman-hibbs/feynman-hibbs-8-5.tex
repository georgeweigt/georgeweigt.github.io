\documentclass[12pt]{article}
\usepackage{amsmath}
\usepackage{amssymb}

\parindent=0pt

\newcommand\U{\vert\Phi_0\vert^2}

\begin{document}

8-5.
A transition element which employs the same wave
function as both the initial and final states is called an expectation value.
Thus the expectation value of $F$ for the ground state $\Phi_0$ of
equation (8.83) is
\begin{equation*}
\langle\Phi_0|F|\Phi_0\rangle
=\int\cdots\int
\Phi_0^*F\Phi_0\,dQ_1\,dQ_2\cdots dQ_{N-1}
\tag{8.84}
\end{equation*}

(The integral over complex variables is defined as equal to the
corresponding integral over real normal coordinates $Q_\alpha^c$
and $Q_\alpha^s$.)
Show that the following expectation values are correct (for $\alpha\ne0$).
\begin{equation*}
\begin{aligned}
\langle\Phi_0^*|Q_\alpha|\Phi_0\rangle
&=\langle\Phi_0^*|Q_\alpha^*|\Phi_0\rangle=0
\\
\langle\Phi_0^*|Q_\alpha^2|\Phi_0\rangle
&=\langle\Phi_0^*|Q_\alpha^{*2}|\Phi_0\rangle=0
\\
\langle\Phi_0^*|Q_\alpha^*Q_\alpha|\Phi_0\rangle
&=\frac{\hbar}{2\omega_\alpha}\langle\Phi_0^*|1|\Phi_0\rangle
\\
\langle\Phi_0^*|Q_\alpha^*Q_\alpha|\Phi_0\rangle&=0,\quad\alpha\ne\beta
\end{aligned}
\tag{8.85}
\end{equation*}

\bigskip
\hrule

\bigskip
From problem 8-4,
\begin{equation*}
\U
=\Phi_0^*\Phi_0
=\exp\left(
-\frac{1}{\hbar}
\sum_{\alpha=1}^{N-1}
\omega_\alpha\left((Q_\alpha^c)^2+(Q_\alpha^s)^2\right)
\right)
\end{equation*}

We will use the following integrals.
\begin{align*}
&\int_{-\infty}^\infty\exp(-ax^2+b)\,dx=\sqrt{\frac{\pi}{a}}\exp(b)
\tag{1}
\\
&\int_{-\infty}^\infty x\exp(-ax^2+b)\,dx=0
\tag{2}
\\
&\int_{-\infty}^\infty x^2\exp(-ax^2+b)\,dx=\frac{1}{2a}\sqrt{\frac{\pi}{a}}\exp(b)
\tag{3}
\end{align*}

Here are some specific examples with $a=\omega_1/\hbar$.
\begin{align*}
&\int_{-\infty}^\infty
\U\,dQ_1^c=\left(\frac{\pi\hbar}{\omega_1}\right)^{1/2}
\exp\left(\frac{\omega_1}{\hbar}(Q_1^c)^2\right)
\U
\tag{4}
\\
&\int_{-\infty}^\infty Q_1^c
\U\,dQ_1^c=0
\tag{5}
\\
&\int_{-\infty}^\infty (Q_1^c)^2
\U\,dQ_1^c
=\frac{\hbar}{2\omega_1}
\left(\frac{\pi\hbar}{\omega_1}\right)^{1/2}
\exp\left(\frac{\omega_1}{\hbar}(Q_1^c)^2\right)
\U
\tag{6}
\end{align*}

Multiplying $\U$ by an exponential cancels a factor, i.e., in (1) and (4),
\begin{equation*}
\exp(b)=\exp\left(\frac{\omega_1}{\hbar}(Q_1^c)^2\right)\U
\end{equation*}

Compute the expectation value for $Q_1$.
Let
\begin{equation*}
I=\int_{-\infty}^\infty\int_{-\infty}^\infty
\left(\frac{Q_1^c-iQ_1^s}{\sqrt2}\right)
\U\,dQ_1^c\,dQ_1^s
\end{equation*}

By integral (2), $I=0$ hence by equation (8.84)
\begin{equation*}
\langle\Phi_0^*|Q_1|\Phi_0\rangle=0
\end{equation*}

Compute the expectation value for $Q_1^*$.
Let
\begin{equation*}
I=\int_{-\infty}^\infty\int_{-\infty}^\infty
\left(\frac{Q_1^c+iQ_1^s}{\sqrt2}\right)
\U\,dQ_1^c\,dQ_1^s
\end{equation*}

As above, $I=0$ hence
\begin{equation*}
\langle\Phi_0^*|Q_1^*|\Phi_0\rangle=0
\end{equation*}

Compute the expectation value for $Q_1^2$.
Let
\begin{equation*}
I=\int_{-\infty}^\infty\int_{-\infty}^\infty
\left(\frac{Q_1^c-iQ_1^s}{\sqrt2}\right)^2
\U\,dQ_1^c\,dQ_1^s
\end{equation*}

Rewrite as
\begin{multline*}
I=-i\int_{-\infty}^\infty\int_{-\infty}^\infty
Q_1^cQ_1^s
\U\,dQ_1^c\,dQ_1^s
\\
{}+\frac{1}{2}
\int_{-\infty}^\infty\int_{-\infty}^\infty
(Q_1^c)^2
\U\,dQ_1^c\,dQ_1^s
-\frac{1}{2}
\int_{-\infty}^\infty\int_{-\infty}^\infty
(Q_1^s)^2
\U\,dQ_1^c\,dQ_1^s
\end{multline*}

The first integral vanishes by (2).
The remaining integrals cancel hence
\begin{equation*}
\langle\Phi_0^*|Q_1^2|\Phi_0\rangle=0
\end{equation*}

Compute the expectation value for $Q_1^{*2}$.
(As above except for a sign change.)
\begin{multline*}
I=i\int_{-\infty}^\infty\int_{-\infty}^\infty
Q_1^cQ_1^s
\U\,dQ_1^c\,dQ_1^s
\\
{}+\frac{1}{2}
\int_{-\infty}^\infty\int_{-\infty}^\infty
(Q_1^c)^2
\U\,dQ_1^c\,dQ_1^s
-\frac{1}{2}
\int_{-\infty}^\infty\int_{-\infty}^\infty
(Q_1^s)^2
\U\,dQ_1^c\,dQ_1^s
\end{multline*}

By the same arguments as $Q_1^2$
\begin{equation*}
\langle\Phi_0^*|Q_1^{*2}|\Phi_0\rangle=0
\end{equation*}

Compute the expectation value for $Q_1^*Q_1$.
Let
\begin{equation*}
I=\int_{-\infty}^\infty\int_{-\infty}^\infty
\left(\frac{(Q_1^c)^2+(Q_1^s)^2}{2}\right)
\U\,dQ_1^c\,dQ_1^s
\end{equation*}

Rewrite as
\begin{equation*}
I=\frac{1}{2}\int_{-\infty}^\infty\left(\int_{-\infty}^\infty (Q_1^c)^2
\U\,dQ_1^c\right)\,dQ_1^s
+\frac{1}{2}\int_{-\infty}^\infty\left(\int_{-\infty}^\infty (Q_1^s)^2
\U\,dQ_1^s\right)\,dQ_1^c
\end{equation*}

By integral (3) with $a=\omega_1/\hbar$,
\begin{multline*}
I=
\frac{\hbar}{4\omega_1}
\left(\frac{\pi\hbar}{\omega_1}\right)^{1/2}
\exp\left(\frac{\omega_1}{\hbar}(Q_1^c)^2\right)
\int_{-\infty}^\infty
\U\,dQ_1^s
\\
{}+\frac{\hbar}{4\omega_1}
\left(\frac{\pi\hbar}{\omega_1}\right)^{1/2}
\exp\left(\frac{\omega_1}{\hbar}(Q_1^s)^2\right)
\int_{-\infty}^\infty
\U\,dQ_1^c
\end{multline*}

By integral (1),
\begin{multline*}
I=
\frac{\hbar}{4\omega_1}
\frac{\pi\hbar}{\omega_1}
\exp\left(\frac{\omega_1}{\hbar}(Q_1^c)^2\right)
\exp\left(\frac{\omega_1}{\hbar}(Q_1^s)^2\right)
\U
\\
{}+
\frac{\hbar}{4\omega_1}
\frac{\pi\hbar}{\omega_1}
\exp\left(\frac{\omega_1}{\hbar}(Q_1^s)^2\right)
\exp\left(\frac{\omega_1}{\hbar}(Q_1^c)^2\right)
\U
\end{multline*}

Integrate over the remaining measure as in (8.84).
\begin{multline*}
\langle\Phi_0^*|Q_1^*Q_1|\Phi_0\rangle
=\int_{-\infty}^\infty\cdots\int_{-\infty}^\infty I\,dQ_2^c\,dQ_2^s\cdots dQ_{N-1}^c\,dQ_{N-1}^s
\\
=\frac{\hbar}{2\omega_1}
\frac{\pi\hbar}{\omega_1}
\prod_{k=2}^{N-1}\frac{\pi\hbar}{\omega_k}
\tag{7}
\end{multline*}

Note that the exponentials cancel with $u$.

\bigskip
Compute the normalization constant.
By integral (1),
\begin{equation*}
\langle\Phi_0^*|1|\Phi_0\rangle
=\prod_{k=1}^{N-1}\frac{\pi\hbar}{\omega_k}
\tag{8}
\end{equation*}

Hence by (7) and (8),
\begin{equation*}
\langle\Phi_0^*|Q_1^*Q_1|\Phi_0\rangle=\frac{\hbar}{2\omega_1}\langle\Phi_0^*|1|\Phi_0\rangle
\end{equation*}

Compute the expectation value for $Q_1^*Q_2$.
Let
\begin{equation*}
I=\int\limits_{-\infty}^\infty\cdots\int\limits_{-\infty}^\infty
\left(\frac{Q_1^cQ_2^c-Q_1^sQ_2^s-iQ_1^cQ_2^s-iQ_1^sQ_2^c}{2}\right)
\U\,dQ_1^c\,dQ_1^s\,dQ_2^c\,dQ_2^s
\end{equation*}

By integral (2) we have $I=0$, hence by equation (8.84)
\begin{equation*}
\langle\Phi_0^*|Q_1^*Q_2|\Phi_0\rangle=0
\end{equation*}

\end{document}
