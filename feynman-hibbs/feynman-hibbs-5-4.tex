\documentclass[12pt]{article}
\usepackage{amsmath}
\usepackage{amssymb}

\parindent=0pt

\newcommand\INT{\int_{\mathbb R^3}}

\begin{document}

5-4.
Suppose the wave function for a system is $\psi(x)$ at time $t_a$.
Suppose further that the behavior of the system described by
the kernel $K(x_b,t_b,x_a,t_a)$ for motions in the interval
$t_b\ge t\ge t_a$.
Show that the probability that the system is found to be in state
$\chi(x)$ at time $t_b$ is given by the square of the integral
\begin{equation*}
\int_{-\infty}^\infty\int_{-\infty}^\infty
\chi^*(x_b)
K(x_b,t_b,x_a,t_a)
\psi(x_a)
\,dx_a\,dx_b
\tag{1}
\end{equation*}

We call this integral the {\it transition amplitude} to go from
state $\psi(x)$ to state $\chi(x)$.

\bigskip
\hrule

\bigskip
From equation (3.42)
\begin{equation*}
\psi(x_b,t_b)=\int_{-\infty}^\infty
K(x_b,t_b,x_a,t_a)
\psi(x_a,t_a)
\,dx_a
\end{equation*}

Hence the integral over $x_a$ in (1) is equivalent to $\psi(x_b)$.
\begin{equation*}
\int_{-\infty}^\infty\int_{-\infty}^\infty
\chi^*(x_b)
K(x_b,t_b,x_a,t_a)
\psi(x_a)
\,dx_a\,dx_b
=\int_{-\infty}^\infty
\chi^*(x_b)
\psi(x_b)
\,dx_b
\end{equation*}

Then by equation (5.32)
\begin{equation*}
P(\mathrm X)=\left|
\int_{-\infty}^\infty
\chi^*(x_b)
\psi(x_b)
\,dx_b
\right|^2
\end{equation*}

where $\mathrm X$ is upper-case $\chi$.

\end{document}
