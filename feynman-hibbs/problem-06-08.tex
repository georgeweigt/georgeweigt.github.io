\input{preamble}

\FBOX{
6-8.
In an atom the potential follows the Coulomb law only for very small radii.
As the radius is increased the atomic electrons gradually shield,
or cancel out, the nuclear charge until, for sufficiently large values of
$r$, the potential is zero.
The shielding effect of atomic electrons can be accounted for in a very
rough approximate manner with the formula
\begin{equation*}
V(r)=-\frac{Ze^2}{r}e^{-r/a}
\tag{6.51}
\end{equation*}

In this expression $a$ is called the radius of the atom. It is not the same
as the outer radius of the atom as used by chemists, but instead is given
by $a_0/Z^{1/3}$, where the Bohr radius is
$a_0=\hbar^2/me^2=0.0529\,\text{nm}$.

\bigskip
Show that in such a potential
\begin{equation*}
v(\breve p)=-\frac{4\pi Ze^2}{(\breve p/\hbar)^2+(1/a)^2}
\tag{6.52}
\end{equation*}

and hence
\begin{equation*}
\frac{d\sigma}{d\Omega}=\frac{Z^2e^4}
{(mu^2/2)^2\left[4\sin^2(\theta/2)+(\hbar/pa)^2\right]^2}
\tag{6.53}
\end{equation*}

The total cross section $\sigma_T$ is defined as the integral of
$d\sigma/d\Omega$ over the unit sphere; thus
\begin{equation*}
\sigma_T=\int_0^{4\pi}\frac{d\sigma}{d\Omega}\,d\Omega
\tag{6.54}
\end{equation*}

In the present example show that
\begin{equation*}
\sigma_T=\pi a^2\frac{(2Ze^2/u\hbar)^2}{1+(\hbar/2pa)^2}
\tag{6.55}
\end{equation*}
}

\begin{equation*}
v(\breve p)=\int_0^{2\pi}\int_0^\pi\int_0^\infty
\exp\left(\frac{i\breve pr\cos\theta}{\hbar}\right)
V(r)\,r^2\sin\theta
\,dr\,d\theta\,d\phi
\end{equation*}

Substitute for $V(r)$.
\begin{equation*}
v(\breve p)=-Ze^2\int_0^{2\pi}\int_0^\pi\int_0^\infty
\exp\left(\frac{i\breve pr\cos\theta}{\hbar}\right)
\exp\left(-\frac{r}{a}\right)r\sin\theta
\,dr\,d\theta\,d\phi
\end{equation*}

Integrate over $\phi$.
\begin{equation*}
v(\breve p)=-2\pi Ze^2\int_0^\pi\int_0^\infty
\exp\left(\frac{i\breve pr\cos\theta}{\hbar}\right)
\exp\left(-\frac{r}{a}\right)r\sin\theta
\,dr\,d\theta
\end{equation*}

%%%%%
\iffalse
Convert the first exponential to rectangular form.
\begin{equation*}
v(\breve p)=-2\pi Ze^2\int_0^\pi\int_0^\infty
\left[\cos\left(\frac{\breve pr\cos\theta}{\hbar}\right)
+i\sin\left(\frac{\breve pr\cos\theta}{\hbar}\right)\right]
\exp\left(-\frac{r}{a}\right)r\sin\theta\,dr\,d\theta
\end{equation*}

By the definite integrals
\begin{equation*}
\int_0^\pi\cos(a\cos\theta)\sin\theta\,d\theta=\frac{2\sin a}{a},\quad
\int_0^\pi\sin(a\cos\theta)\sin\theta\,d\theta=0
\end{equation*}

we have for the integral over $\theta$ (note that $r$ in the integrand cancels)
\begin{equation*}
v(\breve p)=-\frac{4\pi Ze^2\hbar}{\breve p}
\int_0^\infty
\sin\left(\frac{\breve pr}{\hbar}\right)
\exp\left(-\frac{r}{a}\right)\,dr
\end{equation*}

By the definite integral
\begin{equation*}
\int_0^\infty\sin(ay)\exp(-by)\,dy=\frac{a}{a^2+b^2}
\end{equation*}

we have for the integral over $r$
\begin{equation*}
v(\breve p)=-\frac{4\pi Ze^2\hbar}{\breve p}\times
\frac{\breve p/\hbar}
{(\breve p/\hbar)^2+(1/a)^2}
\end{equation*}

Hence
\begin{equation*}
v(\breve p)=-\frac{4\pi Ze^2}
{(\breve p/\hbar)^2+(1/a)^2}
\end{equation*}
\fi
%%%%%

Transform the integral over $\theta$ to an integral over $y$
where $y=\cos\theta$, $dy=-\sin\theta\,d\theta$.
\begin{equation*}
v(\breve p)=-2\pi Ze^2
\int_{-1}^1
\int_0^\infty
\exp\left(\frac{ipry}{\hbar}\right)
\exp\left(-\frac{r}{a}\right)
r\,dr\,dy
\end{equation*}

Solve the integral over $y$ (note $r$ in the integrand cancels).
\begin{equation*}
v(\breve p)=-2\pi Ze^2
\int_0^\infty
\frac{\hbar}{ip}
\left[\exp\left(\frac{ipr}{\hbar}\right)-\exp\left(-\frac{ipr}{\hbar}\right)\right]
\exp\left(-\frac{r}{a}\right)
\,dr
\end{equation*}

Solve the integral over $r$.
\begin{equation*}
v(\breve p)=-2\pi Ze^2
\frac{\hbar}{ip}
\left[
\frac{1}{ip/\hbar-1/a}
\exp\left(\frac{ipr}{\hbar}-\frac{r}{a}\right)
+\frac{1}{ip/\hbar+1/a}
\exp\left(-\frac{ipr}{\hbar}-\frac{r}{a}\right)
\right]_0^\infty
\end{equation*}

Evaluate the limits.
\begin{equation*}
v(\breve p)=-2\pi Ze^2
\frac{\hbar}{ip}
\left[
-\frac{1}{ip/\hbar-1/a}
-\frac{1}{ip/\hbar+1/a}
\right]
=-\frac{4\pi Ze^2}{(p/\hbar)^2+(1/a)^2}
\end{equation*}

The cross section is
\begin{equation*}
\frac{d\sigma}{d\Omega}
=\left(\frac{m}{2\pi\hbar^2}\right)^2|v(\breve p)|^2
=\left(\frac{2mZe^2}{\breve p^2+(\hbar/a)^2}\right)^2
\end{equation*}

Substitute $2mu\sin(\theta/2)$ for $\breve p$.
\begin{equation*}
\frac{d\sigma}{d\Omega}
=\left(\frac{2mZe^2}{4m^2u^2\sin^2(\theta/2)+(\hbar/a)^2}\right)^2
\end{equation*}

Factor out $mu=p$ in the denominator.
\begin{equation*}
\frac{d\sigma}{d\Omega}
=\left(\frac{2mZe^2}{m^2u^2\left[4\sin^2(\theta/2)+(\hbar/pa)^2\right]}\right)^2
\end{equation*}

Hence
\begin{equation*}
\frac{d\sigma}{d\Omega}=\frac{Z^2e^4}
{(mu^2/2)^2\left[4\sin^2(\theta/2)+(\hbar/pa)^2\right]^2}
\end{equation*}

The total cross section is
\begin{equation*}
\sigma_T=\int_0^{2\pi}\int_0^\pi
\frac{Z^2e^4}
{(mu^2/2)^2\left[4\sin^2(\theta/2)+(\hbar/pa)^2\right]^2}\sin\theta\,d\theta\,d\phi
\end{equation*}

Integrate over $\phi$.
\begin{equation*}
\sigma_T=2\pi\int_0^\pi
\frac{Z^2e^4}
{(mu^2/2)^2\left[4\sin^2(\theta/2)+(\hbar/pa)^2\right]^2}\sin\theta\,d\theta
\end{equation*}

Factor out 16 in the denominator and write as
\begin{equation*}
\sigma_T=\frac{2\pi Z^2e^4}{16(mu^2/2)^2}\int_0^\pi
\frac{\sin\theta}{\left(\sin^2(\theta/2)+(\hbar/2pa)^2\right)^2}\,d\theta
\end{equation*}

By the definite integral
\begin{equation*}
\int_0^\pi\frac{\sin\theta}{\left(\sin^2(\theta/2)+a\right)^2}\,d\theta=\frac{2}{a^2+a}
\end{equation*}

we have
\begin{equation*}
\sigma_T=\frac{2\pi Z^2e^4}{16(mu^2/2)^2}
\frac{2}{(\hbar/2pa)^4+(\hbar/2pa)^2}
\end{equation*}

Rewrite as
\begin{equation*}
\sigma_T=\frac{\pi Z^2e^4}{2p^2u^2}
\left(\frac{2pa}{\hbar}\right)^2
\frac{2}{(\hbar/2pa)^2+1}
\end{equation*}

Hence
\begin{equation*}
\sigma_T=\pi a^2\frac{(2Ze^2/u\hbar)^2}{(\hbar/2pa)^2+1}
\end{equation*}

\end{document}
