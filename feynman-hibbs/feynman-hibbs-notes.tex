\documentclass[12pt]{article}
\usepackage{amsmath}
\usepackage{amssymb}

\parindent=0pt

\newcommand\INT{\int_{\mathbb R^3}}

\begin{document}

{\bf Chapter 6}

\bigskip
Start with equation (6.1).
\begin{equation*}
K_V(b,a)=\int_{x_a}^{x_b}\exp\left(
\frac{i}{\hbar}
\int_{t_a}^{t_b}
\left(\tfrac{1}{2}m\dot x^2-V(x(t),t)\right)
\,dt\right)\,\mathcal Dx(t)
\tag{6.1}
\end{equation*}

Partition the integral.
\begin{equation*}
K_V(b,a)=\int_{x_a}^{x_b}
\exp\left(
\frac{i}{\hbar}
\int_{t_a}^{t_b}\tfrac{1}{2}m\dot x^2
\,dt
-
\frac{i}{\hbar}
\int_{t_a}^{t_b}V(x(t),t)
\,dt
\right)\,\mathcal Dx(t)
\end{equation*}

Factor the exponential.
\begin{equation*}
K_V(b,a)=\int_{x_a}^{x_b}
\exp\left(\frac{i}{\hbar}\int_{t_a}^{t_b}\tfrac{1}{2}m\dot x^2\,dt\right)
\exp\left(-\frac{i}{\hbar}\int_{t_a}^{t_b}V(x(t),t)\,dt\right)
\,\mathcal Dx(t)
\end{equation*}

Use $t_c$ for the measure in the second integral.
\begin{equation*}
K_V(b,a)=\int_{x_a}^{x_b}
\exp\left(\frac{i}{\hbar}\int_{t_a}^{t_b}\tfrac{1}{2}m\dot x^2\,dt\right)
\exp\left(-\frac{i}{\hbar}\int_{t_a}^{t_b}V(x(t_c),t_c)\,dt_c\right)
\,\mathcal Dx(t)
\end{equation*}

Make the second exponential a power series.
\begin{align*}
&K_V(b,a)=\int_{x_a}^{x_b}\exp\left(\frac{i}{\hbar}\int_{t_a}^{t_b}\tfrac{1}{2}m\dot x^2\,dt\right)\times{}
\\
&\left(1
-\frac{i}{\hbar}\int_{t_a}^{t_b}V(x(t_c),t_c)\,dt_c
+\frac{1}{2}\left(-\frac{i}{\hbar}\int_{t_a}^{t_b}V(x(t_c),t_c)\,dt_c\right)^2
+\cdots
\right)\,\mathcal Dx(t)
\end{align*}

Expand the product.
\begin{align*}
&K_V(b,a)=\int_{x_a}^{x_b}\exp\left(\frac{i}{\hbar}\int_{t_a}^{t_b}\tfrac{1}{2}m\dot x^2\,dt\right)\,\mathcal Dx(t)
\\
&\quad{}-\frac{i}{\hbar}
\int_{x_a}^{x_b}\exp\left(\frac{i}{\hbar}\int_{t_a}^{t_b}\tfrac{1}{2}m\dot x^2\,dt\right)
\left(\int_{t_a}^{t_b}V(x(t_c),t_c)\,dt_c\right)\,\mathcal Dx(t)
\\
&\quad{}-\frac{1}{2\hbar^2}
\int_{x_a}^{x_b}\exp\left(\frac{i}{\hbar}\int_{t_a}^{t_b}\tfrac{1}{2}m\dot x^2\,dt\right)
\left(\int_{t_a}^{t_b}V(x(t_c),t_c)\,dt_c\right)^2\,\mathcal Dx(t)
+\cdots
\end{align*}

Let
\begin{align*}
K_0(b,a)&=\int_{x_a}^{x_b}\exp\left(\frac{i}{\hbar}\int_{t_a}^{t_b}\tfrac{1}{2}m\dot x^2\,dt\right)\,\mathcal Dx(t)
\tag{6.5}
\\
K^{(1)}(b,a)&=-\frac{i}{\hbar}
\int_{x_a}^{x_b}\exp\left(\frac{i}{\hbar}\int_{t_a}^{t_b}\tfrac{1}{2}m\dot x^2\,dt\right)
\left(\int_{t_a}^{t_b}V(x(t_c),t_c)\,dt_c\right)\,\mathcal Dx(t)
\\
K^{(2)}(b,a)&=-\frac{1}{2\hbar^2}
\int_{x_a}^{x_b}\exp\left(\frac{i}{\hbar}\int_{t_a}^{t_b}\tfrac{1}{2}m\dot x^2\,dt\right)
\left(\int_{t_a}^{t_b}V(x(t_c),t_c)\,dt_c\right)^2\,\mathcal Dx(t)
\end{align*}

Then equation (6.4) follows.
\begin{equation*}
K_V(b,a)=K_0(b,a)+K^{(1)}(b,a)+K^{(2)}(b,a)+\cdots
\tag{6.4}
\end{equation*}

Let us take a closer look at $K^{(1)}$.
By the distributive law we can change the order of integration
and obtain the following.
\begin{equation*}
K^{(1)}(b,a)=-\frac{i}{\hbar}
\int_{t_a}^{t_b}
\int_{x_a}^{x_b}\exp\left(\frac{i}{\hbar}\int_{t_a}^{t_b}\tfrac{1}{2}m\dot x^2\,dt\right)
V(x(t_c),t_c)
\,\mathcal Dx(t)\,dt_c
\end{equation*}

Let
\begin{equation*}
I(t_c)=
\int_{x_a}^{x_b}\exp\left(\frac{i}{\hbar}\int_{t_a}^{t_b}\tfrac{1}{2}m\dot x^2\,dt\right)
V(x(t_c),t_c)
\,\mathcal Dx(t)
\end{equation*}
so that
\begin{equation*}
K^{(1)}(b,a)=-\frac{i}{\hbar}\int_{t_a}^{t_b}I(t_c)\,dt_c
\end{equation*}

We want to rewrite $I(t_c)$ as an integral over $x(t_c)$.

\bigskip
Let $x_c=x(t_c)$ and note that $x_c$ can take on any value.
In other words, for any $x_c\in(-\infty,\infty)$ there is a path from $x_a$ to $x_b$
that goes through $x_c$.
Since $V(x_c,t_c)$ is a function of $c$ only,
the kernel for the path is a free particle from $a$ to $c$ and from $c$ to $b$.
Hence
\begin{equation*}
I(t_c)=\int_{-\infty}^{\infty}
K_0(x_b,t_b;x_c,t_c)V(x_c,t_c)K_0(x_c,t_c;x_a,t_a)
\,dx_c
\end{equation*}

Or more compactly
\begin{equation*}
I(t_c)=\int_{-\infty}^{\infty}
K_0(b,c)V(c)K_0(c,a)
\,dx_c
\end{equation*}

Hence
\begin{equation*}
K^{(1)}(b,a)=-\frac{i}{\hbar}
\int_{t_a}^{t_b}
\int_{-\infty}^{\infty}
K_0(b,c)V(c)K_0(c,a)
\,dx_c\,dt_c
\end{equation*}

\section*{7-1}

``Thus a transition element will be written as $\langle F\rangle_S$
instead of $\langle\chi|F|\psi\rangle_S$.''

\section*{7-2}

$\delta F$ is the differential of $F$.

\bigskip
Consider equation (7.29).
\begin{equation*}
\langle F\rangle_S=\int F\big(x(t)+\eta(t)\big)
\exp\left(\frac{i}{\hbar}S\big(x(t)+\eta(t)\big)\right)
\,\mathcal Dx(t)
\end{equation*}

Consider equation (7.31).
\begin{equation*}
\int\frac{\partial F}{\partial x_k}\exp\left(\frac{i}{\hbar}S\big(x(t)\big)\right)\,\mathcal Dx(t)
\tag{7.31}
\end{equation*}

Integrate by parts. Let
\begin{align*}
u&=\exp\left(\frac{i}{\hbar}S\big(x(t)\big)\right)
\\
dv&=\frac{\partial F}{\partial x_k}\,\mathcal Dx(t)
\end{align*}

Then
\begin{equation*}
\int u\,dv=uv-\int v\,du
\end{equation*}

The $uv$ part vanishes, why?

\end{document}
