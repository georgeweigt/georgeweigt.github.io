\documentclass[12pt]{article}
\usepackage{amsmath}
\usepackage{amssymb}

\parindent=0pt

\newcommand\INT{\int_{\mathbb R^3}}

\begin{document}

6-15.
Recall that in problem 5-4 we defined a particular integral as the
transition amplitude to go from state $\psi(x)$ to state $\chi(x)$.
Show that the function $\lambda_{mn}$ satisfies this definition
when the initial state is the eigenfunction $\phi_n(x)$ and the
final state is the eigenfunction $\phi_m(x)$.

\bigskip
\hrule

\bigskip
From problem 5-4 the transition amplitude is
\begin{equation*}
\langle m|n\rangle=
\int_{-\infty}^\infty\int_{-\infty}^\infty
\phi_m^*(x_b)K_V(b,a)\phi_n(x_a)
\,dx_a\,dx_b
\tag{1}
\end{equation*}

Consider equation (6.68).
\begin{equation*}
K_V(b,a)=\sum_m\sum_n\lambda_{mn}(t_b,t_a)\phi_m(x_b)\phi_n^*(x_a)
\tag{6.68}
\end{equation*}

Substitute (6.68) into (1).
\begin{multline*}
\langle m|n\rangle=
\int_{-\infty}^\infty\int_{-\infty}^\infty
\phi_m^*(x_b)
\\
{}\times
\left(\sum_{m'}\sum_{n'}\lambda_{m'n'}(t_b,t_a)\phi_{m'}(x_b)\phi_{n'}^*(x_a)\right)
\phi_n(x_a)
\,dx_a\,dx_b
\end{multline*}

By distributive law
\begin{equation*}
\langle m|n\rangle=
\sum_{m'}\sum_{n'}\lambda_{m'n'}(t_b,t_a)
\int_{-\infty}^\infty\int_{-\infty}^\infty
\phi_m^*(x_b)\phi_{m'}(x_b)
\phi_{n'}^*(x_a)\phi_n(x_a)
\,dx_a\,dx_b
\end{equation*}

By orthogonality of eigenfunctions
\begin{equation*}
\langle m|n\rangle=
\lambda_{mn}(t_b,t_a)\int_{-\infty}^\infty\int_{-\infty}^\infty
\phi_m^*(x_b)\phi_m(x_b)
\phi_n^*(x_a)\phi_n(x_a)
\,dx_a\,dx_b
\end{equation*}

By normalization of eigenfunctions
\begin{equation*}
\langle m|n\rangle=
\lambda_{mn}(t_b,t_a)
\end{equation*}

\end{document}
