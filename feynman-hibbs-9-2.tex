\documentclass[12pt]{article}
\usepackage{amsmath}
\usepackage{amssymb}

\parindent=0pt

\begin{document}

9-2.
Explain why the charge density corresponding to a single
charge $q$ located at the point $\mathbf x(t)=(x(t),y(t),z(t))$
at time $t$ is
\begin{equation*}
\rho(\mathbf r,t)=q
\delta(r_x-x(t))
\delta(r_y-y(t))
\delta(r_z-z(t))
=q\delta^3(\mathbf r-\mathbf x(t))
\end{equation*}

For a point charge $q$ at $\mathbf x(t)$ the charge density is
\begin{equation*}
\rho(\mathbf r,t)
=\begin{cases}
q & \mathbf r=\mathbf x(t)
\\
0 & \mathbf r\ne\mathbf x(t)
\end{cases}
\end{equation*}

Hence
\begin{equation*}
\rho(\mathbf r,t)=q\delta^3(\mathbf r-\mathbf x(t))
\end{equation*}

(9-2 cont'd)
Show that
\begin{equation*}
\rho_{\mathbf k}(t)=q\exp(-i\mathbf k\cdot\mathbf x(t))
\tag{1}
\end{equation*}

From equation (9.14)
\begin{equation*}
\rho(\mathbf r,t)=\int\rho_{\mathbf k}(t)
\exp(i\mathbf k\cdot\mathbf r)
\frac{d^3\mathbf k}{(2\pi)^3}
\tag{2}
\end{equation*}

Substitute (1) into (2).
\begin{equation*}
\rho(\mathbf r,t)=\frac{q}{(2\pi)^3}
\int_{-\infty}^\infty
\int_{-\infty}^\infty
\int_{-\infty}^\infty
\exp\big(i\mathbf k\cdot(\mathbf r-\mathbf x(t))\big)
\,dk_x\,dk_y\,dk_z
\end{equation*}

Recall the definition of a delta function.
\begin{equation*}
\delta(a)=\frac{1}{2\pi}\int_{-\infty}^\infty\exp(iax)\,dx
\end{equation*}

Hence
\begin{equation*}
\rho(\mathbf r,t)=q\delta^3(\mathbf r-\mathbf x(t))
\end{equation*}

(9-2 cont'd)
Explain why the current density is
\begin{equation*}
\mathbf j(\mathbf r,t)=q\dot{\mathbf x}(t)\delta^3(\mathbf r-\mathbf x(t))
\end{equation*}

(9-2 cont'd)
If we have a number of charges $q_i$ located at $\mathbf x_i(t)$,
the values $\rho_{\mathbf k}$ and $\mathbf j_{\mathbf k}$ are
\begin{equation*}
\rho_{\mathbf k}=\sum_iq_i\exp(-i\mathbf k\cdot\mathbf x_i(t))
\qquad
\mathbf j_{\mathbf k}
=\sum_iq_i\dot{\mathbf x}_i(t)
\exp(-\mathbf k\cdot\mathbf x_i(t))
\tag{9.16}
\end{equation*}

From equation (9.14)
\begin{equation*}
\mathbf j(\mathbf r,t)
=\int\mathbf j_{\mathbf k}(t)
\exp(i\mathbf k\cdot\mathbf r)
\frac{d^3\mathbf k}{(2\pi)^3}
\tag{3}
\end{equation*}

Substitute (9.16) into (3).
\begin{equation*}
\mathbf j(\mathbf r,t)
=\frac{q}{(2\pi)^3}\dot{\mathbf x}(t)
\int_{-\infty}^\infty
\int_{-\infty}^\infty
\int_{-\infty}^\infty
\exp\big(i\mathbf k\cdot(\mathbf r-\mathbf x(t))\big)
\,dk_x\,dk_y\,dk_z
\end{equation*}

Hence
\begin{equation*}
\mathbf j(\mathbf r,t)=q\dot{\mathbf x}(t)\delta^3(\mathbf r-\mathbf x(t))
\end{equation*}

\end{document}
