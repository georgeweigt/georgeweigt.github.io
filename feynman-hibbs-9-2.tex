\documentclass[12pt]{article}
\usepackage{amsmath}
\usepackage{amssymb}

\parindent=0pt

\begin{document}

9-2.
Explain why the charge density corresponding to a single
charge $q$ located at the point $\mathbf x(t)=(x(t),y(t),z(t))$
at time $t$ is
\begin{equation*}
\rho(\mathbf r,t)=q
\delta(r_x-x(t))
\delta(r_y-y(t))
\delta(r_z-z(t))
=q\delta^3(\mathbf r-\mathbf x(t))
\end{equation*}

From equation (9.14)
\begin{equation*}
\rho(\mathbf r,t)=\int\rho_{\mathbf k}(t)
\exp(i\mathbf k\cdot\mathbf r)
\frac{d^3\mathbf k}{(2\pi)^3}
\end{equation*}

For a point charge at $\mathbf x(t)$
\begin{equation*}
\rho_{\mathbf k}(t)=\begin{cases}
q & \mathbf k\cdot\mathbf r=0
\\
0 & \mathbf k\cdot\mathbf r\ne0
\end{cases}
\end{equation*}

Rewrite as
\begin{multline*}
\rho(\mathbf r,t)=
\\
\frac{1}{(2\pi)^3}
\int_{-\infty}^\infty
\int_{-\infty}^\infty
\int_{-\infty}^\infty
\rho_{\mathbf k}(t)
\exp(ik_xr_x)
\exp(ik_yr_y)
\exp(ik_zr_z)
\,dk_x\,dk_y\,dk_z
\end{multline*}

\end{document}
