\documentclass[12pt]{article}
\usepackage[margin=1in]{geometry}
\usepackage{amsmath}
\parindent=0pt
\begin{document}

\section*{Spontaneous emission rate}

What is the spontaneous emission rate for hydrogen state 2$p$?

\bigskip
Let us begin by writing down the wave function $\psi$ for hydrogen.
\begin{equation*}
\psi_{nlm}(r,\theta,\phi)=R_{nl}(r)Y_{lm}(\theta,\phi)
\end{equation*}

where
\begin{gather*}
R_{nl}(r)=
\frac{2}{n^2}
\biggl(\frac{(n-l-1)!}{(n+l)!}\biggr)^{1/2}
\biggl(\frac{2r}{na_0}\biggr)^l
L_{n-l-1}^{2l+1}\biggl(\frac{2r}{na_0}\biggr)
\exp\biggl(-\frac{r}{na_0}\biggr)
a_0^{-3/2}
\\
L_n^m(x)=(n+m)!\sum_{k=0}^n\frac{(-x)^k}{(n-k)!(m+k)!k!}
\\
Y_{lm}(\theta,\phi)=(-1)^m
\biggl(\frac{2l+1}{4\pi}\biggr)^{1/2}
\biggl(\frac{(l-m)!}{(l+m)!}\biggr)^{1/2}
P_l^m(\cos\theta)\exp(im\phi)
\\
P_l^m(x)=\frac{1}{2^l l!}(1-x^2)^{m/2}\frac{d^{l+m}}{dx^{l+m}}(x^2-1)^l
\\
a_0=\frac{4\pi\varepsilon_0\hbar^2}{e^2\mu}\approx0.529\times10^{-10}\,\text{meter}
\end{gather*}

The state $2p$ means that $n=2$ and $l=1$.
For $l=1$ there are three ways to choose $m$ hence all of the following processes correspond to the transition
$2p\rightarrow1s$.
It turns out that all three processes have the same transition rate.
\begin{equation*}
\left.\begin{aligned}
&\psi_{2,1,1}
\\
&\psi_{2,1,0}
\\
&\psi_{2,1,-1}
\end{aligned}\right\}\rightarrow\psi_{100}+\text{photon}
\end{equation*}

The spontaneous emission rate is
\begin{equation*}
A_{21}=\frac{e^2}{3\pi\varepsilon_0\hbar c^3}\omega_{21}^3|r_{21}|^2
\tag{1}
\end{equation*}
where
\begin{gather*}
\omega_{21}=\frac{E_2-E_1}{\hbar},\quad E_n=-\frac{e^2}{8\pi\varepsilon_0a_0n^2}
\\
|r_{21}|^2=|x_{21}|^2+|y_{21}|^2+|z_{21}|^2
\\
x_{21}=\int\limits_{0}^{2\pi}\int\limits_{0}^{\pi}\int\limits_{0}^{\infty}xf_{21}\,dV,
\quad
y_{21}=\int\limits_{0}^{2\pi}\int\limits_{0}^{\pi}\int\limits_{0}^{\infty}yf_{21}\,dV,
\quad
z_{21}=\int\limits_{0}^{2\pi}\int\limits_{0}^{\pi}\int\limits_{0}^{\infty}zf_{21}\,dV
\\
x=r\sin\theta\cos\phi,
\quad
y=r\sin\theta\sin\phi,
\quad
z=r\cos\theta
\\
f_{21}=\psi_{100}^*\psi_{210}=\frac{r\cos\theta}{4\sqrt2\pi a_0^4}\exp\biggl(-\frac{3r}{2a_0}\biggr)
\\
dV=r^2\sin\theta\,dr\,d\theta\,d\phi
\end{gather*}

For the calculation of $|r_{21}|^2$ we obtain
\begin{equation*}
x_{21}=0,
\quad
y_{21}=0,
\quad
z_{21}=\frac{2^7}{3^5}\sqrt2a_0
\end{equation*}
hence
\begin{equation*}
|r_{21}|^2=|z_{21}|^2=\frac{2^{15}}{3^{10}}a_0^2=\frac{32768}{59049}a_0^2
\end{equation*}

By equation (1) the spontaneous emission rate is
\begin{equation*}
A_{21}=6.26\times10^8\,\text{second}^{-1}
\end{equation*}

The mean interval is
\begin{equation*}
\frac{1}{A_{21}}=1.60\times10^{-9}\,\text{second}
\end{equation*}

\end{document}
