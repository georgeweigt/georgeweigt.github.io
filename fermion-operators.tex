\documentclass[12pt]{article}
\usepackage[margin=2cm]{geometry}
\usepackage{amsmath}

\begin{document}

\section{Anticommutation}
Consider the following eigenstates of a hypothetical quantum system.\footnote{
Adapted from problem 16.1.1 of ``Quantum Mechanics for Scientists and Engineers.''\\
{\tt https://ee.stanford.edu/{\textasciitilde}dabm/QMbook.html}}
\begin{align*}
|00\rangle&=(\text{1 0 0 0})\qquad\text{no fermions}\\
|10\rangle&=(\text{0 1 0 0})\qquad\text{one fermion in state 1}\\
|01\rangle&=(\text{0 0 1 0})\qquad\text{one fermion in state 2}\\
|11\rangle&=(\text{0 0 0 1})\qquad\text{two fermions, one in state 1, one in state 2}
\end{align*}

\noindent
Creation and annihilation operators are formed from outer products of state vectors.
Sign changes make the operators antisymmetric.
\begin{align*}
\hat{b}_1^\dag&=|10\rangle\langle00|-|11\rangle\langle01| \qquad\text{Create one fermion in state 1}
\\
\hat{b}_1&=|00\rangle\langle10|-|01\rangle\langle11| \qquad\text{Annihilate one fermion in state 1}
\\
\hat{b}_2^\dag&=|01\rangle\langle00|+|11\rangle\langle10| \qquad\text{Create one fermion in state 2}
\\
\hat{b}_2&=|00\rangle\langle01|+|10\rangle\langle11| \qquad\text{Annihilate one fermion in state 2}
\end{align*}

\noindent
The operators in matrix form.
\begin{equation*}
\hat{b}_1^\dag=\begin{pmatrix}0&0&0&0\\1&0&0&0\\0&0&0&0\\0&0&-1&0\end{pmatrix}
\quad
\hat{b}_1=\begin{pmatrix}0&1&0&0\\0&0&0&0\\0&0&0&-1\\0&0&0&0\end{pmatrix}
\quad
\hat{b}_2^\dag=\begin{pmatrix}0&0&0&0\\0&0&0&0\\1&0&0&0\\0&1&0&0\end{pmatrix}
\quad
\hat{b}_2=\begin{pmatrix}0&0&1&0\\0&0&0&1\\0&0&0&0\\0&0&0&0\end{pmatrix}
\end{equation*}

\noindent
Verify anticommutation relations of the operators.
\begin{align*}
\hat{b}_j\hat{b}_k+\hat{b}_k\hat{b}_j&=0
\\[2ex]
\hat{b}_j^\dag\hat{b}_k^\dag+\hat{b}_k^\dag\hat{b}_j^\dag&=0
\\[2ex]
\hat{b}_j\hat{b}_k^\dag+\hat{b}_k^\dag\hat{b}_j&=\delta_{jk}
\end{align*}

\newpage

\section{Wavefunction operator}
Consider the following eigenstates of a hypothetical quantum system.\footnote{
Adapted from problem 16.2.1 of ``Quantum Mechanics for Scientists and Engineers.''\\
{\tt https://ee.stanford.edu/{\textasciitilde}dabm/QMbook.html}}
\begin{align*}
|00\rangle&=(\text{1 0 0 0})\qquad\text{no fermions}\\
|10\rangle&=(\text{0 1 0 0})\qquad\text{one fermion in state $\phi_1$}\\
|01\rangle&=(\text{0 0 1 0})\qquad\text{one fermion in state $\phi_2$}\\
|11\rangle&=(\text{0 0 0 1})\qquad\text{two fermions, one in state $\phi_1$, one in state $\phi_2$}
\end{align*}

\noindent
Let fermion states $\phi_n$ be modeled by a one dimensional box of length $L$.
\begin{equation*}
\phi_n(x)=\sqrt{\frac{2}{L}}\sin\left(\frac{n\pi x}{L}\right)
\end{equation*}

\noindent
Creation and annihilation operators are formed from outer products of state vectors.
Sign changes make the operators antisymmetric.
\begin{align*}
\hat{b}_1^\dag&=|10\rangle\langle00|-|11\rangle\langle01| \qquad\text{Create one fermion in state $\phi_1$}
\\
\hat{b}_1&=|00\rangle\langle10|-|01\rangle\langle11| \qquad\text{Annihilate one fermion in state $\phi_1$}
\\
\hat{b}_2^\dag&=|01\rangle\langle00|+|11\rangle\langle10| \qquad\text{Create one fermion in state $\phi_2$}
\\
\hat{b}_2&=|00\rangle\langle01|+|10\rangle\langle11| \qquad\text{Annihilate one fermion in state $\phi_2$}
\end{align*}

\noindent
Given the wavefunction operator
\begin{equation*}
\hat{\psi}=\frac{1}{\sqrt{2}}\sum_{n,m}\phi_n(x)\phi_m(y)\hat{b}_n\hat{b}_m
\end{equation*}

\noindent
show that
\begin{equation*}
\hat{\psi}|11\rangle=\frac{1}{\sqrt{2}}\big(\phi_1(x)\phi_2(y)-\phi_1(y)\phi_2(x)\big)|00\rangle
\end{equation*}

\newpage

\section{Position operator}
Consider the following eigenstates of a hypothetical quantum system.
\begin{align*}
|00\rangle&=(\text{1 0 0 0})\qquad\text{no fermions}\\
|10\rangle&=(\text{0 1 0 0})\qquad\text{one fermion in state $\phi_1$}\\
|01\rangle&=(\text{0 0 1 0})\qquad\text{one fermion in state $\phi_2$}\\
|11\rangle&=(\text{0 0 0 1})\qquad\text{two fermions, one in state $\phi_1$, one in state $\phi_2$}
\end{align*}

\noindent
Let fermion states $\phi_n$ be modeled by a one dimensional box of length $L$.
\begin{equation*}
\phi_n(x)=\sqrt{\frac{2}{L}}\sin\left(\frac{n\pi x}{L}\right)
\end{equation*}

\noindent
Creation and annihilation operators are formed from outer products of state vectors.
Sign changes make the operators antisymmetric.
\begin{align*}
\hat{b}_1^\dag&=|10\rangle\langle00|-|11\rangle\langle01| \qquad\text{Create one fermion in state $\phi_1$}
\\
\hat{b}_1&=|00\rangle\langle10|-|01\rangle\langle11| \qquad\text{Annihilate one fermion in state $\phi_1$}
\\
\hat{b}_2^\dag&=|01\rangle\langle00|+|11\rangle\langle10| \qquad\text{Create one fermion in state $\phi_2$}
\\
\hat{b}_2&=|00\rangle\langle01|+|10\rangle\langle11| \qquad\text{Annihilate one fermion in state $\phi_2$}
\end{align*}

\noindent
Let $\hat{r}$ be the position operator
\begin{equation*}
\hat{r}=\sum_{n,m}r_{nm}\hat{b}_n^\dag\hat{b}_m
\end{equation*}

\noindent
where
\begin{equation*}
r_{nm}=\int_0^L\phi_n^*(x)x\phi_m(x)\,dx
\end{equation*}

\noindent
Note that for a one dimensional box
\begin{equation*}
r_{nn}=\langle x\rangle=\tfrac{1}{2}L
\end{equation*}

\noindent
Verify that
\begin{align*}
\langle10|\hat{r}|10\rangle&=r_{11}\\
\langle10|\hat{r}|01\rangle&=r_{12}\\
\langle01|\hat{r}|10\rangle&=r_{21}\\
\langle01|\hat{r}|01\rangle&=r_{22}
\end{align*}

\end{document}
