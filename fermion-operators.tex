\documentclass[12pt]{article}
\usepackage[margin=2cm]{geometry}
\usepackage{amsmath}

\title{Fermion Operators}
\date{}
\author{}

\begin{document}

\maketitle

\tableofcontents

\newpage

\section{Anticommutation}
Consider the following eigenstates of a hypothetical quantum system.\footnote{
Adapted from problem 16.1.1 of ``Quantum Mechanics for Scientists and Engineers.''\\
{\tt https://ee.stanford.edu/{\textasciitilde}dabm/QMbook.html}}
\begin{align*}
|00\rangle&=(\text{1 0 0 0})\qquad\text{no fermions}\\
|10\rangle&=(\text{0 1 0 0})\qquad\text{one fermion in state 1}\\
|01\rangle&=(\text{0 0 1 0})\qquad\text{one fermion in state 2}\\
|11\rangle&=(\text{0 0 0 1})\qquad\text{two fermions, one in state 1, one in state 2}
\end{align*}

\noindent
Creation and annihilation operators are formed from outer products of state vectors.
Sign changes make the operators antisymmetric.
\begin{align*}
\hat{b}_1^\dag&=|10\rangle\langle00|-|11\rangle\langle01| \qquad\text{Create one fermion in state 1}
\\
\hat{b}_1&=|00\rangle\langle10|-|01\rangle\langle11| \qquad\text{Annihilate one fermion in state 1}
\\
\hat{b}_2^\dag&=|01\rangle\langle00|+|11\rangle\langle10| \qquad\text{Create one fermion in state 2}
\\
\hat{b}_2&=|00\rangle\langle01|+|10\rangle\langle11| \qquad\text{Annihilate one fermion in state 2}
\end{align*}

\noindent
The operators in matrix form.
\begin{equation*}
\hat{b}_1^\dag=\begin{pmatrix}0&0&0&0\\1&0&0&0\\0&0&0&0\\0&0&-1&0\end{pmatrix}
\quad
\hat{b}_1=\begin{pmatrix}0&1&0&0\\0&0&0&0\\0&0&0&-1\\0&0&0&0\end{pmatrix}
\quad
\hat{b}_2^\dag=\begin{pmatrix}0&0&0&0\\0&0&0&0\\1&0&0&0\\0&1&0&0\end{pmatrix}
\quad
\hat{b}_2=\begin{pmatrix}0&0&1&0\\0&0&0&1\\0&0&0&0\\0&0&0&0\end{pmatrix}
\end{equation*}

\noindent
Verify anticommutation relations of the operators.
\begin{align*}
\hat{b}_j\hat{b}_k+\hat{b}_k\hat{b}_j&=0
\\[2ex]
\hat{b}_j^\dag\hat{b}_k^\dag+\hat{b}_k^\dag\hat{b}_j^\dag&=0
\\[2ex]
\hat{b}_j\hat{b}_k^\dag+\hat{b}_k^\dag\hat{b}_j&=\delta_{jk}
\end{align*}

\newpage

\section{Wavefunction operator}
Consider the following eigenstates of a hypothetical quantum system.\footnote{
Adapted from problem 16.2.1 of ``Quantum Mechanics for Scientists and Engineers.''\\
{\tt https://ee.stanford.edu/{\textasciitilde}dabm/QMbook.html}}
\begin{align*}
|00\rangle&=(\text{1 0 0 0})\qquad\text{no fermions}\\
|10\rangle&=(\text{0 1 0 0})\qquad\text{one fermion in state $\phi_1$}\\
|01\rangle&=(\text{0 0 1 0})\qquad\text{one fermion in state $\phi_2$}\\
|11\rangle&=(\text{0 0 0 1})\qquad\text{two fermions, one in state $\phi_1$, one in state $\phi_2$}
\end{align*}

\noindent
Let fermion states $\phi_n$ be modeled by a one dimensional box of length $L$.
\begin{equation*}
\phi_n(x)=\sqrt{\frac{2}{L}}\sin\left(\frac{n\pi x}{L}\right)
\end{equation*}

\noindent
Creation and annihilation operators are formed from outer products of state vectors.
Sign changes make the operators antisymmetric.
\begin{align*}
\hat{b}_1^\dag&=|10\rangle\langle00|-|11\rangle\langle01| \qquad\text{Create one fermion in state $\phi_1$}
\\
\hat{b}_1&=|00\rangle\langle10|-|01\rangle\langle11| \qquad\text{Annihilate one fermion in state $\phi_1$}
\\
\hat{b}_2^\dag&=|01\rangle\langle00|+|11\rangle\langle10| \qquad\text{Create one fermion in state $\phi_2$}
\\
\hat{b}_2&=|00\rangle\langle01|+|10\rangle\langle11| \qquad\text{Annihilate one fermion in state $\phi_2$}
\end{align*}

\noindent
Given the wavefunction operator
\begin{equation*}
\hat{\psi}=\frac{1}{\sqrt{2}}\sum_{n,m}\phi_n(x)\phi_m(y)\hat{b}_n\hat{b}_m
\end{equation*}

\noindent
show that
\begin{equation*}
\hat{\psi}|11\rangle=\frac{1}{\sqrt{2}}\big(\phi_1(x)\phi_2(y)-\phi_1(y)\phi_2(x)\big)|00\rangle
\end{equation*}

\newpage

\section{Position operator}
Consider the following eigenstates of a hypothetical quantum system.
\begin{align*}
|00\rangle&=(\text{1 0 0 0})\qquad\text{no fermions}\\
|10\rangle&=(\text{0 1 0 0})\qquad\text{one fermion in state $\phi_1$}\\
|01\rangle&=(\text{0 0 1 0})\qquad\text{one fermion in state $\phi_2$}\\
|11\rangle&=(\text{0 0 0 1})\qquad\text{two fermions, one in state $\phi_1$, one in state $\phi_2$}
\end{align*}

\noindent
Let fermion states $\phi_n$ be modeled by a one dimensional box of length $L$.
\begin{equation*}
\phi_n(x)=\sqrt{\frac{2}{L}}\sin\left(\frac{n\pi x}{L}\right)
\end{equation*}

\noindent
Creation and annihilation operators are formed from outer products of state vectors.
Sign changes make the operators antisymmetric.
\begin{align*}
\hat{b}_1^\dag&=|10\rangle\langle00|-|11\rangle\langle01| \qquad\text{Create one fermion in state $\phi_1$}
\\
\hat{b}_1&=|00\rangle\langle10|-|01\rangle\langle11| \qquad\text{Annihilate one fermion in state $\phi_1$}
\\
\hat{b}_2^\dag&=|01\rangle\langle00|+|11\rangle\langle10| \qquad\text{Create one fermion in state $\phi_2$}
\\
\hat{b}_2&=|00\rangle\langle01|+|10\rangle\langle11| \qquad\text{Annihilate one fermion in state $\phi_2$}
\end{align*}

\noindent
Let $\hat{r}$ be the position operator
\begin{equation*}
\hat{r}=\sum_{n,m}r_{nm}\hat{b}_n^\dag\hat{b}_m
\end{equation*}

\noindent
where
\begin{equation*}
r_{nm}=\int_0^L\phi_n^*(x)x\phi_m(x)\,dx
\end{equation*}

\noindent
Note that for a one dimensional box
\begin{equation*}
r_{nn}=\langle x\rangle=\tfrac{1}{2}L
\end{equation*}

\noindent
Verify that
\begin{align*}
\langle10|\hat{r}|10\rangle&=r_{11}\\
\langle10|\hat{r}|01\rangle&=r_{12}\\
\langle01|\hat{r}|10\rangle&=r_{21}\\
\langle01|\hat{r}|01\rangle&=r_{22}
\end{align*}

\newpage

\section{Exchange energy}
Let $\psi(x,y)$ be the antisymmetrized wave function for two electrons in a box of length $L$.
\begin{align*}
\psi(x,y)&=\frac{1}{\sqrt{2}}
\big(\phi_1(x)\phi_2(y)-\phi_1(y)\phi_2(x)\big)
\\[2ex]
\phi_n(x)&=\sqrt{\frac{2}{L}}\sin\left(\frac{n\pi x}{L}\right)
\end{align*}

\noindent
For $L=10^{-9}$ meter the expected potential energy is
\begin{equation*}
V=\frac{e^2}{4\pi\epsilon_0}\int_0^L\int_0^L\frac{\psi^*(x,y)\psi(x,y)}{|x-y|}\,dx\,dy
=4.67\,\text{eV}
\end{equation*}

\noindent
Next calculate the potential energy for a wave function that is not antisymmetrized.
\begin{equation*}
V_0=\frac{e^2}{4\pi\epsilon_0}
\int_0^L\int_0^L\frac{\phi_1^*(x)\phi_2^*(y)\phi_1(x)\phi_2(y)}{|x-y|}\,dx\,dy
=12.80\,\text{eV}
\end{equation*}

\noindent
The difference is the exchange energy.
\begin{equation*}
V_{ex}=V-V_0=-8.13\,\text{eV}
\end{equation*}

\noindent
Note that the formula for $V_0$ has a singularity at $x=y$.
The computed value shown above is the result of an arbitrary cutoff in numerical integration.
The actual value of $V_0$ goes to infinity.

\bigskip
\noindent
Note also that there is a singularity at $x=y$ in the formula for $V$.
However, due to antisymmetry we have $\psi(x,x)=0$ and hence the integral converges.

\newpage

\section{Energy matrix}
Consider a system with the following eigenstates.
\begin{align*}
|0\rangle&=(\text{1 0 0 0})\qquad\text{no electrons}\\
|1\rangle&=(\text{0 1 0 0})\qquad\text{one electron in state $\phi_1$}\\
|2\rangle&=(\text{0 0 1 0})\qquad\text{one electron in state $\phi_2$}\\
|3\rangle&=(\text{0 0 0 1})\qquad\text{two electrons, one in state $\phi_1$, one in state $\phi_2$}
\end{align*}

\noindent
Let electron states $\phi_n$ be modeled by a one dimensional box of length $L$.
\begin{equation*}
\phi_n(x)=\sqrt{\frac{2}{L}}\sin\left(\frac{n\pi x}{L}\right)
\end{equation*}

\noindent
Let $|\xi\rangle$ be an arbitrary normalized state vector.
\begin{equation*}
|\xi\rangle=c_0|0\rangle+c_1|1\rangle+c_2|2\rangle+c_3|3\rangle,\qquad\langle\xi|\xi\rangle=1
\end{equation*}

\noindent
Let us determine an energy matrix $\hat{E}$ such that the expected
energy $\langle E \rangle$ in state $|\xi\rangle$ is
\begin{equation*}
\langle E\rangle=\langle\xi|\hat{E}|\xi\rangle
\end{equation*}

\noindent
Energy matrix $\hat{E}$ is the sum of kinetic and potential energy matrices.
\begin{equation*}
\hat{E}=\hat{K}+\hat{V}
\end{equation*}

\noindent
Kinetic energy matrix $\hat{K}$ can be computed from energy eigenvalues of the box model.
\begin{equation*}
\hat{K}=\begin{pmatrix}
0 & 0 & 0 & 0\\
0 & E_1 & 0 & 0\\
0 & 0 & E_2 & 0\\
0 & 0 & 0 & E_1+E_2
\end{pmatrix},
\qquad
E_n=\frac{n^2\pi^2\hbar^2}{2mL^2}
\end{equation*}

\noindent
Potential energy matrix $\hat{V}$ has one entry due to Coulomb interaction in the two electron state.
\begin{equation*}
\hat{V}=
\begin{pmatrix}
0 & 0 & 0 & 0\\
0 & 0 & 0 & 0\\
0 & 0 & 0 & 0\\
0 & 0 & 0 & V
\end{pmatrix}
\end{equation*}

\noindent
Let $\psi(x,y)$ be the antisymmetrized wavefunction of the two electrons.
\begin{equation*}
\psi(x,y)=\frac{1}{\sqrt{2}}\big(\phi_1(x)\phi_2(y)-\phi_1(y)\phi_2(x)\big)
\end{equation*}

\noindent
Then
\begin{equation*}
V=\frac{e^2}{4\pi\epsilon_0}\int_0^L\int_0^L
\psi^*(x,y)\left(\frac{1}{|x-y|}\right)\psi(x,y)\,dx\,dy
\end{equation*}

\noindent
Let us now choose $L=10^{-9}$ meters and compute numerical values.
For $\hat{K}$ we have
\begin{equation*}
\hat{K}=\begin{pmatrix}
0 & 0 & 0 & 0\\
0 & 0.38\,\text{eV} & 0 & 0\\
0 & 0 & 1.50\,\text{eV} & 0\\
0 & 0 & 0 & 1.88\,\text{eV}
\end{pmatrix}
\end{equation*}

\noindent
Computing $V$ by numerical integration we have
\begin{equation*}
\hat{V}=\begin{pmatrix}
0 & 0 & 0 & 0\\
0 & 0 & 0 & 0\\
0 & 0 & 0 & 0\\
0 & 0 & 0 & 4.67\,\text{eV}
\end{pmatrix}
\end{equation*}

\noindent
Hence
\begin{equation*}
\hat{E}=\hat{K}+\hat{V}=\begin{pmatrix}
0 & 0 & 0 & 0\\
0 & 0.38\,\text{eV} & 0 & 0\\
0 & 0 & 1.50\,\text{eV} & 0\\
0 & 0 & 0 & 6.55\,\text{eV}
\end{pmatrix}
\end{equation*}

\newpage

\section{Superposition of eigenstates}
Consider a system with the following eigenstates.
\begin{align*}
|0\rangle&=(\text{1 0 0 0})\qquad\text{no electrons}\\
|1\rangle&=(\text{0 1 0 0})\qquad\text{one electron in state $\phi_1$}\\
|2\rangle&=(\text{0 0 1 0})\qquad\text{one electron in state $\phi_2$}\\
|3\rangle&=(\text{0 0 0 1})\qquad\text{two electrons, one in state $\phi_1$, one in state $\phi_2$}
\end{align*}

\noindent
Then for the wavefunction basis
\begin{equation*}
\phi_n(x)=\sqrt{\frac{2}{L}}\sin\left(\frac{n\pi x}{L}\right)
\end{equation*}

\noindent
and for $L=10^{-9}$ meters we have
\begin{equation*}
\hat{E}=\begin{pmatrix}
0&0&0&0\\
0&0.38\,\text{eV}&0&0\\
0&0&1.50\,\text{eV}&0\\
0&0&0&6.55\,\text{eV}
\end{pmatrix}
\end{equation*}

\noindent
Let $|\xi\rangle$ be the state vector
\begin{equation*}
|\xi\rangle
=\frac{1}{2}|0\rangle+\frac{1}{2}|1\rangle+\frac{1}{2}|2\rangle+\frac{1}{2}|3\rangle
=\begin{pmatrix}1/2\\1/2\\1/2\\1/2\end{pmatrix}
\end{equation*}

\noindent
The expected energy is
\begin{equation*}
\langle\xi|\hat{E}|\xi\rangle
=\frac{0\,\text{eV}}{4}+\frac{0.38\,\text{eV}}{4}+\frac{1.50\,\text{eV}}{4}+\frac{6.55\,\text{eV}}{4}
=2.11\,\text{eV}
\end{equation*}

\noindent
For the system we are considering, the result of a single measurement is either
0~eV, 0.38~eV, 1.50~eV, or 6.55~eV.
%The probability of observing each eigenvalue is 0.25 when the system is in state $|\xi\rangle$.
The value 2.11 eV is the expected average across multiple measurements.
Recall that a measurement causes the system to exit state $|\xi\rangle$
and enter an eigenstate $|0\rangle$, $|1\rangle$, $|2\rangle$, or $|3\rangle$
corresponding to the measured eigenvalue.
The system must be put back in state $|\xi\rangle$ before the next measurement.

\bigskip
\noindent
To use a slot machine analogy, state $|\xi\rangle$ is like the wheels spinning.
Observing the system makes the wheels stop.
The stopped wheels are in an eigenstate $|0\rangle$, $|1\rangle$, $|2\rangle$, or $|3\rangle$.
Once they are stopped the wheels don't change, they remain in the same eigenstate.
You have to pull the lever to get the wheels spinning again.

\end{document}
