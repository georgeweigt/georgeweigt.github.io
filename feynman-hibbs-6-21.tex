\documentclass[12pt]{article}
\usepackage{amsmath}
\usepackage{amssymb}

\parindent=0pt

\newcommand\INT{\int_{\mathbb R^3}}

\begin{document}

6-21.
Consider the special case that the perturbing potential
$V$ has no matrix elements except between the two states
1 and 2; and further, suppose these states are degenerate,
that is, suppose $E_1=E_2$.
Let $V_{12}=V_{21} = v$ and let $V_{11}$, $V_{22}$, and
all other $V_{mn}$ be zero.
Show that
\begin{align*}
\lambda_{11}&=1-\frac{v^2T^2}{2\hbar^2}+\frac{v^4T^4}{24\hbar^4}-\cdots&=\cos\frac{vT}{\hbar}
\tag{6.81}
\\
\lambda_{12}&=-i\frac{vT}{\hbar}+i\frac{v^3T^3}{6\hbar^3}-\cdots&=-i\sin\frac{vT}{\hbar}
\tag{6.82}
\end{align*}

\bigskip
\hrule

\bigskip
Consider equation (6.75).
\begin{multline*}
\lambda_{mn}(t_b,t_a)=\delta_{mn}
\exp\left(-\frac{i}{\hbar}E_m(t_b-t_a)\right)
\\
{}-\frac{i}{\hbar}
\int_{t_a}^{t_b}
\exp\left(-\frac{i}{\hbar}E_m(t_b-t_a)\right)
\sum_jV_{mj}(t_c)\lambda_{jn}(t_c,t_a)\,dt_c
\tag{6.75}
\end{multline*}

Let $E=E_1=E_2$ and $T=t_b-t_a$.
Then by (6.75) we have
\begin{align*}
\lambda_{11}(t_b,t_a)&=
\exp\left(-\frac{iET}{\hbar}\right)
-\frac{i}{\hbar}
\int_{t_a}^{t_b}
\exp\left(-\frac{i}{\hbar}E(t_b-t_c)\right)
v(t_c)\lambda_{21}(t_c,t_a)\,dt_c
\\
\lambda_{12}(t_b,t_a)&=
-\frac{i}{\hbar}
\int_{t_a}^{t_b}
\exp\left(-\frac{i}{\hbar}E(t_b-t_c)\right)
v(t_c)\lambda_{21}(t_c,t_a)\,dt_c
\\
\lambda_{21}(t_b,t_a)&=
-\frac{i}{\hbar}
\int_{t_a}^{t_b}
\exp\left(-\frac{i}{\hbar}E(t_b-t_c)\right)
v(t_c)\lambda_{12}(t_c,t_a)\,dt_c
\\
\lambda_{22}(t_b,t_a)&=
\exp\left(-\frac{iET}{\hbar}\right)
-\frac{i}{\hbar}
\int_{t_a}^{t_b}
\exp\left(-\frac{i}{\hbar}E(t_b-t_c)\right)
v(t_c)\lambda_{12}(t_c,t_a)\,dt_c
\end{align*}

\end{document}
