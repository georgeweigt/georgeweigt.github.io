\documentclass[12pt]{article}
\usepackage{amsmath}
\usepackage{amssymb}

\parindent=0pt

\begin{document}

8-1.
The amplitude to go from any state $\psi(x)$ to another
state $\chi(x)$ is the transition amplitude $\langle\chi|1|\psi\rangle$
as defined in equation (7.1).
Suppose $\psi(x)$ and $\chi(x)$ are expanded in terms of the orthogonal
functions $\phi_n(x)$, the energy solutions to the wave equation associated
with the kernel $K(b,a)$, as discussed in section 4-2. Thus
\begin{equation*}
\psi(x)=\sum_n\psi_n\phi_n(x),\qquad\chi(x)=\sum_n\chi_n\phi_n(x)
\tag{8.23}
\end{equation*}

Using the coefficients $\psi_n$ and $\chi_n$ and equation (4.59),
show that the transition amplitude can be written as
\begin{equation*}
\int_{-\infty}^\infty\int_{-\infty}^\infty
\chi^*(x_b)K(x_b,T;x_a,0)\psi(x_a)\,dx_a\,dx_b
=\sum_n\chi_n^*\psi_n\exp\left(-\frac{i}{\hbar}E_nT\right)
\tag{8.24}
\end{equation*}

This is equation (4.59).
\begin{equation*}
K(x_b,t_b;x_a,t_a)=\begin{cases}
\sum_{n=1}^\infty\phi_n(x_b)\phi_n^*(x_a)
\exp\left(-\frac{i}{\hbar}E_n(t_b-t_a)\right) & t_b>t_a
\\
0 & t_b<t_a
\end{cases}
\tag{4.59}
\end{equation*}

Let
\begin{equation*}
I=\int_{-\infty}^\infty\int_{-\infty}^\infty
\chi^*(x_b)K(x_b,T;x_a,0)\psi(x_a)\,dx_a\,dx_b
\end{equation*}

By (8.23) and (4.59) we have
\begin{multline*}
I=\sum_j\sum_n\sum_k\int_{-\infty}^\infty\int_{-\infty}^\infty
\\
\chi_j^*\phi_j^*(x_b)
\times
\phi_n(x_b)\phi_n^*(x_a)\exp\left(-\frac{i}{\hbar}E_nT\right)
\times
\psi_k\phi_k(x_a)
\,dx_a\,dx_b
\end{multline*}

The $\phi_j^*\phi_n$ and $\phi_n^*\phi_k$ vanish by orthogonality for $j\ne n$ and $k\ne n$, hence
\begin{multline*}
I=\sum_n\int_{-\infty}^\infty\int_{-\infty}^\infty
\\
\chi_n^*\phi_n^*(x_b)
\times
\phi_n(x_b)\phi_n^*(x_a)\exp\left(-\frac{i}{\hbar}E_nT\right)
\times
\psi_n\phi_n(x_a)
\,dx_a\,dx_b
\end{multline*}

The integrals over $\phi_n^*\phi_n$ are unity by normalization hence
\begin{equation*}
I=\sum_n\chi_n^*\exp\left(-\frac{i}{\hbar}E_nT\right)\psi_n
\end{equation*}

\end{document}
