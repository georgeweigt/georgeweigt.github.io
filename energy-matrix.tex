\documentclass[12pt]{article}
\usepackage{amsmath}
\parindent=0pt
\begin{document}

Consider a system of two electrons with the following eigenstates.
\begin{align*}
|0\rangle&=(1,0,0,0)\quad\text{no electrons}\\
|1\rangle&=(0,1,0,0)\quad\text{one electron in state $\phi_1$}\\
|2\rangle&=(0,0,1,0)\quad\text{one electron in state $\phi_2$}\\
|3\rangle&=(0,0,0,1)\quad\text{two electrons, one in state $\phi_1$, one in state $\phi_2$}
\end{align*}

Let electron states $\phi_n$ be modeled by a one dimensional box of length $L$.
\begin{equation*}
\phi_n(x)=\sqrt{\frac{2}{L}}\sin\left(\frac{n\pi x}{L}\right)
\end{equation*}

Let $|\xi\rangle$ be an arbitrary normalized state vector.
\begin{equation*}
|\xi\rangle=c_0|0\rangle+c_1|1\rangle+c_2|2\rangle+c_3|3\rangle,\quad\langle\xi|\xi\rangle=1
\end{equation*}

Let us determine an energy matrix $\hat{E}$ such that the expected
energy $\langle E \rangle$ in state $|\xi\rangle$ is
\begin{equation*}
\langle E\rangle=\langle\xi|\hat{E}|\xi\rangle
\end{equation*}

Energy matrix $\hat{E}$ is the sum of kinetic and potential energy matrices.
\begin{equation*}
\hat{E}=\hat{K}+\hat{V}
\end{equation*}

Kinetic energy matrix $\hat{K}$ can be computed from energy eigenvalues of the box model.
\begin{equation*}
\hat{K}=\begin{pmatrix}
0 & 0 & 0 & 0\\
0 & E_1 & 0 & 0\\
0 & 0 & E_2 & 0\\
0 & 0 & 0 & E_1+E_2
\end{pmatrix},
\quad
E_n=\frac{n^2\pi^2\hbar^2}{2mL^2}
\end{equation*}

Potential energy matrix $\hat{V}$ has one entry due to Coulomb interaction in the two electron state.
\begin{equation*}
\hat{V}=
\begin{pmatrix}
0 & 0 & 0 & 0\\
0 & 0 & 0 & 0\\
0 & 0 & 0 & 0\\
0 & 0 & 0 & V
\end{pmatrix}
\end{equation*}

Let $\psi(x,y)$ be the antisymmetrized wavefunction of the two electrons.
\begin{equation*}
\psi(x,y)=\frac{1}{\sqrt{2}}\big(\phi_1(x)\phi_2(y)-\phi_1(y)\phi_2(x)\big)
\end{equation*}

Then
\begin{equation*}
V=\frac{e^2}{4\pi\epsilon_0}\int_0^L\int_0^L
\psi^*(x,y)\left(\frac{1}{|x-y|}\right)\psi(x,y)\,dx\,dy
\end{equation*}

Let us now choose $L=10^{-9}$ meters and compute numerical values.
For $\hat{K}$ we have
\begin{equation*}
\hat{K}=\begin{pmatrix}
0 & 0 & 0 & 0\\
0 & 0.38\,\text{eV} & 0 & 0\\
0 & 0 & 1.50\,\text{eV} & 0\\
0 & 0 & 0 & 1.88\,\text{eV}
\end{pmatrix}
\end{equation*}

Computing $V$ by numerical integration we have
\begin{equation*}
\hat{V}=\begin{pmatrix}
0 & 0 & 0 & 0\\
0 & 0 & 0 & 0\\
0 & 0 & 0 & 0\\
0 & 0 & 0 & 4.67\,\text{eV}
\end{pmatrix}
\end{equation*}

Hence
\begin{equation*}
\hat{E}=\hat{K}+\hat{V}=\begin{pmatrix}
0 & 0 & 0 & 0\\
0 & 0.38\,\text{eV} & 0 & 0\\
0 & 0 & 1.50\,\text{eV} & 0\\
0 & 0 & 0 & 6.55\,\text{eV}
\end{pmatrix}
\end{equation*}

\end{document}
