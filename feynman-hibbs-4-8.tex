\documentclass[12pt]{article}
\usepackage{amsmath}
\usepackage{amssymb}

\parindent=0pt

\newcommand\INT{\int_{\mathbb R^3}}

\begin{document}

4-8.
Show from the fact that $H$ is hermitian that $E$ is real.
Hint: Choose $f=g=\phi$ in equation (4.30).

\bigskip
The Hamiltonian $H$ is an eigenfunction with corresponding eigenvalue $E$.
\begin{equation*}
H\phi(x)=E\phi(x)
\tag{4.42}
\end{equation*}

Since $H$ is hermitian we have from equation (4.30)
\begin{equation*}
\int_{-\infty}^\infty(Hg)^*f\,dx=\int_{-\infty}^\infty g^*(Hf)\,dx
\tag{4.30}
\end{equation*}

Substitute $\phi$ into $f$ and $g$.
\begin{equation*}
\int_{-\infty}^\infty(H\phi)^*\phi\,dx=\int_{-\infty}^\infty \phi^*(H\phi)\,dx
\end{equation*}

Replace $H$ with eigenvalue $E$.
\begin{equation*}
\int_{-\infty}^\infty (E\phi)^*\phi\,dx=\int_{-\infty}^\infty \phi^*E\phi\,dx
\end{equation*}

Since $E$ is a constant it can be factored out of the integrands.
\begin{equation*}
E^*\int_{-\infty}^\infty\phi^*\phi\,dx=E\int_{-\infty}^\infty \phi^*\phi\,dx
\end{equation*}

The integrals are identical hence
\begin{equation*}
E^*=E
\end{equation*}

\end{document}
