\documentclass[12pt]{article}
\usepackage{amsmath}

\parindent=0pt

\begin{document}

Feynman and Hibbs problem 3-12

\bigskip
If the wave function for a harmonic oscillator (at $t=0$) is
\begin{equation*}
\psi(x,0)=\exp\left(-\frac{m\omega}{2\hbar}(x-a)^2\right)
\end{equation*}

then, using equation (3.42) and the results of problem 3-8, show that
\begin{multline*}
\psi(x,T)=
\\
\exp\left(
-\frac{i\omega T}{2}-\frac{m\omega}{2\hbar}
\left(x^2-2ax\exp(-i\omega T)+a^2\cos(\omega T)\exp(-i\omega T)\right)
\right)
\end{multline*}

and find the probability density $|\psi|^2$.

\bigskip
Adapted from equation (3.42)
\begin{equation*}
\psi(x,T)=\int_{-\infty}^\infty K(x,T;x_c,0)\psi(x_c,0)\,dx_c
\end{equation*}

Adapted from problem 3-8
\begin{equation*}
K=\left(\frac{m\omega}{2\pi i\hbar\sin(\omega T)}\right)^{1/2}
\exp\left(
\frac{im\omega}{2\hbar\sin(\omega T)}\left((x^2+x_c^2)\cos(\omega T)-2xx_c\right)
\right)
\end{equation*}

Hence
\begin{align*}
\psi(x,T)&=\left(\frac{m\omega}{2\pi i\hbar\sin(\omega T)}\right)^{1/2}
\\
&\qquad{}\times
\int_{-\infty}^\infty \exp\left(
\frac{im\omega}{2\hbar\sin(\omega T)}\left((x^2+x_c^2)\cos(\omega T)-2xx_c\right)
\right)
\\
&\qquad\qquad\qquad{}\times
\exp\left(-\frac{m\omega}{2\hbar}(x_c-a)^2\right)\,dx_c
\end{align*}

Rewrite as
\begin{equation*}
\psi(x,T)=\left(\frac{m\omega}{2\pi i\hbar\sin(\omega T)}\right)^{1/2}
\int_{-\infty}^\infty\exp(Ax_c^2+Bx_c+C)
\tag{1}
\end{equation*}

where
\begin{align*}
A&=\frac{m\omega}{2\hbar}\left(\frac{i\cos(\omega T)}{\sin(\omega T)}-1\right)
\tag{2}
\\
B&=\frac{m\omega}{\hbar}\left(a-\frac{ix}{\sin(\omega T)}\right)
\tag{3}
\\
C&=\frac{m\omega}{2\hbar}\left(\frac{ix^2\cos(\omega T)}{\sin(\omega T)}-a^2\right)
\tag{4}
\end{align*}

Solve the integral.
\begin{gather*}
\int_{-\infty}^\infty\exp(Ax_c^2+Bx_c+C)
=\left(-\frac{\pi}{A}\right)^{1/2}
\exp\left(-\frac{B^2}{4A}+C\right)
\\
-\frac{\pi}{A}
=-\frac{2\pi\hbar\sin(\omega T)}{im\omega\cos(\omega T)-m\omega\sin(\omega T)}
\tag{5}
\\
-\frac{B^2}{4A}+C
=-\frac{m\omega}{2\hbar}
\left(x^2-2ax\exp(-i\omega T)+a^2\cos(\omega T)\exp(-i\omega T)\right)
\tag{6}
\end{gather*}

It can be shown that
\begin{equation*}
\underset{\substack{\\[1ex]\text{from equation (1)}}}
{\frac{m\omega}{2\pi i\hbar\sin(\omega T)}}
\times
\left(-\frac{\pi}{A}\right)
=\exp(-i\omega T)
\tag{7}
\end{equation*}

Hence from equation (1)
\begin{multline*}
\psi(x,T)=\exp\left(-\frac{i\omega T}{2}\right)
\\
{}\times\exp\left(
-\frac{m\omega}{2\hbar}
\left(x^2-2ax\exp(-i\omega T)+a^2\cos(\omega T)\exp(-i\omega T)\right)
\right)
\end{multline*}

\end{document}