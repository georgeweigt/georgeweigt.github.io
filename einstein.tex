\documentclass[12pt]{article}
\usepackage{amsmath}
\parindent=0pt

\newcommand\BNM{B_{nm}} % absorption
\newcommand\BMN{B_{mn}} % induced emission
\newcommand\AMN{A_{mn}} % spontaneous emission
\newcommand\RHO{\rho}

\newcommand\ABSORPTION{\substack{\\[1ex] \text{absorption}}}
\newcommand\INDUCED{\substack{\\[1ex] \text{induced}\\ \text{emission}}}
\newcommand\SPONTANEOUS{\substack{\\[1ex] \text{spontaneous}\\ \text{emission}}}

\begin{document}

%In his 1917 paper, ``On the Quantum Theory of Radiation,''
%Einstein uses the following argument to derive Planck's law.
%The argument requires the existence of induced emission, a process which was a theoretical discovery by Einstein.
%Prior to Einstein, no one was aware that induced emission existed.

Consider a gas at temperature $T$.
Let $N$ be the number of gas molecules
and let $N_n$ be the number of molecules with energy $E_n$.
By the Maxwell-Boltzmann distribution we have
\begin{equation*}
\frac{N_n}{N}=p_n\exp\left(-\frac{E_n}{kT}\right)
\tag{1}
\end{equation*}

\noindent
Coefficient $p_n$ is a statistical weighting factor that does not depend on $T$.

\bigskip
\noindent
Let us now consider the processes by which an atom or molecule transitions between energy levels.
The processes are absorption, induced emission, and spontaneous emission.
Let $E_m$ and $E_n$ be energy levels such that $E_m>E_n$.
Let $N_{m\rightarrow n}/\Delta t$ be the number of atoms or molecules that transition from energy level $E_m$ to $E_n$ in time $\Delta t$.
Finally, let $\BNM$, $\BMN$, and $\AMN$ be coefficients such that
\begin{equation*}
\frac{N_{n\rightarrow m}}{\Delta t}
=\underset{\ABSORPTION}{\RHO\BNM N_n},
\qquad
\frac{N_{m\rightarrow n}}{\Delta t}
=\underset{\INDUCED}{\RHO\BMN N_m}
+
\underset{\SPONTANEOUS}{\AMN N_m}
\end{equation*}

\noindent
Absorption and induced emission are proportional to radiant energy density $\rho$.
The $A$ and $B$ coefficients do not depend on $T$.

\bigskip
\noindent
At equilibrium, the transition rates are equal.
\begin{equation*}
\frac{N_{n\rightarrow m}}{\Delta t}=\frac{N_{m\rightarrow n}}{\Delta t}
\end{equation*}

\noindent
Hence
\begin{equation*}
\underset{\ABSORPTION}{\RHO\BNM N_n}
=\underset{\INDUCED}{\RHO\BMN N_m}
+\underset{\SPONTANEOUS}{\AMN N_m}
\end{equation*}

\noindent
Divide through by $N$.
\begin{equation*}
\underset{\ABSORPTION}{\RHO\BNM \frac{N_n}{N}}
=\underset{\INDUCED}{\RHO\BMN \frac{N_m}{N}}
+\underset{\SPONTANEOUS}{\AMN \frac{N_m}{N}}
\end{equation*}

\noindent
Then by the Maxwell-Boltzmann distribution (1) we have
\begin{equation*}
\underset{\ABSORPTION}{\RHO\BNM p_n\exp\left(-\frac{E_n}{kT}\right)}
=\underset{\INDUCED}{\RHO\BMN p_m\exp\left(-\frac{E_m}{kT}\right)}
+\underset{\SPONTANEOUS}{\AMN p_m\exp\left(-\frac{E_m}{kT}\right)}
\tag{2}
\end{equation*}

\noindent
Multiply both sides by $\exp(E_m/kT)$.
\begin{equation*}
\underset{\ABSORPTION}{\RHO\BNM p_n\exp\left(\frac{E_m-E_n}{kT}\right)}
=\underset{\INDUCED}{\RHO\BMN p_m}
+\underset{\SPONTANEOUS}{\AMN p_m}
\end{equation*}

\noindent
Note that for increasing $T$ we have
\begin{equation*}
\lim_{T\rightarrow\infty}\exp\left(\frac{E_m-E_n}{kT}\right)=1
\end{equation*}

\noindent
It follows that for $T\rightarrow\infty$ the equilibrium formula is
\begin{equation*}
\RHO\BNM p_n
=\RHO\BMN p_m
+\AMN p_m
\end{equation*}

\noindent
Divide through by $\RHO$.
\begin{equation*}
\BNM p_n=\BMN p_m+\frac{\AMN p_m}{\RHO}
\end{equation*}

\noindent
Energy density $\RHO$ increases with temperature $T$
hence $\AMN p_m/\RHO$ vanishes for $T\rightarrow\infty$ leaving
\begin{equation*}
\BNM p_n=\BMN p_m
\tag{3}
\end{equation*}

\noindent
Einstein reasoned that equation (3) is true in general based on the assumption that
the factors involved do not depend on $T$.
Hence we can substitute $B_{mn}p_m$ into the absorption term and write
\begin{equation*}
\underset{\ABSORPTION}{\RHO\BMN p_m\exp\left(\frac{E_m-E_n}{kT}\right)}
=\underset{\INDUCED}{\RHO\BMN p_m}
+\underset{\SPONTANEOUS}{\AMN p_m}
\end{equation*}

\noindent
Divide both sides by $\BMN p_m$.
\begin{equation*}
%\underset{\ABSORPTION}{\RHO\exp\left(\frac{E_m-E_n}{kT}\right)}
%=\underset{\INDUCED}{\RHO}+\underset{\SPONTANEOUS}{\frac{\AMN}{\BMN}}
\RHO\exp\left(\frac{E_m-E_n}{kT}\right)
=\RHO+\frac{\AMN}{\BMN}
\end{equation*}

\noindent
Rearrange terms.
\begin{equation*}
%\underset{\ABSORPTION}{\RHO\exp\left(\frac{E_m-E_n}{kT}\right)}
%-\underset{\INDUCED}{\RHO}=\underset{\SPONTANEOUS}{\frac{\AMN}{\BMN}}
\RHO\exp\left(\frac{E_m-E_n}{kT}\right)
-\RHO=\frac{\AMN}{\BMN}
\end{equation*}

\noindent
Factor out $\RHO$.
\begin{equation*}
\RHO\left(\exp\left(\frac{E_m-E_n}{kT}\right)-1\right)
=\frac{\AMN}{\BMN}
\end{equation*}

\noindent
Solve for $\RHO$.
\begin{equation*}
\RHO
=\frac{\AMN}{\BMN}\,\frac{1}{\exp\left(\frac{E_m-E_n}{kT}\right)-1}
\end{equation*}

\noindent
We now consider the case of large exponentials such that
\begin{equation*}
\exp\left(\frac{E_m-E_n}{kT}\right)\approx\exp\left(\frac{E_m-E_n}{kT}\right)-1
\end{equation*}

Hence for large exponentials
\begin{equation*}
\RHO\approx
\frac{\AMN}{\BMN}\exp\left(-\frac{E_m-E_n}{kT}\right)
\end{equation*}

\noindent
By equivalence with Wien's law (which is accurate for large $\nu$) we have
\begin{equation*}
\RHO=\frac{2h\nu^3}{c^2}\exp\left(-\frac{h\nu}{kT}\right)
\end{equation*}

Hence
\begin{equation*}
\frac{\AMN}{\BMN}=\frac{2h\nu^3}{c^2}
\tag{4}
\end{equation*}
and
\begin{equation*}
E_m-E_n=h\nu
\end{equation*}

\noindent
Then by substitution we obtain Planck's law.
\begin{align*}
\RHO
&=\frac{\AMN}{\BMN}\,\frac{1}{\exp\left(\frac{E_m-E_n}{kT}\right)-1}
\\[1.5ex]
&=\frac{2h\nu^3}{c^2}\,\frac{1}{\exp\left(\frac{h\nu}{kT}\right)-1}
\end{align*}

%Note that Wien's law is an empirical formula.

\noindent
Let us now consider the values of the $A$ and $B$ coefficients.
The coefficient for spontaneous emission can be computed from quantum mechanics.
For example, for hydrogen transition $2p\rightarrow1s$ we have
\begin{equation*}
A_{21}=\frac{16e^8}{6561\varepsilon_0^4h^4c^3a_0}
=6.26\times10^8\,\text{second}^{-1}
\end{equation*}

\noindent
The coefficient for induced emission can be obtained from equation (4).
\begin{equation*}
\BMN=\frac{c^2}{2h\nu^3}\,\AMN
\end{equation*}

\noindent
The coefficient for absorption can be computed from equation (3).
\begin{equation*}
\BNM=\frac{p_m}{p_n}\,\BMN
\end{equation*}

\noindent
The ratio $p_m/p_n$ is equal to $g_m/g_n$ where $g$ is the multiplicity
for quantum numbers $\ell$ and $m_s$.
\begin{equation*}
g=(2\ell+1)(2m_s+1)
\end{equation*}

\noindent
Hence for hydrogen $2p\rightarrow1s$ we have
\begin{align*}
g_1&=2\qquad(\ell=0,\,m_s=1/2)
\\
g_2&=6\qquad(\ell=1,\,m_s=1/2)
\end{align*}

\noindent
(Recall that $\ell=0$ for orbital $s$ and $\ell=1$ for orbital $p$.)

\end{document}
