\documentclass[12pt]{article}
\usepackage[margin=2cm]{geometry}
\usepackage{amsmath}

\newcommand\BNM{B_{nm}} % absorption
\newcommand\BMN{B_{mn}} % induced emission
\newcommand\AMN{A_{mn}} % spontaneous emission
\newcommand\RHO{\rho(\nu,T)}

\newcommand\ABSORPTION{\substack{\phantom{0}\\ \text{absorption}\\ \varepsilon_n\rightarrow\varepsilon_m}}
\newcommand\INDUCED{\substack{\phantom{0}\\ \text{induced emission}\\ \varepsilon_m\rightarrow\varepsilon_n}}
\newcommand\SPONTANEOUS{\substack{\phantom{0}\\ \text{spontaneous emission}\\ \varepsilon_m\rightarrow\varepsilon_n}}

\begin{document}

\noindent
In his 1917 paper, ``On the Quantum Theory of Radiation,''
Einstein uses the following argument to derive Planck's law.
The argument requires induced emission, a process which was a theoretical discovery by Einstein.
Prior to Einstein, no one was aware that induced emission existed.

\bigskip
\noindent
Consider a gas at temperature $T$.
Let $N$ be the number of molecules in the gas
and let $N_n$ be the number of molecules with internal energy $\varepsilon_n$.
By the Maxwell-Boltzmann distribution we have
\begin{equation*}
\frac{N_n}{N}=p_n\exp\left(-\frac{\varepsilon_n}{kT}\right)
\end{equation*}
where $k$ is Boltzmann's constant.
The coefficient $p_n$ is a statistical weighting factor that does not depend on $T$.

\bigskip
\noindent
Let us now consider the processes by which a molecule transitions between energy levels.
The processes are absorption, induced emission, and spontaneous emission.
Let $\varepsilon_m$ be an energy level such that $\varepsilon_m>\varepsilon_n$.
Let $\BNM$, $\BMN$, and $\AMN$ be coefficients of transition rates such that
\begin{equation*}
\underset{\ABSORPTION}{\frac{dN_n}{dt}=\BNM N_n \RHO}
\qquad
\underset{\INDUCED}{\frac{dN_m}{dt}=\BMN N_m \RHO}
\qquad
\underset{\SPONTANEOUS}{\frac{dN_m}{dt}=\AMN N_m}
\end{equation*}
Absorption and induced emission are proportional to $\RHO$
which is the radiant energy density of the gas
as a function of radiant frequency $\nu$ and temperature $T$.
The $A$ and $B$ coefficients are presumed to not depend on temperature $T$.

\bigskip
\noindent
At equilibrium, transition rates between $\varepsilon_m$ and $\varepsilon_n$ are equal.
\begin{equation*}
\underset{\ABSORPTION}{\BNM N_n \RHO}
=\underset{\INDUCED}{\BMN N_m \RHO}
+\underset{\SPONTANEOUS}{\AMN N_m}
\end{equation*}

\noindent
Divide through by $N$ to obtain
\begin{equation*}
\underset{\ABSORPTION}{\BNM p_n \RHO\exp\left(-\frac{\varepsilon_n}{kT}\right)}
=\underset{\INDUCED}{\BMN p_m \RHO\exp\left(-\frac{\varepsilon_m}{kT}\right)}
+\underset{\SPONTANEOUS}{\AMN p_m\exp\left(-\frac{\varepsilon_m}{kT}\right)}
\tag{1}
\end{equation*}

\noindent
Multiply both sides by $\exp(\varepsilon_m/kT)$.
\begin{equation*}
\underset{\ABSORPTION}{\BNM p_n \RHO\exp\left(\frac{\varepsilon_m-\varepsilon_n}{kT}\right)}
=\underset{\INDUCED}{\BMN p_m \RHO}
+\underset{\SPONTANEOUS}{\AMN p_m}
\end{equation*}

\noindent
Note that for increasing $T$ we have
\begin{equation*}
\lim_{T\rightarrow\infty}\exp\left(\frac{\varepsilon_m-\varepsilon_n}{kT}\right)=1
\end{equation*}

\noindent
It follows that for $T\rightarrow\infty$ the equilibrium formula is
\begin{equation*}
\BNM p_n \RHO
=\BMN p_m \RHO
+\AMN p_m
\end{equation*}

\noindent
Divide through by $\RHO$.
\begin{equation*}
\BNM p_n=\BMN p_m+\frac{\AMN p_m}{\RHO}
\end{equation*}

\noindent
Energy density $\RHO$ increases with temperature $T$
hence $\AMN p_m/\RHO$ vanishes for $T\rightarrow\infty$ leaving
\begin{equation*}
\BNM p_n=\BMN p_m
\tag{2}
\end{equation*}

\noindent
Einstein reasoned that equation (2) is true in general based on the assumption that
the factors involved do not depend on $T$.
By substitution in the absorption term we can now eliminate $\BNM p_n$ and obtain
\begin{equation*}
\underset{\ABSORPTION}{\BMN p_m \RHO\exp\left(\frac{\varepsilon_m-\varepsilon_n}{kT}\right)}
=\underset{\INDUCED}{\BMN p_m \RHO}
+\underset{\SPONTANEOUS}{\AMN p_m}
\end{equation*}

\noindent
Divide both sides by $\BMN p_m$.
\begin{equation*}
\underset{\ABSORPTION}{\RHO\exp\left(\frac{\varepsilon_m-\varepsilon_n}{kT}\right)}
=\underset{\INDUCED}{\RHO}+\underset{\SPONTANEOUS}{\frac{\AMN}{\BMN}}
\end{equation*}

\noindent
Rearrange terms.
\begin{equation*}
\underset{\ABSORPTION}{\RHO\exp\left(\frac{\varepsilon_m-\varepsilon_n}{kT}\right)}
-\underset{\INDUCED}{\RHO}=\underset{\SPONTANEOUS}{\frac{\AMN}{\BMN}}
\end{equation*}

\noindent
Factor out $\RHO$.
\begin{equation*}
\RHO\left(\exp\left(\frac{\varepsilon_m-\varepsilon_n}{kT}\right)-1\right)
=\frac{\AMN}{\BMN}
\end{equation*}

\noindent
Solve for $\RHO$.
\begin{equation*}
\RHO
=\frac{\AMN}{\BMN}\,\frac{1}{\exp\left(\frac{\varepsilon_m-\varepsilon_n}{kT}\right)-1}
\end{equation*}

\noindent
We now consider the limit of $\RHO$ as $\varepsilon_m-\varepsilon_n\rightarrow\infty$.
\begin{equation*}
\lim_{\varepsilon_m-\varepsilon_n\rightarrow\infty}\RHO
=\frac{\AMN}{\BMN}\exp\left(-\frac{\varepsilon_m-\varepsilon_n}{kT}\right)
\end{equation*}

\noindent
Then by equivalence with Wien's law (which is accurate for large $\nu$)
\begin{equation*}
\rho_\text{wien}(\nu,T)=\frac{2h\nu^3}{c^2}\exp\left(-\frac{h\nu}{kT}\right)
\end{equation*}
we have
\begin{equation*}
\frac{\AMN}{\BMN}=\frac{2h\nu^3}{c^2}
\tag{3}
\end{equation*}
and
\begin{equation*}
\varepsilon_m-\varepsilon_n=h\nu
\end{equation*}

\noindent
Then by substitution we obtain Planck's law.
\begin{align*}
\RHO
&=\frac{\AMN}{\BMN}\,\frac{1}{\exp\left(\frac{\varepsilon_m-\varepsilon_n}{kT}\right)-1}
\\[2ex]
&=\frac{2h\nu^3}{c^2}\,\frac{1}{\exp\left(\frac{h\nu}{kT}\right)-1}
\end{align*}

\bigskip
\noindent
The coefficient for spontaneous emission can be computed from quantum mechanics.
For example, for hydrogen we have
\begin{equation*}
A_{21}=\frac{e^{10}m_e}{26244\,\pi^5\varepsilon_0^5\hbar^6 c^3}
=6.27\times10^8\,\text{second}^{-1}
\end{equation*}

\noindent
The coefficient for induced emission can be obtained from equation (3).
\begin{equation*}
\BMN=\frac{c^2}{2h\nu^3}\,\AMN
\end{equation*}

\noindent
The coefficient for absorption can be computed from equation (2).
\begin{equation*}
\BNM=\frac{p_m}{p_n}\,\BMN
\end{equation*}

\noindent
The ratio $p_m/p_n$ is equal to $g_m/g_n$
where $g_m$ is the multiplicity (number of degenerate states)
associated with energy level $m$.
Hence $p_m/p_n$ is determined by the atomic species.

\end{document}
