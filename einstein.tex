\documentclass[12pt]{article}
\usepackage[margin=2cm]{geometry}
\usepackage{amsmath}

\newcommand\BNM{B_n} % absorption
\newcommand\BMN{B_m} % induced emission
\newcommand\AMN{A_m} % spontaneous emission

\newcommand\ABSORPTION{\substack{\phantom{0}\\ \text{absorption}\\ \varepsilon_n\rightarrow\varepsilon_m}}
\newcommand\INDUCED{\substack{\phantom{0}\\ \text{induced emission}\\ \varepsilon_m\rightarrow\varepsilon_n}}
\newcommand\SPONTANEOUS{\substack{\phantom{0}\\ \text{spontaneous emission}\\ \varepsilon_m\rightarrow\varepsilon_n}}

\begin{document}

\noindent
Consider a gas at temperature $T$.
Let $N$ be the number of molecules in the gas
and let $N_n$ be the number of molecules with internal energy $\varepsilon_n$.
By the Maxwell-Boltzmann distribution we have
\begin{equation*}
\frac{N_n}{N}=p_n\exp\left(-\frac{\varepsilon_n}{kT}\right)
\end{equation*}
where $k$ is Boltzmann's constant.
The coefficient $p_n$ is a statistical weighting factor that does not depend on $T$.

\bigskip
\noindent
Let us now consider the processes by which a molecule transitions between energy levels.
The processes are absorption, induced emission, and spontaneous emission.
Let $\varepsilon_m$ be an energy level such that $\varepsilon_m>\varepsilon_n$.
Let $\BNM$, $\BMN$, and $\AMN$ be coefficients of transition rates such that
\begin{equation*}
\underset{\ABSORPTION}{\frac{dN_n}{dt}=\BNM\rho(\nu)N_n}
\qquad
\underset{\INDUCED}{\frac{dN_m}{dt}=\BMN\rho(\nu)N_m}
\qquad
\underset{\SPONTANEOUS}{\frac{dN_m}{dt}=\AMN N_m}
\end{equation*}
Absorption and induced emission are proportional to $\rho(\nu)$
which is the radiant energy density of the gas
as a function of radiant frequency $\nu$.
The $A$ and $B$ coefficients are presumed to not depend on temperature $T$.

\bigskip
\noindent
At equilibrium, transition rates between $\varepsilon_m$ and $\varepsilon_n$ are equal.
\begin{equation*}
\underset{\ABSORPTION}{\BNM\rho(\nu)N_n}
=\underset{\INDUCED}{\BMN\rho(\nu)N_m}
+\underset{\SPONTANEOUS}{\AMN N_m}
\end{equation*}

\noindent
Divide through by $N$ to obtain
\begin{equation*}
\underset{\ABSORPTION}{\BNM\rho(\nu)p_n\exp\left(-\frac{\varepsilon_n}{kT}\right)}
=\underset{\INDUCED}{\BMN\rho(\nu)p_m\exp\left(-\frac{\varepsilon_m}{kT}\right)}
+\underset{\SPONTANEOUS}{\AMN p_m\exp\left(-\frac{\varepsilon_m}{kT}\right)}
\end{equation*}

\noindent
Note that for increasing $T$ we have
\begin{equation*}
\lim_{T\rightarrow\infty}\exp\left(-\frac{\varepsilon_n}{kT}\right)=1
\end{equation*}

\noindent
It follows that for $T\rightarrow\infty$ the equilibrium formula is
\begin{equation*}
\BNM\rho(\nu)p_n
=\BMN\rho(\nu)p_m
+\AMN p_m
\end{equation*}

\noindent
Divide through by $\rho(\nu)$.
\begin{equation*}
\BNM p_n=\BMN p_m+\AMN p_m/\rho(\nu)
\end{equation*}

\noindent
Energy density $\rho(\nu)$ increases with temperature $T$
hence $\AMN p_m/\rho(\nu)$ vanishes for $T\rightarrow\infty$ leaving
\begin{equation*}
\BNM p_n=\BMN p_m
\end{equation*}

\noindent
Einstein reasoned that the above relation is true in general based on the assumption that
the factors involved do not depend on $T$.
By substitution in the absorption term we now have
\begin{equation*}
\underset{\ABSORPTION}{\BMN\rho(\nu)p_m\exp\left(-\frac{\varepsilon_n}{kT}\right)}
=\underset{\INDUCED}{\BMN\rho(\nu)p_m\exp\left(-\frac{\varepsilon_m}{kT}\right)}
+\underset{\SPONTANEOUS}{\AMN p_m\exp\left(-\frac{\varepsilon_m}{kT}\right)}
\end{equation*}

\noindent
Divide through by $\BMN p_m$ and rearrange terms.
\begin{equation*}
\underset{\ABSORPTION}{\rho(\nu)\exp\left(-\frac{\varepsilon_n}{kT}\right)}
-\underset{\INDUCED}{\rho(\nu)\exp\left(-\frac{\varepsilon_m}{kT}\right)}
=\underset{\SPONTANEOUS}{(\AMN/\BMN)\exp\left(-\frac{\varepsilon_m}{kT}\right)}
\end{equation*}

\noindent
Solve for energy density $\rho(\nu)$.
\begin{equation*}
\rho(\nu)=\frac
{\displaystyle(\AMN/\BMN)\exp\left(-\frac{\varepsilon_m}{kT}\right)}
{\displaystyle\exp\left(-\frac{\varepsilon_n}{kT}\right)-\exp\left(-\frac{\varepsilon_m}{kT}\right)}
=\frac{\AMN/\BMN}{\displaystyle\exp\left(\frac{\varepsilon_m-\varepsilon_n}{kT}\right)-1}
\end{equation*}

\noindent
From Wien's law
$\rho(\nu)=\alpha\nu^3\exp(-h\nu/kT)$ which is accurate for large $\nu$, we have
\begin{equation*}
\AMN/\BMN=\alpha\nu^3
\end{equation*}
and
\begin{equation*}
\varepsilon_m-\varepsilon_n=h\nu
\end{equation*}

\noindent
Planck's law follows directly.
\begin{equation*}
\rho(\nu)=\frac{\alpha\nu^3}{\displaystyle\exp\left(\frac{h\nu}{kT}\right)-1}
\end{equation*}

\end{document}
