\documentclass[12pt]{article}
\usepackage[margin=1in]{geometry}
\usepackage{amsmath}
\parindent=0pt
\begin{document}

\section*{Dirac equation}

This is Dirac's equation.
\begin{equation*}
i\hbar\left(
\frac{\gamma^0}{c}\frac{\partial}{\partial t}+
\gamma^1\frac{\partial}{\partial x}+
\gamma^2\frac{\partial}{\partial y}+
\gamma^3\frac{\partial}{\partial z}
\right)\psi
=mc\psi
\end{equation*}

Gamma matrices for the ``Dirac representation'' are
\begin{align*}
\gamma^0&=\begin{pmatrix}1&0&0&0\\0&1&0&0\\0&0&-1&0\\0&0&0&-1\end{pmatrix}
& \gamma^1&=\begin{pmatrix}0&0&0&1\\0&0&1&0\\0&-1&0&0\\-1&0&0&0\end{pmatrix}
\\
\\
\gamma^2&=\begin{pmatrix}0&0&0&-i\\0&0&i&0\\0&i&0&0\\-i&0&0&0\end{pmatrix}
& \gamma^3&=\begin{pmatrix}0&0&1&0\\0&0&0&-1\\-1&0&0&0\\0&1&0&0\end{pmatrix}
\end{align*}

Let $\phi$ be the field
\begin{equation*}
\phi=p_xx+p_yy+p_zz-Et
\end{equation*}
where
\begin{equation*}
E=\sqrt{p_x^2c^2+p_y^2c^2+p_z^2c^2+m^2c^4}
\end{equation*}

The four positive wave solutions to the Dirac equation are
\begin{align*}
\psi_1&=\begin{pmatrix}E/c+mc\\0\\p_z\\p_x+ip_y\end{pmatrix}
\exp\left(\frac{i\phi}{\hbar}\right)
& \psi_2&=\begin{pmatrix}0\\E/c+mc\\p_x-ip_y\\-p_z\end{pmatrix}
\exp\left(\frac{i\phi}{\hbar}\right)
\\
\\
\psi_3&=\begin{pmatrix}p_z\\p_x+ip_y\\E/c-mc\\0\end{pmatrix}
\exp\left(\frac{i\phi}{\hbar}\right)
& \psi_4&=\begin{pmatrix}p_x-ip_y\\-p_z\\0\\E/c-mc\end{pmatrix}
\exp\left(\frac{i\phi}{\hbar}\right)
\end{align*}

The four negative wave solutions are
\begin{align*}
\psi_5&=\begin{pmatrix}E/c-mc\\0\\p_z\\p_x+ip_y\end{pmatrix}
\exp\left(-\frac{i\phi}{\hbar}\right)
& \psi_6&=\begin{pmatrix}0\\E/c-mc\\p_x-ip_y\\-p_z\end{pmatrix}
\exp\left(-\frac{i\phi}{\hbar}\right)
\\
\\
\psi_7&=\begin{pmatrix}p_z\\p_x+ip_y\\E/c+mc\\0\end{pmatrix}
\exp\left(-\frac{i\phi}{\hbar}\right)
& \psi_8&=\begin{pmatrix}p_x-ip_y\\-p_z\\0\\E/c+mc\end{pmatrix}
\exp\left(-\frac{i\phi}{\hbar}\right)
\end{align*}

Negative wave solutions flip the sign of the $mc$ term.

\bigskip
The following solutions are used for fermion fields.
\begin{center}
\begin{tabular}{ll}
$\psi_1$ & fermion, spin up\\
$\psi_2$ & fermion, spin down\\
\\
$\psi_7$ & anti-fermion, spin up\\
$\psi_8$ & anti-fermion, spin down
\end{tabular}
\end{center}

Here is a check of physical units.
The momenta $p_x$, $p_y$, and $p_z$ have units of
\begin{equation*}
\frac{\text{kilogram}\,\text{meter}}{\text{second}}
\end{equation*}

Hence
\begin{equation*}
p_xx\propto\frac{\text{kilogram}\,\text{meter}^2}{\text{second}}
\end{equation*}

For the time-dependent term
\begin{equation*}
Et\propto\frac{\text{kilogram}\,\text{meter}^2}{\text{second}^2}\times\text{second}
=\frac{\text{kilogram}\,\text{meter}^2}{\text{second}}
\end{equation*}

We have for the reduced Planck constant
\begin{equation*}
\hbar\propto\frac{\text{kilogram}\,\text{meter}^2}{\text{second}}
\end{equation*}

Hence $\phi/\hbar$ is dimensionless as required by the exponential function.
\begin{equation*}
\frac{p_xx-Et}{\hbar}\propto\frac{\text{kilogram}\,\text{meter}^2}{\text{second}}
\times\frac{\text{second}}{\text{kilogram}\,\text{meter}^2}=1
\end{equation*}

The derivatives introduce inverse units.
\begin{equation*}
\frac{\partial\psi}{\partial t}\propto\frac{1}{\text{second}}
\qquad
\frac{\partial\psi}{\partial x}\propto\frac{1}{\text{meter}}
\end{equation*}

Hence
\begin{equation*}
\frac{\hbar}{c}\frac{\partial\psi}{\partial t}
\propto\frac{\text{kilogram}\,\text{meter}}{\text{second}}
\end{equation*}

and
\begin{equation*}
\hbar\frac{\partial\psi}{\partial x}\propto\frac{\text{kilogram}\,\text{meter}}{\text{second}}
\end{equation*}

The resulting units match the right-hand side of the Dirac equation.
\begin{equation*}
mc\propto\frac{\text{kilogram}\,\text{meter}}{\text{second}}
\end{equation*}

\end{document}
