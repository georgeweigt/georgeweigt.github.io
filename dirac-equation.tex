\documentclass[12pt]{article}
\usepackage[margin=2cm]{geometry}
\usepackage{amsmath}

\begin{document}

\noindent
This is the Dirac equation.
\begin{equation*}
i\hbar\left(\frac{1}{c}
\gamma^0\frac{\partial}{\partial t}+
\gamma^1\frac{\partial}{\partial x}+
\gamma^2\frac{\partial}{\partial y}+
\gamma^3\frac{\partial}{\partial z}
\right)\psi
=mc\psi
\end{equation*}

\noindent
The following set of gamma matrices are known as the ``Dirac representation.''
{\small
\begin{gather*}
\gamma^0=\begin{pmatrix}1&0&0&0\\0&1&0&0\\0&0&-1&0\\0&0&0&-1\end{pmatrix}\quad
\gamma^1=\begin{pmatrix}0&0&0&1\\0&0&1&0\\0&-1&0&0\\-1&0&0&0\end{pmatrix}\quad
\gamma^2=\begin{pmatrix}0&0&0&-i\\0&0&i&0\\0&i&0&0\\-i&0&0&0\end{pmatrix}\quad
\gamma^3=\begin{pmatrix}0&0&1&0\\0&0&0&-1\\-1&0&0&0\\0&1&0&0\end{pmatrix}
\end{gather*}
}

\noindent
Let $\phi$ be the field
\begin{equation*}
\phi(x,y,z,t)=p_xx+p_yy+p_zz-Et
\end{equation*}
where
\begin{equation*}
E=\sqrt{p_x^2c^2+p_y^2c^2+p_z^2c^2+m^2c^4}
\end{equation*}

\noindent
The four positive wave solutions to the Dirac equation are
{\small
\begin{equation*}
\psi_1=\begin{pmatrix}E+mc^2\\0\\p_zc\\p_xc+ip_yc\end{pmatrix}
\exp\left(\frac{i\phi}{\hbar}\right)
\qquad
\psi_2=\begin{pmatrix}0\\E+mc^2\\p_xc-ip_yc\\-p_zc\end{pmatrix}
\exp\left(\frac{i\phi}{\hbar}\right)
\end{equation*}
\begin{equation*}
\psi_3=\begin{pmatrix}p_zc\\p_xc+ip_yc\\E-mc^2\\0\end{pmatrix}
\exp\left(\frac{i\phi}{\hbar}\right)
\qquad
\psi_4=\begin{pmatrix}p_xc-ip_yc\\-p_zc\\0\\E-mc^2\end{pmatrix}
\exp\left(\frac{i\phi}{\hbar}\right)
\end{equation*}
}

\noindent
The four negative wave solutions are
{\small
\begin{equation*}
\psi_5=\begin{pmatrix}E-mc^2\\0\\p_zc\\p_xc+ip_yc\end{pmatrix}
\exp\left(-\frac{i\phi}{\hbar}\right)
\qquad
\psi_6=\begin{pmatrix}0\\E-mc^2\\p_xc-ip_yc\\-p_zc\end{pmatrix}
\exp\left(-\frac{i\phi}{\hbar}\right)
\end{equation*}
\begin{equation*}
\psi_7=\begin{pmatrix}p_zc\\p_xc+ip_yc\\E+mc^2\\0\end{pmatrix}
\exp\left(-\frac{i\phi}{\hbar}\right)
\qquad
\psi_8=\begin{pmatrix}p_xc-ip_yc\\-p_zc\\0\\E+mc^2\end{pmatrix}
\exp\left(-\frac{i\phi}{\hbar}\right)
\end{equation*}
}

\noindent
The negative wave solutions flip the sign of the $mc^2$ term.

\bigskip
\noindent
The following solutions are used in quantum electrodynamics.
\begin{center}
\begin{tabular}{ll}
$\psi_1$ & fermion, spin up\\
$\psi_2$ & fermion, spin down\\
\\
$\psi_7$ & anti-fermion, spin up\\
$\psi_8$ & anti-fermion, spin down
\end{tabular}
\end{center}

\noindent
Here is a check of physical units.
The momenta $p_x$, $p_y$, and $p_z$ have units of
\begin{equation*}
\frac{\text{kilogram}\,\text{meter}}{\text{second}}
\end{equation*}

\noindent
Hence
\begin{equation*}
p_xx\propto\frac{\text{kilogram}\,\text{meter}^2}{\text{second}}
\end{equation*}

\noindent
For the time-dependent term
\begin{equation*}
Et\propto\frac{\text{kilogram}\,\text{meter}^2}{\text{second}^2}\times\text{second}
=\frac{\text{kilogram}\,\text{meter}^2}{\text{second}}
\end{equation*}

\noindent
We have for the reduced Planck constant
\begin{equation*}
\hbar\propto\frac{\text{kilogram}\,\text{meter}^2}{\text{second}}
\end{equation*}

\noindent
Hence $\phi/\hbar$ is dimensionless as required by the exponential function.
\begin{equation*}
\frac{p_xx-Et}{\hbar}\propto\frac{\text{kilogram}\,\text{meter}^2}{\text{second}}
\times\frac{\text{second}}{\text{kilogram}\,\text{meter}^2}=1
\end{equation*}

\noindent
The derivatives introduce inverse units.
\begin{equation*}
\frac{\partial\psi}{\partial t}\propto\frac{1}{\text{second}}
\qquad
\frac{\partial\psi}{\partial x}\propto\frac{1}{\text{meter}}
\end{equation*}

\noindent
Hence
\begin{equation*}
\frac{\hbar}{c}\frac{\partial\psi}{\partial t}
\propto\frac{\text{kilogram}\,\text{meter}}{\text{second}}
\end{equation*}

\noindent
and
\begin{equation*}
\hbar\frac{\partial\psi}{\partial x}\propto\frac{\text{kilogram}\,\text{meter}}{\text{second}}
\end{equation*}

\noindent
The resulting units match the right-hand side of the Dirac equation.
\begin{equation*}
mc\propto\frac{\text{kilogram}\,\text{meter}}{\text{second}}
\end{equation*}

\end{document}
